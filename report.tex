\documentclass[12pt]{muthesis}
\usepackage{graphicx}
\usepackage{a4wide}
\usepackage[dvipsnames]{xcolor}
\usepackage{color}
\tolerance=1
\emergencystretch=\maxdimen
\hyphenpenalty=10000
\hbadness=10000
\usepackage[square,sort,comma,numbers]{natbib} %References
\usepackage{amsmath,amssymb,amsthm,hyperref,enumitem}
\usepackage{graphicx}
\usepackage{subcaption}
\usepackage{changepage}
\newtheorem{theorem}{Theorem}[section]
\newtheorem{corollary}{Corollary}[theorem]
\newtheorem{lemma}[theorem]{Lemma}
\usepackage{caption}
\usepackage{wrapfig}
\usepackage{tikz,pgfplots}
\usepackage{pgf}
\usetikzlibrary{arrows}
\setcounter{tocdepth}{2}
\setcounter{secnumdepth}{3}
\usepackage{titlesec}
\usepackage[T1]{fontenc}
\usepackage[utf8]{inputenc}
\usepackage{lmodern}
\usepackage{pdfpages}
\usepackage{minted}
\newcommand{\ie}{\textit{i.e.}\ }
\newcommand{\eg}{\textit{e.g.}\ }
\newcommand{\LRP}{\textsc{lrp}}
\newcommand{\CTC}{\textsc{ctc}}
\newcommand{\CTN}{\textsc{ctn}}
\newcommand{\RAP}{\textsc{rap}}
\newcommand{\REVEAL}{\textsc{Reveal}}
\newcommand{\DBSCAN}{\textsc{Dbscan}}
\newcommand{\SAM}{\textsc{Sam}}
% \usepackage[ruled,vlined]{algorithm2e}
% \usepackage{algpseudocode}

\usepackage{algpseudocode}
\newcommand{\abs}[1]{\left|{#1}\right|}
% \newcommand{\passto}{\hookrightarrow}
\newcommand{\NN}{\mathcal{N}}
\newcommand{\HM}{\mathcal{H}}

\titlespacing*{\section}{0pt}{0\baselineskip}{0\baselineskip}
\titlespacing*{\subsection}{0pt}{0\baselineskip}{0\baselineskip}
\titlespacing*{\subsubsection}{0pt}{0\baselineskip}{0\baselineskip}
\newcommand{\bbR}{\mathbb{R}}
\newcommand{\passto}{\hookrightarrow}
\newcommand{\bbH}{\mathbb{H}}
\newcommand{\act}{\varphi}
%\newcommand{\net}[1][j]{\mathit{net}_{#1}}
\newcommand{\net}[1][k]{\vec{net}_{#1}}
\newcommand{\norm}[1]{\|{#1}\|}
\newcommand{\dom}[1]{\mathrm{dom}(#1)}
\newcommand{\HMhat}{\hat{\HM}}
\newcommand{\HMtilde}{\tilde{\HM}}
\newcommand{\ReLU}{\mathsf{ReLU}}
\newcommand{\Htilde}{\tilde{H}}
\newcommand{\Hhat}{\hat{H}}
\newcommand{\tr}[1]{#1^\intercal}
\setlength{\parskip}{\baselineskip}
\usepackage{multirow}
\titleformat{\paragraph}
{\normalfont\normalsize\bfseries}{\theparagraph}{1em}{}
\titlespacing*{\paragraph}
{0pt}{3.25ex plus 1ex minus .2ex}{1.5ex plus .2ex}
\setlist[description]{leftmargin=\parindent,labelindent=\parindent}
% Default fixed font does not support bold face
\DeclareFixedFont{\ttb}{T1}{txtt}{bx}{n}{10} % for bold
\DeclareFixedFont{\ttm}{T1}{txtt}{m}{n}{10}  % for normal

% Custom colors
\usepackage{color}
\definecolor{deepblue}{rgb}{0,0,0.5}
\definecolor{deepred}{rgb}{0.6,0,0}
\definecolor{deepgreen}{rgb}{0,0.5,0}

\usepackage{listings}

% Python style for highlighting
\newcommand\pythonstyle{\lstset{
language=Python,
basicstyle=\ttm,
otherkeywords={self},             % Add keywords here
keywordstyle=\ttb\color{deepblue},
emph={MyClass,__init__},          % Custom highlighting
emphstyle=\ttb\color{deepred},    % Custom highlighting style
stringstyle=\color{deepgreen},
frame=tb,                         % Any extra options here
showstringspaces=false            %
}}


% Python environment
\lstnewenvironment{py_excerpt}[1][]
{
\pythonstyle
\lstset{#1}
}
{}

% Python for external files
\newcommand\pythonexternal[2][]{{
\pythonstyle

\lstinputlisting[#1]{#2}}}

% Python for inline
\newcommand\py[1]{{\pythonstyle\lstinline!#1!}}
\lstset{showstringspaces=false}
\usepackage{ragged2e}
\justifying
\usepackage{empheq}

%% Margin notes from CH and ML
\usepackage{todonotes}
\newcommand{\chris}[1]{\todo[color=Purple!40]{\footenotesize CH: #1}}
\newcommand{\mike}[1]{\todo[color=red!40]{\footenotesize ML: #1}}
\newcommand{\chrislong}[1]{\todo[color=Purple!40,inline]{\footenotesize CH: #1}}

\usepackage{amsthm}
\theoremstyle{definition}
\newtheorem{definition}{Definition}[section]
\usepackage{tabularx}
\usepackage[utf8]{inputenc}
\usepackage[T1]{fontenc}
\newcommand{\cnet}[1][k]{\text{$c$-}\net[#1]}
\setstretch{1.5} 
% \usepackage{nomencl}
% \makenomenclature

% \renewcommand{\nomname}{Notation}

% \usepackage{etoolbox}
% \renewcommand\nomgroup[1]{%
%   \item[\bfseries
%   \ifstrequal{#1}{P}{Physics Constants}{%
%   \ifstrequal{#1}{N}{Number Sets}{%
%   \ifstrequal{#1}{O}{Other Symbols}{}}}%
% ]}
\usepackage[framemethod=TikZ]{mdframed}
\usepackage{amsmath}
\usepackage{algorithm}
\usepackage{algpseudocode}
\usepackage{cleveref}
\usepackage[most]{tcolorbox}

\usepackage{amsthm}
% Define the custom Definition environment
\Crefformat{tcb@cnt@Definition}{Definition~#2#1#3}

\newtcbtheorem[auto counter, number within=chapter]{Definition}{Definition}{
  enhanced,
  attach boxed title to top left={
    yshift=-0.1in,xshift=0.15in
  },
  colback=white,
  colframe=blue!25,
  fonttitle=\bfseries,
  coltitle=black,
  boxed title style={
    sharp corners,
    size=small,
    colback=blue!10,
    colframe=blue!10,
  }
}{def}

\newtcbtheorem[auto counter, number within = chapter]{Desideratum}{Desideratum}{
  breakable,
  enhanced,
  attach boxed title to top left={
    yshift=-0.1in,xshift=0.15in
  },
  colback=white,
  colframe=red!25,
  fonttitle=\bfseries,
  coltitle=black,
  boxed title style={
    sharp corners,
    size=small,
    colback= red!10,
    colframe=red!10,
  } 
}{prf}


\newtcbtheorem[auto counter, number within = chapter]{Conjecture}{Conjecture}{
  breakable,
  enhanced,
  attach boxed title to top left={
    yshift=-0.1in,xshift=0.15in
  },
  colback=white,
  colframe=orange!25,
  fonttitle=\bfseries,
  coltitle=black,
  boxed title style={
    sharp corners,
    size=small,
    colback= orange!10,
    colframe=orange!10,
  } 
}{prf}


\newtcbtheorem[auto counter, number within = chapter]{Example}{Example}{
  breakable,
  enhanced,
  attach boxed title to top left={
    yshift=-0.1in,xshift=0.15in
  },
  colback=white,
  colframe=ForestGreen!25,
  fonttitle=\bfseries,
  coltitle=black,
  boxed title style={
    sharp corners,
    size=small,
    colback= ForestGreen!20,
    colframe=ForestGreen!20,
  } 
}{prf}

\usepackage{longtable}
\usepackage{footnote}
\usepackage{etoolbox}
\BeforeBeginEnvironment{Example}{\savenotes}
\AfterEndEnvironment{Example}{\spewnotes}

%\renewcommand{\chris}[1]{}
%\renewcommand{\mike}[1]{}

%%%%%%%%%%%%%%%%%%%%%%%%%%%%%%%%%
% Front Matter - project title, name, supervisor name and date
%%%%%%%%%%%%%%%%%%%%%%%%%%%%%%%%%


% Word count with: texcount template.tex -inc -incbib -sum
% Spell check with: aspell -t -c template.tex

\hbadness=10000
\makeindex

\begin{document}
\title{Towards More Interpretable and Faithful Explanations in Deep Neural Networks}
\author{Sophia Vesselinova Kalanovska}
\date{December 2023}
\dept{Department of Informatics}
\faculty{Natural and Mathematical Sciences}
\university{King's College London}
\crest{KCL_crest_hires.pdf}
\maketitle


\prefacesection{Abstract}
Deep Neural Networks (DNNs) have gained significant attention for their ability to learn features from data, leading to improved performance across various tasks. However, the features they learn are often not understandable to humans, making it challenging to ensure that the models' decisions are not based on irrelevant patterns in the data. This has sparked a growing interest in making DNNs more interpretable. Yet, attempts to explain these models often face a trade-off between accuracy of the explanation (faithfulness) and ease of understanding (interpretability). Explanations that closely reflect the model's internal workings tend to be too complex for human comprehension, while simpler explanations may not accurately represent the model's behavior.

\noindent
This issue is particularly pronounced with image data due to its high dimensionality. Images consist of thousands to millions of pixels, each contributing subtly to the overall picture, making it difficult to determine which features most influence the model's decisions. Existing faithful methods assign importance scores to every individual pixel, resulting in detailed but overwhelming explanations. Conversely, more interpretable methods simplify this by reducing features to basic importance levels, which can obscure critical information.

\noindent
This thesis introduces a method to enhance the ease of interpretability without sacrificing faithfulness. By grouping  features together and assigning a single importance value to each group, the approach reduces the number of elements one needs to consider, making explanations more manageable. The more challenging issue is assigning a single value to a group of features that is representative of the importance of that group given the model’s function.

\noindent
To achieve this, two sets of rules for propagating importance for a group of features is proposed. The first set reverses the well-known technique of Layer-wise Relevance Propagation (LRP). While theoretically sound, this approach is prohibitively computationally demanding. In response, this thesis presents a second comprehensive set of propagation rules that distribute importance through network layers by focusing only on the grouped features. This required adjusting the scaling of learned parameters, like biases, to prevent the importance signals from becoming too large or too small, which could distort the explanations. The proposed method keeps the contributions within the normal range of layer outputs, ensuring a balanced influence of all parameters.

\noindent
The effectiveness of this method is demonstrated using large-scale, pre-trained convolutional neural networks, comparing performance with existing relevance techniques. The results indicate a promising solution for interpreting deep neural networks, offering valuable insights into their inner workings while maintaining both human comprehensibility and fidelity to the original model.

\prefacesection{Acknowledgements}

First and foremost, I would like to thank to my partner, Charles Higgins. Without your constant support and encouragement, the completion of this thesis would not have been possible. I find it hard to sufficiently express just how much your help has meant to me and how thankful I am to have you by my side.

\noindent
My supervisor, Dr. Christopher Hampson, deserves special mention for his support and readiness to share his extensive knowledge. His numerous enlightening whiteboarding sessions have been instrumental in shaping my academic development. Further thanks are owed to Dr. Daniele Magazzeni for giving me the opportunity to do a PhD and Dr. Michael Luck for his critical role in overseeing part of my PhD journey.

\noindent
I am grateful for the support provided by the Center for Doctoral Training in Safe and Trusted AI, which offered me a sponsorship essential for undertaking this PhD.

\noindent
I also extend my thanks to the undergraduate and postgraduate students I have had the privilege of teaching over the years. Your presence and support have been crucial in making these past few years not just productive, but also enjoyable. 

\noindent
Above all, I owe a heartfelt thanks to my family for their endless support and understanding throughout my long education. To my parents, whose belief in me has been unwavering from my earliest days in school to the completion of this PhD. Your faith and love have been a constant source of motivation and strength.

\noindent
Thank you all for your part in this journey.
\contentspage

% \printnomenclature

%%%%%%%%%%%%%%
% Report Content
%%%%%%%%%%%%%%%%%%%%%%%%%%%%%%%%%
% You can write each chapter directly here or in a separate .tex file and use the include command.
 % Include this in the document preamble


\prefacesection{Summary of Notation}

\begin{longtable}{p{0.2\textwidth} p{0.8\textwidth}}

    \hline
    \textbf{Symbol} & \textbf{Meaning} \\
    \hline
    \endfirsthead

    \hline
    \textbf{Symbol} & \textbf{Meaning} \\
    \hline
    \endhead
    \endfoot
    \endlastfoot

$\mathbb{R}$ & Set of real numbers \\
$\mathbb{N}$ & Set of natural numbers \\
$\NN$ & Neural network \\
$\Lambda$ & Set of layers in the neural network \\
$k$, $j$ & Indices over layers \\
$N$ & Output layer index \\
$\lambda_i$ & Function of layer $i$ \\
$f_k$ & Function associated with layer $k$ \\
$n_j$, $m_k$ & Input and output dimensions of layer functions \\
$j \passto k$ & Layer precedence relation (layer $j$ precedes layer $k$) \\
$[\vec{a}_j]_{j \passto k}$ & Activations from layers preceding layer $k$ \\
$[\vec{c}_j]_{j \passto k}$ & Contributions from layers preceding layer $k$ \\
$\vec{x}$ & Input vector to a neuron or layer \\
$x_i$ & $i$-th component of $\vec{x}$ \\
$d$ & Dimension of input vector or neuron \\
$\vec{w}_j$ & Weight vector for neuron $j$ \\
$w_{ji}$ & Weight from input $x_i$ to neuron $j$ \\
$w_{j0}$ & Bias term for neuron $j$ \\
$W_j$ & Weight matrix from layer $\Lambda_j$ to $\Lambda_k$ \\
$W_k$ & Weight matrix at layer $k$ \\
$W_j^\intercal$ & Transposed weight matrix from $\Lambda_j$ to $\Lambda_k$ \\
$\vec{b}_k$ & Bias vector at layer $k$ \\
$t$ & Transfer function \\
$t_i$ & Transfer function of neuron $i$ \\
$\act$ & Activation function \\
$\act_i$ & Activation function of neuron $i$ \\
$\net$ & Output of transfer function of neuron k \\
$y_j$ & Output of neuron $j$ \\
$\vec{a}_j$ & Activation vector at layer $\Lambda_j$ \\
$X$ & Input to a layer or function \\
$W$ & Kernel (filter) in convolution \\
$\ast$ & Convolution operation \\
$b$ & Bias term in convolution \\
$s_x$, $s_y$ & Stride values (horizontal and vertical) \\
$f_h$, $f_w$ & Filter height and width \\
$\operatorname{Conv}$ & Convolution operation \\
$t_{conv}$ & Convolution transfer function \\
$\operatorname{MaxPooling}$ & Max pooling operation \\
$t_{max}$ & Max pooling transfer function \\
$\operatorname{AvgPooling}$ & Average pooling operation \\
$t_{avg}$ & Average pooling transfer function \\
$d_1, d_2, \ldots, d_n$ & Dimensions of a feature map or tensor \\
$H$, $W$, $D$ & Height, width, and depth of image data \\
$F$ & Input feature map or complex input feature \\
$F_i$ & Feature vector at layer $i$ \\
$F_{i,j,k}$ & Value at position $(i,j,k)$ in a feature map \\
$i$, $j$, $c$ & Output indices (spatial and channel) \\
$\mathcal{P}_l$ & Set of pixels in a pooling patch \\
$|\mathcal{P}_l|$ & Number of pixels in $\mathcal{P}_l$ \\
$\operatorname{ReLU}$ & Rectified Linear Unit activation function \\
$\operatorname{tanh}$ & Hyperbolic tangent activation function \\
$\operatorname{sigmoid}$ & Sigmoid activation function \\
$\epsilon$ & Small constant for numerical stability \\
$\gamma$, $\beta$ & Scale and shift parameters in batch normalization \\
$\mu$, $\sigma^2$ & Mean and variance in batch normalization \\
$\sigma$ & Standard deviation in batch normalization \\
$\hat{x}$ & Normalized input in batch normalization \\
$t_{norm}$ & Batch normalization transfer function \\
$\sigma_x^2$ & Variance of input $\vec{x}$ in batch normalization \\
$\vec{R}_k$, $R_j$ & Relevance vectors at layer $\Lambda_k$, $\Lambda_j$ \\
$R_j^\prime$ & New relevance vector at layer $\Lambda_j$ \\
$R_{jk}$, $R_{jk}^\prime$ & Relevance matrices between layers $\Lambda_j$ and $\Lambda_k$ \\
$P_{jk}$ & Proportion matrix of relevance from $\Lambda_j$ to $\Lambda_k$ \\
$\alpha$, $\beta$ & Parameters in the Alpha-Beta rule in LRP \\
$R_j^+$, $R_j^-$ & Positive and negative relevance components at $\Lambda_j$ \\
$\vec{c}_j$, $\vec{c}_k$ & Contribution vectors at layers $j$, $k$ \\
$\vec{c}_0$ & Contribution vector at the input layer \\
$\cnet^-$ & Contribution net input before bias at layer $k$ \\
$P_k$ & Scaling matrix for adjusting bias at layer $k$ \\
$\vec{s}_k$ & Relevance scaling factors at layer $\Lambda_k$ \\
$\vec{t}_k$ & Transformation function from $\Lambda_j$ to $\Lambda_k$ \\
$J_{kj}(\vec{x})$ & Jacobian matrix of $\vec{t}_k$ with respect to $\vec{x}$ \\
$\vec{m}$, $\vec{m}_k$ & Mask vectors over inputs or at layer $k$ \\
$Q_k$ & Ratio of new to old relevance at layer $\Lambda_k$ \\
$\odot$ & Element-wise multiplication operator \\
$||\cdot||$ & Element-wise absolute value function \\
$\nabla$ & Gradient operator \\
$\vec{1}$, $\vec{0}$ & Vectors of ones and zeros \\
\end{longtable}


\prefacesection{Summary of Terms and Acronyms}

\begin{longtable}{p{0.25\textwidth} p{0.75\textwidth}}

    \hline
    \textbf{Term/Acronym} & \textbf{Meaning} \\
    \hline
    \endfirsthead

    \hline
    \textbf{Terms/Acronym} & \textbf{Meaning} \\
    \hline
    \endhead
    \endfoot
    \endlastfoot
        ANN & Artificial Neural Network \\
        DNN & Deep Neural Network \\
        MLP & Multi-Layer Perceptron \\
        CNN & Convolutional Neural Network \\
        RNN & Recurrent Neural Network \\
        LSTM & Long Short-Term Memory network \\
        DQN & Deep Q-Network \\
        VGP & Vanishing Gradient Problem\\
        BN & Batch Normalisation \\
        IFM & Input Feature Map \\
        ICS & Internal Covariate Shift \\
        ReLU & Rectified Linear Unit activation function\\ 
        \text{Leaky~ReLU} & Leaky Rectified Linear Unit activation function \\
        LRP & Layer-wise Relevance Propagation method \\
        VGG16 & Visual Geometry Group 16-layer network (a CNN architecture) \\
        OOM & Out of Memory Error \\
        Jacobian Matrix & Matrix of all first-order partial derivatives of a vector-valued function \\
        AD & Automatic Differentiation \\
        CTC & Contribution to Classification \\
        CTN & Contribution to Neuron \\
        HM & Relevance heatmap \\
        SAM & Segment Anything Method \\
        DBSCAN & Density-Based Spatial Clustering of Applications with Noise \\
\end{longtable}


\chapter{Introduction}
\chaptermark{Introduction}
\label{chap:intro}
\justifying
\setlength{\parindent}{0em}
% \textbf{The primary goal of this PhD is to create a method for generating explanations of a deep neural network’s (DNN's) classification that are both human-understandable (interpretable) and accurate (faithful) with respect to how  the  network reached  the classification}

\section{Machine Learning}
In the early days of \textit{artificial intelligence} (AI), the field was concerned with solving problems that are well-defined, formal and can be expressed with mathematical rules~\cite{bernstein1958computer, minsky1956heuristic, minsky1962problems}. These types of problems are computationally and intellectually difficult for humans to perform, which is why the field of artificial intelligence was interested in solving them. However, despite these types of problems posing difficulties for humans, they were easy to encode, so were relatively uncomplicated for a computer to reason about.

The challenge of current AI systems is to perform tasks that come intuitively to humans (\eg recognising objects and faces, interpreting speech, text generation \textit{etc.}). Such tasks are difficult to perform as they require a huge amount of knowledge about the world. This makes it difficult to formulate the problem in a well-defined mathematical way with enough complexity that allows the system to reason about the information and infer knowledge from the input. Finding ways of encoding the environment in an accurate and detailed way is still an active area of research. 

A way to avoid this problem of \textit{knowledge engineering} is to let the AI algorithm obtain its own knowledge through learning from experience. These types of AI systems are referred to as \textit{machine learning} (ML) systems. Machine learning algorithms aim to obtain their own knowledge by \textit{extracting information} from data, \textit{recognising useful patterns} in data or \textit{making decisions} based on data. The ease of application of a number of ML techniques comes from each system's ability to perform specific decision-making tasks without requiring explicit human instruction. This makes them applicable to a wide range of areas~\cite{forthcoming} from content filtering and recommendations on social networks and e-commerce websites~\cite{de2010combining} to protein folding~\cite{varadi2022alphafold}, autonomous vehicle control~\cite{kuutti2020survey}, financial markets~\cite{cavalcante2016computational} and cancer diagnostics~\cite{huang2020artificial}. ML techniques have existed since the 1960s~\cite{minsky1969introduction, samuel1959some, nilsson1965learning}, but it was not until recently (circa 2010) that the integration into ML systems into wider society became noteworthy. This is mainly due to the the increase of available information, as well as other developments, including hardware improvements and new optimisation algorithms.  
\section{Feature Engineering Problem}
\label{featureeng}
Powerful ML models trained on large amounts of data can achieve great performance, and on some cognitive tasks record results on a par with humans~\cite{brown2017libratus, zhai2023can, silver2017mastering}. Many tasks, previously thought to be so computationally demanding as to be unattainable, such as image detection and recognition~\cite{HeZRS16}, strategic game planning~\cite{SilverHMGSDSAPL16} and natural language processing~\cite{DengHK13}, have seen rapid development~\cite{LeCunBH15}. 

To perform such tasks well, machine learning systems need a large amount of data to learn how to perform a specific task. The performance of all machine learning algorithms is strongly influenced by the representation and structure of this data. Hence, after the necessary data is collected some aspects, qualities, or characteristics of the data are often chosen that make the task easier to learn. These are often referred to as \textit{features} and finding the right set which allow the algorithm to learn a task well is a difficult and intricate process. Moreover, finding such features can be a waste of human time and effort, as features for more complicated tasks may take decades and a community of researchers to design~\cite{goodfellow2016deep}. 

To overcome this, some machine learning algorithm not only learn a mapping from input to output, but also the features they should focus on from the input. This is called \textit{representation learning} and the type of machine learning algorithms that are capable of performing representation learning are various types of \textit{artificial neural networks (ANN)}.  

\section{Black-box Nature of Artificial Neural Networks}

Algorithms that learn their own feature representation (\ie ANN) often perform much better than algorithms that use human-deigned features~\cite{goodfellow2016deep}. However, the problem with leaving ANNs to learn their own features is that the knowledge learned and encoded in the network is not comprehensible and that makes the decisions taken by neural networks hard to trust. 

Currently most ANNs are deployed and expected to behave as intended if they show a high degree of predictive accuracy on test data. However, in some cases, the high predictive accuracy of  neural networks can be the result of erroneous exploration of artefacts in the data (\ie the presence of systematic bias in data that the system bases its decision upon) rather than the result of correctly identified parts of the input that \textit{should} lead to the decision~\cite{leek2010tackling, SzegedyZSBEGF13, corr, taylor2006methods}. AI systems that exploit such spurious correlations from the input to make their decisions are referred to as ``Clever Hans'' Predictors~\footnote{Clever Hans was a horse that was believed to be able to count and was declared a scientific marvel in the years around the 1900s. Later it turned out that the horse did not learn how to count, but was rather deriving the answer from the questioner's reactions~\cite{lapuschkin2019unmasking}}. Moreover, since accuracy is calculated on test data, there is no guarantee that once deployed on real-world data it will remain high. Hence the lack of transparent decision making in deep learning systems not only limits their trustworthiness, but also poses problems to their \emph{acceptance} and \emph{applicability}, due to concerns over poor performance in safety critical applications, potentially leading to complex enquiries regarding \emph{accountability} (\ie who or what is at fault) and industrial \emph{liability} (\ie who or what is liable for costly mistakes made by an algorithm)~\cite{NguyenHHTK14, abs, kucharski2016study, WolfMG17}. The problem of \emph{non-transparency} of neural networks is even more pressing in life-critical situations where it is essential to robustly verify a system's decision-making process (\eg in areas including medicine and the criminal justice system, etc.)~\cite{CaruanaLGKSE15, BojarskiYCCFJM17}.

In recent years there has been growing recognition beyond academia of the need to develop explainable machine learning methods. In an attempt to prevent the problems that could occur from the integration and the rising importance of uninterpretable algorithms in a wide range of areas, the European Union (EU) introduced a right to explanation in the General Data Protection Regulation Act (GDPR)~\cite{GoodmanF17} in May 2018. The EU further introduced an EU’s Artificial Intelligence Act, which enforces that high-risk AI systems shall be designed and developed in such a way to ensure that their operation are sufficiently transparent to enable users to interpret the system’s output and use it appropriately. While the legal requirement for explanations is recent and only in specific geographic areas (the EU), the term explainable artificial intelligence was coined in 2004~\cite{LentFM04} and the problem of creating explanations in AI can be traced back to the mid-1970s~\cite{moore1988explanation}. 

\section{ Artificial Neural Networks Explainability Methods}

To develop methods which will make neural networks deemed \emph{trusted}, one must have clear criteria of when they have succeeded. However, trust is a concept that is difficult to quantify and formalize~\cite{doshi2017towards, Lipton18}, making it challenging to use as a guiding principle in developing explainable and safe machine learning algorithms.Consequently, the primary focus has shifted to achieving human-understandable (\textit{interpretable}) explanations that are also accurate (\textit{faithful}) representations of the classifier's decision-making process. Achieving these properties is a crucial stepping stone for building trust~\cite{Lipton18}. The research area that addresses the problem of neural networks explainability is part of the broader area of explainable artificial intelligence (XAI).

The techniques developed for explaining neural networks fall within two broad categories of intrinsic explainability and post-hoc explainability~\cite{ArrietaRSBTBGGM20, DuLH20, IbrahimS23}(see Chapter~\ref{chap:lit}). Intrinsic explainability methods involve designing models that are interpretable by nature, while post-hoc explainability techniques seek to interpret models after training without altering their internal structures. Intrinsic methods often face a trade-off between interpretability and performance. Highly interpretable models may not achieve the same level of accuracy as complex ones. Post-hoc explainability techniques aim to interpret models after they have been trained without altering their internal structure. The challenge is that post-hoc methods may produce explanations that lack fidelity, meaning they do not accurately reflect the true reasoning of the model. This lack of fidelity can result in misleading interpretations, undermining the trustworthiness of the explanations.

Balancing interpretability and faithfulness in explanations is a significant challenge in the field. Currently post-hoc explainability methods, which are interpretable create abstractions that lower the faithful of the interpretation. Conversely, methods that prioritise faithfulness tend to decrease interpretability by overwhelming the user with excessive information. Interpreting models that process high-dimensional inputs, such as images, requires simplifying vast amounts of information without compromising the faithfulness of the explanation.

Given these challenges, this research seeks to bridge the gap by developing techniques that simplify explanations through the grouping of correlated input features. This approach aims to make explanations more accessible to human understanding while accurately reflecting the model's decision-making process.

\section{Thesis Aims}

The primary aim of this thesis is to develop methods that enhance the interpretability of neural network models without compromising the fidelity of their internal decision-making processes.

Human cannot comprehend more than three to five meaningful items at once~\cite{cowan2001magical, starkey1995development, morris2018human}. Therefore, interpretability can be enhanced by presenting fewer features in explanations.

In high-dimensional inputs such as images, simply showing the most important features may not be effective, as these features might be strongly correlated. For example, when looking at a picture of a cat, no single feature (i.e., pixel) is more important than any other pixel in isolation. However, when combined, they form complex features (e.g., ears), which are significant in identifying the overall image as that of a cat.

This thesis proposes developing an approach that groups correlated input features into \emph{complex input features}, thereby reducing the complexity of explanations. These features must be meaningful to the explainee and significant for the neural network.

Some existing methods assign a value of relevance to groups of features~\cite{Ribeiro0G16, LundbergL17}. However, these methods often create linear approximations of the decision boundary for those features and, therefore, suffer from fidelity issues. This thesis aims to propose rules for assigning importance to complex input features without oversimplifying the neural network's decision boundary. 

The method should assign a single importance value to each complex input feature, ensuring that the explanation reflects the model's true reasoning. Furthermore, the method should be computationally feasible for large-scale neural networks, enabling practical applicability.

The thesis further aims to evaluate the efficacy of the proposed method compared to other widely used explainability methods.

\section{Thesis Structure and Contributions}

This thesis is organised into several chapters, each building toward the goal of enhancing the interpretability of neural networks, with a particular focus on convolutional neural networks (CNNs) while maintaining fidelity. Below is an overview of the thesis structure and the main contributions of each chapter.

\subsection{Chapter 1: Introduction}

Chapter~\ref{chap:intro}, introduces the motivation behind the research, the importance of explainability in neural networks, and outlines the thesis aims and objectives.

\subsubsection*{Contribution of the Chapter}
It sets the stage for the thesis by highlighting the need for methods that enhance interpretability by preserving fidelity.

\subsection{Chapter 2: Background}

In Chapter~\ref{chap:background}, the foundational knowledge on artificial neural networks is provided, focusing on their mathematical foundations and computational structures. It covers essential concepts such as neurons, layers, activation functions, and delves into convolutional neural networks (CNNs) and their common layers.

\subsubsection*{Contribution of the Chapter}
The Chapter establishes the necessary background for understanding the proposed rules for assigning importance to complex input features from Chapters~\ref{chapter:revLRP} and Chapters~\ref{chapter:REVEAL} and prepares the reader for the technical discussions in subsequent chapters.

\section*{Chapter 3: Literature Review}

This chapter provides a comprehensive review of interpretability methods for deep neural networks within the field of explainable artificial intelligence (XAI). The methods are categorized into two main groups:

\begin{enumerate}
    \item \textbf{Intrinsic Interpretability}: Focuses on designing inherently transparent models. Techniques discussed include:
    \begin{itemize}
        \item Adding interpretability constraints within neural networks~\cite{ZhangWZ18a, SabourFH17}.
        \item Model extraction methods~\cite{VandewieleJOTH16, BastaniKB17a}.
        \item Attention-based approaches~\cite{BahdanauCB14, XiaoXYZPZ15}.
        \item Loss-based methods that focus on specific input parts or learn concrete patterns~\cite{ShiXXCLLG21, LinsleySES19}.
    \end{itemize}
    However, intrinsic methods often face limitations like a trade-off between interpretability and performance.

    \item \textbf{Post-hoc Explanations}: Interpret complex, pre-trained models without altering their architecture or compromising performance~\cite{MarkusKR21, AdebayoGMGHK18}. These are further divided into:
    \begin{itemize}
        \item \textbf{Global Explanations}: Such as activation maximization~\cite{SimonyanVZ13, NguyenDYBC16} and methods for sequential data~\cite{KarpathyJL15, KadarCA17}, which help visualize preferred inputs or internal representations.
        \item \textbf{Local Explanations}: Particularly relevant to this thesis. Methods include:
        \begin{itemize}
            \item LIME~\cite{Ribeiro0G16} and SHAP~\cite{LundbergL17}, which create interpretable models approximating complex models near specific input features.
            \item Class activation mapping~\cite{ZhouKLOT16, SelvarajuCDVPB20}.
            \item Back-propagation-based methods like Layer-wise Relevance Propagation (LRP)~\cite{bach2015pixel}, detailed further in Chapter~\ref{chapter:revLRP} as a foundation for the proposed methods.
        \end{itemize}
    \end{itemize}
\end{enumerate}

The chapter identifies key properties essential for effective explanations:
\begin{itemize}
    \item \textbf{Fidelity}: Accuracy in reflecting the model's true reasoning~\cite{SundararajanTY17}.
    \item \textbf{Input Invariance}: Consistency in explanations for similar inputs~\cite{AnconaCOG19}.
    \item \textbf{Handling of Saturation}: Addressing issues where gradients may vanish~\cite{NielsenDRRB22}.
    \item \textbf{Sensitivity}: The explanation's responsiveness to changes in input~\cite{SundararajanTY17}.
\end{itemize}

Challenges such as achieving high interpretability without sacrificing fidelity are discussed~\cite{KindermansHAASDEK19, AdebayoGMGHK18}, along with the importance of computational efficiency for practical applications~\cite{GhorbaniAZ19}.

\subsubsection*{Contribution of the Chapter}

The literature review lays the groundwork for the thesis by:
\begin{itemize}
    \item Highlighting the limitations of existing interpretability methods, especially the trade-offs involved.
    \item Emphasizing the need for post-hoc local explanation techniques that do not compromise model performance.
    \item Identifying gaps in current research, particularly in balancing interpretability, fidelity, and computational efficiency for high-dimensional data like images~\cite{YehHSIR19}.
\end{itemize}

Building on these insights, the chapter justifies the thesis's focus on developing post-hoc local methods that enhance interpretability while maintaining fidelity, aligning with the overarching goals of advancing practical and effective XAI solutions.


\subsection{Chapter 4: Multi-faceted Clustering Approaches
for Isolating Complex Input Features}

In Chapter~\ref{chap:clustering}, novel clustering methods for grouping correlated input features into complex input features are introduces. It details the algorithms developed for clustering and their application to image data.

\subsubsection*{Contribution of the Chapter}

\begin{enumerate}
\item Provided a formal definition of \emph{Complex Input Features} for image data, establishing a theoretical foundation for subsequent methods and ensuring a clear understanding of what constitutes a complex feature within the context of neural network interpretability.
\item Introduced a novel heatmap-based clustering approach that uses relevance heatmaps to detect and cluster coherent groups of pixels. This method includes a two-stage filtering process to remove irrelevant features, thereby enhancing the quality and relevance of the clusters before applying DBSCAN for clustering. The approach was evaluated qualitatively using a range of interpretability techniques and neural networks, demonstrating its effectiveness in identifying meaningful feature groups.
\item Developed an enhanced method that integrates object detection algorithms with the heatmap-based clustering technique to improve human interpretability. This method identifies distinct objects or regions within an image and selects the most relevant ones based on heatmap-derived significance. It includes multiple augmentation steps to enhance the applicability of object detection for interpretability tasks and introduces three distinct techniques for combining object detection with heatmap-based clustering. This integration ensures that the identified clusters correspond to semantically meaningful objects, thereby bridging the gap between automated feature detection and human-understandable interpretations.
\item Implemented comprehensive preprocessing steps for object detection, including generating masks for undetected regions and resolving overlapping clusters. These steps ensure comprehensive coverage of the image and eliminate redundancies, resulting in more accurate and meaningful clustering outcomes. The use of Fisher-Jenks natural breaks and thresholding techniques further refines the selection of relevant pixels, enhancing the overall quality of the clusters.
 
\item Conducted qualitative evaluations across multiple neural network architectures (e.g., VGG16~\cite{SimonyanZ14a}, VGG19~\cite{SimonyanZ14a}, ResNet50~\cite{he2015deep}, InceptionV3~\cite{szegedy2015rethinking}) and interpretability techniques, showcasing the versatility and robustness of the proposed clustering methods. This broad evaluation highlights the method's applicability to various models and its ability to consistently identify meaningful feature clusters regardless of the underlying network architecture.
\end{enumerate}

The two-stage filtering process significantly reduces the number of pixels to be clustered, the approach remains effective even for large input. This ensures that the method can be applied to real-world, high-dimensional data without prohibitive computational costs.

By combining object detection with heatmap clustering, the chapter enhances the human interpretability of neural network decisions. The resulting clusters correspond to recognisable objects or regions, making it easier for users to understand and trust the model's predictions. It sets the foundations for next chapters, where the complex input features identified using the techniques in Chapter~\ref{chap:clustering} are used as the input for the rules defined in Chapter~\ref{chapter:revLRP} and Chapter~\ref{chapter:REVEAL}.

\subsection{Chapter 5: Relevance Distribution Tracing}

Chapter~\ref{chapter:revLRP} introduces a novel approach to reversing Layer-wise Relevance Propagation (LRP) to assign importance values to groups of features.

The chapter begins by examining the limitations of naive aggregation methods for relevance scores, noting how simple aggregation can obscure critical nuances and interdependencies among features. This limitation impacts the accuracy of interpretations by failing to capture the complex interactions within neural networks. Recognising the inherent interdependence of features, the chapter explains how transformations and interactions across neural network layers influence the relevance attributed to input features. Following this, a detailed explanation of the basic LRP rule is provided, with a step-by-step breakdown of the mathematical operations involved, establishing a foundation for the proposed reversal method.

\subsubsection*{Contribution of the Chapter}

\begin{enumerate} 
    \item It introduces the method of \emph{Reverse Relevance Distribution Tracing}, which reverses the LRP process to assign a single relevance value to each cluster of features. This novel approach enhances interpretability while preserving the original faithfulness of LRP.
    \item Vector-based definitions for LRP rules are presented, moving from neuron-level descriptions to a more generalized and accessible framework. A set of rules is developed to reverse the relevance propagated by basic LRP rules, enabling backward tracing of relevance distribution throughout the network.
    \item The method is further extended to incorporate the Alpha Beta rule, allowing for a more nuanced distribution of relevance scores. This extension improves adaptability across various neural network layers and architectures, accommodating both positive and negative feature contributions.
    \item The chapter critically evaluates the computational complexity and memory requirements of the proposed method, particularly focusing on the challenges posed by Jacobian calculations in large, multi-layered networks. Practical limitations, such as memory constraints encountered on high-capacity GPUs, especially with networks like VGG16, are also discussed.

\end{enumerate}
Potential strategies to mitigate these computational challenges are suggested, including simplifications of model architectures, approximation techniques, and the use of parallel processing. This discussion lays the groundwork for the following chapter, which introduces a more computationally efficient method.

\subsection{Chapter 6: Contribution to Classification of Complex Input Features}
Chapter~\ref{chapter:REVEAL} introduces a novel technique called \emph{Contribution to Classification} \CTC\/, which is an interpretability technique designed to accurately trace the influence of complex input features through a neural network without the computational burden associated with high-dimensional Jacobians. It further presents the results of applying the proposed method to pre-trained convolutional neural networks using large-scale image datasets. It includes qualitative and quantitative evaluations.

\subsubsection*{Contribution of the Chapter}

\begin{enumerate} 
\item A formal definition of contribution to classification (\CTC\/) of a feature is provided within the neural network context, explaining how to inductively compute contributions at each layer based on the activations and contributions of preceding layers.
\item A comprehensive set of propagation rules is developed for distributing contributions throughout various types of layers in a neural network, ensuring accurate tracing from input to output layers, including the first layer, masking after each layer, dense layers, convolutional layers, max pooling layers, batch normalization layers, concatenation layers, and average pooling layers.
\item Presented an in-depth analysis of the differences between the \CTC\/ of a feature and classifying a feature in isolation, highlighting how \CTC\/ offers a more granular understanding of feature influence compared to traditional relevance-based methods.
\item Demonstrated the necessity of carefully adjusting the scaling of learned parameters (e.g., biases) to prevent exponential growth or shrinkage of the contribution signal, which could lead to distorted or misleading interpretations, especially in deep networks and proposing a method that created a scaling matrix to keep contributions within the natural variability of the layer outputs.
\item Extension of the iNNvestigate Library: A major contribution of this thesis is the expansion of the iNNvestigate library to incorporate the newly developed \CTC\/ method. By integrating these techniques into a widely used GitHub repository, this work enhances the accessibility of this method.
\item Empirical Validation Across Major Networks: The \CTC\/ method is empirically validated using the ILSVRC 2012 dataset~\cite{ILSVRC15} across several well-known convolutional neural network architectures, including VGG16~\cite{SimonyanZ14a}, VGG19~\cite{SimonyanZ14a}, ResNet50~\cite{he2015deep}, InceptionV3~\cite{szegedy2015rethinking} and DenseNet121~\cite{huang2018densely}. This extensive evaluation shows the practical utility of these techniques, involving both qualitative and quantitative analyses. The results show that these methods significantly improve the interpretability of neural network decisions while preserving the faithfulness of the explanations.
\end{enumerate}

The chapter includes illustrative examples and figures to visually demonstrate key concepts, enhancing the accessibility and understanding of complex ideas presented. It further ensures computational efficiency and faithfulness of the \CTC\/ method by only requiring two passes through the network, one for classification and one for the explanation. The method further propagates contributions only through neurons active during the inference step, thereby providing accurate and faithful interpretations without unnecessary computational overhead.

\subsection{Chapter 7: Conclusion and Future Work}

Finally, Chapter~\ref{chapter:conclusion} summarizes the findings of the thesis, discusses the implications of the research, and outlines potential directions for future work.

\subsubsection*{Contribution of the Chapter}

\begin{enumerate}
    \item The chapter situates the research within the broader context of explainable AI, drawing parallels with existing methods and highlighting the unique contributions of the thesis.
    \item It outlines several avenues for future work, including enhancing feature isolation techniques, addressing computational challenges in reverse relevance distribution, extending forward pass retracing to other neural network architectures, and improving evaluation metrics.
    \item This thesis is drawing on research that underscores human cognitive limitations. These limitations, notably the difficulty in processing more than five significant items simultaneously, are highlighted in seminal works by Cowan~\textit{et al.}\cite{cowan2001magical}, Starkey\textit{et al.}\cite{starkey1995development}, and Morris\textit{et al.}~\cite{morris2018human}. Work by~\cite{Ribeiro0G16} which conducted studies with human subjects (graduate ML students and non-technical users from Mechanical Turk) to evaluate interpretability methods that show limited number of complex input features found that explanations of this sort are preferred by users, and are able to lead to better decision-making in model-selection, feature-engineering and identifying irregularities in model-classification. This thesis focuses on finding a faithful value for complex input features, the effects of the extra information introduced of the importance of each feature has not been evaluated. The final chapter suggests conducting user studies to assess the effectiveness and usability of the proposed methods compared to other post hoc local interpretability methods.
\end{enumerate}





\chapter{Artificial Neural Networks Background}
\label{chap:background}
\section{Introduction}

Neural networks have emerged from a combination of mathematical concepts, applied statistics and computational structures~\cite{du2013neural}. This chapter delves into the essential foundations of neural networks, shedding light on the motivation for their use, notation, basic principles, and architectural elements that underpin their functioning.

Neural networks, inspired by the human brain~\cite{arbib2003handbook, abraham2005artificial}, are computational models designed to \textit{extract information} from data, \textit{recognise useful patterns} in data or \textit{make decisions} based on data~\cite{basu2010use, perlovsky2001neural}. In this chapter the notational conventions commonly used to describe neural networks are defined. Subsequently, the fundamental building block of neural networks is explored: the neuron. The methods by which artificial neurons process information through weighted connections and activation functions is described.

This is followed by a discussion on layers, which are formed by interconnected neurons. The mathematical operations within each layer are shown and then the underlying principles of neural network training that lead to the network's ability to generalise from training data to unseen instances are presented. The role of loss functions as guides for optimisation is presented, and backpropagation --- the technique that computes gradient information to update the weights of the neurons connections --- is explained.

This chapter lays the foundation for the forthcoming discussions on explainability methods for neural networks in Chapter~\ref{chap:lit}.

\section{Neural Network Motivation and General Structure}

Artificial neural networks represent a simplified view of the brain. They consist of \textit{processing units}, called neurons, which communicate though \textit{weighted connections} (expressed as directed edges) (see Figure~\ref{fig:neural}). A neural network can have million or even billion of neurons, each one performing a simple computation (see Section~\ref{Sec:process}). A neuron's output of a computation is fed though the weighted connections into the subsequent connected neurons. These neurons in turn use that output as their input for the simple computation they perform. This operation, performed by all neurons jointly, implements a non-linear mapping from the input to the output and allows neural networks to perform complex tasks. 

Neurons are organised into sequential layers, forming the architecture of a neural network. Each layer is a function that receives input from the previous layer and produces output that serves as input for the subsequent layer. The layer's function is implemented by all neurons that form it. Figure~\ref{fig:neural} shows four functions connected in a chain $f_{0}$, $f_{1}$, $f_{2}$ and $f_{3}$, which form the overall neural network function $f(x) = f_{3}(f_{2}(f_{1}(f_{0}(x))))$, where $f_{0}$ is the input layer and $f_{3}$ is the final output layer. The length of the chain determines the depth of the model. 

\begin{Definition}{Neural Network}{}
A \emph{(trained) neural network} is a tuple $\NN=\big(\Lambda,\passto,(f_k)_{k\in \Lambda}\big)$ where $(\Lambda,\passto)$ is a directed graph over a set of \emph{layers} $\Lambda=\{0,\dots, N\}$, with a single source $0\in \Lambda$, referred to as the \emph{input layer} and single sink $N\in \Lambda$, referred to as the \emph{output layer}. Each $f_k:\bbR^{n_j}\to \bbR^{m_k}$ describes a vector transformation associated with each layer, with the dimensionality conditions that $m_k=\sum_{j\passto k} n_j$, for all $k=1,\dots N$.
%
We say that layer $j$ \emph{precedes} layer $k$ if $j\passto k$, and that layers $j$ and $k$ are \emph{consecutive}.
\end{Definition}
The network $\NN$ computes a function $f_{\NN}:\bbR^{d}\to \bbR^{d'}$ on inputs of dimension $d=n_0$ to output of dimension $d'=m_N$, by successively computing the output of each layer, starting from the input layer. 

The arrangement and function of layers and the number of neurons within each layer determine the network's capacity to learn and generalise from data. Layers in neural networks can be divided into two groups, \textit{visible layers} and \textit{hidden layer(s)}. Visible layers, namely the \textit{input layer}, which receives inputs from the environment (\textit{e.g.\ }an image of a cat to be classified) and the \textit{output layer}, which sends outputs to the environment (\textit{e.g.\ }the label ``bengal cat'' --- the classification of the input). The hidden layers are the layers between the input and the output and they can be any natural number $n \in \mathbb{N}$ (see Figure~\ref{fig:wide} B). 

\begin{figure}[ht!]
	\begin{center}
		\includegraphics[width=\linewidth]{Figures/neural_network.pdf}
	\end{center}
	\caption{The image represents a five-layer neural network. The input layer receives the vector [x1, x2, x3, x4], which passes through three hidden layers connected through weighted connections, demonstrating the complex data processing within the network. The final output layer computes the vector [y1, y2, y3].}
	\label{fig:neural}
\end{figure} 

A network without hidden layers (\ie a two-layer neural network with only an input and an output layer) can only compute linear mappings between the input and the output. In order to deal with more complex tasks artificial neural networks have to be able to encode complex mappings (\ie non-linear) between the input and the output. To express such mappings the network needs at least one hidden layer and enough parameters in the hidden layer(s) (\ie neurons and weighted connections between neurons) to allow for the information to be learned and later stored and used. 


In principle, artificial neural networks can implement any computable function using only three-layers (\ie a network containing only one input, hidden and output layer)~\cite{Hornik91}. However, given that the input layer is fixed, as the input the network receives should always be the same size and type, and the output layer is fixed to all the possible types of outputs the network can produce, the only layer left to learn the task is the hidden layer. Therefore, to implement a complex task using only a three-layer neural network, the hidden layer of the network must contain a lot of neurons to allow for more parameters that will learn the task in question. The example below shows the increasingly more complicated decision boundary a neural network can learn, the more neurons are being added to the hidden layer of the network.


\begin{Example}{ANN decision boundary}{}
An example of how a network can have an increasingly more complex non-linear decision boundary, when more non-linear hidden neurons are added to its design is illustrated in Figure~\ref{Fig:decision}. The first sub-figure Figure~\ref{Fig:decision} (a) shows a non-linearly separable data set, where the light blue circles belong to class 1 and the dark blue circles to class 2. Figure~\ref{Fig:decision} (b), (c) and (d) illustrate the decision boundaries of neural networks with 1, 2 and 3 non-linear hidden neurons respectively, where the points in the purple region are classified by the neural network as belonging to class 1 and the points in the yellow regions as class~2. 
Figure~\ref{Fig:decision} (b) shows a neural network with a single non-linear hidden neuron and illustrates the linear decision boundary that this network has. Figure~\ref{Fig:decision} (c) shows the same neural network with one more non-linear hidden neuron ($\ie$ two in total) and illustrates the reduction in classification error and the overall non-linear (and discontinuous) decision boundary of that neural network. Finally, Figure~\ref{Fig:decision}~(d) shows a neural network with 3 non-linear hidden neurons that also achieves 100\% accuracy on this dataset.


\begin{figure}[H]
	\begin{center}
		\includegraphics[width=0.9\linewidth]{Figures/decision_bounderies.png}
	\end{center}
	\caption{Illustration of decision boundaries of a neural network where hidden neurons with a non-linear activation function are added incrementally (the non-linear activation function used is sigmoid (see Equation~\ref{eq:sigmoid}))}
	\label{Fig:decision}
\end{figure} 

\end{Example}

\subsection{Feature Representation}
Machine learning systems require an extensive volume of data to acquire the expertise needed for specific tasks. The performance of all machine learning algorithms is heavily contingent on the manner in which data is represented and structured. Finding the optimal set of features that enables the algorithm to grasp a task proficiently is challenging and stands as the main motivation for neural networks, which not only acquire a mapping from input to output, but also the pertinent features from the input.

Whenever the number of hidden layers of the network is small and the hidden layers have many neurons, then the neural network often memorises the training data rather than learns features from the input~\cite{PoggioMRML17}. The architecture of such neural network is referred to as wide (see Figure~\ref{fig:wide}~A). The limited amount of layers in such architectures prevents the network from learning concepts and patterns from the data and leads to a failure of the network to perform well (\textit{generalise}) to unseen samples. This problem is referred to as \textit{memorisation of the training data} or (\textit{overfitting}). To deal with overfitting the vast majority of neural networks used and designed are deep neural networks (DNNs) with a number of hidden layers (see Figure~\ref{fig:wide}~B). Neural networks that are created with more that one hidden layer, have the capability of learning their own complex features, capturing the natural hierarchy that many machine learning tasks have. They can learn complicated concepts by building them out of simpler ones, which allows for complex mappings to be learned without memorisation of the training data. An example of this hierarchy of abstractions can be seen in the process of arriving at a class label for an image recognition task. The input layer recognises pixels, the hidden layers following the input layer can learn to recognise edges in the image and deeper into the network the algorithm may detect more complex things such as object and textures, until it finally arrives at a label that describes the object identified. 


\begin{figure}[ht!]
	\begin{center}
		\includegraphics[width=0.9\linewidth]{Figures/neural_network_wide_deep.pdf}
	\end{center}
	\caption{Illustration of a wide neural network and a deep neural network.}
	\label{fig:wide}
\end{figure} 


Many machine learning algorithm need to have their features selected (a process commonly referred to as feature engineering) to make their task easier to learn. However, as briefly discussed (see Section~\ref{featureeng}) this can be a challenging endeavour and the ability of deep neural networks to learn their own feature representations is one of the main reasons behind their popularity. The inputs to a neural network is referred to as a an \textit{input feature map} $F$ and is a $n$-dimensional array that can take any form. 

\begin{Definition}{Input Feature Map}{}
Input feature map $F \in \mathbb{R}^{d_1 \times d_2 \times \ldots \times d_n}$, where $d_1, d_2, \ldots, d_n$ are the dimensions along each axis of the tensor and the number of dimensions $n$ could vary based on the type of data.
\end{Definition}

In image data $F \in \mathbb{R}^{H \times W \times D}$, where $H$ is the height of the feature map, $W$ is the width and $D$ is the depth, usually corresponding to the number of colour channels (\eg three for an RGB image) and a single value in the feature map is $F_{i, j, k}$, where $i$ is the row index ($1 \leq i \leq H$), $j$ is the column index ($1 \leq j \leq W$), an $k$ is the depth index ($1 \leq k \leq D$). When learning from text data each word or token could be represented as a vector in a high-dimensional space, often called an embedding. A sequence of words can then be viewed as $F \in \mathbb{R}^{N \times D}$, where $N$ is the number of tokens in the sequence and $D$ is the dimensionality of the embeddings. For time-series data (\eg stock prices, weather data, etc.), the feature map might be a two-dimensional array $F \in \mathbb{R}^{ T \times F}$, where $T$ is the number of time steps and $F$ is the number of features at each time step (\eg opening price, closing price, volume for stock data).


Deep neural networks transform and extract information from the input feature map $F$ through their layers. This higher-level, abstract representations of input data is referred to as a \textit{learned feature}~\cite{hinton2006reducing}. Depending on the task the learned features can be discrete (\eg labels such as ``ears”, ``mouth”, etc.) or continuous (\eg numeric values representing height, pixel values, etc.). As data passes through layers of the neural network, it undergoes a series of transformations (discussed further in Section~\ref{Sec:process}). These operations create new feature maps at each layer, which capture increasingly complex and abstract characteristics of the original input. The learned feature vector after each layer is denoted as $F_i$ that is a result of the transformation of the layer $\lambda_i$, represented as:

\begin{eqnarray}
    F_i = \lambda_i(F_{i-1})
\end{eqnarray}

\begin{Definition}{Feature Vector}{}
Each layer $\lambda_i$ has a feature vector $F_i \in \mathbb{R}^{d_1 \times d_2 \times \ldots \times d_n}$, which is a result of the transformation of the previous layer's output feature map. The input feature vector is a special type of a feature vector $F_0$ that doesn't result from a transformation, but is given to the network as input.
\end{Definition}

All the possible values of the feature vector forms a $d$-dimensional space called the \textit{feature space} (Figure~\ref{fig:space} A). Each \textit{sample} $X$, also called a data point or exemplar consist of a set of features from the feature space. All data points that are created through any combination of the $d$ features can be represented as points in the feature space (Figure~\ref{fig:space} B). The way the samples are distributed across the space dictates how easy it is for a function to be learned that performs the task well. For example, in a classification task, choosing features that allow exemplars from the same class to be close together and exemplars from different classes to be far apart will allow the classifier to discriminate between the classes more easily. Figure~\ref{fig:space}~C shows a hypothetical feature space of a feature vector deeper in the network, which have transformed the original features vector $F_0$ and its space that can be seen in Figure~\ref{fig:space}~B to one where a decision function can easily separate the two classes. 

% \begin{Definition}{Learned Features}{}
% Given an input $I\in\bbR^d$, we define a \emph{complex feature} of $I$ to be ...
% % a subset $F\subseteq \bbR^2$.
% \end{Definition}


\begin{figure}[ht!]
	\begin{center}
		\includegraphics[width=\linewidth]{Figures/feature_space.pdf}
	\end{center}
	\caption{Illustration of a feature space and datapoints in feature space. In practice neural networks have a feature space dimentionality far greater than the one illustrated.}
	\label{fig:space}
\end{figure} 

\subsection{Specialised Neural Network Architectures}
Neural networks are versatile and can learn from data that is obtained from the physical world (\textit{e.g.\ }camera) or comes from online sources (\eg databases, surveys). Input data can be in a number of forms (\textit{e.g.\ }video, sound, images, EEG recordings, salary etc.) depending on the task. The foundational architecture that can be trained on various inputs is the multi-layer perceptron (MLPs). Characterised by their feed forward structure and multiple layers, MLPs can capture complex relationships in data and have been used for variety of tasks. They have since been subsumed by more specialised architectures aimed at specific tasks. Specific neural network architectures are better equipped and applied to a subset of problems. For instance, when processing image data, a \textit{convolutional neural network} (CNN) is often the architecture of choice~\cite{egmont2002image}. CNNs contain layers, such as convolutional layers and pooling layers (see Section~\ref{sec:conv} \& Section~\ref{section:max}), that can detect and learn hierarchical patterns in images. On the other hand, for sequential data like time series or natural language, \textit{recurrent neural networks} (RNNs), Long Short-Term Memory (LSTM) networks, and increasingly in recent years, transformer networks are often chosen~\cite{husken2003recurrent, langkvist2014review, zhou2021informer}. These architectures have the capability to process sequential inputs allowing them to establish connections between past and present data points. Networks like \textit{autoencoders} are specialised in data compression and noise reduction. The autoencoder architecture starts wide with layers containing many neurons followed by middle layers with fewer neurons and ending with wide layers again. This forces the network to compress the input into a compact representation and then reconstructing it. Neural networks are also deployed in situations where an agent learns to perform actions in an environment to achieve certain objectives (\ie reinforcement learning tasks), in such cases Deep Q-Networks (DQNs) or Actor-Critic networks are often employed. It is important to note, that here are listed only the most popular architectures and that there are many more neural network architectures available and a wide range of values that the parameters in these networks can have.


\section{Neurons Transforming Data from Layer to Layer}
\label{Sec:process}

In a neural network, the layers serve as building blocks, stringing together neurons to help the network process and understand complex patterns in data. This capability does not solely lie in the architecture or the number of neurons, but critically in how these neurons transform their input into output. 
\subsection{A Neuron}
A neuron $j$ in an artificial neural network receives an input vector $\vec{x}$ either from the outputs of the connected neurons from the previous layer or from the external environment. The neuron then performs a simple calculation, which transforms the input and determines the response of the neuron. Such functions, are known as \textit{response functions} and they play a pivotal role in defining the behaviour and learning capacity of a neural network. By applying a response function on the input vector $\vec{x}$ the neuron produces the output $y_j$ (response) (see Figure~\ref{fig:neuron}). This response function is comprised of two sequential functions --- a transfer function (see Section~\ref{section:trans}) and an activation function (see Section~\ref{section:act}).
% \mike{intuition}


\begin{Definition}{Neuron}{}
A \emph{neuron} is a tuple $j = \langle t, \act\rangle$ comprising a transfer function $t:\mathbb{R}^d \to \mathbb{R}$ and an activation function $\act:\mathbb{R}\to \mathbb{R}$, where $d \in \mathbb{N}$. The neuron takes as input a vector $\vec{x}\in \mathbb{R}^d$ and applies the transfer function $t$ to produce a real value $t(\vec{x})=\net\in \mathbb{R}$. The activation function $\act$ is then applied to $\net$ to produce a real-valued output $\act(\net)=y_j\in \mathbb{R}$.
\end{Definition}


The input vector $\vec{x}$ contains values from $x_1$ to $x_d$, where $d$ is the number of elements in the vector. Each value $x_i\in \mathbb{R}$ is a real number and the vector of inputs $\vec{x}\in \mathbb{R}^d$ is a vector of real number of dimension $d\in \mathbb{N}$. Each input $x_i$ can be connected (\textit{i.e.\ }be an input) to more than one neuron. 

The neuron takes the inputs from the previous layer and applies the \textit{response function}. The response function computation consists of two parts: a \textit{transfer function} and an \textit{activation function}. The transfer function sometimes has associated parameters with it (discussed in Section~\ref{section:trans}), where the most commonly used parameter is a weight vector $\vec{w}_{j}$, where each element is a connection weight between input $x_i$ and neuron $j$, defined as $w_{ji}$. Each weight $w \in \mathbb{R}$ is a real number and it determines the strength of influence that input has on the neuron $j$. If $w_{ji}>0$ then neuron $i$ has a positive (\emph{excitatory}) influence on neuron $j$, while if $w_{ji}<0$ then $i$ has a negative (\emph{inhibitory}) influence on $j$.

The transfer function integrates all inputs and weights (if there are weights associated with the function) into a single value $net_j$ (see Section~\ref{section:trans}). The activation function takes $net_j$ as input, applies a function (see Section~\ref{section:act}) and determines the response $y_j$. Figure~\ref{fig:neuron} illustrates the computation that is performed by neuron $j$.


% \mike{intuition}
% \chrislong{The notation allows you to be more precise here: for example ``If $w_{ji}>0$ then neuron $i$ has a positive (excitatory) influence on neuron $j$, while if $w_{ji}<0$ then $i$ has a negative (inhibitory) influence on $j$''}
\begin{figure}[ht!]
	\begin{center}
		\includegraphics[width=\linewidth]{Figures/neuron.pdf}
	\end{center}
	\caption{Illustration of a neuron $j$}
	\label{fig:neuron}
\end{figure} 



\subsection{A Layer}

Each level of interconnected neurons in a neural network forms a layer in the network. All neurons that belong to a layer perform the same calculation, by applying the layer's associated response function.

\begin{Definition}{Layer}{}
A layer is a function $\lambda: \mathbb{R}^d \to \mathbb{R}^t$, where $\lambda(\vec{x}) = [\act_1(t_1(\vec{x})), \act_2(t_2(\vec{x})) \dots, \act_k(t_k(\vec{x}))] $, where the input is a vector $\vec{x}\in \mathbb{R}^d$ and each element in the output vector is the output of a neuron $j_i = \langle t_i, \act_i \rangle$, where each transfer function $t_i:\mathbb{R}^{d_i} \to \mathbb{R}$, with the dimensionality
conditions that $d_1 = d_2 = d_3 \dots = d_k$.
\end{Definition}

The response function that neurons in a layer utilise is not just a mathematical tool for data transformation, but rather a defining aspect of the layer's purpose and behaviour within the neural network. 

\subsection{Response function}

The choice of a response function can shape the tasks a layer is most adept at handling. For instance, a layer which has an activation function that squashes the input into a range can be ideal for outputting probabilities. On the other hand, a non-linear activation function can capture complex patterns in data, allowing the network to model non-linear relationships. Without non-linear activation functions the neural network layers would just be a composition of linear transformations which would itself be a linear transformation, so complex relationships would not be learned. 

The response function can prioritise kinds of patterns during training (\eg edge detection), it can make the network more robust to different input ranges (\eg by normalising the input) and it is crucial for the successful training (\ie convergence) of deep neural networks. Given the variety and use of both transfer and activation functions, they can be combined in countless ways to yield desired behaviours in neural networks. In this section, we will first explain some of the most prominent transfer and activation functions. Subsequently we explore prevalent combinations employed in widely recognised types of layers, with a particular focus on computer vision and CNNs.

\subsubsection{Transfer function}
\label{section:trans}
The transfer function is the first step in calculating the response function. It determines how the inputs from the previous layer or the external environment are integrated and transformed. 

\begin{Definition}{Transfer function}{}
A neuron's \emph{transfer function} $t:\mathbb{R}^d \to \mathbb{R}$ takes as input an input vector $\vec{x}\in \mathbb{R}^d$ and integrates those values into a single real value $net_j$ by applying the transfer function $t(\vec{x})=net_j\in \mathbb{R}$.
\end{Definition}


% The function takes the input vector $\vec{x}\in \mathbb{R}^d$
% , and the associated weight vector $\vec{w}_j\in \mathbb{R}^d$, which are the inputs for neuron $j$. The function also has an associated parameter, which is a constant, called a threshold or bias $w_{j 0}\in \mathbb{R}$ or $\theta \in \mathbb{R}$. 

\textbf{Weighted Sum}
\label{sec:weightedsum}
Many layer's transfer functions have an associated weight vector $\vec{w}_j\in \mathbb{R}^d$, where each element from the vector weights the inputs for neuron $j$ and has an associated parameter, called a threshold or bias a threshold or bias $\theta= \beta= w_{j,0}\in \mathbb{R}$. The most common way of integrating all the inputs of the transfer function is by taking the weighted sum of the inputs. This is done by taking the sum of all inputs $x$ multiplied by their corresponding weight $w$ and then adding the threshold $w_{j 0}$ to that sum to produce the final integrated input $net_j$ for a given neuron $j$. The integration of the constant term $w_{j 0}$ allows for faster convergence and provides a minimum activation value for the neuron regardless of the input. 

The type of transfer function that integrates the input values by taking the weighted sum of the inputs is therefore expressed, as:
\begin{eqnarray}
    \notag t_{wsum}(\vec{x}) & = & x_{1} w_{j 1}+x_{2} w_{j 2}+\cdots+x_{d} w_{j d}+w_{j 0} \\
    \notag & = & \sum_{i=1}^{d} x_{i} w_{j i}+w_{j 0} \\
     & = & \sum_{i=0}^{d} x_{i} w_{j i} \ = \  \vec{{w}_j}^T\, \vec{x}
    \label{eq:weightredsum}
\end{eqnarray}

where the subscript $i$ denotes the index of the neurons in the source layer, $j$ denotes the index of the current neuron, $w_{j0}$ is the bias, $x_0$ is 1 (to allow for the augmented notation), $x_1$ to $x_d$ are the values contained in the input vector $\vec{x}$ and $w_{ji}$ denotes the weight of source-to-current neuron at a neuron $j$ and 
\begin{equation*}
  \vec{w}_{j}=\left[\begin{matrix}w_{j 0}\\ w_{j 1} \\ w_{j 2} \\ \vdots  \\ w_{j d }\end{matrix}\right] 
  \qquad \mbox{and}\qquad
  \vec{x}=\left[\begin{matrix} 1\\ x_{1} \\ x_{2} \\ \vdots \\ x_{d} \end{matrix}\right]
\end{equation*}
and ${T}$ in $\vec{{w}_j}^T$ is the transpose of $\vec{w}_{j}$.


The bias is often a one dimensional vector that has the size of the feature dimension of the activation as shown in Figure~\ref{fig:dimentionality_bias}. 
\begin{figure}[ht!]
	\begin{center}
		\includegraphics[width=0.8\linewidth]{Figures/Tensor_bias.pdf}
	\end{center}
	\caption{This figure shows the addition of a one dimensional bias vector with the the size of the feature dimension of the $\cnet^-$ matrix.}
	\label{fig:dimentionality_bias}
\end{figure} 

When looking at the addition of the bias on a layer level rather than on a neuron level, it represents a constant shift of the mean of the activations' distributions as shown in Figure~\ref{fig:addingbias} (for a single dimension). This follows from a more general formula $E[a*X+b] = a*E[X] + b$ for the linearity of the expectation/mean of a random variable $X$, where $a =1$ in the case of the bias addition. 
\begin{figure}[ht!]
	\begin{center}
		\includegraphics[width=0.9\linewidth]{Figures/distribution.pdf}
	\end{center}
	\caption{The figure shows for a single feature dimension how the mean changes from the one of the weighted activation to the the mean of the weighted activation plus the bias constant.}
	\label{fig:addingbias}
\end{figure} 

\textbf{Max Pooling Transformation}
The max pooling transformation function operates on small regions of the input feature map and picks the maximum value from each region. Each small region is fed as input to a neuron and the neuron picks the max value and outputs it as $net_j$.
\label{section:maxout}
\begin{equation}
t_{max}(\vec{x}) =\max _{j \in \vec{x}} x_j, 
\label{eq:max}
\end{equation}
This transfer function is commonly used by max pooling layers (see Section~\ref{section:max})

\textbf{Average Transformation}
\label{section:avg}
The layer's average transformation function also operates on small regions (sub-matrices) of the input feature map, but calculates the average value for each region rather than the maximum one. Each region is fed into a neuron from the layer and the transformation that neuron performs is defined as:
\begin{equation}
f(\Vec{x})=\frac{1}{n} \sum_{i=1}^n x_i,
\label{eq:avg}
\end{equation}
where $\vec{x}$ is the input vector (or the region that is fed tot h neuron) and $n$ is the number of elements in the input vector $\vec{x}$. This transformation function is often used by average pooling layers (see Section~\ref{section:avglayer})

\textbf{Convolutional Transformation}
\label{section:conv}
The convolutional transformation operates on small regions (sub-matrices) of the input feature map and calculates the dot product between the input and a learned filter associated with this transfer function, often referred to as a kernel. Each neuron from the layer has an associated region slice of the input feature map $X$ of dimensions $h \times w$ that the filter $W$ of dimensions \( f_h \times f_w \) is covering. For simplicity, let's assume that both the input feature map and filter have only one channel. The neuron computes the following weighted sum of its inputs (which is the convolution operation):

\begin{equation}
t_{conv}(X) = \sum_{i=0}^{f_h - 1} \sum_{j=0}^{f_w - 1} X(i, j) \cdot W(i, j) + b
\label{eq:conv}
\end{equation}

This is the output of that particular neuron for the given region of the input feature map. This transformation function is often used by convolutional layers (see Section~\ref{sec:conv})

\textbf{Batch Normalisation Transformation}
The batch normalisation transformation has four associated learned parameters, mean $m$, variance $\sigma$, scale $\gamma$ and a shift $\beta$. The desired effect of the batch normalisation is to first shift the input to have a mean of zero and a standard deviation of one and then re-scale it appropriately using scale $\gamma$ and a shift $\beta$.
During training the values of the mean and the standard deviation are calculated as usual by:

\begin{equation}
\mu^\prime = \frac{1}{N} \sum_{i=1}^{N} x_i \qquad \sigma^{2\prime} = \frac{1}{N} \sum_{i=1}^{N} (x_i - \mu)^2
\end{equation}

Using those values a learnable mean $m$ and variance $\sigma$ are updated by using back propagation (see Section~\ref{sec:backprop}). Each neuron in this layer has as input a single input value $x$, which has to be normalised and then scaled and shifted. The normalisation is achieved through :

\begin{equation}
\hat{x} = \frac{x - \mu}{\sqrt{\sigma^2 + \epsilon}},
\end{equation}
where $\epsilon $ is a small constant added for numerical stability.

Finally, the normalised output $\hat{x}$ is scaled and shifted using the parameters  $\gamma$  and  $\beta$ :

\begin{equation}
net_j = \gamma \hat{x} + \beta
\end{equation}


The batch normalisation transformation for a given neuron can be summarised to:
\begin{equation}
    t_{norm}(x) = \gamma \frac{(x-\mu)}{\sqrt{\sigma^2 +\epsilon}} + \beta ,
\end{equation}

\subsubsection{Activation function}
\label{section:act}
The activation function is the final step in calculating the response function of a neuron. It transforms the integrated neurons input $\net$ and determines the proportion of the transformed value that should be transmitted from one layer to the next by calculating the output of neuron~$j$. 
\begin{Definition}{Activation function}{}
A neuron's \emph{activation function} $\act:\mathbb{R} \to \mathbb{R}$ takes as input $\net$ and returns a real-valued overall response (activation) $y_j$ of neuron~$j$ $\act(\net)=y_j\in \mathbb{R}$ .
\end{Definition}

The final layer of neural networks often needs to output a probability. This layer's neurons often apply activation functions that transform the input into a number within a range (referred to as a \textit{squashification function}). One such activation function is the \textit{sigmoid function} defined by:

\begin{equation}
\label{eq:sigmoid}
\act(x)=\frac{1}{1+\exp (-x)}
\end{equation}

and can be visualised as a S-shaped curve with a range of activation for this function is $(0,1)$ (see Figure~\ref{Fig:act}~(a).

Another squashification function is the \textit{hyperbolic tangent function} (\textit{tanh function}), defined as
\begin{equation*}
\act(x)=\frac{exp(x) - exp (-x)}{exp (x)+\exp (-x)}.
\end{equation*}
The tanh function is just a scaled sigmoid function, it also has an S-shape, but with a range between $(-1, 1)$ (see Figure~\ref{Fig:act}~B). 

Squashification functions are useful, as they allow to determine how positive the input to the function is. A drawback of using this type of activation function in the hidden layers of neural networks is that the output will not respond to or will respond little to changes in the value of $\net$ that is on either side of the curve. This makes neural networks using this type of activation function harder to train ($\ie$ the network stops learning or learns very slowly). This problem is referred to as the \textit{vanishing gradient problem}.

An example of non-linear activation function that overcomes the vanishing gradient problem is \textit{rectifier activation function (ReLu)}. In modern neural networks, the default recommendation is to use ReLU~\cite{JarrettKRL09, NairH10, Heaton18}, which is defined by the activation function 

\begin{equation}
\act(x)=\max \left(x, 0\right)= \begin{cases}x &  \text { if } x>0 \\ 0 & \text { otherwise }\end{cases}
\label{eq:relu}
\end{equation}

that is a linear in the positive axis and $0$ otherwise (see Figure~\ref{Fig:act}~C). Because of the linearity in the positive axis the function is unbound and can produce an output $y_j$ in the rage of $[0,\infty)$, which may cause the activations to blow up. An advantage of the ReLu function is that some neurons remain inactivate in comparison to the the sigmoid or tanh activation functions where the neurons fire in an analog way. The inactivate neurons when using ReLu make the activations more sparse, which in turn requires the network to perform fewer computations and increases efficiency. The disadvantage of ReLu is that it can stop responding to training, as the rate of change may become $0$, because of the constant output for any negative value of $\net$. This is overcome by allowing small negative values for for all negative values of $\net$. This is referred to as \textit{leaky ReLu}, defined by the activation function 
\begin{equation*}
\act(x) =\max \left(x, 0\right)= \begin{cases} x & \text { if } x>0 \\ \epsilon x & \text { otherwise }\end{cases}
\end{equation*}
which has a slightly inclined line rather than horizontal line for all negative values of $\net$ (see Figure~\ref{Fig:act}~D), the $\epsilon$ value is often set to 0.01.


\begin{figure}[ht!]
	\begin{center}
		\includegraphics[width=\linewidth]{Figures/act.pdf}
	\end{center}
	\caption{Illustration of non-linear activation functions}
	\label{Fig:act}
\end{figure} 



While many more activation functions exist and can potentially be employed in neural networks, those discussed here are among the most widely adopted in the field. Below we discuss some of the layers that deploy them and the effect their application has on the network learning. 

\subsection{Types of Layers Based on Response Functions}

\subsubsection{Dense Layer}
\label{section:dense}
In dense layers, often referred to as fully connected layers, each neuron is connected to every other neuron in the previous layer. In other words, the outputs of all neurons in the previous layer are inputs to each neuron in the following layer. This connectivity pattern allows information to flow freely and interact between neurons, giving the network the ability to learn intricate relationships and make complex predictions. They have been used extensively in various architectures from traditional multi-layer perceptrons to more modern structures like convolutional neural networks and recurrent neural networks.

Each neuron in a dense layer performs a weighted sum of its inputs (see Section~\ref{eq:weightredsum}) as a transformation function, where the weights associated with each input determine the strength of the connection between neurons. After the linear transformation (dot product of inputs with weights and addition of bias), the result is typically passed through a non-linear activation function, commonly ReLU (Rectified Linear Unit) (see equation~\eqref{eq:relu}), in some cases the activation function is just bypassed through an identity function $f(x) = x$.

\begin{Definition}{Dense Layer}{}
The \emph{dense layer function} $f_k:\bbR^{n_j}\to \bbR^{m_k}$, which describes the vector transformation associated with the layer $\lambda_k$. The transfer function of dense layer neurons is a weighted sum transformation defined in equation~\eqref{eq:weightredsum}. The activation function part of dense layers may be any function of the ones defined in Section~\ref{section:act}
\end{Definition}

\subsubsection{Max Pooling Layer}
\label{section:max}

Max pooling layers play a pivotal role in extracting essential features from input data, especially in convolutional neural networks. They divide the input of the layer into non-overlapping regions and select the maximum value within each region. The size of the pooling operation or filter is smaller than the size of the feature map. Commonly, the filter is $2\times 2$ pixels applied with a stride of 2 pixels. Stride denotes by how many pixels apart are all filters. As max pooling doesn't have overlapping regions, the stride and the filter dimension are the same. In the case where we have a filter of $2\times 2$, the resulting layer will have each dimension halved, reducing the number of pixels or values in each feature map to one quarter the size. 

More concisely max pooling layer divides the input $X$ in to into groups of $k$ values. Each group is fed as input to a neuron $j_i$ from the max pooling layer, which performs a max pooling transformation $t_{max}$ (see Section~\ref{section:maxout}) and outputs the maximum element from the group. The overall function the layer performs can be summarised by 

\begin{equation}
\operatorname{MaxPooling}(X)= [t_{max}(X[i \cdot s_x, j \cdot s_y, c])]_{i,j,c}
\label{eq:maxlayer}
\end{equation}

where is the input, $(i,j)$ are the indices of the output, $c$ is the channel index, $s_x$ and $s_y$ are the stride values in the horizontal and vertical directions, respectively. Figure~\ref{Fig:maxpooling} shows an example of the output of a max pooling layer.

The max-pooling process has many advantages. Notably, taking the maximum value across every group of $k$ features, reduces the parameters that the subsequent layer operates with by a factor of $k$. This allows the layer to capture the most salient features present in the input while discarding less significant information. Max pooling is also known for removing invariances, meaning the network can still recognise features regardless of their position, rotation and scale. As a result the max pooling operation enhances the network's ability to generalise and makes it more robust to noise. Max pooling layers don't have learnable parameters.

\begin{figure}[ht!]
	\begin{center}
		\includegraphics[width=0.9\linewidth]{Figures/maxpool.pdf}
	\end{center}
	\caption{A max pooling transformation, where the $t_{max}$ function is performed and the workings are shown on the first filter-sized part of the input (in pink).}
	\label{Fig:maxpooling}
\end{figure} 


\begin{Definition}{Max Pooling Layer}{}
The \emph{max pooling function} $f_k:\bbR^{n_j}\to \bbR^{m_k}$, which describes the vector transformation associated with the layer $\lambda_k$, with the dimensionality conditions that $n > m$. The transfer function neurons in the layer performs is max pooling transformation described in equation~\eqref{eq:max}. The activation function that follows is often ReLU~\eqref{eq:relu}, but could also be linear activation. This type of layer has an associated filter size and no learnable parameters.
\end{Definition}

\subsubsection{Average Pooling Layer}
\label{section:avglayer}
Average pooling layers, similarly to max pooling layers, are a type of down-sampling or sub-sampling layer. These layers reduce the spatial dimensions of the input volume during the forward pass by taking, as the name suggests, an average of a group of neighbouring pixels and outputting this average as the new value for the corresponding region in the output feature map.

More concisely average pooling layer divides the input $X$ in to into groups of $k$ values. Each group is fed as input to a neuron $j_i$ from the average pooling layer, which performs a average transformation $t_{avg}$ (see Section~\ref{section:avg}) and outputs the group's mean. The overall function the layer performs, can be summarised by 

\begin{equation}
\operatorname{AvgPooling}(X)= [t_{avg}(X[i \cdot s_x, j \cdot s_y, c])]_{i,j,c}
\label{eq:maxlayer}
\end{equation}

where is the input, $(i,j)$ are the indices of the output, $c$ is the channel index, $s_x$ and $s_y$ are the stride values in the horizontal and vertical directions, respectively. Figure~\ref{Fig:avg} shows an example of the output of a average pooling layer. Average pooling layers also don't have learnable parameters.


\begin{figure}[ht!]
	\begin{center}
		\includegraphics[width=0.9\linewidth]{Figures/avgpooling.pdf}
	\end{center}
	\caption{A average pooling transformation, where the $t_{avg}$ function is performed and the workings are shown on the first filter-sized part of the input (in pink).}
	\label{Fig:avg}
\end{figure} 
Similarly to max pooling layers,  they reduce the spatial dimensions of the input volume, which subsequently reduces the computational complexity of the model. This can make it easier to train the model and can help prevent overfitting. Contrary to max pooling, which only keeps the maximum value in a local region, average pooling is less aggressive in reducing the input dimensions. This means that more information can be preserved when down-sampling, although this can also be a disadvantage if too much irrelevant information is retained. Average pooling has a smoothing effect on the input. This is sometimes beneficial for tasks where the fine-grained details are not as important as the broader structure. 

\begin{Definition}{Average Pooling Layer}{}
The \emph{average pooling function} $f_k:\bbR^{n_j}\to \bbR^{m_k}$, which describes the vector transformation associated with the layer $\lambda_k$, with the dimensionality conditions that $n > m$. The transfer function neurons in the layer performs is average transformation described in equation~\eqref{eq:avg}. There is often no activation function that follows. This type of layer has an associated filter size and no learnable parameters.
\end{Definition}
\subsubsection{Convolutional Layer}
\label{sec:conv}
Convolutional layers are a type of layer often used for processing image information and a main building block of convolutional neural networks. They are designed to automatically learn spatial hierarchies of features, focusing on local regions at first and later capturing more global information as the layers progress. The stacking of convolutional layers enables the learning of higher-level features by composing them from lower-level ones. For instance, the first layer might learn edges, the second layer shapes by combining edges, and deeper-level layers might learn more complex structures or objects. Figure~\ref{Fig:faces} shows the amalgamation of feature learned in a CNN as a result of using convolutional layers.

\begin{figure}[ht!]
	\begin{center}
		\includegraphics[width=1\linewidth]{Figures/faces.pdf}
	\end{center}
	\caption{Learned features from a Convolutional Deep Belief Neural Network. Source~\cite{LeeGRN09}}
	\label{Fig:faces}
\end{figure} 

Convolutional layers learn the features through a \textit{kernel} (also known as filter), which is associated with the transfer function. A kernel is a small matrix of numbers, which is passed over the input and performs a transformation based on the the values in the filter. The kernel can have predefined values, so that it is specialised at detecting particular patters (\eg edges) or can be sampled from a distribution, such as normal or uniform distribution or set to a constant like 0 or 1. The kernel values (when not pre-defined) are usually learnable parameters of a convolutional layer.

More concisely convolutional layer divide the input $X$ in to into groups of $k$ values. Each group is fed as input to a neuron $j_i$ from the convolutional layer, which performs a a convolution  $t_{conv}$ (see Section~\ref{section:conv}) outputs the dot product of the group input with the associated kernel $W$. The overall function the layer performs can be summarised by 

\begin{equation}
\operatorname{Conv}(X)= [t_{conv}(X[i \cdot s_x, j \cdot s_y, c])]_{i,j,c}
% \underset{m, n}{t_{conv}}(X_{i \cdot s_x+m, j \cdot s_y+n, k}, h)
\label{eq:maxlayer}
\end{equation}

where $X$ is the input, $(i,j)$ are the indices of the output, $c$ is the channel index, $s_x$ and $s_y$ are the stride values in the horizontal and vertical directions, respectively. Figure~\ref{Fig:conv} shows an example of the output of a convolutional layer.

\begin{figure}[ht!]
	\begin{center}
		\includegraphics[width=1\linewidth]{Figures/conv.pdf}
	\end{center}
	\caption{A convolutional transformation, where the $t_{conv}$ function is performed and the workings are shown on the first kernel-sized part of the input (in pink). The stride values $s_x$ and $s_y$ are set to $1$, which means the convolutional operation is performed to each part of the input that is $1$ position away from the previous part. In this example the bias term is set to zero.}
	\label{Fig:conv}
\end{figure} 

Another advantage of this layer is that the only parameters that need learning are the ones in the kernel and the same kernel is implied to different parts of the input. The reduction in parameters not only makes the network easier to train, but also helps to avoid overfitting to some extent.

\begin{Definition}{Convolutional Layer}{}
The \emph{convolutional function} $f_k:\bbR^{n_j}\to \bbR^{m_k}$, which describes the vector transformation associated with the layer $\lambda_k$, with the dimensionality conditions that $n > m$. The transfer function neurons in the layer performs is convolution in equation~\ref{eq:conv}. The activation function that follows is often ReLU~\eqref{eq:relu}, but could also be linear activation. This type of layer has an associated filter size and a stride and learnable kernel values $W$. 
\end{Definition}

\subsubsection{Batch Normalisation Layer}
\label{section:bn}
The purpose of batch normalisation is to address the issue of internal co-variate shift. Internal co-variate shift refers to the change in the distribution of layer inputs that occurs during the training process. Batch normalisation tackles this problem by first normalising the inputs to the layer and second scaling and shifting the normalised activations. This allows the model to learn the optimal scale and offset for each feature. The normalisation step is achieved by computing the mean and standard deviation of the activations within a mini-batch and using these statistics to normalise the activations. The normalisation is applied independently to each feature dimension, ensuring that the mean becomes zero and the standard deviation becomes one. The scale and shift operation is performed on the normalised activations and uses learned parameters called gamma and beta. 

In practice batch normalisation works differently during training and during inference. The mean and standard deviation for each mini-batch are calculated only during training, which sets the mean and the standard deviation of the activations precisely to 0 and 1 respectively. Meanwhile values for a typical mean and standard deviation are being learned and updated as each mini batch is passed through for training. Those values are then used during the forward pass for ease of computation.

\begin{Definition}{Batch Normalisation Layer}{}
The \emph{batch normalisation function} $f_k:\bbR^{n_j}\to \bbR^{m_k}$, which describes the vector transformation associated with the layer $\lambda_k$, with the dimensionality conditions that $n = m$. The transfer function is batch normalisation $t_{norm}$ and there is often no activation function that follows.
\end{Definition}


The batch normalisation layer $k\in \Lambda$ during the forward pass can be expressed as a vector-valued function $f_k:\bbR^{n_k}\to \bbR^{m_k}$ ($n_k=m_k$) given~by:
\begin{equation}
    f_k(\vec{x}) = \gamma \frac{(\vec{x}-\mu)}{\sqrt{\sigma^2 +\epsilon}} + \beta ,
\end{equation}
where $\frac{(\vec{x}-\mu)}{\sqrt{\sigma^2 +\epsilon}}$ normalises the input, the multiplication with $\gamma$ scales the values within the distribution, while the addition of $\beta$ shifts them. The normalisation component of the layer has an $\epsilon$ value which is a small constant to avoid division by zero and maintain numerical stability, $\mu$ is the learned mean and $\sigma$ is the learned standard deviation from all the examples seen during training.

% \subsubsection{Concatenation Layer}

% \begin{Definition}{Concatenation Layer}{}

% \end{Definition}

\section{Training}
\label{sec:training}
As discussed, the non-linear mapping between the input and the output in a neural network is enabled through the combination of all hidden neurons’ decision boundaries. The position of neuron $j$'s decision boundary is defined through the parameters associated with the layer neuron $j$ belongs to. Hence, to create a network that performs a specific task one needs to find a combination of weights and biases for all neurons, kernels, layer' means and variances or any other learnable parameter a layer may have that enable the network to perform the task in question. The process of finding the appropriate values is called \emph{training} and is the main focus of this section. 

Hypothetically, if the task is very simple an exhaustive search of the weights and biases could be possible, but this way of `learning' quickly becomes impractical~\cite{BianchiniS14}. 
In general, an $n$-layer neural network, where all neurons are connected to the next layer (\ie it is fully connected) with $l_1,l_2,\dots l_n$ be the number of neurons in each of the $n$ layers and $l_{bi}$ is the number of biases for layer $i$, the overall number of parameters is given by:

\begin{equation}
\text{Total Parameters} = \sum_{i=1}^{N} \left( (l_{i-1} \times l_{i}) + l_{bi} \right)
\end{equation}


The term $l_{i-1} \times l_{i}$ accounts for the weights between each neuron from $l_{i-1}$ and each neuron in $l_{i}$. For convolutional neural networks, recurrent neural networks, or any specialised architectures, the calculation would be more complex and would depend on additional factors like filter sizes, strides, padding, etc., for convolutional layers, or the type of recurrent unit (LSTM, Gated Recurrent Unit (GRU), etc.) for recurrent layers.


\begin{Example}{ANN number of parameters}{}
For instance, a fully connected neural network trained to recognise digits from the MNIST dataset that has 784 neurons as input (the images are 28 $\times$ 28 pixels), two hidden layers each with 32 neurons (this is arbitrary) and an output layer of size 10 (encoding the digits in the range 0--9). The number of parameters in this minimal classifier is $784 \times 32$ ($\ie$ the weights connecting the input and the hidden layer ($n \times l$)) + $32 \times 32$ ($\ie$ the weights connecting the hidden layers $(l \times l)^{m-1}$) + $32 \times 10$ ($\ie$ the weights connecting the hidden layer and the output layer ($l \times k$)) + $32 \times 2$ ($\ie$ the biases of all neurons in the hidden layers ($l \times m$)) + 10 ($\ie$ the  biases in the final layer ($k$)) = 26 506 overall parameters. Even in such a small network the number of parameters is too big for an exhaustive search of the weights and biases to be feasible, much less in the current state-of-the-art neural networks for image classifications, where the neural network can have as many as 480 million parameters~\footnote{The number of parameters on the FixEfficientNet neural network trained for the ImageNet Large Scale Visual Recognition Challenge (ILSVRC)~\cite{FixEff}.}
or language generation models, where the parameters can be as 220 billion parameters~\footnote{The number of parameters on the ChatGPT 4 model, which is said to have  as many as eight models with 220 billion parameters each for a total of about 1.76 trillion parameters.}
\end{Example}

For more complex tasks, neural networks tune their parameters by using a training data to improve their performance. This is called \textit{training} (also referred as \textit{learning} and \textit{parameter tuning}) and is analogous to learning in animals, which have central nervous system and learn to improve their performance through experience. Before training can commerce one must choose the neural networks size, the number of hidden layer and the type of activation functions the neurons perform. The weights and biases or any other learnable paramethers in the network must also be initialised with pseudo-random values. The values of the weights and biases can then be iteratively updated until the non-linear mapping between the input and the output in a neural network performs as expected on the pattern recognition task in question. 


% \subsection{Supervised learning}



\subsection{Backpropagation}
\label{sec:backprop}
During this type of training each input vector $\vec{x}$ has the associated approximated or ground truth output. This output can be a class label in a classification task, the desired number in a regression task or a reward in reinforcement learning. During this type of training, the network is fed an input vector $\vec{x}$ without the given output. The network then tries to perform the task in training to produce an output $\vec{z}$, which is then compared to the input's target output $\vec{t}$. If the target output is different from $\vec{z}$, the algorithm adjusts the learnable parameters so that the network gives the correct output in the future. 

The most general way of adjusting the learnable parameters is by \textit{backpropagation}. Each iteration during training consists of two modes. In the first mode (the feed-forward mode), the input vector is fed through the network, where neurons at each layer perform their response function and pass the output to the subsequent connected neurons until the last layer of the network is reached (explained in Section~\ref{Sec:process}). In the second mode (the learning mode), the algorithm computes the difference between the computed output $\vec{z}$ and the target output $\vec{t}$ of the network and then minimises this difference (the error). 

\subsection{Training Error}
\begin{Definition}{Training error}{}
The \emph{training error} is computed using a cost function $cost: \mathbb{R}^d \times \mathbb{R}^d \to \mathbb{R}$, which takes as input the constant target output vector $\vec{t} \in \mathbb{R}^d$ and the computed output vector $\vec{z}\in \mathbb{R}^d$ and calculates the single real valued training error $cost(\vec{t},\vec{z})= C \in \mathbb{R}$
\end{Definition}

The training error, or cost $C$ is computed at each iteration by taking the distance between the target output vector $\vec{t}$ of the neural network and the computed output vector $\vec{z}$ of the neural network. The distance is calculated using a cost function $cost(\vec{t}, \vec{z})$ (is also referred to as a \textit{objective function} or a \textit{loss function}). The error is dependant on the value of the computed vector $\vec{z}$, as the target output vector $\vec{t}$ is static for each instance. The computed vector $\vec{z}$ is a result of where the input sits with respect to the current neural network's decision boundary and the position of the decision boundary is set by the learnable parameters vector $\vec{w}$. Therefore, the objective function $cost(\vec{t}, \vec{z})$ calculates the error not only of the computed vector $\vec{z}$, but also the overall error of all weights, biases, kernels, layer’ means and variances or any other learnable parameters. For that reason the cost function is sometimes expressed as $cost(\vec{w})$.

\subsubsection{Loss Functions}
The cost $C$ can be computed using any distance function. The most widely used one is the \textbf{Mean Squared Error} $\mathrm{MSE_n}: \mathbb{R}^n \times \mathbb{R}^n \to \mathbb{R}$, given by: 

\begin{equation*}
\mathrm{MSE_n}(\vec{x}, \vec{y})= \frac{1}{n} \sum_{k=1}^{n}\left(x_k-y_k\right)^{2} 
\end{equation*}


where $n$ is the number of data points, the subscript $k$ denotes the index of the element in the vector of observed values $\vec{x}$ and the vector of predicted values $\vec{y}$ . 

When computing the training error of the neural network, the vector of observed values $\vec{x}$ is equal to the computed output vector of the neural network $\vec{z}$, the vector of predicted values $\vec{y}$ is equal to the target output vector $\vec{t}$. Therefore the training error of the neural network, when using mean squared error is given by:


% where $n$ is the number of data points, the subscript $k$ denotes the index of the element in the vector of computed values $\vec{z}$ and the vector of target values $\vec{t}$. When the mean squared error is applied to the neural network the 


\begin{eqnarray*}
    cost(\vec{t}, \vec{z}) & = & \frac{1}{2} \sum_{k=1}^{n}\left(z_{k} - t_k\right)^{2} \\
    & = & \frac{1}{2}\norm{\vec{t}-\vec{z}}^{2} 
\end{eqnarray*}
% \chrislong{
% Check this matches which your previous definition? This seems to suggest that you are only working with 2-dimensional vectors $\vec{t},\vec{z}\in \bbR^2$?
% }

where the subscript $k$ denotes the index of the output in the target output vector $\vec{t}$ and the computed output vector $\vec{z}$, $n$ refers to the size of the $\vec{t}$ and $\vec{z}$ vectors and the $\norm{\vec{t}-\vec{z}}$ denotes the \textit{L2-norm operator}, which is the square root of the sum of the squares of $\vec{t}$ and $\vec{z}$.

\textbf{Mean Absolute Error} $\mathrm{MAE_n}: \mathbb{R}^n \times \mathbb{R}^n \to \mathbb{R}$ is another often used loss function. It calculates the average of the absolute differences between the predicted and actual values. It is less sensitive to outliers compared to $\mathrm{MSE_n}$ and is given by:

\begin{eqnarray*}
\mathrm{MAE_n}(\vec{x}, \vec{y})= \frac{1}{n} \sum_{k=1}^{n} |x_k - y_k|
\end{eqnarray*}

\textbf{Root Mean Squared Error} $\mathrm{RMSE_n}: \mathbb{R}^n \times \mathbb{R}^n \to \mathbb{R}$ is the square root of $\mathrm{MSE_n}$ and provides an error metric in the original units of the output variable. Because it squares the errors before averaging, $\mathrm{RMSE_n}$ gives a relatively high weight to large errors. This is useful when large errors are particularly undesirable. $\mathrm{RMSE_n}$ is just the square root of $\mathrm{MSE_n}$, so it retains many of the beneficial properties of $\mathrm{MSE_n}$ while being in the same units as the original data, making interpretation easier.

\begin{eqnarray*}
\mathrm{RMSE_n}(\vec{x}, \vec{y})=\sqrt{\frac{1}{n} \sum_{k=1}^{n} (x_k - y_k)^2}
\end{eqnarray*}

\textbf{Cross-Entropy Loss} $\mathrm{CrossEntropy}: \mathbb{R}^n \times \mathbb{R}^n \to \mathbb{R}$ also known as log loss~\cite{shannon1948mathematical}, is the most common loss function for classification problems. It measures the performance of a classification model whose output is a probability value between 0 and 1.

\begin{eqnarray*}
\mathrm{CrossEntropy}(\vec{x}, \vec{y})= -\sum_{k=1}^{n} [x_i \log(y_k) + (1-x_k) \log(1-y_k)]
\end{eqnarray*}

\subsection{Learnable parameter updating}

A trained neural network should perform the tasks in question with a low error and high accuracy (see Section~\ref{sec:eval}). During training the objective is to minimise the training error $C$ computed in the previous step. The value $C$ is minimised by changing the computed vector $\vec{z}$ to be similar to the target vector $\vec{t}$. The values in the computed vector $\vec{z}$ are determined by the values of the learnable parameter vector $\vec{w}$. Hence, the training error $C$ is reduced by updating the learnable parameters in a way that minimises the difference between the computed vector $\vec{z}$ and the target vector~$\vec{t}$. The learnable parameter updates are done iteratively in small steps (called a \textit{learning rate} $\eta$) until either:

\begin{enumerate}
    \item The lowest value of the objective function $cost(\vec{w})$ is found in the neighbourhood of the initial value. This is referred to as a \emph{local minima} and the value found may not be the optimum value. This is the case for non-convex functions ($\ie$ functions where not all local minimas are global minima ($\ie$ the optimum value)) (see Figure~\ref{Fig:local_global}~A).
    \item The lowest value of the objective function $cost(\vec{w})$ is found that is also the optimum value of $cost(\vec{w})$. This is referred to as a \emph{global minima} and can be guaranteed only if the objective function $cost(\vec{w})$ is a convex function ($\ie$~any local minima is a global minima)(see Figure~\ref{Fig:local_global}~B). 
\end{enumerate}


\begin{figure}[ht!]
	\begin{center}
		\includegraphics[width=\linewidth]{Figures/graphs_local_global.pdf}
	\end{center}
	\caption{Illustration of a non-convex and convex function}
	\label{Fig:local_global}
\end{figure} 

At time step $m$, the vector containing the learnable parameter in the network $\vec{w}_m$ is changed in a direction that reduces the value of the overall training error $cost(\vec{w}_m)$. The new value of the learnable parameter vector $\vec{w}_{m+1}$ is calculated by taking the learnable parameters in the network $\vec{w}_m$ and adding a change $\Delta\vec{w}_m$ that reduces the error $C(\vec{w}_m)$. Hence, the weights and biases update is given by:

\begin{equation*}
\vec{w}_{m+1}=\vec{w}_{m}+\Delta \vec{w}_{m}
\end{equation*}

where $\Delta \vec{w_m}$ is a vector and each element in the vector is a number that corresponds to the change of the learnable parameters in the vector $\vec{w}_m$. Using this rule for updating the learnable parameter vector, one can calculate at time step $m+1$ the training error $C$ and then repeat the update until the $\vec{w}$ stops changing. Figure~\ref{Fig:finding_minima} illustrates four iterations of backpropagation algorithm, where the blue circle is the current position and each arrow show the step taken in the direction that reduces the training error $C$  until a local minima is reached. 

\begin{figure}[ht!]
	\begin{center}
		\includegraphics[width=0.85\linewidth]{Figures/finding_minima.pdf}
	\end{center}
	\caption{Illustration of four iterations of backpropagation algorithm, where the learnable parameter vector $\vec{w}$ is updated at each time step to reduce the training error $cost(\vec{w})$ }
	\label{Fig:finding_minima}
\end{figure} 

An alternative way of thinking about minimising the training error $C$ is to think of it as moving the current position of the error in the direction of the slope of $cost(\vec{w})$ (as illustrated in Figure~\ref{Fig:finding_minima}). The direction of the slope is computed using \textit{gradients} --- a vector that points in the direction of greatest increase of a function. As the objective is to minimise the cost function $cost(\vec{w})$, one can take the negative gradient and then move iteratively in the direction of greatest decrease of the function. 

The gradient is calculated by taking the partial derivatives of all variables in the function. The variables in the case of the objective function $cost(\vec{w})$ are all learnable parameter $\vec{w}$. The derivatives of the objective function $cost(\vec{w})$ measure the sensitivity to change of the $cost(\vec{w})$ function's output $\vec{z}$ with respect to the change in the variables $\vec{w}$. Calculating the partial derivatives computes how much each \textit{single} parameter $w_i$ from the vector $\vec{w}$ contributes to the overall error. The change in the objective function $cost(\vec{w})$ with respect to the change of $w_i$ is given by:


\begin{equation*}
\Delta w_i = -\eta \frac{\partial cost(\vec{w})}{\partial w_{i}}
\end{equation*}

where $\eta$ is a learning rate, most commonly in the range $\eta \in [0,1]$, which determines the size of the steps taken in the direction of the greatest decrease of the function. When the element $w_i$ is updated using the equation described above ${w_i(m+1)=w_i(m)+\Delta w_i(m)}$, the error of that learnable parameter is decreased leading to an overall decrease in the training error $cost(\vec{w})$ .

The partial derivatives are calculated using the \textit{chain rule}. The chain rule looks at the dependencies that the learnable parameter $w_i$ has and expands the partial derivatives equation to include those dependencies. Therefor the partial derivative of $w_i$ can be expanded to:

\begin{equation*}
\frac{\partial cost(\vec{w})}{\partial w_{i}} = \frac{\partial cost(\vec{w})}{\partial net_i} \frac{\partial net_i}{\partial w_{i}} 
\end{equation*}

% \chrislong{it was at this point where I recalled the dependence of $J$ on $net$; this could be made more transparent when this is first defined. Is this the only nested dependency needed to calculate the $\nabla J(\vec{w})$? You claim below that it is not but it is not immediately apparent where these additional dependencies are introduced in $J$ --- note that, the precise dependency will affect what `version' of the multivariate chain rule is needed.}

When the partial derivatives is computed with respect to a parameter, such as weight or bias, deeper in the network, the dependencies are all the weights and activation function's net values between that weight and the output have to be computed. This is due to the series of equations performed in a sequential manner when executing the forward pass ($\ie$ nested functions as described in Section~\ref{Sec:process}). The chain rule is just the way to compute the partial derivative of the error with respect to any variable in the nested function. The operation of updating all learnable parameters is called backpropagation because the operation starts from the last layer, as the amount of dependencies is lower and then reuses some of the calculation when determining the partial derivative of weights and biases deeper in the network. 



% \begin{equation*}
% \quad \Delta \mathbf{w}(m)=-\eta \frac{\partial J(\mathbf{w}(m))}{\partial \mathbf{w}(m)}
% \end{equation*}
\subsection{Regularisation techniques}
Training a machine learning model extensively on a given dataset can often lead to overfitting, a process where the model becomes too tailored to the training data and performs poorly on new, unseen data. In essence, the model learns the noise and random fluctuations in the training data as if they are meaningful patterns. This is particularly likely to happen when the model is complex, with numerous parameters relative to the size of the dataset, or when the training process is not properly regularised. Here we present couple of the most common regularisation techniques used during training to avoid overfitting.

\subsubsection{L1 and L2 Regularisation}
L1 and L2 regularisation are techniques used in machine learning to prevent overfitting~\cite{ng2004feature} by adding a penalty term to the loss function that the algorithm optimises. By doing so, regularisation constrains the complexity of the model, discouraging it from fitting the training data too closely. This helps the model generalise better to new, unseen data.

\textbf{L1 Regularisation}
L1 regularisation adds an additional term to the loss function of a model. The loss function is a measure of how well the model fits the training data. By adding this regularisation term, the model is penalised for having large weights (parameters). The key characteristic of L1 regularisation is its tendency to drive some weights to exactly zero, leading to sparse models. This happens because the penalty for having non-zero weights is linear in the weight's values.
  
\begin{equation*}
   C = C^- + \lambda \sum |w_i|, 
\end{equation*}
where $C^-$ is the cost without the additional term and $\lambda$ is a hyper-parameter that controls the strength of the regularisation. L1 regularisation is particularly useful when you have a high-dimensional data set (many features) and you suspect that not all features are relevant for prediction. By penalising the magnitude of the learned parameters, L1 regularisation helps simplifying the model by selecting a subset of the available features.

\textbf{L2 Regularisation}
L2 regularisation also works by adding a penalty term to the model's loss function. The regularisation term in L2 regularisation is the sum of the squares of the model parameters to the loss function. L2 regularisation tends to shrink the parameter values towards zero but, unlike L1 regularisation, it does not set any of them exactly to zero. This shrinking effect helps in reducing overfitting by ensuring that no single weight dominates the model. However, it does not result in feature selection, as L1 regularisation does.
  
\begin{equation*}
   C = C^- + \lambda \sum w_i^2, 
\end{equation*}

where the additional term $\lambda$ is a hyper-parameter that controls the strength of the regularisation. A $\lambda=0$ reduces the model to its original form, without any regularisation and a $\lambda>0$ penalises large weights, leading to a simpler, more constrained model.

\subsubsection{Dropout}
Dropout involves randomly setting a fraction of the input units to zero at each update during training time, effectively dropping out those units. By doing this, the neural network becomes less sensitive to the specific weights of neurons and becomes more robust, improving its generalisation capabilities~\cite{srivastava2014dropout}.

During each training iteration, dropout randomly selects some neurons and sets their outputs to zero during the forward and backward passes. The percent of neurons dropped is determined by a \textit{rate} hyperparamether which is set between 0 and 1. To compensate for the dropped neurons, the output from the remaining neurons is often scaled by $1/(1 - rate)$. For example, if dropout rate  is set to 0.5 (meaning approximately 50\% of neurons are dropped), the remaining neuron activations are scaled by a factor of 2 ( $2 = 1/(1 - 0.5)$ ) to maintain the expected output level. Dropout is applied only during training. 

Dropout forces the neural network layers to learn redundant representations that are robust to the absence of some neurons, effectively preventing overfitting~\cite{srivastava2014dropout}. By randomly dropping out neurons, dropout essentially creates a simpler version of the neural network during each training iteration, mimicking the effect of training multiple simpler networks. Dropout is computationally inexpensive compared to other regularisation methods, making it faster and easier to implement.


\section{Evaluation}

\label{setup}
A practice not only implemented when training neural networks, but when training any machine learning method in general, is to divide the dataset into a training, validation, and test subsets.

The training data serves as the foundational set on which the neural network learns the patterns or relationships in the data. Through multiple epochs, the model adjusts its internal parameters to minimise the loss function, which measures the difference between its predictions and the actual outcomes (see Section~\ref{sec:training}).Typically, the largest portion of the data is allocated for training. For example, in an 80-10-10 or 70-15-15 split, 70-80\% of the data might be used for training. It's essential that the training set is representative of the task that needs to be learned to ensure that the model learns the general trends and nuances across all classes or outcomes. The validation set serves as an unbiased benchmark during the training phase to evaluate the model's performance. It is also used for hyperparameter tuning and model selection. Typically the validation set is smaller than the training set but large enough to be statistically significant, perhaps 10-15\% of the total dataset. If the model learns too well on the training data, it may overfit and perform poorly on new data. Assessing if that is the case is the main purpose of the validation set. The test set provides the final evaluation metric for the model. Since it has not been seen by the model during training or validation, it represents how well the model generalises to completely new, unseen data. Like the validation set, the test set is generally smaller than the training set but large enough to offer statistically significant results, , again it is often around 10-15\% of the total data. It's crucial that the test data is entirely independent of the training and validation sets to offer an unbiased evaluation metric.

Before splitting, it’s often good to shuffle the data to ensure randomness and that each split is representative. Despite this, some tasks impose restrictions on the way data can be shuffled. For instance, in classification problems, maintaining the same ratio of classes in each subset is often important, especially when the dataset is imbalanced. In time-series problems, random shuffling is generally not done, and chronological splits are used to simulate real-world scenarios. Understanding how to appropriately divide and use these subsets of data is critical for training and evaluating robust and effective neural network models that not only learns how to perform on the training data but also generalises well to new data.


\subsection{Evaluation Metrics}
\label{sec:eval}
Evaluation serves as a feedback mechanism, helping to identify the aspects of the model that need improvement. In critical applications like healthcare, autonomous driving, and finance, a poorly evaluated neural network can have dire consequences, including risking human lives. Rigorous evaluation is not just a good practice but a moral imperative in such cases. The following section will delve into the various evaluation metrics commonly used to assess the performance of neural networks. These metrics provide quantitative ways to measure the effectiveness, reliability, and overall quality of the model.

\subsubsection{Accuracy}

Accuracy is defined as the ratio of correctly classified samples to the total number of samples. 
\begin{equation*}
    \text{Accuracy} = \frac{\text{Number of Correct Predictions}}{\text{Total Number of Predictions}} 
\end{equation*}

Accuracy is most useful when the classes are balanced -- if one class vastly outnumbers the other, a high accuracy can be misleading. For example, if 95\% of the samples are of Class A, predicting Class A for all samples would still yield a 95\% accuracy. This metric also imposes that the misclassification for one class should be no worse than the misclassification for another. Note that this is not the case for all domains. For instance in medical diagnostics, missing a positive case for a critical condition like cancer is generally considered far worse than falsely identifying someone as having cancer when they do not. In this case, misclassification for one class (false negatives for cancer presence) is weighted more heavily than misclassification for false positives.

Similarly, in fraud detection, failing to identify a fraudulent transaction could have more severe repercussions, such as financial loss or legal complications, compared to falsely flagging a legitimate transaction as fraudulent. Hence, in such applications the misclassification costs are different and accuracy can not give the required information to evaluate the model in the desired way. 

\subsubsection{Precision, Recall and F1 Score}

Precision, Recall and F1 Score are used when the dataset is imbalanced and when the cost of false positives and false negatives are different, precision and recall offer a more nuanced evaluation. 

\textbf{Precision}, which is known as the positive predictive value, measures the number of true positives divided by the number of true positives and false positives. A high precision value indicates a low rate of false-positive errors, meaning that when the model predicts a positive class, it is likely correct. A low precision score implies that many of the instances the model predicts as positive actually belong to the negative class, making it a less reliable predictor for the positive class.
\begin{equation*}
\text{Precision} = \frac{\text{True Positives (TP)}}{\text{True Positives (TP)} + \text{False Positives(FP)}} 
\end{equation*}
While precision gives an idea of how reliable the positive predictions are, it does not show how many actual positive cases are missing (false negatives). Therefore, it is often used in conjunction with other metrics like recall and F1-score for a more comprehensive evaluation.

\textbf{Recall}, which is known as sensitivity, measures the number of true positives divided by the number of true positives and false negatives. Recall measures the ability of a classifier to correctly identify all relevant instances, essentially asking, "Of all the possible positive labels, how many did the model correctly identify?". A high recall value means that the false negative rate is low, implying the model is good at identifying positive instances as such. A low recall score indicates a high number of false negatives, meaning the model often fails to identify positive instances.
\begin{equation*}
\text{Recall} = \frac{\text{True Positives(TP)}}{\text{True Positives(TP)} + \text{False Negatives(FN)}} 
\end{equation*}
Recall is particularly useful when the cost of false negatives is high. For example, in medical diagnostics, failing to identify a disease could have severe consequences. Generally, improving recall may result in a decrease in precision, and vice versa. This trade-off occurs because increasing the sensitivity of the model for identifying positive instances often leads to more false positives. This makes it challenging to optimise for both. While they offer a more nuanced view, using just precision or recall alone can still be misleading, which is why the F1 score is often used as a single metric that combines both.

\textbf{F1 Score}, a harmonic mean of precision and recall, combines both precision and recall into a single value, aiming to provide a more comprehensive view of a model's performance. It is especially useful when neither precision nor recall can be ignored. 
\begin{equation*}
\text{F1 Score} = 2 \times \frac{\text{Precision} \times \text{Recall}}{\text{Precision} + \text{Recall}} 
\end{equation*}
An F1 score close to 1 indicates a good balance of precision and recall, and thus a good model, whereas close to 0 indicates a poor model, possibly heavily biased or just inaccurate. The F1 score assumes equal importance of the two, which may not always be the case in specific applications. In datasets where one class is significantly outnumbered by the other, F1 score can give a better measure of the model's performance compared to accuracy.

\subsubsection{AUC-ROC (Area Under the Receiver Operating Characteristic Curve)}

The ROC curve is a graphical representation of the true positive rate (Recall) against the false positive rate, at various threshold settings (see Figure~\ref{Fig:Roc}). The method measures the area under the curve, providing a scalar value that quantifies the model's performance across all possible classification thresholds. A value of 1 represents a perfect model, while a value of 0.5 represents a random classifier. Typically used in binary classification problems and useful for comparing different models' performances. AUC-ROC may provide an overly optimistic view of the model's performance if there is a significant class imbalance. While AUC-ROC gives a single scalar value, it doesn't provide specifics on what threshold should be used for classification, which may require further investigation.

\begin{figure}[ht!]
	\begin{center}
		\includegraphics[width=0.6\linewidth]{Figures/ROC.png}
	\end{center}
	\caption{Illustration of a ROC curve~\cite{ROC}}
	\label{Fig:Roc}
\end{figure} 


These metrics are foundational tools for evaluating classification models. The choice of which metric to use often depends on the specific application, the nature of the data, and the cost of different types of errors. By understanding the pros and cons of each metric, one can better evaluate the model's performance in a manner most aligned with the objectives of the application.


\section{Conclusion}

Artificial neural networks are widely used for their computational capabilities --- they can learn and generalise even if they are trained on incomplete or conflicting data sets and do not need explicit human instructions. Importantly, neural networks possess remarkable fault tolerance once deployed. They can produce reliable outputs even if significant portions of the network are corrupted or lost. This makes them highly adaptable and useful in a range of applications where traditional algorithms fall short.

Starting with the underlying motivations and general structure, this chapter established the foundation for understanding how data is processed and transformed through the layers of a neural network. A detailed look into the constituent elements like neurons, layers, and activation functions shed light on their specific roles in the data transformation process.

The training subsection presents backpropagation, as the main algorithm through which neural networks learn. Possible regularisation changes to the traditional loss function are also explored, as a way to avoid overfitting. Next, we delved into the metrics needed to assess the performance of neural networks effectively. These metrics primarily inform us about how well the model has been trained, but not necessarily what the model has learned or how it arrives at its decisions. While measures like accuracy, precision, and F1-score provide valuable insights into the model's performance, they don't shed light on the model's internal workings.

In the following chapter, we will delve into explainability metrics, which are designed to address this very issue. These metrics aim to provide insights into what the neural network has actually learned and how it makes its decisions, offering a more comprehensive understanding of the model beyond its performance indicators. Understanding what a model has learned is crucial for trust, validation, and potential debugging of neural network models, particularly in critical applications where explainability is of paramount importance.
\chapter{Literature Review}
\label{chap:lit}
\section{Introduction}

Building upon the foundational understanding of artificial neural networks' capabilities an structure discussed in the previous Chapter~\ref{chap:background}, this chapter transitions into a critical examination of deep neural networks explainability. While these networks exhibit high adaptability and robustness, often surpassing traditional algorithms in complex scenarios~\cite{HeZRS16, SilverHMGSDSAPL16, DengHK13, LeCunBH15}, their intrinsic decision-making processes remain largely obscure. This opacity presents a significant challenge, especially in applications with far-reaching consequences, necessitating a deeper exploration into the explainability of these models. As neural networks become more involved in critical areas like healthcare, finance, and autonomous systems, the need to comprehend their decision-making processes intensifies. Understanding what the model has learned is pivotal not just for model validation and trust-building but also for identifying potential biases and ensuring ethical AI practices. 

The literature review chapter covers the general directions of research in explainable deep neural networks (XAI). It delves into various methodologies and frameworks that have been developed to make these complex models more interpretable. Further, the review explores the different metrics and benchmarks used to evaluate the effectiveness of XAI methods, as well as the ongoing open questions and challenges in the field.

\section{Evolution of Explainable AI and Categorisation of Explainability Methods}

The evolution of Explainable AI (XAI) as a field is vividly captured by the increasing number of scholarly publications over recent years, as illustrated in Figure \ref{Fig:XAI}. This trend underscores the growing academic and industry interest in making AI systems more transparent and understandable, paralleling the rapid advancements in artificial intelligence and machine learning.

The initial stages of XAI can be traced back to simpler AI models, where explainability was often a byproduct of the model's simplicity and transparency. Early AI systems, such as rule-based systems~\cite{hayes1985rule} and linear models~\cite{searle1997linear}, were relatively easy to interpret. More recently, the research focus has shifted towards more complex models like neural networks and deep learning, which provided better performance at many tasks~\cite{de2010combining,varadi2022alphafold,kuutti2020survey,cavalcante2016computational,huang2020artificial} but at the cost of reduced interpretability.

This transition marked a pivotal moment in the evolution of XAI. The complexity of deep learning models, characterised by their deep architectures and non-linear transformations, made it challenging to understand their decision-making processes. This complexity barrier sparked a surge in XAI research, aiming to demystify these ``black box'' models~\cite{VandewieleJOTH16, BastaniKB17a, ZhangWZ18a, NguyenYC16, SabourFH17, LinsleySES19, ShiXXCLLG21, SimonyanVZ13, SpringenbergDBR14, Ribeiro0G16, LundbergL17, ElenbergDFK17, Ribeiro0G18, ShrikumarGK17, SundararajanTY17, SmilkovTKVW17, ChattopadhyaySH18, bach2015pixel}.

\begin{figure}[ht!]
\begin{center}
\includegraphics[width=0.9\textwidth]{Figures/documents_bar_graph.png}
\end{center}
\caption{Evolution of the number of total publications whose title, abstract and/or keywords refer to the field of XAI during the last years. Data retrieved from Scopus (November 20th, 2023)~\cite{Scopus} by using the search terms: XAI, Interpretable Artificial Intelligence and Explainable Artificial Intelligence.}
\label{Fig:XAI}
\end{figure} 

Previous research has categorised explainable AI models using various criteria that relied on factors like structure, scope, dependence, and dataset~\cite{IbrahimS23}. Structure wise, AI systems can be explainable by nature (\ie intrinsic) or explainable by interpretability methods after the black-box model is already trained (\ie post-hoc)~\cite{DuLH20, abs-1901-04592}. Intrinsic explainability is inherent in the model's architecture, making it transparent and interpretable from the outset. In contrast, post-hoc explainability involves techniques applied to already trained, opaque models to elucidate their decision-making process. Regarding scope, XAI models can either focus on individual data points for specific explanations (local) or assess overall network behaviour (global)~\cite{LiCSBGQWGZXC22, ArrietaRSBTBGGM20, DuLH20}. Local models are crucial for understanding and justifying individual decisions, whereas global models offer a broader perspective on the network's decision patterns.
The dependency of XAI models also varies. Some are designed for specific AI systems (model-specific), offering detailed insights tailored to particular network architectures. Others are more versatile, applicable across various networks (model-agnostic), though they may sacrifice some specificity in their explanations~\cite{ArrietaRSBTBGGM20, DuLH20}. When categorised by the dataset, explainability models could be separated by the intput data they require, such as images, text, and tabular data~\cite{ArrietaRSBTBGGM20}.


This chapter's literature review adopts a similar structure. The aim of this literature review is not to present the whole of the explainable artificial intelligence literature, but to provide a focused overview of the methods that are either specifically designed or applicable for solving the problem of interpretability in deep neural networks. It explores the nuances of intrinsic and post-hoc explainability within the context of deep learning, highlighting how each approach contributes to understanding complex neural architectures. The review also examines the local and global scopes of explainable artificial neural networks, analysing how these perspectives offer insights into individual predictions and overall network behaviour. By focusing on these aspects, the chapter provides a comprehensive review of the current research in explainable deep learning methods, delving into the various metrics for evaluating it, and concluding with an analysis of future prospects and inherent limitations.

\section{Intrinsically Interpretable Models}

Intrinsic interpretability in neural networks focuses on the inherent transparency of their architecture. Such models are intentionally designed for simplicity and clarity, ensuring that the process of transforming input data into outputs is easily understandable. This approach is in stark contrast to more complex, opaque `black-box' models where the decision-making mechanism are far from obvious. The significance of intrinsic interpretability becomes especially pronounced in sensitive areas, where comprehending the rationale behind decisions is just as vital as the decisions themselves.

To illustrate, the most basic models are naturally endowed with global interpretability. Consider logistic and linear regression models, which are grounded in linear equations. These equations establish a direct, transparent connection between input variables and the output. This linear relationship allows for a readily comprehensible view of how alterations in input features influence predictions. Decision Trees provide another example. They function by segmenting the input space based on input feature values, constructing a tree-like set of decision rules. The journey from the tree's root to a leaf symbolises the decision-making path, offering a clear and traceable route to the final decision. Similarly, K-Nearest Neighbours (KNN) adheres to a straightforward principle. It assigns classifications to new instances based on the predominant class among its `k' nearest neighbours in the feature space, drawing its interpretability from the intuitive concept of similarity to known data points. Rule-based Learners, operating on explicit if-then rule sets, also exemplify this clarity, as each rule directly maps specific conditions to outcomes.

\subsection{Intrinsic Globally Interpretable Models}
\label{sec:Intrinsic}

Building upon these foundational models, intrinsic interpretability extends beyond inherently clear-cut systems. One can enhance a model's interpretability by integrating constraints within the network (discussed in Section~\ref{sec:Constraints}). This approach moulds the network into an inherently interpretable form. Additionally, interpretability can be achieved by employing simpler, understandable models like decision trees, rule-based models, or linear models to approximate the functionality of a complex system (as explored in Section~\ref{sec:extraction}). This strategy bridges the gap between high performance and transparency, harnessing the strengths of both simple and complex models in a cohesive framework.

\subsubsection{Adding Interpretability Constraints}
\label{sec:Constraints}

This approach is rooted in the integration of specific constraints into the neural network's architecture. These constraints are designed not just to direct the learning process, but also to ensure that the model's internal mechanisms remains comprehensible. This is a significant departure from traditional deep learning models, where the focus is predominantly on optimising performance, often at the expense of transparency. In this subsection, we delve deeper into the application of interpretability constraints in neural networks, highlighting the work of Zhang~\textit{et al.}~\cite{ZhangWZ18a} and the development of Capsule Networks (CapsNet)~\cite{SabourFH17} as key examples.


Zhang~\textit{et al.}'s~\cite{ZhangWZ18a} work proposes a method to transform traditional convolutional neural networks into interpretable convolutional neural networks. Interpretable convolutional neural networks have each filter in a deep convolutional layer represent a specific object part, which is automatically assigned during training. This is unlike traditional CNNs where a filter might represent a mixture of patterns~\cite{NguyenYC16} and the training process doesn't impose the patterns learned. The method is versatile and can be applied to various types of CNNs with different structures. It also uses the same training data as ordinary CNNs without requiring any annotations of object parts or textures for supervision. The method achieves its objective by introducing a special loss function, which pushes the filters towards representing an object part. A masking operation is also used to aid in filtering out noisy activations. This results in disentangled representations, resulting in interpretable CNN filters' activations that could detect semantically meaningful natural objects. The location of filter activation is also more stable and consistent than in traditional CNNs, meaning when a filter is trained to recognise a specific part, it activates in the same or very similar locations across different images when that part is present. However, there was a slight decrease in discrimination power, which the researchers aimed to keep within a small range.

A different approach introduced a novel type of neural network architecture known as a Capsule Network (CapsNet)~\cite{SabourFH17}. It combines a group of neurons, and a vector, called an activity vector, which represents various properties of a specific type of entity such as an object or an object part. The length of this activity vector indicates the probability of the entity's presence, and its orientation represents initial parameters like pose, deformation, texture, etc. Capsules use a dynamic routing process to determine where their outputs should be sent in the network hierarchy. This offers more interpretability than standard CNNs and avoids the exponential inefficiencies in representing spatial hierarchies, common in CNNs. The network's architecture also allows for effective segmentation and recognition of overlapping or complex objects, which is shown on the MNIST dataset~\cite{deng2012mnist}, especially excelling at recognising highly overlapping digits. Similar to generative models, CapsNets also tend to account for all aspects of an image. Therefore, the varied backgrounds in the CIFAR-10 dataset~\cite{cifar} challenged the model, contributing to poorer performance compared to datasets with less complex backgrounds.


The integration of interpretability constraints within neural network models, as exemplified by the approaches of Zhang ~\textit{et al.}'s~\cite{ZhangWZ18a} work and the Capsule Network~\cite{SabourFH17}, reveals a fundamental trade-off between interpretability and prediction accuracy. While these methods enhance the clarity and understanding of network decisions by making each filter or capsule represent specific, identifiable object parts, this increase in transparency often comes at the cost of reduced discriminative power. Interpretable models like the interpretable CNN and CapsNet excel in providing meaningful, disentangled representations and handling complex visual tasks with greater human-like understanding. However, their performance, in terms of raw prediction accuracy, may not always match that of less interpretable, traditional neural network architecture.

\subsubsection{Model Extraction}
\label{sec:extraction}

Model extraction offers an alternative route to achieving intrinsically interpretable models. This method involves simplifying complex, often non-transparent models into more understandable forms without major performance trade-offs. By distilling a complex model like a deep neural network into more interpretable structures such as decision trees, rule-based systems, or linear models, model extraction maintains prediction accuracy while enhancing interpretability. This makes it easier for users to trust and comprehend the model's decisions.

One notable example is GENESIM~\cite{VandewieleJOTH16}, an algorithm that converts an ensemble of decision trees into a single decision tree. This new tree is not only more interpretable but also maintains sometimes even improves, accuracy compared to traditional decision tree induction techniques. GENESIM stands out for its balance of low model complexity and high interpretability, matching the performance of ensemble techniques.

Bastani~\textit{et al.}~\cite{BastaniKB17a} introduced a method to create an interpretable model, like a decision tree, from a complex, opaque model. Their algorithm employs active learning to build an interpretable model from a complex model. It actively samples numerous training data points, labelled using the complex model, to ensure that the interpretable model doesn't overfit to a limited initial dataset. Their method outperforms the CART decision tree learning algorithm~\cite{BreimanFOS84} in approximating complex models and has practical applications in understanding biased features and comparing models trained on the same data.

Another significant approach is the Classification Accuracy Reduction (CAR) method~\cite{VandewieleJOTH16}. It focuses on making deep convolutional neural networks (CNNs) more interpretable without significantly compromising accuracy. CAR employs a greedy structural compression scheme that prunes less impactful filters, resulting in smaller, more interpretable CNNs. Despite fewer filters, these networks retain a diverse filter range and maintain nearly original accuracy levels. The importance of each image category to each CNN filter is also quantified. Tested on various datasets and CNN architectures like LeNet~\cite{LeCunBDHHHJ89}, AlexNet~\cite{KrizhevskySH12}, and ResNet-50~\cite{HeZRS15}, CAR shows its effectiveness in reducing filters without losing functional diversity or accuracy.

Model extraction methods present a compelling approach for creating interpretable models from complex, opaque systems. Techniques like GENESIM~\cite{VandewieleJOTH16}, Bastani et al.'s method~\cite{BastaniKB17a}, and the CAR approach~\cite{VandewieleJOTH16} effectively simplify models such as deep neural networks into more transparent structures like decision trees or reduced CNNs. These methods aim to maintain the accuracy of the original models while significantly improving their interpretability. This increased transparency facilitates user trust and understanding, crucial in fields where decision-making processes need to be transparent and justifiable.

However, there are inherent drawbacks to model extraction. A critical aspect to consider in model extraction techniques is that even when these simplified models match the accuracy of their original counterparts, they may not provide a true representation of what the original model has learned. This discrepancy arises because accuracy alone does not capture the entire learning process of a complex model. The original models, especially deep neural networks, often capture intricate, multi-dimensional patterns and relationships within the data. When these models are distilled into simpler forms, such as decision trees or pruned neural networks, the subtleties and complexities of these relationships can be lost or overly simplified. Additionally, while these techniques aim to maintain the original model's performance, there can be instances where the extracted model's accuracy is slightly compromised, especially in scenarios involving highly intricate data patterns.


\subsection{Intrinsic Locally Interpretable Models}

Unlike globally interpretable models which aim to provide a broad understanding of a model's overall functioning, intrinsic locally interpretable models ensure that learning methods focus on specific parts of the input or learn concrete patterns. The two types of methods deployed in this category are \emph{attention based} methods and \emph{loss based} methods.


\subsubsection{Attention-based explainability}

This approach relies on the concept of \textit{attention mechanisms} primarily used in neural networks. In essence, these mechanisms allow the model to focus on specific parts of the input data when making a decision, much like how humans pay more attention to certain aspects of a problem when solving it. In the context of neural networks, especially those used for natural language processing and image recognition, attention mechanisms can provide insights into which parts of the input data (such as words in a sentence or areas in an image) the model considers most important for a particular prediction. This localised focus helps in understanding the model's decision-making process on a case-by-case basis. For example, in text analysis, an attention-based model can highlight specific words or phrases that were pivotal in determining the sentiment of a sentence. Similarly, in image recognition tasks, such models can illustrate which regions of an image were most influential in classifying the image. This granularity not only aids in interpreting individual predictions but also in identifying potential biases or errors in the model's learning process.

Bahdanau et al.'s work~\cite{BahdanauCB14} exemplifies this by introducing an attention-based model that autonomously describes image content. It visualises how the model learns to focus on salient objects while generating corresponding words. The paper addresses the challenge of combining the computer vision task of object detection with the natural language processing task of generating coherent sentences. The model uses convolutional neural networks (CNNs) to encode images into feature vectors and recurrent neural networks (RNNs) to decode these vectors into natural language sentences.

Another domain where this type of explainability is used is fine-grained classification. Fine-grained classification involves recognising subordinate-level categories, such as different types of birds or dog breeds, which is challenging due to subtle and local differences among categories. The problem is further compounded by variances in pose, scale, and rotation. Xiao~\textit{et al.}'s~\cite{XiaoXYZPZ15} work proposes applying visual attention to fine-grained classification tasks using deep convolutional neural networks. The model integrates bottom-up attention (\ie attention shaped by current features) for proposing candidate patches, object-level top-down attention (\ie attention shaped by prior knowledge) for selecting relevant patches, and part-level top-down attention (\ie directed by knowledge about which parts of an object are most informative for the classification task) for localising discriminative parts (see Figure~\ref{Fig:loss-based-grained}). The approach improved fine-grained classification accuracy significantly, particularly under weak supervision settings.

\begin{figure}[ht!]
	\begin{center}
		\includegraphics[width=1\linewidth]{Figures/Xiau.png}
	\end{center}
	\caption{Complete classification pipeline of the Xiao~\textit{et al.}'s~\cite{XiaoXYZPZ15} method. The photo shown is from~\cite{XiaoXYZPZ15}}
	\label{Fig:loss-based-grained}
\end{figure} 

The works of Bahdanau~\textit{et al.}\cite{BahdanauCB14} and Xiao~\textit{et al.}\cite{XiaoXYZPZ15} exemplify the practical application and effectiveness of attention-based models. Bahdanau ~\textit{et al.'s }\cite{BahdanauCB14} model, which combines object detection in images with sentence generation, showcases the model's ability to focus on relevant objects and articulate them in natural language. Xiao~\textit{et al.}~\cite{XiaoXYZPZ15} approach to fine-grained classification shows how visual attention can enhance accuracy in identifying subtle differences among categories, particularly in challenging scenarios with minimal supervision. However, integrating attention mechanisms into a model's architecture does alter the way the model performs inference and can change the learned function, but it does not fundamentally overhaul the entire inference process or the core structure of the model. This does mean that this types of methods, similarly to Section~\ref{sec:extraction} do not provide a true representation of what the original model has learned. 


\subsubsection{Loss-based explainability}

This class of methods focuses on adjusting the loss function (\ie the core metric that measures the difference between the predicted and actual outcomes) to guide the network towards learning features that are more interpretable to humans or align with preselected aspects deemed important. In traditional machine learning models, the loss function primarily drives the accuracy and performance of the model, often at the cost of interpretability. However, in loss-based explainability, the loss function is intentionally designed or modified to incorporate interpretability constraints. These constraints can be in the form of regularization terms that penalize the model for learning complex, non-interpretable patterns, or they can explicitly encourage the model to focus on specific, predefined features that are easier for humans to understand.

Unlike previous attention mechanisms, Loss-based attention~\cite{ShiXXCLLG21} does not add attention layers to CNN. The authors introduce a mechanism that uses the same parameters to learn both patch weights and image predictions simultaneously, linking the attention mechanism with the loss function.  The proposed loss-based attention mechanism is designed to focus on significant patches during training, preserving the spatial relationship of these patches and avoiding reliance on additional annotations (see Figure~\ref{Fig:loss-based}). The proposed deep architectures remove max-pooling or stride operations in convolutional layers as it boosts the accuracy of patch identification.

\begin{figure}[ht!]
	\begin{center}
		\includegraphics[width=1\linewidth]{Figures/Loss-based.png}
	\end{center}
	\caption{Images highlighting the areas found relevant by the loss-based attention. The photos shown are samples from~\cite{ShiXXCLLG21}}
	\label{Fig:loss-based}
\end{figure} 



Global-and local attention (GALA) was integrated with neural networks like ResNet-50 supervised by ClickMe maps to encourage the selection of visual features favoured by humans~\cite{LinsleySES19}. ClickMe.ai is an online game for large-scale data acquisition. It involved participants playing with convolutional neural network partners to recognise images from the ILSVRC12 challenge. Players selected image parts informative for recognising the category, creating top-down attention maps (\ie attention shaped by prior knowledge). ClickMe maps were used to supervise GALA modules, introducing an additional loss to the training process. Models supervised with ClickMe maps demonstrated improved object categorisation accuracy and predicted ClickMe maps more effectively (see Figure~\ref{Fig:GALA}). This led to improved interpretability as ClickMe supervision led to features that were more local and consistent with human-selected features.

\begin{figure}[ht!]
	\begin{center}
		\includegraphics[width=1\linewidth]{Figures/GALA.pdf}
	\end{center}
	\caption{Images showing the areas found relevant by GALA both supervised by ClickMe maps and without. The photos shown are samples from~\cite{LinsleySES19}}
	\label{Fig:GALA}
\end{figure} 

\subsection{Discussion: Intrinsically Interpretable Models}

The methods in the category of intrinsically interpretable models focus primarily on either learning an interpretable version of the model, modifying the existing model to enhance its interpretability or changing the learning process to produce a more interpretable model. These types of approaches offer a compromise, allowing the use of accurate, complex models while still gaining some level of interpretability. It makes it possible to apply AI in critical domains without completely sacrificing the model's predictive power. They are generally easier to implement and require less computational power than complex models. Their simplicity makes it easier to communicate findings and understand the learned function by the model.


This approach is crucial in fields where the ability to understand and trust AI decisions is as important as the decisions themselves. However, it's important to acknowledge the trade-offs involved, particularly in terms of the accuracy of the model. The introduction of interpretability often impacts the model's accuracy. For instance, when modifying convolutional neural networks to enhance interpretability by having each filter represent a distinct object part, there's often a slight compromise in the model's ability to discriminate effectively. This phenomenon is also evident in scenarios where a new model is designed to be inherently interpretable, such as with decision trees. In these cases, to preserve a level of accuracy comparable to the original model, the decision tree may incorporate highly detailed and complex information. As a result, despite the model's intrinsically interpretable nature, the complexity of the information it encodes can still render it challenging to comprehend easily. Conversely, if the learned model is overly simplistic, it can overlook complex interactions in the data. The challenge lies in finding a balance between interpretability and accuracy. Too much emphasis on one can significantly diminish the other. Ongoing research aims to mitigate this issue, seeking methods that retain high accuracy while providing clear insights into the model's decision-making process.

 
Given the challenges associated with intrinsically interpretable models, recent literature is increasingly concerned with interpreting models post-training~\cite{MarkusKR21}. These approaches involves applying techniques to already trained models to extract insights and understand their decision-making processes. They provide insights into complex models like deep neural networks without altering the underlying model architecture and need for retraining.

\section{Post-hoc Explanations}
\label{sec:post-hoc}

This section shifts focus to post-hoc explanations in machine learning, representing a departure from intrinsic interpretability. Unlike the direct integration of transparency into a model's architecture, post-hoc explanations aim to clarify the decision-making processes of models that have already been trained. These methods arise in response to the limitations and compromises found in intrinsically interpretable models, especially when applied to complex systems.

The objective of post-hoc explanations is to retain the performance and pattern recognition abilities of complex models while revealing the mechanisms behind their decisions. This goal is pursued without altering the structural integrity of the models, thereby preserving their accuracy and intricacy. Exploration in this section includes an examination of various methodologies and tools used in post-hoc explanations, focusing on their role in improving the interpretability of complex models. The effectiveness of these techniques across different application scenarios, along with their limitations and the challenges they face, is assessed.


\subsection{Post-hoc Global Explanations}
\label{section:postglobal}

Post-hoc global explanations attempt to map the overall logic and patterns that a model uses to make its decisions across a wide range of inputs. This involves distilling the complex, often multi-dimensional decision processes of a model into more comprehensible, overarching principles or patterns. The methods employed here typically aim to identify general trends, rules, or dependencies that are indicative of the model's behaviour on a macro level.

\subsubsection{Activation Maximisation Methods}

A key method for providing a global explanation involves identifying the preferred inputs for neurons at specific layers, commonly framed within the activation maximisation (AM) framework as outlined by Simonyan~\textit{et al.}~\cite{SimonyanVZ13}. This work involves generating inputs that, when fed into the model, yield high scores for a target category. These scores are determined by the model’s classification layer for each input, with the goal of finding an input that not only maximises the class score but also retains a degree of normal appearance. This is achieved through the incorporation of L2 regularisation. The technique leverages back-propagation, typically used for training neural networks to adjust internal parameters, but with a slight modification. In this scenario, the network's weights remain fixed post-training, and the process starts with a blank input. Iterative adjustments are made until the input is identified that maximises the class score while appearing normal (see Figure~\ref{Fig:AM}). This technique effectively reveals the network's internal representation of a class, derived from its training, and can be applied to neurons at any neural network level.

\begin{figure}[ht!]
	\begin{center}
		\includegraphics[width=1\linewidth]{Figures/MA.pdf}
	\end{center}
	\caption{This images are showing the class appearance, learnt by a ConvNet, trained on ILSVRC-2013. The photos shown are samples from~\cite{SimonyanVZ13}}
	\label{Fig:AM}
\end{figure} 

However, the apparent simplicity of this approach hides the challenges it faces, particularly in generating interpretable inputs. The optimisation process can often produce unrealistic and noisy explanations. Without adequate regularisation, this process might yield results that activate neurons but remain unrecognisable. Overcoming this challenge involves constraining the optimisation process with natural priors, ensuring synthetic inputs resemble natural ones more closely. Researchers have proposed various hand-crafted priors like total variation norm and Gaussian blur~\cite{NguyenDYBC16}. More robust regularisation can be achieved using natural image priors from generative models like Generative Adversarial Networks (GANs) or Variational Autoencoders (VAEs)~\cite{NguyenDYBC16}. In Nguyen~\textit{et al.}'s paper~\cite{NguyenDYBC16}, the results (see Figure~\ref{Fig:VAE}) underscore the effectiveness of using generative model priors in enhancing visualisation quality when generating images. These visualisations provide deep insights into CNN processing, especially when trained on datasets like ImageNet~\cite{deng2009imagenet}.

\begin{figure}[ht!]
	\begin{center}
		\includegraphics[width=1\linewidth]{Figures/VAE.png}
	\end{center}
	\caption{Images synthesised from scratch to highly activate output neurons in the CaffeNet deep
neural network. The photos shown are samples from~\cite{NguyenDYBC16}}
	\label{Fig:VAE}
\end{figure} 
Using traditional activation maximisation techniques shows only one type of feature that a neuron detects, research by Nguyen~\textit{et al.}~\cite{NguyenYC16} shows the multiple facets detected by a neuron. For instance, a neuron linked to the concept of a ``grocery store'' may be activated by diverse visual cues, ranging from rows of produce to images of storefronts (see Figure~\ref{Fig:VAE_2}). This nuanced understanding of neuronal functionality marks a significant departure from earlier methods, which often produced images with unnatural colour distributions and incoherent, repetitive fragments, failing to capture the diverse stimuli each neuron can respond to. The paper delves into two principal deep visualisation techniques: activation maximisation and code inversion. Activation maximisation is focused on identifying an image that maximally activates a particular neuron, thereby shedding light on the specific features that the neuron is sensitive to like in the work by Simonyan et al~\textit{et al.}~\cite{SimonyanVZ13}. Code inversion, in contrast, involves creating an image that yields an activation vector similar to that produced by a specific real image, thereby unravelling the image-specific information encoded by the DNN at a certain layer. To address the issue of fragmented and inconsistent image outputs seen in previous techniques, Nguyen~\textit{et al.} introduce a novel centre-biased regularisation approach. This method strategically biases the image optimisation process to favour the formation of a single, central object in the generated image. The result of this adjustment is a more coherent and structurally integrated visual output, enhancing the overall interpretability and coherence of the images produced through these deep visualisation processes.

\begin{figure}[ht!]
	\begin{center}
		\includegraphics[width=1\linewidth]{Figures/Nyuyen.png}
	\end{center}
	\caption{A multifaceted visualisation of example neuron feature detectors from the eight layer. The photos shown are samples from~\cite{NguyenYC16}}
	\label{Fig:VAE_2}
\end{figure} 

Work by Zhou~\textit{et al.}~\cite{ZhouKLOT14} further shows that when CNNs are trained to classify scenes, they inherently develop object detectors within their structure. This is significant because these object detectors form without explicit training or supervision on objects. This finding implies that a single network can perform multiple tasks like scene recognition and object localisation in one forward pass, without being explicitly taught about objects.

\subsubsection{Sequential Data Methods}

Other approaches aim to understand the performance and limitations of LSTMs, a type of RNN that has gained popularity due to its success in various machine learning tasks involving sequential data. Karpathy~\textit{et al.}~\cite{KarpathyJL15} use character-level language models to analyse what most strongly activate individual units in this layer. These investigations reveal that certain units within RNNs can effectively grasp intricate linguistic features, such as syntax and semantics. LSTMs that can even track long-range dependencies like line lengths, quotes, and brackets. In work by Kádár~\textit{et al.}~\cite{KadarCA17}, the authors use a word-level language model to examine the linguistic attributes encoded by the individual hidden units of a Recurrent Neural Network (RNN). Visual analysis from this work reveals that select units primarily respond to distinct semantic types. Other units, however, are adept at identifying specific syntactic categories or fulfilling particular dependency roles. Additionally, a notable observation is that certain hidden units are capable of maintaining their activation levels into subsequent temporal stages. This functionality sheds light on the RNN's proficiency in understand long-term dependencies and multifaceted linguistic elements. 

Work by Peters and colleagues introduces ELMo~\cite{PetersNIGCLZ18}, a new type of word representation that models both complex characteristics of word use (like syntax and semantics) and how these uses vary across linguistic contexts. It shows that similar to CNNs, Recurrent Neural Networks (RNNs) are capable of learning hierarchical representations by inspecting different hidden layers. The research also shows that different layers of deep bidirectional RNNs (biRNNs) encode distinct types of information. For example, lower layers of a deep LSTM model were found to be more effective at tasks like part-of-speech (POS) tagging, indicating their ability to capture syntactic information. On the other hand, higher layers were better at encoding semantic information, consistent with machine translation encoders.


\subsubsection{Concept-Based Global Explanations}

Testing with Concept Activation Vectors (TCAV), bridges the gap between model feature representations and human-understandable concepts. Proposed by Kim et al., TCAV introduces Concept Activation Vectors (CAVs) to quantify the influence of predefined human concepts on a model's predictions~\cite{KimWGCWVS18}. CAVs are derived by training a linear classifier to distinguish activations of neural network layers associated with a specific concept from random examples. This representation allows the use of directional derivatives to compute a metric of sensitivity, quantifying the importance of each concept in decision-making. Unlike traditional attribution methods, TCAV operates at the conceptual level, aligning with human intuition and enabling domain experts to evaluate models without requiring extensive technical knowledge.

This method employs randomisation-based significance testing to ensure robustness against spurious correlations. TCAV has been validated across applications such as diabetic retinopathy diagnosis and image classification, revealing model biases and enhancing user trust. It effectively complements methods like saliency maps~\cite{SpringenbergDBR14} and Grad-CAM~\cite{SelvarajuCDVPB20} (discussed in Section~\ref{cam} \& ~\ref{decov}) by providing insights into a model's global decision-making patterns through user-defined concepts.

\subsection{Post-hoc Local Explanations}

Shifting from the overarching perspectives of global explanations, this subsection delves into the realm of post-hoc local explanations in machine learning. Contrasting with global explanations, local explanations focus on providing insights into specific individual predictions made by a model, offering a detailed understanding of the decision-making process on a specific instance.

Local explanations are particularly crucial in scenarios where the reasoning behind a single prediction needs to be understood and justified. This level of granularity is essential in fields like medicine, where understanding the rationale behind a specific diagnosis can be as critical as the diagnosis itself, or in financial services, where individual loan approval or rejection decisions must be transparent.


\subsubsection{Local Approximation Based Explanation}

Local approximation based explanation methods operate on the fundamental premise that the predictions made by a machine learning model for inputs in the vicinity of a specific query instance can be effectively represented using a simpler, interpretable model. This approach does not necessitate the interpretable, or ``white-box,'' model to demonstrate accurate predictions across the entire input space of the original, more complex ``black-box'' model. Instead, its primary requirement is to closely approximate the behaviour of the black-box model within a constrained, localised region surrounding the initial input of interest. In this context, the local approximation is achieved by constructing a white-box model that mimics the decision-making process of the black-box model for a subset of data points that are in close proximity to the query instance. This subset is often selected based on similarity measures or other criteria that ensure their relevance to the query instance. Once the local white-box model is trained and its performance in approximating the black-box model within the defined neighbourhood is validated, the next step involves dissecting the white-box model to understand its decision-making process. This is typically achieved by analysing the model's parameters or decision rules, which are inherently more interpretable than those of the original model.

Ribeiro~\textit{et al.}'s work on LIME (Local Interpretable Model-Agnostic Explanation)~\cite{Ribeiro0G16} is an algorithm designed to explain the predictions of any classifier in an interpretable and faithful manner. It approximates a complex model locally with an interpretable model, providing insights into individual predictions. Alongside LIME, the paper proposes SP-LIME, a method to select a set of representative instances and explanations in a non-redundant way, thereby helping users understand and trust the model as a whole. The technique starts by perturbing the input, which can either be done with a human-in-the-loop or autonomously. In the former case, the method starts by asking the user to perturb the input in a way that makes sense for them (\eg  letting the user remove a contiguous region of pixels in an image (\textit{super pixel}) or important words or phrases from text). In the latter case perturbing of the input is done randomly. This process creates instances that do not have some feature that is present in the original instance and are in the local region (with respect to the space of instances) of the original instance. These perturbations result in a new, local dataset around the instance being explained. This local dataset consists of the perturbed versions of the original instance, along with the predictions of the classifier for these perturbed instances. LIME then trains an interpretable model, such as a linear regression or decision tree, on this local dataset. The interpretable model is designed to approximate the predictions of the complex model as closely as possible, but only in the vicinity of the instance being explained. The contribution of each feature is then computed by passing it through the new explainable classifier. If a change in a feature changes the way the instance is classified, the contribution of that feature is assigned a value of one and if it does not the contribution of the feature is zero (binary feature's contribution). If the feature changes the classification then its presence constitutes a part of the explanation for why the original instance is classified the way it is. In this way, a feature can either be important or not important to the classification. The entirety of the explanation in LIME amounts to a binary vector where each element is a feature assigned one or zero (see Figure~\ref{Fig:LIME}). The authors conduct extensive experiments to validate the effectiveness of LIME and SP-LIME. These include simulated user experiments to measure the impact of explanations on trust, as well as evaluations with human subjects on tasks like choosing the best classifier, improving an untrustworthy classifier, and gaining insights into classifier behaviour. Subsequent implementations of LIME extend this framework to extract pseudo-relevance scores based on the relevance of the classification to the surrogate models. However, given that this is importance to an over-fitted surrogate, the relevance scores are somewhat misleading, and are often combined with some form of regularisation. As such, the default explanation is typically binary. 

\begin{figure}[ht!]
	\begin{center}
		\includegraphics[width=1\linewidth]{Figures/LIME.png}
	\end{center}
	\caption{Explanations for an image classification prediction made by Google’s Inception neural network for electric guitar, acoustic guitar and labrador. The photos shown are samples from~\cite{Ribeiro0G16}}
	\label{Fig:LIME}
\end{figure} 


LIME is foundational work in the area of ML interpretability and many methods have either been inspired by LIME or address some of its limitations~\cite{ElenbergDFK17, Ribeiro0G18, WhiteG20}. One limitation of the method is the turnaround time. In order to compute each feature contribution, LIME has to sample a multitude of instances in the case when there is no human-in-the-loop, each of them perturbing a different area of the original instance (a feature) and then classify all those perturbed instances and compare them to the original classification. This results in slow turnaround time, making the method unusable for explanations on-the-fly. The problem of \emph{slow explanation generation}, led Elenberg~\textit{et al.}'s~\cite{ElenbergDFK17} work for an efficient streaming algorithm, named Streak. The algorithm is efficient in terms of memory usage and does not require prior knowledge. When compared to LIME is shown to be produced at a far greater speed with a turnaround time up to 10 times faster than LIME. This improvement in speed is particularly significant for large neural networks like Inception V3. Another problem of Lime is the inability of the method to create counterfactual explanations, which are critical in scenarios requiring a nuanced understanding of model behaviour, as they indicate how a model's predictions could change with alterations in feature values. Methods like CLEAR (Counterfactual Local Explanations via Regression)~\cite{WhiteG20} have been developed to address this shortcoming, offering high-fidelity counterfactuals that outperform LIME in various domains. One of the significant limitations of LIME is its lack of a fidelity measure, as defined in the CLEAR paper, where fidelity reflects how accurately an explanation model represents the underlying decision boundary and generates reliable counterfactuals.

Another limitation of LIME that was noted by its authors is that there is a lack of clarity whether the explanation produced can be used only for the original instance, or for the original instance as well as other similar instances (i.e.\ instances which are in the local region with respect to the space of instances). This is referred to as the problem of ``\emph{unclear region of explanation}''

This has been addressed in subsequent work by Ribeiro~\textit{et al.} in a method called Anchors~\cite{Ribeiro0G18}. The method uses a concept called "anchors," which are high-precision rules representing local "sufficient" conditions for predictions. These explanations are designed to be model-agnostic, meaning they can be applied to any black-box model, providing high-probability guarantees for their accuracy. It finds all the features that matter for the classification and then creates if--then rules, where the antecedent part of the statement lists the features that can lead to a change in classification and the consequent part contains the classification itself. If a different data instance is presented that contains all the features in the ``if'' statement, then that feature is in the region of explanation of the original instance and the explanation will remain the same. The rules apply only when all the features are present in the data instance (all \emph{conditions} are met). The if statement is seen as the anchor of this local prediction, which is where the method takes its name. 
The process involves defining a precision level for an anchor and ensuring that the anchor achieves this precision with high probability (see Figure~\ref{Fig:Anchors}). This approach takes into account the probabilistic nature of predictions in machine learning, where absolute certainty is often unattainable. The goal is to maximise the coverage of an anchor, defined as the probability that it applies to samples from a perturbation distribution. To address the limitations of a greedy approach in finding anchors, the authors propose a beam-search method. This method maintains a set of candidate rules and selects the ones with the highest coverage, optimising the search for anchors that describe a larger part of the input space effectively. Sometimes, even the local behaviour of a model may be
extremely non-linear, so linear explanations like the ones used by LIME~\cite{Ribeiro0G16} could lead to poor performance. Anchors~\cite{Ribeiro0G18} could model the non-linear relationship as the local approximation due to the utilisation of the "if-then" rules. One of the problems of the Anchors approach is that in some more complex domains they may create very specific anchors (if conditions), which have very low coverage or are specific to the instances the model is trained on. Despite this limitation, the approach manages to achieve its original aim to have clear coverage and high precision for interpretable explanations of the local model's behaviour. 


\begin{figure}[ht!]
	\begin{center}
		\includegraphics[width=1\linewidth]{Figures/Anchors.pdf}
	\end{center}
	\caption{Explanations produced by Anchors for both image and text input. The photos shown are samples from~\cite{Ribeiro0G18}}
	\label{Fig:Anchors}
\end{figure} 



\subsubsection{Loss-based explainability}

One of the reasons to create explanations is to see if the classifier is a ``Clever Hans predictor'', meaning that it makes correct predictions but for the wrong reasons. If it turns out that it is, (i.e.\ it has erroneously explored features in the data that \textit{should be} irrelevant) the classifier is retrained under changed conditions (e.g.\ without the part of the data that contained spurious correlations). An alternative way to solve this issue is by feature selection. Feature selection is a technique that only exposes the machine learning algorithm to ‘good’ input features.

Work by Ross~\textit{et al.} focuses on creating ML algorithms that are “right for the right reasons” (RRR)~\cite{RossHD17}. The method uses LIME to produce explanations of how the model creates it's decisions and uses a human-in-the-loop to specify whether the feature should be relevant to the classification. If the feature is not relevant, the method constrains the input gradients to be small in irrelevant areas of the input space, based on an annotation matrix that indicates whether certain dimensions should be irrelevant for predicting specific observations. RRR uses binary masks to encompass the user's specification on whether the features are relevant or not, and so prevents the machine learning algorithm from learning spurious correlations which might be present within the training data. The paper shows that input gradient penalties enable learning of generalisable decision logic, even in datasets with inherent ambiguities. This method is especially effective for continuous inputs and provides a basis for further advancements in explanation optimisation.



\subsubsection{Class Activation Mapping Methods}
\label{cam}

Class Activation Map (CAM)~\cite{ZhouKLOT16} is introduced as a technique to generate visual explanations for CNN-based classification decisions. It highlights the discriminative regions of an image used by the CNN to identify a specific category. The authors show that CNNs, even when trained only with image-level labels (not object locations), can localise objects. This is attributed to the use of global average pooling layer (GAP), which calculates the average contribution of each feature map in the last convolutional layer (see Figure~\ref{Fig:CAM}). The paper proves that the use of GAP preserves the spatial information discarded by fully-connected layers. In the end, CAM overlays the activation map on the input image to identify the areas of interest the CNN used to make its prediction. CAM allows CNNs trained for classification to localise class-specific image regions in a single forward pass. This is significant for understanding and interpreting the decisions made by the network. The method achieves a top-5 error of 37.1\% for object localisation on ILSVRC 2014, close to the 34.2\% error rate of a fully supervised CNN. The paper compares global average pooling layer (GAP) over global max pooling layer (GMP), concluding that GAP is more effective for localisation tasks as it encourages the network to consider the entire extent of the object rather than focusing on the most discriminative part. The paper includes extensive experiments demonstrating the efficacy of GAP and CAM for weakly-supervised object localisation, comparing the performance with other methods on the ILSVRC dataset. They show that the localisation ability of GAP-trained networks is generic and applicable to various tasks beyond those it was trained for, including scene recognition and concept discovery. Application of the method to fine-grained recognition tasks (like bird species identification) shows significant improvement in accuracy when using localised regions identified by CAM. The technique is also used for discovering common elements in images and localising high-level concepts from weakly labelled images. The primary drawback of CAM is that it modifies the architecture of the model that is being interpreted, which does effect the prediction accuracy.

\begin{figure}[ht!]
	\begin{center}
	\includegraphics[width=0.8\linewidth]{Figures/CAM.png}
	\end{center}
	\caption{Explanations produced by CAM. The photos shown are samples from~\cite{ZhouKLOT16}}
	\label{Fig:CAM}
\end{figure} 

Grad-CAM~\cite{SelvarajuCDVPB20} was proposed to overcome the drawbacks of CAM and it does not change the CNN architecture. This technique uses the gradients of any target concept (\ie classification in a classification network) flowing into the final convolutional layer to produce a localisation map. This map highlights important regions in the image for predicting the concept. Unlike CAM, Grad-CAM does not require any specific model architecture. It can be applied to a broad range of CNN models, including those with fully connected layers and complex architectures designed for various tasks beyond simple classification. Grad-CAM is evaluated for localisation tasks and for its ability to faithfully represent the model's decision-making process. The technique provides insights into model failures and helps identify biases in datasets.

Grad-CAM++~\cite{ChattopadhyaySH18} is a more generalised version of Grad-CAM, designed to address some of its limitations, particularly in cases of localising multiple instances of the same object in an image and fully capturing single objects. Grad-CAM++ modifies the approach used in Grad-CAM by focusing on the positive gradients of the class score with respect to feature maps and incorporating higher-order derivatives to calculate the weights for the activation maps. This approach allows for a more detailed and class-specific heatmap generation, leading to better visualisation of the regions in the image that contribute to the model's decision.


Grad-CAM++~\cite{ChattopadhyaySH18} is further improved by Smooth Grad-CAM++~\cite{abs-1908-01224}, which combines elements from SmoothGrad~\cite{SmilkovTKVW17} --- methods for reducing the noise in the explanation by adding random noise to the input image and averaging the sensitivity maps across multiple such noisy images, and Grad-CAM++~\cite{ChattopadhyaySH18}. It provides sharper and more precise visual explanations of deep CNN model decisions. Smooth Grad-CAM++ allows visualisation at the level of specific layers, subsets of feature maps, or even subsets of neurons within a feature map, which is a significant advancement over previous techniques. The technique integrates gradient smoothing into Grad-CAM++, where noise is added to the sample image of interest and the average of all gradient matrices generated from each noised image is taken (see Figure~\ref{Fig:Grad-CAM}). This process leads to visually sharper maps. Similar to Grad-CAM++, it involves pixel-wise weighting of the gradients of the output with respect to a particular spatial position in the final convolutional feature map of the CNN, providing a measure of the importance of each pixel.

\begin{figure}[ht!]
	\begin{center}
		\includegraphics[width=0.7\linewidth]{Figures/Grad-CAM.png}
	\end{center}
	\caption{Explanations produced by Grad-CAM, Grad-CAM++ and Smooth Grad-CAM++. The photos shown are samples from~\cite{abs-1908-01224}}
	\label{Fig:Grad-CAM}
\end{figure} 
\subsubsection{Deconvolution Based Methods}
\label{decov}

While the advancements in CAM methods like CAM, Grad-CAM, and their successors have significantly improved the ability to interpret CNNs through localization and visualization of discriminative regions, another approach to interpretability emerges with deconvolution based methods. Unlike CAM methods, which focus on generating class activation maps to highlight important regions, deconvolution based methods like Deconv and its variants focus on reconstructing input images from learned feature maps. This shift marks a transition from methods primarily enhancing object localization to those unraveling the CNN's learned image representations, offering a complementary perspective on understanding deep neural networks. 

The deconvolutional network model (Deconv)~\cite{ZeilerKTF10} aims to build robust low and mid-level image representations beyond simple edge primitives, addressing the challenge of capturing mid-level cues like edge intersections, parallelism, and symmetry. This approach uses the filters that the CNN has learned from its training phase to break down an input image into a set of feature maps. These feature maps represent the essential components or features of the image as understood by the CNN. The process then involves reconstructing the original image from these feature maps, effectively demonstrating the CNN's understanding and internal representation of the input image. The process typically involves reversing the operations of the CNN. For each convolutional layer, a corresponding deconvolutional layer is used, where the forward and backward passes are swapped. The presence of pooling layers (especially max pooling) in CNNs presents a challenge for deconvolution. This is because pooling operations are not reversible without additional information (\eg forward pass). In standard deconvolution approaches, `switches' that record the positions of the maximum values in the pooling layers during the forward pass are often used to guide the deconvolution.

To analyse the network and understand what features it learns, the authors of~\cite{SpringenbergDBR14} introduce a variant of the ``deconvolution approach" called Guided Backpropagation. This new method is applicable to a broader range of networks compared to existing techniques and helps in visualising features learned by CNNs. The proposed networks only use convolutional layers with occasional dimensionality reduction achieved by strided convolutions. Fully connected layers are replaced by $1\times 1$ convolutions. The paper shows that max-pooling layers can be effectively replaced with convolutional layers that have increased stride, without compromising the network's accuracy. This leads to an architecture that is more homogeneous and possibly easier to analyse and understand. Since their network architecture lacks pooling layers, their deconvolution approach doesn't need to deal with the typical challenges posed by max-pooling. This simplifies the deconvolution process. The model used deconvolutional layers and guided backpropagation to generate saliency maps. However, choosing to drop or keep max-pooling layers is challenging as it depends on several factors, such as domain area, dataset, and network architecture. 

\subsubsection{Back-propagation Based Methods}
\label{sec:backprop}

Back-propagation based methods calculate the gradient of a specific output with respect to the input . In the simplest case, the gradient can be back-propagated~\cite{SimonyanVZ13}. The method introduced the concept of using gradients to create saliency maps, which visualise the importance of each pixel for the classification decision of a convolutional neural network (see Figure~\ref{Fig:Saliency_Method}). This paper is often cited as introduction of Input$\times$Gradient, as it laid significant groundwork for gradient-based interpretability methods in deep learning. This saliency maps highlight which parts of the image are most important for classifying it into a particular category. The authors show that class saliency maps can be used for object segmentation in images, even with weak supervision (i.e., without detailed annotations for training). This implies that convolutional neural networks trained for classification can also assist in locating objects in images. The paper establishes a relationship between the gradient-based CNN visualisation methods and deconvolutional networks, showing how the former can be considered a generalisation of the latter.

\begin{figure}[ht!]
	\begin{center}
		\includegraphics[width=0.8\linewidth]{Figures/Gradient.pdf}
	\end{center}
	\caption{Explanations produced by Saliency Method. The photos shown are samples from~\cite{SimonyanVZ13}}
	\label{Fig:Saliency_Method}
\end{figure} 


While DeepLIFT offers a significant advancement in interpreting neural network decisions by back-propagating the contribution of each neuron to the final output, a parallel approach known as Integrated Gradients emerges to further refine the interpretability of deep learning models. Integrated Gradients~\cite{SundararajanTY17}, unlike DeepLIFT, which focuses on comparing the activation of each neuron to a reference input, integrates over a path of inputs between a baseline and the actual input. This method provides a more continuous and comprehensive attribution, capturing the importance of each input feature across a spectrum of changes. This shift from the discrete, reference-based approach of DeepLIFT to the path-integrated framework of Integrated Gradients addresses the issues like vanishing gradients.

Layer-wise Relevance Propagation~\cite{bach2015pixel}, widely referred to as \LRP, is a type of backward propagation that is widely applicable to general network structures~\cite{LapuschkinBMMS16} and has been specifically designed for explanations. The method propagates the prediction backwards using a set of rules that are subject to a \textit{relevance conservation property}, which refers to the equal redistribution of weightings from the network's output to the preceding neurons. Techniques that implement such a \textit{relevance conservation property} can also identify negatively relevant areas from the input. There are many different \LRP\/ propagation rules that one can use at any given step, and depending on the rule used one can extract a different quality explanations. The development of propagation rules and the choice of which rule should be applied at any given situation are research areas in their own right~\cite{MontavonLBSM17}. One type of a \LRP\/ propagation rule used in~\cite{bach2015pixel}, known as $\alpha\beta$-rule, has been proven to work well. However as mentioned, rules that deal with specialised layers (e.g.\ an input layer, pooling layers, normalisation layer, etc.) exist, but they are omitted in this literature review for the sake of brevity (for further propagation rules, see~\cite{MontavonLBSM17}).


Deep Taylor decomposition is a special type of backward propagation technique, where the \LRP\ rule, namely the $\alpha\beta$-rule is applied with parameters $\alpha = 1$ and $\beta = 0$ ( denoted $\alpha_{1}\beta_{0}$). This parameter setting simplifies the propagation rule by only distributing the positive relevance though the network. When applying this rule to a deep ReLU networks (piecewise linear network) for a single layer it is equivalent to computing a Taylor decomposition for that same layer~\cite{MontavonLBSM17}. When applying a simple Taylor decomposition the classification is explained by approximating the deep neural network's function around the instance one is trying to explain using a Taylor series. A Taylor series find an approximation of a non-polynomial function by summing the derivatives of that function. The more the series are expanded the better the approximation is. When computing a simple Taylor decomposition, a Taylor series is used to only compute the first order derivative (i.e a linear approximation) of the function returned by the algorithm around the instance one is trying to explain. However, it is often the case that the higher order terms are not 0. Hence, the approximations provided by a simple Taylor decomposition, which takes only the first-order expansion, often provides an incomplete explanation of the function used to classify the given instance. There is an exception to this, when the ML function is piecewise linear\footnote{the instance lies on the same linear trajectory as 0, which makes the second and higher order terms zero}, however, often this is not the case and the generated explanation may be incorrect or incomplete. Taylor decomposition is then repeated for all layers starting from the last one until the input is reached, which is where deep Taylor decomposition takes it's name from.

RAP~\cite{NamGCWL20} is a development on \LRP~\cite{bach2015pixel} designed to decompose the output predictions of DNNs. It introduces a new perspective by separating positive and negative attributions based on their relative influence between the layers of the network. The technique assigns a bi-polar importance score to each neuron relative to the output, ranging from highly positive to highly negative. This approach allows for a more nuanced understanding of each neuron's influence on the network's decisions. This is in contrast to all other techniques where neurons have either positive or negative influence. The method shows clearer and more attentive visualisations of separated attributions compared to conventional explaining methods (see Figure~\ref{Fig:LRP}).


\begin{figure}[ht!]
	\begin{center}
		\includegraphics[width=\linewidth]{Figures/LRP.pdf}
	\end{center}
	\caption{Explanations produced by Gradient, Input$\times$Gradient, Integrated Gradients, Guided Backpropagation, Pattern Attribution, \LRP$\alpha_1\beta_0$ (Deep Taylor decomposition), \LRP$\alpha_2\beta_1$ and RAP. The photos shown are samples from~\cite{NamGCWL20}}
	\label{Fig:LRP}
\end{figure} 



In a line of research undertaken by Lundberg~\textit{et al.}~\cite{LundbergL17} called SHAP (SHapley Additive exPlanations) the authors note a similarity and prove that LIME~\cite{Ribeiro0G16}, DeepLIFT~\cite{ShrikumarGK17}, and Layer-Wise Relevance Propagation~\cite{bach2015pixel} can all be classified under additive feature attribution methods, where an effect is attributed to each feature and the sum of these effects approximates the output of the original model. The authors introduce SHAP values as a unified measure of feature importance. These values are based on the concept of Shapley values from cooperative game theory and provide a consistent and unique solution for additive feature attribution. SHAP values assign each feature an importance value for a particular prediction, which is done by sub-sampling the feature inputs in varying combinations to understand their interactions and effects on the decision. The paper provides theoretical results showing that there is a unique solution within the class of additive feature attribution methods that has a set of desirable properties. These properties include local \emph{accuracy}, \emph{missingness}, and \emph{consistency}. The paper includes experiments to demonstrate the computational efficiency and the alignment of SHAP values with human intuition. 


% The EBAnO engine~\cite{abs-1908-04348} aims to increase the transparency of black-box algorithms, particularly in image processing and classification. It explains the inner workings of these algorithms by analysing the impact of each interpretable input feature on the final outcome. To identify portions of image to be used as interpretable features EBAnO performs a Simultaneous Detection and Segmentation (SDS) analysis~\cite{HariharanAGM14} based on hypercolumns~\cite{HariharanAGM15} and cluster analysis via the K-Means algorithm~\cite{JuangR90}. EBAnO performs iterative input perturbation and classification to analyse the impact of each interpretable feature. The paper presents preliminary results of EBAnO applied to a dataset of 85 images, demonstrating its effectiveness in providing insights into the relationship between interpretable features and the classification made by the CNN model. The EBANO model was assessed using two indices: IR (Image Retention) and IRP (Image Retention Perturbation). The IR index determined the likelihood of a class in the original image relative to the perturbed image. This contrasts with the IRP index, which evaluated how each feature impacted the group of classes. However, setting the initial value of 'k' in the k-means clustering algorithm presents difficulties, especially when dealing with medical images and extensive datasets. This initialisation challenge is crucial because it influences the effectiveness of the clustering process in these complex and large-scale data environments.


\subsection{Discussion: Post-hoc Explanations}
\label{lit:discussion}


The effectiveness of post-hoc explanations plays a crucial role in understanding and interpreting the behaviour of neural networks. These methods are tasked with identifying the significance of different input features in the network's decision-making process. However, the reliance on post-hoc methods for explaining the behaviour of machine learning models based on single data points can be problematic due to the fragility of these explanations~\cite{abs-1806-08049}. Such an approach risks drawing misleading conclusions about the model's overall performance. This issue arises because explanations generated for individual data points may not accurately represent the model's behaviour in a broader context. Furthermore, attempting to comprehend the intricacies of complex models through one or even several point-wise explanations shouldn't be without the support of theoretically sound metrics. 


To ensure the reliability and usefulness of these methods, several key properties such as fidelity, input invariance, handling saturation and sensitivity must be rigorously maintained~\cite{NielsenDRRB22}. Table~\ref{tab:comparison} provides a comparative overview of various explanation methods across these properties, highlighting their respective strengths and weaknesses. Section~\ref{sec:comp} conducts an in-depth analysis of each method, examining how they address these critical aspects.
\newpage


\begin{table}[h!]
\small
\centering
\begin{tabularx}{\textwidth}{cXXXXXXXXX}
\hline
\textbf{Method} & \textbf{Input Invariant} & \textbf{Handles Saturation} & \textbf{Input Sensitive} & \textbf{Fast Generation} & \textbf{Easy Interpretability}\\
\hline
Saliency Maps & no & no & yes  & yes & no\\
Input$\times$Gradient & no & no & yes  & yes & no\\
SmoothGrad & partially & partially & yes  & no & no\\
Guided Backprop. & no & no & yes  & yes & no\\
Grad-CAM & yes & partially & yes & yes & no\\
Integrated Grad. & no & yes & yes  & no & no\\
DeepLIFT & partial & partially & yes  & yes & no\\
LRP & yes & yes & yes  & yes & no\\
\hline
\end{tabularx}
\caption{Comparison of explanation methods across different properties.}
\label{tab:comparison}
\end{table}

At the core of robust post-hoc explanation methods is the principle of \emph{fidelity}~\cite{TomsettHCGP20} of an explainability method, which is is closely tied to its ability to accurately identify and quantify the relevance of different input features. High-fidelity methods are characterised by their precision in assigning significant importance scores to features that critically influence the network's performance. Conversely, features that have a minimal impact on the output should receive low attribution scores. This accuracy in feature relevance assessment is crucial for deciphering the decision-making process within the network. Another critical aspect of attribution methods is \emph{input transformations resilience}~\cite{KindermansHAASDEK19, 11700}. Neural networks often exhibit invariance to modifications such as constant shifts in the input data. Explainability methods in this category, therefore, must also demonstrate this invariance, ensuring that their output remains stable and reliable under such transformations. This stability is essential for understanding the network's behaviour in dynamic environments where input data might vary. In ML models, especially those involving nonlinear activation functions like Sigmoids, input features can cause \emph{saturation} in the network~\cite{SundararajanTY17}. For instance, in a network using a Sigmoid activation function, an input greater than a certain threshold might not alter the network's output, leading to a zero gradient. Explainability methods especially the ones using gradients to produce their explanation must effectively address this saturation phenomenon. One approach is to incorporate reference inputs or baselines, such as zero values, random numbers, or averages calculated from the input dataset. This helps in accurately estimating attributions even in saturated networks. Lastly, the concept of \emph{sensitivity} is pivotal in attribution methods~\cite{AnconaCOG19, 11700}. It requires that the model's output for a given input can be broken down into the sum of contributions from individual input features. In practice, this means that an attribution method should assign non-zero scores to features that uniquely distinguish between two similar inputs. Moreover, features that have no impact on the network's output should be attributed a zero score. This property, also known as \emph{completeness} or \emph{summation to delta}~\cite{SundararajanTY17, ShrikumarGK17}, ensures that the attribution reflects the true influence of each feature in the model's decision process.


The inherent difficulty in validating attribution methods in machine learning stems from the absence of definitive, ground truth explanations for the decisions made by these models. This challenge is central to the field: a comprehensive evaluation of attribution methods ideally requires a complete understanding of the decision-making process of the ML model, which is precisely what these methods aim to solve. Moreover, distinguishing between inaccuracies originating from the ML models themselves and those from the attribution methods further complicates this validation process~\cite{SundararajanTY17}. Since explanations are often tailored to align with human visual perception, the assessments of these attribution methods have predominantly been based on subjective interpretations. This subjective approach, while valuable, is limited in its ability to provide a robust and thorough understanding. The importance of objective evaluation in this context cannot be overstated. Objective measures are crucial for establishing a solid theoretical framework for attribution methods~\cite{yeh2019infidelity}. They enable a more systematic comparison and contrast of different approaches, shedding light on their respective strengths and weaknesses. 

\subsubsection{Methods Comparison}
\label{sec:comp}

The fidelity of an interpretability method is a critical aspect, often considered the most crucial criterion. When an interpretability method lacks fidelity, its interpretations can be misleading or completely incorrect, potentially obscuring the true capabilities and limitations of the model. This can result in false assessments of the model's reliability, effectiveness, or potential biases. For example, a method with low fidelity might misleadingly suggest that the model is focusing on relevant features, whereas in reality, it could be relying on irrelevant or spurious correlations. To quantitatively assess the fidelity of an explainability model, \emph{infidelity metrics} have been proposed. One such metric requires the explanations to capture the function values~\cite{PlumbMT18}, another assess how well an explanation captures the changes in a model's output in response to significant alterations in the input features~\cite{YehHSIR19}. The work by~\cite{YehHSIR19} contends that the technique of perturbing images by randomly removing subsets of pixels could be of limited significance. This stems from the observation that such perturbations often result in minimal information loss, considering that the surrounding, non-removed pixels still retain much of the image's information. Additionally, the approach of considering every possible subset of pixels for removal, as employed in methods like SHAP~\cite{LundbergL17} and LIME~\cite{Ribeiro0G16}, becomes impractical in the context of high-dimensional images due to the vast number of potential combinations.


As part of checking if the model is faithful a ```sanity'' check was introduced. The research conducted by Adebayo~\textit{et al.} proposed two important sanity checks to assess the responsiveness of attribution methods in relation to changes in model parameters and the dataset used~\cite{AdebayoGMGHK18}. These checks are designed to evaluate whether the attribution methods accurately reflect the underlying mechanics of the neural network and its training data. The first sanity check involves replacing the learned neural network parameters with random values. The attribution maps generated from this modified network are then compared with those from the original, unaltered network. The comparison employs various correlation metrics to determine if the attribution maps effectively capture the impact of these changes in network parameters. If the attribution maps do not significantly change despite the drastic alteration in network parameters, this raises questions about their the methods faithfulness. The second sanity check involves using neural networks trained on data with randomly shuffled labels. The objective is to see if the attribution methods produce different attribution maps under these conditions. These sanity checks serve as critical tools for verifying the robustness of attribution methods. They ensure that these methods are not just producing visually appealing or superficially consistent maps, but are genuinely representative of the model's learning and operational dynamics. If an attribution method fails these checks, it suggests that the method may be unreliable, potentially offering misleading insights into how the model is processing its inputs and making predictions. Adebayo~\textit{et al.} found that Guided Backpropagation~\cite{SpringenbergDBR14} and Guided Grad-CAM~\cite{SelvarajuCDVPB17} were insensitive to the learned parameters in the layers near to the output~\cite{AdebayoGMGHK18}. 

\emph{Input invariance} refers to an attribution method's ability to remain unaffected by specific transformations to the input that do not alter the model's predictions~\cite{YehHSIR19}. A good attribution method will exhibit low sensitivity to such changes (\ie produce the same explanations for minor variations in the input). The Saliency method~\cite{SimonyanVZ13} is based on the gradient of the model’s output with respect to the input, it is therefore not input invariant. Input$\times$Gradient~\cite{SimonyanVZ13}, similar to Saliency, is also based on gradients. Since it involves multiplying the gradient by the input it is even less input invariant. Guided Backpropagation modifies standard backpropagation to focus on positive gradients, this does not make it any more input invariant. Integrated Gradients~\cite{SundararajanTY17} calculates attributions by integrating the gradients along a path from a baseline input to the actual input. The method is therefore generally not input invariant. DeepLIFT~\cite{ShrikumarGK17} compares activations to a reference activation and assigns contribution scores accordingly. Its input invariance depends on the choice of the reference input and how it relates to the transformed inputs. SmoothGrad~\cite{SmilkovTKVW17} enhances traditional gradient-based methods by averaging the gradients of multiple noisy versions of the input. This averaging process can potentially reduce sensitivity to small changes in the input, thereby increasing input invariance to some extent. LRP~\cite{bach2015pixel} redistributes the output prediction back through the network layers. Its input invariance largely depends on how it handles different layers and their activations and the rules deployed. Input that changes the activation drastically is however unlikely to not alter the model's predictions, so \LRP\ is seen as mostly input invariant. Grad-CAM~\cite{SelvarajuCDVPB20} uses the gradients flowing into the last convolutional layer to produce a localisation map. Therefore it is input invariant, as insignificant changes do not lead to changes in the gradient in the last layers~\cite{AdebayoGMGHK18}.



\emph{Saturation} is another critical aspect to consider when evaluating interpretability methods in machine learning. It refers to how the attribution method deals with the non-linearities in a network, especially in the context of activation functions like ReLU (Rectified Linear Unit). In simple terms, saturation occurs when changes in the input feature do not affect the output of the network due to the nature of the activation function. The Saliency method~\cite{SimonyanVZ13} is based on the gradient of the output with respect to the input. In cases of saturation, where gradients may become zero or near-zero, the saliency method might face challenges in accurately capturing the influence of input features on the output. Similar to Saliency, Input$\times$Gradient~\cite{SimonyanVZ13} also uses gradients but multiplies them by the input, so it still could struggle in regions of saturation where gradients are minimal. SmoothGrad~\cite{SmilkovTKVW17} enhances saliency maps by averaging the gradients of multiple noisy versions of the input. This approach can reduce the noise in the gradient-based explanations, potentially providing more clarity even in saturated regions. However, its effectiveness in saturation still largely depends on how well the underlying gradient method copes with saturation. Guided Backpropagation~\cite{SpringenbergDBR14} modifies standard backpropagation by only allowing positive gradients to flow backward through the network. While this can create clearer visualisations, it doesn't have an effect on the saturation problem and can similarly to the other methods have this problem in regions where the gradient is close to zero. Grad-CAM~\cite{SelvarajuCDVPB20} uses the gradients of the target concept (like a class label) flowing into the final convolutional layer to produce a coarse localisation map highlighting important regions in the image for predicting the concept. While Grad-CAM can be effective in highlighting relevant areas in the input space, its performance in the context of saturation is more dependent on the behaviour of the final convolutional layers and may vary depending on the network architecture. DeepLIFT~\cite{ShrikumarGK17} compares the activation of each neuron to its 'reference activation' and assigns contribution scores accordingly. The method is specifically designed to address the saturation issue, as it considers the difference from a reference point. Despite this it may still struggle in highly saturated regions. Integrated Gradients~\cite{SundararajanTY17} improves DeepLIFT~\cite{ShrikumarGK17} by integrating the gradient along the path from a baseline to the actual input. This approach helps in capturing the contribution of features even in regions where the activation functions of the network are saturated. Finally, \LRP~\cite{bach2015pixel} redistributes the output prediction value back through the layers of the network, which could potentially handle saturation effectively. It considers the contribution of each neuron relative to others, which helps in understanding the impact of saturated features.


Sensitivity in the context of attribution methods refers to how responsive these methods are to changes in the input features. A sensitive attribution method should ideally change its output significantly if the input features that are important for the model's decision change. Integrated Gradients~\cite{SundararajanTY17}, Saliency~\cite{SimonyanVZ13} Input$\times$Gradient\cite{SimonyanVZ13} and Guided Backpropagation~\cite{SpringenbergDBR14} all compute gradients with respect to the input and therefore directly reflect the immediate impact of input changes on the output. SmoothGrad~\cite{SmilkovTKVW17} averages the gradients of multiple noisy versions of the input and retains sensitivity to input changes but attempts to provide a clearer signal by averaging out the effects of minor, non-critical variations. DeepLIFT~\cite{ShrikumarGK17} compares the activation of each neuron to a reference activation, which can make it sensitive to changes in input, particularly those that significantly alter neuron activations compared to the reference state. \LRP~\cite{bach2015pixel} redistributes the output back through the layers of the network, it is designed to be sensitive to changes in inputs that significantly affect the network’s output. Grad-CAM~\cite{SelvarajuCDVPB20} uses the gradients flowing into the last convolutional layer to produce a localization map. Its sensitivity is more focused on the spatial aspects of the input (like in image data) and is sensitive to changes in features that strongly influence these convolutional layers. The majority of methods are sensitive to the input. However some interpretability methods can lack sensitivity to changes in class labels, when computing gradients. This issue is evident when, for a given input image, similar attribution maps are generated regardless of the class label.


% For instance, methods like Saliency Maps~\cite{SimonyanVZ13}, Input$\times$Gradient~\cite{SimonyanVZ13}, Guided Backpropagation~\cite{SpringenbergDBR14} and Integrated Gradients~\cite{SundararajanTY17} often show this insensitivity, as demonstrated in Figure~\ref{Fig:sensitivity}.


% \begin{figure}[ht!]
% 	\begin{center}
% 		\includegraphics[width=1\linewidth]{Figures/methods_insesitive.png}
% 	\end{center}
% 	\caption{This figure shows the lack of sensitivity of some methods to the class being interpreted. Initially, the input image is showcased in the first column, followed by various attribution maps in the subsequent columns. The top of the first column's image highlights the target class and its soft-max probability. The photos shown are samples from~\cite{NielsenDRRB22}}
% 	\label{Fig:sensitivity}
% \end{figure} 


The discussion in this section has highlighted several crucial aspects of post-hoc explanation methods in neural networks, emphasising their significance in interpreting the behaviour of these complex models. The section delves into the principles of fidelity, input invariance, saturation, and sensitivity, underscoring the importance in ensuring reliable and meaningful explanations. However, despite the progress made in developing and evaluating these explanation methods, several challenges and limitations persist. These challenges not only pertain to the technical aspects of the explanation methods but also to their interpretation and integration into broader machine learning workflows.

\section{Future Directions}

As machine learning models are increasingly used in critical decision-making, ensuring that these models are fair and unbiased becomes paramount. Inherently interpretable models focus primarily on either learning an interpretable version of the model, modifying the existing model to enhance its interpretability or changing the learning process to produce a more interpretable model. This type of approaches offer a compromise, allowing the use of somewhat accurate, complex models while still gaining some level of interpretability. Work by Markus~\textit{et al}~\cite{MarkusKR21} suggests that instead of focusing solely on developing inherently interpretable models, which may have lower predictive performance, post-hoc explanations can be used to provide insights into the workings of an AI model. This allows for the full accuracy and learned complex function of these models to be used.


In order for the post-hoc explanations to be useful they should not only be interpretable, but also have a very high degree of faithfulness. Faithfulness in this context indicates how accurately an explanation reflects the true reasoning process of the model. Higher faithfulness means the explanation is more precise and reliable in representing how the model actually works. Interpretability on the other hand, refers to how well a human can understand the reasons behind a model's decision or prediction. Higher interpretability often means the explanation is more accessible and comprehensible to non-experts. 


Early methods for post-hoc explanations started around 2014 and 2015. They were not very interpretable and lacked fidelity. Saliency~\cite{SimonyanVZ13}, Input$\times$Gradient~\cite{SimonyanVZ13} and Guided Backpropagation~\cite{SpringenbergDBR14} produced noisy explanations which were not easy to comprehend by a human user, while also not being input invariant, having saturation issues and lacking sensitivity to changes in the output being interpreted. 


The next generation of methods are easier for humans to understand but might not accurately represent the model's internal workings. Methods include LIME~\cite{Ribeiro0G16} and SHAP~\cite{LundbergL17}, as well as others~\cite{ElenbergDFK17, Ribeiro0G18}. They choose sub-parts of the input that are already meaningful to a human and then present their importance to the made classification. By grouping individual units (\eg pixels in an image) into a bigger feature (\eg a superpixel) these types of methods select a limited amount of information to show to the user. It has been shown that a human can not comprehend more than three to five meaningful item at once~\cite{cowan2001magical, starkey1995development, morris2018human}, so showing higher level explanations in the way that input perturbing-based techniques do, makes them highly interpretable methods. Input perturbing-based explanations, assign a value of importance to each feature based on whether an alteration of that feature leads to a change in the classification or a lowered classification certainty. These methods do not qualify as input invariant. It also inevitably leads to unfaithful explanations, as the methods assume that if a feature doesn't change the activation of the classification that means that the feature doesn't contribute to the classification, which doesn't hold, as proven by the saturation problem explained earlier. From this it follows that these methods don't detect all the features that lead to a classification. Aside from that the input perturbing methods sometimes assign importance to features that are irrelevant. This occurs whenever the data instance is classified with a low degree of certainty, as it is easier to identify features that when perturbed lead to a different classification. 


The identification of these fidelity issues is a research direction in its own right~\cite{KindermansHAASDEK19, DombrowskiAAAMK19, GhorbaniAZ19} and the preceding methods described are focused on either fixing fidelity problems in the explanation or making the explanation more interpretable (\eg in the case of convolutional neural networks making the heatmaps ``less noisy''). For example, DeepLift~\cite{ShrikumarGK17}, which focuses on comparing the activation of each neuron to a reference input, reduces the saturation problem. It also is less input variant to insignificant changes as long as the neuron chosen for comparison is not the one being varied. The method still shows sensitivity to big changes in the output or in the input. Integrated Gradients~\cite{SundararajanTY17} improve DeepLift~\cite{ShrikumarGK17} by integrating the change over a path of inputs between a baseline and the actual input. Taking more than one reference point makes the algorithm even more resilient to non-linearities in the network. However, it does take a step back in terms of input invariance and sensitivity. SmooothGrad~\cite{SmilkovTKVW17} in comparison, doesn't suffer from sensitivity issues and due to taking the average of perturbed inputs also tends to improve input invariance. However, similar to older approaches it suffers from the saturation problem. GradCAM~\cite{ShrikumarGK17} and GradCAM++~\cite{ChattopadhyaySH18} use the gradients flowing into the last convolutional layer to produce a localisation map. They are input invariant and still sensitive to big changes. Their
performance in the context of saturation is dependent on the behaviour of the final convolutional layers and may vary depending on the network architecture, but it not consistently struggling like other methods. Finally \LRP~\cite{bach2015pixel} works by tracing the output prediction value backward through the network's layers. This approach could effectively manage saturation issues. \LRP\ assesses the contribution of each neuron in relation to others, aiding in the comprehension of the influence exerted by saturated features. It is also structured to be highly responsive to alterations in inputs that have a substantial impact on the network’s output and not have too big of an impact to small input changes. 

Another consideration, when choosing to deploy interpretability methods is computational resources. Gradient-based methods in machine learning are notably efficient in terms of computational resources, Saliency maps~\cite{SimonyanVZ13}, Input$\times$Gradient~\cite{SimonyanVZ13}, Guided Backpropagation~\cite{SpringenbergDBR14}, \LRP~\cite{bach2015pixel} and  Grad-CAM~\cite{SelvarajuCDVPB20} and its successors~\cite{SelvarajuCDVPB20, ChattopadhyaySH18, abs-1908-01224, SmilkovTKVW17} require one forward and one backpropagation step to generate an explanation. This efficiency is particularly important when considering the growing need for explainability in machine learning models. As these models become more complex and are used in critical applications, there is an increasing demand for methods that can provide insights into how decisions are made without significantly adding to computational costs. In contrast, the process of generating an explanation for Integrated Gradients~\cite{SundararajanTY17} and SmoothGrad~\cite{SmilkovTKVW17} requires 50 to 200 steps depending upon the problem domain, dataset, and the scope of explanation. Methods like SHAP~\cite{LundbergL17} and LIME~\cite{Ribeiro0G16} require numerous perturbation as well. Such methods might not be as feasible in scenarios where computational resources are limited or when quick explanations are needed~\cite{GhorbaniAZ19}.


The future direction in machine learning interpretability is arguably moving towards a blend of approaches, where explanations are both highly interpretable like LIME~\cite{bach2015pixel} and SHAP~\cite{LundbergL17} and extremely faithful to the actual workings of the model, similar to \LRP~\cite{bach2015pixel}, while not requiring too much computational resources like gradient based methods~\cite{SimonyanVZ13, SimonyanVZ13, SpringenbergDBR14, bach2015pixel, SelvarajuCDVPB20, SelvarajuCDVPB20, ChattopadhyaySH18, abs-1908-01224, SmilkovTKVW17}. The goal is to create techniques that encompass all key properties discussed so far: fidelity, input invariance, and effective handling of saturation and sensitivity issues. Incorporating these aspects into future interpretability methods will require innovative approaches that combine the best of both worlds:  human-friendly explanations and model-faithful insights. This would lead to more robust, understandable, and trustworthy AI systems, crucial for their successful deployment in critical decision-making domains.


\section{Conclusion}
This literature review has extensively explored the dynamic field of Explainable Artificial Intelligence (XAI), delving into its trajectory and the various methodologies developed to enhance the interpretability of complex neural networks. The review highlighted two primary categories of explainability methods: intrinsically interpretable models (see Section~\ref{sec:Intrinsic}) and post-hoc explanations (see Section~\ref{sec:post-hoc}), each with their unique approaches and implications.

Intrinsically interpretable models, both globally and locally focused, underscore the importance of integrating transparency directly into the AI model's design. This approach is particularly vital in sensitive applications where decisions must be transparent and justifiable. However, the trade-off between interpretability and model performance remains a significant challenge, emphasising the need for continued innovation in designing models that balance these two critical aspects effectively. Post-hoc explanations, encompassing both global and local perspectives, offer an alternative way to interpret complex neural networks. These methods, applied to already-trained models, aim to shed light on the decision-making processes of AI systems without compromising their performance. This approach has proven particularly useful in scenarios where intrinsic transparency is not feasible or where detailed insights into specific predictions are necessary.

The ongoing development and refinement of these explainability methods are crucial in advancing the field of AI, particularly as these models become increasingly integrated into critical domains such as healthcare, finance, and autonomous systems. The trajectory of interpretability in machine learning seems to be converging towards a harmonious integration of methodologies. Future research will aim for explanations that are not only highly interpretable but also true to the model's inner workings. Moreover, this evolution is mindful of minimising computational demands, a challenge often encountered in gradient-based methods. The objective is to create techniques that embody the essential characteristics discussed: fidelity, input invariance, and adeptness in addressing saturation and sensitivity complexities. This demands innovative strategies that combine easily comprehensible explanations with faithful reflections of the model’s processes.

While interpretability is important across various data domains, the urgency and complexity of this need vary. In natural language processing, textual data inherently possess a granular and interpretable structure—words, phrases, and sentences carry explicit meanings. Techniques like topic modeling or sentiment analysis often provide straightforward insights without requiring extensive additional methods.

Similarly, audio data, especially in controlled environments, tends to be less complex. Key features such as pitch, tone, and rhythm are more directly interpretable, and the dimensionality of audio data is generally lower. This makes understanding and interpreting models in this domain more manageable without necessitating advanced clustering techniques.

In contrast, image data presents unique challenges due to its high dimensionality. A single image comprises of three dimensions and thousands to millions of pixels, each contributing subtly to the overall picture, making it difficult to discern which features are most influential in a model's decision. Critical applications like medical diagnostics, autonomous driving, and surveillance rely heavily on computer vision systems. Misinterpretations in these domains can have profound consequences.

In this thesis, the aim is to harmonise faithfulness and interpretability in the context of image data. Instead of offering importance values for each pixel or distilling outputs into binary judgements, we propose ways to cluster related features together, thereby reducing dimensionality and enhancing interpretability. Chapter ~\ref{chap:clustering} will focus on clustering techniques to group similar features in image data, providing a foundation for assigning a single importance scores to each cluster in Chapter~\ref{chapter:revLRP} and~\ref{chapter:REVEAL}.



\chapter{Multi-faceted Clustering Approaches for Isolating Complex Input Features}
\chaptermark{Isolating Complex Input Features}
\label{chap:clustering}
\section{Introduction}

This chapter introduces two novel approaches that group input features into complex input features specifically tailored for image data, as defined in~\Cref{def:cifv}. This abstraction aligns closely with how humans naturally process visual information~\cite{fiantika2018internal}, thereby making the interpretation of neural networks more intuitive. By grouping pixels into meaningful patterns or regions, we enable the development of methods that assign a single relevance value to these complex input features as a whole, rather than treating each pixel individually. See Chapters~\ref{chapter:revLRP} and~\ref{chapter:REVEAL} for how to assign a single importance score to a complex input feature.

Isolating complex input features enhances neural network interpretability in image classification tasks by reducing the high dimensionality of pixel-level data and providing consolidated representations that highlight important visual patterns or regions. This approach simplifies user focus, thereby promoting standardised interpretations and minimising the risk of misinterpreting data due to noise or insignificant details. By offering a clearer and more distilled view of the image, users gain confidence in understanding the model's decisions, leading to more informed actions. Additionally, feature clustering simplifies comparisons between different models by focusing on a few key complex features rather than numerous individual pixels, aiding in understanding and identifying differences in model behaviour.

The contributions of this chapter are as follows:

\begin{enumerate} 
\item Provided a formal definition of \emph{Complex Input Features} for image data (\Cref{def:cifv}), establishing a foundation for subsequent methods.

\item Introduced a novel heatmap-based clustering approach using relevance heatmaps to detect and cluster coherent groups of pixels. The algorithm applies two stages of filtering to remove irrelevant features before employing DBSCAN for clustering. Evaluated the method qualitatively using various interpretability techniques and neural networks.

\item Developed an enhanced method combining object detection algorithms with the heatmap-based clustering approach to improve human interpretability. This method identifies distinct objects or regions within an image and selects the most relevant ones based on heatmap-derived importance. Proposed multiple augmentation steps to improve the applicability of object detection algorithms for interpretability tasks, and introduced three distinct techniques for integrating object detection with heatmap-based clustering. 
\end{enumerate}

The clustering techniques in this chapter aim to achieve a dual objective: minimise computational overhead while simultaneously augmenting the comprehensibility of interpretative mechanisms. The central premise is to focus on groups of pixels that collectively form features in the deeper layers of the network and contribute to the CNN's decision-making process.

\subsection{Complex Input Features}

The objective is to group pixels that are close in some feature space into meaningful clusters representing complex input features. We define a complex input feature vector to be a cluster of pixels from the input image, as:

\begin{Definition}{Complex Input Feature Vector}{cifv}
Given an input feature map (image) $F \in \mathbb{R}^{d_1 \times d_2 \times \ldots \times d_n}$ as defined in~\Cref{def:inputfeature}, where $d_1, d_2, \ldots, d_n \in \mathbb{N}$ are the dimensions along each axis of the tensor (e.g., height, width, and color channels), we define a \emph{complex input feature vector} of $F$ to be a subset $CF \subseteq \mathbb{R}^n$ encompassing some subset of pixels $CF \cap (d_1 \times d_2 \times \ldots \times d_n)$ from the input feature map.
\end{Definition}

In image data, individual pixels or simple edges alone often lack significant meaning until they form coherent shapes or objects. Clustering can help identify these higher-level visual patterns. For example, one complex input feature vector may correspond to the edges forming the contour of an object, while another might represent texture patterns, or even entire objects like the muzzle of a tiger. This grouping makes it easier to understand why an image was classified a certain way by the neural network. Since images are inherently high-dimensional, feature clustering reduces this complexity into a set of meta-features, simplifying the interpretability of the model's decision-making process.


\section{Heatmap-Based Clustering}
\label{heatmap_clustering}

\subsection{Motivation for Heatmap-Based Clustering}

Post-hoc locally interpretable methods function by identifying parts of the input which most affect the output. Pixels that are ascribed extreme relevance values are often indicative of their association with deeper learned feature vectors within the network's architecture. By directing attention to these specific pixels and their associated values, a more in-depth insight into the network's interpretative processes and decision-making mechanisms can be attained.

Input features get transformed and when they reach the deeper layers of the network they do not just function as individual entities, but rather operate together, forming interconnected feature vectors. When an input feature receives a high relevance score through a relevance propagation technique, it does not do so in isolation. Instead, this relevance is the result of the interaction the feature had with other input features during the forward pass that together lead to the activation of higher level concepts deeper in the network.

ERIC~\cite{townsend2020ericextractingrelationsinferred} further highlights the role of convolutional kernels, demonstrating that these kernels encapsulate semantically meaningful concepts and their interrelations. This underscores the importance of grouping features influenced by the same deep feature vector to reveal the network's reliance on symbolic patterns and abstract representations.

\begin{Conjecture}{Highly Relevant Input Clusters}{}
Input features with pronounced relevance scores (as attributed by an interpretation technique) and exhibit proximity in the feature space are conjectured to derive their relevance from the \textit{same} deep feature vector. 
\end{Conjecture} 

From a human perspective, individual pixels gain significance when collectively representing an entity. In CNNs, convolutional layers, as detailed in Section~\ref{sec:conv}, amalgamate activations of features within the same kernel. Based on this, we propose that when a heatmap contains a region of pronounced relevance, the significance attributed to this group of input features likely originates from the same deep feature vector. By grouping such features together the concept of feature hierarchy and dependency within the network is emphasised.

\subsection{Heatmap-Based Clustering Approach}

The heatmap-based clustering approach is achieved based on heatmap relevance scores. Notably, this method does not give equal weight to all pixels. Instead, it particularly prioritises those pixels with either notably high or significantly low relevance scores. The rationale behind this prioritisation lies in the behaviour observed within deeper network layers. By employing this clustering approach, the complex input features generated are a direct reflection of what the neural network deems relevant. More critically, they encapsulate input features that have had mutual interactions within the network. This methodology sidesteps human biases regarding perceived importance and centres exclusively on the network's intrinsic decision-making and feature relevance.

This understanding of pixel relevance can be further articulated mathematically. Consider an input to a convolutional neural network to be a \emph{($C$-channel) image} of dimensions \( H \times V \), described by a real-valued function \( I: H \times V \times C \to \mathbb{R} \), associating each colour channel \( c \in C \) of each pixel \( (x, y) \in H \times V \) with a colour value (typically in the range \([0,255]\)).

\begin{Definition}{Relevance Heatmap}{rh}
A relevance heatmap over an image \( I \) is a real-valued function \( \HM: \dom{I} \to \mathbb{R} \) over the domain of \( I \), describing the relevance values \( \vec{R}_0 \) associated with each pixel/input neuron of the input layer \( 0 \in \Lambda \).
\end{Definition}

Given Definition~\ref{def:rh}, we may define a three-dimensional point space, described as:
\begin{equation*}
    S = \big\{ (x, y, \max_{c \in C} |\HM(x, y, c)| ) : x \in H, y \in V \big\} \subseteq \mathbb{R}^3
\end{equation*}

Each pixel coordinate \( (x, y) \) within the dimensions \( H \times V \), has its position elevated based on the maximum absolute relevance it holds across all colour channels. This process essentially 'lifts' the pixel into a three-dimensional space, represented as \( \mathbb{R}^3 \). In Figure~\ref{Fig:3D}(A), an image of a chimpanzee is shown. The relevance values \( \vec{R}_0 \) associated with each pixel in the image's three colour channels are determined, leading to the creation of three heatmaps, as seen in Figure~\ref{Fig:3D}(B). The max value for each colour channel is then extracted, resulting in a single heatmap as displayed in Figure~\ref{Fig:3D}(C). The heatmap can subsequently be represented in three dimensions by lifting each pixel based on its importance value, as depicted in Figure~\ref{Fig:3D}(D).

\begin{figure}[ht!]
	\begin{center}
		\includegraphics[width=1\linewidth]{Figures/3D_plot.pdf}
	\end{center}
	\caption{In the presented figure: (A) showcases an image of a chimpanzee; (B) depicts the relevance values of the input, which are determined using layer-wise relevance propagation in the context of the VGG16 classification; (C) displays a consolidated heatmap formed by extracting the maximum values from each pixel in the initial three-channel allocation of relevances; (D) represents this heatmap in a three-dimensional point space.}
	\label{Fig:3D}
\end{figure}

Once the pixel relevances are transformed into this 3D point distribution, the aim is to separate the space into clusters of pixels that are not only in close spatial proximity but also exhibit similar relevance values. This allows for the identification of complex input features which have a high probability of being identified together from the same deeper layer feature vector.

A drawback of adopting a simplistic approach of clustering with an existing algorithm is computational complexity. Clustering algorithms struggle with efficiency due to the vast volume and high dimensionality of the data. Recognising these limitations, an alternative approach becomes necessary to enhance the scalability and effectiveness of the process. Therefore, first a filtering of the unimportant pixels is employed (see Section~\ref{sec:filtering}), followed by a clustering over the selected pixels (see Section~\ref{sec:clustering}).

\section{Filtering Relevances}
\label{sec:filtering}

To address computational complexity and poor scalability, a filtering of the input data is proposed before clustering. This filtering reduces the dataset to a manageable size by retaining only pixels with highly positive or negative relevance values, ensuring that clustering focuses on truly meaningful data points. This approach not only improves clustering efficiency but also yields more relevant and informative results.

\subsection{Threshold}

In this stage, areas of the relevance heatmap that do not meet a minimum relevance threshold, defined as \( \theta > 0 \), are discarded. Specifically, pixels for which the maximum relevance across all channels \( \max_{c \in C} |\HM(x, y, c)| \geq \theta \) (as defined in Definition~\ref{def:rh}) are retained, while the rest are filtered out. The threshold \( \theta \) is determined by the \( p \)th percentile of the relevance values, meaning that \( p\% \) of the relevance points are below or equal to \( \theta \), while \( (100 - p)\% \) are greater than or equal to it.

The choice of the \( p \) value is crucial in determining which pixels are retained for clustering. However, identifying an optimal \( p \) value is challenging due to the varying distribution of relevance scores across different inputs. A lower \( p \) value is effective for inputs rich in features, as it allows the inclusion of most significant regions. However, in feature-sparse inputs, a low \( p \) may retain too many irrelevant pixels, complicating the clustering process by adding unnecessary noise.

On the other hand, higher \( p \) values focus solely on the most dominant features, potentially overlooking subtler yet important regions. For example, in Figure~\ref{Fig:Fisher}, at \( p = 98\% \) (C), a smaller spider is excluded due to its lower number of relevant pixels. When inputs have more evenly distributed relevance scores, higher \( p \) values can create sparse and fragmented heatmaps, as shown in Figure~\ref{Fig:Fisher} at \( p = 98\% \) (D, E), leading to incomplete representation of relevant pixels within objects, which complicates the clustering process.

Rather than aiming for a universally optimal \( p \) value—which may be impossible even within a single input, where some areas are densely packed with features while others have few—the method proposed in this thesis adopts a more flexible approach. It starts with a lower \( p \) value of 90 to capture a broader set of relevant pixels and then applies further filtering using Fisher-Jenks Natural Breaks to refine the selection.

\subsection{Fisher-Jenks Natural Breaks}
\label{sec:fjnb}

To further increase input in the denser portions of the image, the method segments the input into \( n \) number of regions. Each region vector \( \HMhat_i: U_i \to \mathbb{R} \), where \( \HMhat_i(x, y) = \max_{c \in C} |\HM(x, y, c)| \) is the maximum absolute relevance across all colour channels for the pixel in position \( (x, y) \). 

Each region \( U_i \) is further divided into a tuple of subsets \( (U_{i1}, \dots, U_{i\ell}) \), for some \( \ell \geq 1 \), where the highest values in region \( U_i \) are in the subset \( U_{i1} \). The segmentation or division of \( U_i \) into these smaller sections is done using a method called the Fisher-Jenks natural breaks algorithm~\cite{fisher1958grouping}. This algorithm is a well-known method used to classify or segment data into natural classes or bins. Its main purpose is to determine the best arrangement of values into different classes in such a way that the variance within each class is minimised.

The primary objective of the method is to minimise the sum of the squared deviations from the mean in each subset. Consider each \( U_{ij} \) subset having its own mean value \( \mu_{ij} \). The squared deviation from the mean for a data point in this subset is the square of the difference between the data point and \( \mu_{ij} \). Calculating the squared deviation for all data points within \( U_{ij} \) and summing them up results in the total squared deviations from the mean. Using the algorithm, the resulting subsets \( (U_{i1}, \dots, U_{i\ell}) \) are such that the total of the sum of the squared deviations from the mean \( \mu_{ij} \) for each subset \( U_{ij} \) is minimised, across all \( 1 \leq j \leq \ell \).

We define a set of points for each region \( U_i \) that are in the subset \( U_{i1} \) and are above the \( p \)th percentile value:
\begin{equation*}
    U_{i_p} = \left\{ (x, y) \in U_{i1} \mid \HMhat(x, y) \geq p_{\text{value}} \right\},
\end{equation*}
where \( p_{\text{value}} \) represents the value at the \( p \)th percentile for \( \HMhat \) over the entire input \( I \).

In Figure~\ref{Fig:Fisher}, we present two distinct regions form an input image. Initially, these regions underwent segmentation via the Fisher-Jenks natural breaks algorithm. Subsequently, a segmentation based on the \( p_{\text{value}} \), corresponding to the 90th percentile of the input values, was executed. For the first region, the input content of high relevance is limited. As a result, even though a significant portion of the input is classified into the top segment post the Fisher-Jenks application, the non-significant values are eliminated when filtered by the \( p_{\text{value}} \). This outcome essentially mirrors the result of directly selecting the top 90 percentile. Contrarily, in the second region, a substantial number of pixels surpass the top 90 percentile threshold. Hence, applying the Fisher-Jenks natural breaks to this segment results in more dispersed regions.

\begin{figure}[ht!]
	\begin{center}
		\includegraphics[width=1\linewidth]{Figures/Fisher.pdf}
	\end{center}
	\caption{This figure illustrates the effect of selecting a \( p \) value in combination with Fisher-Jenks natural breaks.}
	\label{Fig:Fisher}
\end{figure}

Next, we consider the space of points \( S_{i\HM} \subseteq \mathbb{R}^4 \), under the Euclidean metric, given by
\begin{equation*}
    S_{i\HM} = \left\{ (x, y, c, a \cdot \HMhat(x, y)) \mid (x, y) \in U_{i_p} \right\},
\end{equation*}
where \( a > 0 \) is a hyper-parameter that can be tuned to linearly scale the relative contributions of spatial distance and relevance; lower values of \( a \) will prioritise spatial similarity, while higher values of \( a \) will prioritise similar relevance scores. We define the space of points \( S_{\HM} \subseteq \mathbb{R}^4 \), under the Euclidean metric, to be the union of all \( S_{i\HM} \), defined as:
\begin{equation}
    S_{\HM} = \bigcup_{i=1}^{n} S_{i\HM}
    \label{eq:set_of_points}
\end{equation}

It represents all the points that the combined method takes into account. This set of points contains the highest or lowest relevance points within each of the regions that also conform to being in the top \( p \)th percentile.

\begin{figure}[ht!]
	\begin{center}
		\includegraphics[width=0.9\linewidth]{Figures/NB.pdf}
	\end{center}
	\caption{This image showcases four contrasting outputs derived from different methods: the low \( p \)th percentile values, the high \( p \)th percentile values, the Fisher-Jenks natural breaks, and their combined technique.}
	\label{Fig:NB}
\end{figure}

In Figure~\ref{Fig:NB}, we show the results of using the \( p \)th percentile values, Fisher-Jenks natural breaks, and their combined approach. The combined technique shows clear improvements over using only the \( p \) value. While Fisher-Jenks avoids overly concentrated or sparse outputs, it sometimes selects features that are not globally relevant, as seen in Figure~\ref{Fig:NB} (A) and (D), where less important areas still have selected pixels. The combined approach ensures that only pixels from the 90th percentile are retained while de-noising feature-dense regions.


\subsection{Filtering Algorithm Pseudo-code}

% The pseudo-code for the filtering process can be seen here. This function reduces the dataset to a manageable size by retaining only pixels with highly positive or negative relevance values, ensuring that clustering focuses on truly meaningful data points.


\begin{algorithm}[ht]
\caption{Filtering Relevances}
\begin{algorithmic}[1]
\Require Relevance heatmap \( \HM \), percentile threshold \( p \), number of regions \( n \)
\Ensure Filtered set of points \( S_{\HM} \)
\Procedure{FilterRelevances}{}
    \State Compute the maximum absolute relevance per pixel:
    \ForAll{\( (x, y) \in \dom{I} \)}
        \State \( \HMhat(x, y) \gets \max_{c \in C} |\HM(x, y, c)| \)
    \EndFor
    \State Calculate the \( p \)th percentile value \( p_{\text{value}} \) of \( \HMhat \) over the entire input.
    \State Divide the image into \( n \) regions \( \{ U_1, U_2, \dots, U_n \} \).
    \State Initialize \( S_{\HM} \gets \emptyset \)
    \Comment{Prepare to collect the filtered points}
    \For{\( i \gets 1 \) to \( n \)}
        \Comment{Loop through each region \( U_i \)}
        \State Apply Fisher-Jenks natural breaks to \( \HMhat \) in region \( U_i \)
        \Comment{Partition the region based on relevance values}
        
        \State Obtain relevance subsets \( (U_{i1}, \dots, U_{i\ell}) \) using Fisher-Jenks.
        
        \State Select the subset with the highest relevance \( U_{i1} \).
        \Comment{Choose the top segment with the highest relevance scores}
        
        \State Define \( U_{i_p} \gets \{ (x, y) \in U_{i1} \mid \HMhat(x, y) \geq p_{\text{value}} \} \).
        \Comment{Keep only pixels above the relevance threshold \( p_{\text{value}} \)}
        
        \ForAll{\( (x, y) \in U_{i_p} \)}
            \Comment{Convert each selected pixel into a 4D point}
            \State \( S_{i\HM} \gets (x, y, c, a \cdot \HMhat(x, y)) \)
        \EndFor
        
        \State Add all 4D points from region \( U_i \) to the final set \( S_{\HM} \).
        \State \( S_{\HM} \gets S_{\HM} \cup S_{i\HM} \)
    \EndFor
    \State \Return \( S_{\HM} \)
\EndProcedure
\end{algorithmic}
\end{algorithm}
% \begin{algorithm}[H]
% \caption{Filtering Relevances (Part 1)}
% \begin{algorithmic}[1]
% \Require Relevance heatmap \( \HM \), percentile threshold \( p \), number of regions \( n \)
% \Ensure Filtered set of points \( S_{\HM} \)
% \Procedure{FilterRelevances}{}
%     \State Compute the maximum absolute relevance per pixel:
%     \ForAll{\( (x, y) \in \dom{I} \)}
%         \State \( \HMhat(x, y) \gets \max_{c \in C} |\HM(x, y, c)| \)
%     \EndFor
%     \State Calculate the \( p \)th percentile value \( p_{\text{value}} \) of \( \HMhat \) over the entire input.
%     \State Divide the image into \( n \) regions \( \{ U_1, U_2, \dots, U_n \} \).
%     \State Initialize \( S_{\HM} \gets \emptyset \)
%     \Comment{Prepare to collect the filtered points}
%     \EndProcedure {(Continued on next page)}
% \end{algorithmic}
% \end{algorithm}

% \begin{algorithm}[H]
% \caption{Filtering Relevances (Part 2)}
% \begin{algorithmic}[1]
% \For{\( i \gets 1 \) to \( n \)}
%     \Comment{Loop through each region \( U_i \)}
%     \State Apply Fisher-Jenks natural breaks to \( \HMhat \) in region \( U_i \)
%     \Comment{Partition the region based on relevance values}
    
%     \State Obtain relevance subsets \( (U_{i1}, \dots, U_{i\ell}) \) using Fisher-Jenks.
    
%     \State Select the subset with the highest relevance \( U_{i1} \).
%     \Comment{Choose the top segment with the highest relevance scores}
    
%     \State Define \( U_{i_p} \gets \{ (x, y) \in U_{i1} \mid \HMhat(x, y) \geq p_{\text{value}} \} \).
%     \Comment{Keep only pixels above the relevance threshold \( p_{\text{value}} \)}
    
%     \ForAll{\( (x, y) \in U_{i_p} \)}
%         \Comment{Convert each selected pixel into a 4D point}
%         \State \( S_{i\HM} \gets (x, y, c, a \cdot \HMhat(x, y)) \)
%     \EndFor
    
%     \State Add all 4D points from region \( U_i \) to the final set \( S_{\HM} \).
%     \State \( S_{\HM} \gets S_{\HM} \cup S_{i\HM} \)
% \EndFor
% \State \Return \( S_{\HM} \)
% \EndProcedure
% \end{algorithmic}
% \end{algorithm}

\section{Clustering}
\label{sec:clustering}

After filtering, only the most relevant input features remain, and a clustering algorithm is needed to identify complex input features as defined in Definition~\ref{def:cifv}. This thesis proposes partitioning the set \( S_{\HM} \) into \( k \) clusters, denoted \( (C_1, \dots, C_k) \), where \( k \geq 1 \). The clustering algorithm used in the examples is \DBSCAN\ (Density-Based Spatial Clustering of Applications with Noise)~\cite{EsterKSX96}.

\DBSCAN\ has several advantages over traditional clustering methods. First, it does not require the number of clusters to be specified in advance, which is ideal since the number of features varies depending on the input. Second, \DBSCAN\ can identify clusters of arbitrary shapes, offering flexibility, unlike many algorithms that assume clusters are centered around a centroid, which can be limiting for image features that do not follow simple shapes. Additionally, \DBSCAN\ has a built-in mechanism to handle noise, allowing it to differentiate between core points, border points, and noise.

The primary drawback of \DBSCAN\ is its poor scalability. However, this limitation is mitigated by applying it only to the most relevant pixels, which typically account for less than 5\% of the total input. This ensures the algorithm remains efficient in practice. In Figure~\ref{Fig:DBSCAN}, the performance of \DBSCAN\ is shown on top of the pixels selected by the combined approach of taking the top 90th percentile and Fisher-Jenks on the input.

\begin{figure}[ht!]
	\begin{center}
		\includegraphics[width=0.85\linewidth]{Figures/DBSCAN.pdf}
	\end{center}
	\caption{Clusters found by applying DBSCAN on top of the selected parts of the input from the combined approach of the 90th percentile and Fisher-Jenks natural breaks.}
	\label{Fig:DBSCAN}
\end{figure}
\subsection{Clustering Pseudo-code}

After the mathematical definitions, we provide the pseudo-code for the clustering process.

\begin{algorithm}[H]
\caption{Clustering Filtered Relevances}
\begin{algorithmic}[1]
\Require Filtered set of points \( S_{\HM} \), \DBSCAN\ parameters \( \epsilon \), \textit{min\_samples}
\Ensure Clusters \( \mathcal{C} = \{ C_1, C_2, \dots, C_k \} \)
\Procedure{ClusterRelevances}{}
    \State Apply \DBSCAN\ to \( S_{\HM} \) with parameters \( \epsilon \), \textit{min\_samples}.
    \State Obtain clusters \( \mathcal{C} = \{ C_1, C_2, \dots, C_k \} \).
    \State \Return \( \mathcal{C} \)
\EndProcedure
\end{algorithmic}
\end{algorithm}


To conclude, we present the complete algorithm that combines the filtering and clustering processes described above.

\begin{algorithm}[H]
\caption{Heatmap-Based Clustering Approach}
\begin{algorithmic}[1]
\Require Image \( \mathbf{I} \in \mathbb{R}^{H \times W \times C} \), percentile threshold \( p \), number of regions \( n \), scaling factor \( a \), \DBSCAN\ parameters \( \epsilon \), \textit{min\_samples}
\Ensure Clusters \( \mathcal{C} = \{ C_1, C_2, \dots, C_k \} \)
\Procedure{HeatmapBasedClustering}{}
    \State Compute the relevance heatmap \( \HM \) for the image \( \mathbf{I} \).
    \State Apply \textsc{FilterRelevances} to obtain the filtered set of points \( S_{\HM} \).
    \State Apply \textsc{ClusterRelevances} to \( S_{\HM} \) to obtain clusters \( \mathcal{C} \).
    \State \Return \( \mathcal{C} \)
\EndProcedure
\end{algorithmic}
\end{algorithm}

\subsection{Comparison based on the neural network}

When evaluating feature relevance in a given input, it is important to recognise that different deep neural networks (DNNs) can interpret and classify data in distinct ways due to variations in their architecture and depth. This section compares heatmap generation across three prominent DNN architectures: ResNet50, Inception V3, and VGG16, highlighting both the differences and similarities observed.

\begin{figure}[ht!]
	\begin{center}
\includegraphics[width=\textwidth]{Figures/CLUSTERING_NETWORKS.pdf}
\end{center}
\caption{Heatmap comparisons for a chimpanzee image across DNN architectures: ResNet50, Inception V3, and VGG16. While all networks correctly identify the chimpanzee, the areas deemed relevant vary across architectures, highlighting the impact of network design on feature interpretation.}
\label{Fig:CLUSTERING_NETWORKS}
\end{figure}

ResNet50's residual connections enable it to train deeper networks without vanishing gradients, while Inception V3's wider architecture offers greater capacity for memorising inputs. VGG16, known for its simplicity, provides a different perspective in comparison. Despite all networks correctly classifying a chimpanzee image, Figure~\ref{Fig:CLUSTERING_NETWORKS} reveals notable differences in which parts of the image each network considers relevant. 

Interestingly, while architectural differences affect feature relevance to some extent, the choice of heatmap generation method has a far greater impact on the interpretation of relevance. This suggests that the method used for generating heatmaps plays a more critical role than the differences between network architectures.

The method’s effectiveness in identifying key features depends on both the DNN used and the heatmap generation technique. While the results may vary across configurations, the approach provides valuable insights into the network’s decision-making. 

\section{Object Detection and Heatmap Clustering Approach}

Heatmap clustering approach allows for the identification of regions that are important for the network given a relevance propagation approach. However, challenges arise when extracted features are significant to the network but not easily interpretable, indicating the need for more advanced interpretability techniques beyond basic feature extraction.

The Object Detection and Heatmap Clustering Approach emerges as a response to the challenge of interpretability in feature extraction. This section delves into how this approach not only identifies key features in images but also represents them in a manner that enhances one's understanding of what the network considers important. By integrating object detection principles with heatmap clustering, the methodology provides a clearer, more comprehensive view of how networks process and prioritise information. 

\subsection{Segment Anything Model (SAM)}
To bridge this gap between significant feature extraction and human interpretability, we propose the use of object detection. The method used in this chapter is "Segment Anything Method" (\SAM)~\cite{Kirillov2023SegmentA}. \SAM\ focuses on breaking down the input into segments or regions in a manner that aligns with human understanding. By doing so, \SAM\ translates the intricate and sometimes abstract features highlighted by the network into more understandable visual components that are recognisable to human observers. 

\begin{figure}[ht!]
\begin{center}
\includegraphics[width=\textwidth]{Figures/Clusters_SAM.pdf}
\end{center}
\caption{Demonstration of the advanced object identification technique \SAM\/, showcasing the segmentation of images with 18 to over 170 distinct objects in six different inputs.}
\label{Fig:Sam_many_masks}
\end{figure} 

A challenge that emerges by using SAM is the volume of objects that can be present in a single image. This problem is vividly illustrated in Figure~\ref{Fig:Sam_many_masks}, where six inputs are shown, varying from 18 to over 170 distinct objects in a single input. In these instances, the presentation of such a vast number of objects becomes problematic, particularly in terms of human interpretability and visual clarity. The high number of objects leads to indistinguishable colours, making it difficult to differentiate between objects and correlate them with the legend, especially when assisted contribution values are considered (See  Section~\ref{chapter:revLRP} and Section~\ref{chapter:REVEAL} for how relevance values are assigned to complex input features).

This challenge highlights a disconnect between advanced image processing algorithms and human cognitive limits. While computer vision systems can segment many objects, presenting this data in a human-readable way remains difficult.

A practical solution is to simplify by displaying only the top 5 largest objects, reducing cognitive load and focusing on the most prominent elements. However, this approach does not always align size with object relevance, as shown in Fig~\ref{Fig:SAM_nor_right}, where important objects identified by relevance detection differ from the largest.

\begin{figure}[ht!]
\begin{center}
\includegraphics[width=\textwidth]{Figures/SAM_SELECTION.pdf}
\end{center}
\caption{Only the largest 5 object selected in each input image. This can leads to not selecting the top most relevant objects in the image.}
\label{Fig:SAM_nor_right}
\end{figure} 

The Segment Anything Method (\SAM) excels in segmenting objects, while heatmap clustering highlights key features for neural network decision-making. This chapter proposes integrating \SAM\ with heatmap clustering to link interpretable object masks with deep network features. Some refinements are needed to ensure compatibility and effectiveness.

\subsection{Preprocessing the output of SAM for Integration with Heatmap Clustering}

Before fully examining the integration between SAM and Heatmap Clustering (discussed in Section~\ref{sec:integration}), it is essential to undertake a series of preprocessing steps. These steps are aimed at overcoming certain inherent limitations of \SAM\ and ensures that the object masks $\mathcal{M}_1$ generated by \SAM\ are adequately prepared and formatted to complement and enhance the insights provided by the set of heatmap clustering masks $\mathcal{M}_2$.

\subsubsection{Generating Masks for Undetected regions}

A key limitation of SAM is its occasional failure to detect certain parts of an image, potentially missing critical areas. To address this, the first preprocessing step is to scan the entire image to identify self-contained regions that SAM might have overlooked.

Let $\mathcal{M}_1$ represent the set of binary masks produced by SAM, where $\mathcal{M}_1 = \{M_1^1, M_1^2, \ldots, M_1^k\}$, with each $M_1^i$ being a binary matrix (size $n \times m$), where $1$ indicates detected objects and $0$ represents undetected regions. The sum of all masks is calculated as:
\begin{equation*}
S_1 = \sum_{i=1}^{k} M_1^i
\end{equation*}
where \(S_1\) shows the total coverage of detected objects, with zero entries marking areas missed by SAM.

Next, a complementary mask \( \overline{M} \) is created to highlight these undetected regions:
\begin{equation*}
\overline{M} = 1 - \min(S_1, 1)
\end{equation*}
This operation ensures \( \overline{M} \) marks undetected areas, using connected component analysis to identify distinct regions. Each component \(C_j\) forms a new mask $M'_j$ defined as:
\begin{equation} 
M'_{j,ij} = \begin{cases} 
   1 & \text{if } (i, j) \text{ belongs to } C_j \\
   0 & \text{otherwise} 
   \end{cases}
\end{equation}

The process yields new masks \( \{M'_1, M'_2, \ldots, M'_p\} \) that cover previously undetected regions. This complements SAM’s initial segmentation, ensuring comprehensive coverage of the image.

To manage the number of masks, a size threshold \( \tau \) (set at $0.1\%$ of the input image size $I$) is introduced:
\begin{equation} 
\tau = 0.001I 
\end{equation}
Only masks corresponding to connected components larger than this threshold are retained, resulting in a curated set of significant masks. The final set \( \mathcal{M}_1 = \{M_1^1, M_1^2, \ldots, M_1^n\} \) includes both SAM's original masks and the newly generated ones, ensuring comprehensive detection.

\begin{algorithm}[ht]
\caption{Generate Masks for Undetected Regions}
\begin{algorithmic}[1]
\Require Set of SAM masks $\mathcal{M}_1 = \{M_1^1, M_1^2, \ldots, M_1^k\}$, Image size $I$
\Ensure Updated set of masks $\mathcal{M}_1$ with new masks for undetected regions
\State Initialize $S_1 \gets 0$ matrix of size of the image
\For{each mask $M_1^i$ in $\mathcal{M}_1$}
    \State $S_1 \gets S_1 + M_1^i$
\EndFor
\State Compute the complementary mask: $\overline{M} \gets 1 - \min(S_1, 1)$
\State Identify connected components in $\overline{M}$: $CC \gets \text{ConnectedComponents}(\overline{M})$
\State Initialize $\text{new\_masks} \gets \emptyset$
\For{each connected component $C_j$ in $CC$}
    \If{$\text{Size}(C_j) \geq \tau$}
        \State Create new mask $M'_j$ where $M'_j(p, q) = 1$ if $(p, q) \in C_j$, else $0$
        \State Add $M'_j$ to $\text{new\_masks}$
    \EndIf
\EndFor
\State Update $\mathcal{M}_1 \gets \mathcal{M}_1 \cup \text{new\_masks}$
\State \Return $\mathcal{M}_1$
\end{algorithmic}
\end{algorithm}

\subsubsection{Enhancing Object Detection with Edge Consideration}

CNNs often emphasize edges as critical features during image analysis, whereas SAM typically segments objects without considering edge details. To better align SAM's segmentation with the neural network's focus, the boundaries of each object mask are expanded by approximately $0.001\%$ in all directions. This expansion incorporates edge features, making the segmented objects more consistent with CNN-detected patterns.

The dilation process expands the regions of interest (the '1' values in $M$) by convolving the mask with a square matrix $K$ of ones, where $K$ has dimensions $p \times p$ and $p$ is $0.001\%$ of the input image size. This operation effectively broadens the area around the original points of interest, capturing edge details (see Section~\ref{sec:conv} for convolution details).

\begin{algorithm}[H]
\caption{Enhance Masks with Edge Consideration}
\begin{algorithmic}[1]
\Require Set of masks $\mathcal{M}_1 = \{M_1^1, M_1^2, \ldots, M_1^n\}$, Expansion size $p$
\Ensure Expanded set of masks $\mathcal{M}_1$
\State Create a structuring element $K$ of ones with size $p \times p$
\For{each mask $M_1^i$ in $\mathcal{M}_1$}
    \State $M_1^i \gets \text{Dilate}(M_1^i, K)$
\EndFor
\State \Return $\mathcal{M}_1$
\end{algorithmic}
\end{algorithm}


\subsubsection{Resolving Overlapping Clusters}

Overlapping segments in the segmentation process can create redundant information and increase complexity. SAM may detect the same object multiple times with slight variations. 

An iterative algorithm is applied to refine the set of masks, ensuring each represents a distinct image segment. The refinement starts with the largest mask and progressively removes overlapping regions. Given a set of masks \(\mathcal{M}_1 = \{M_1^1, M_1^2, \ldots, M_1^n\}\), where each is a binary matrix, the masks are ordered by size (the number of '1's). The largest mask, \(M_1^1\), is used as the primary mask, and in each iteration, \(M_1^i\) is refined by removing overlaps with all larger masks \(M_1^j\) (where \(j > i\)) using bitwise operations:
\begin{equation}
M_1^i \leftarrow M_1^i \oplus (M_1^i \land M_1^j),
\end{equation}
where \( \oplus \) and \( \land \) are the XOR and AND operations, respectively.


\begin{algorithm}
\caption{Resolve Overlapping Clusters}
\begin{algorithmic}[1]
\Require Set of masks $\mathcal{M}_1 = \{M_1^1, M_1^2, \ldots, M_1^n\}$
\Ensure Refined set of masks $\mathcal{M}_1$ without overlaps
\State Sort $\mathcal{M}_1$ in descending order based on the size of masks (number of ones)
\For{$i = 1$ to $n$}
    \For{$j = i+1$ to $n$}
        \State $\text{Overlap} \gets M_1^i \land M_1^j$ \Comment{Bitwise AND}
        \If{any overlap exists}
            \State $M_1^j \gets M_1^j \oplus \text{Overlap}$ \Comment{Remove overlapping pixels using XOR}
        \EndIf
    \EndFor
\EndFor
\State \Return $\mathcal{M}_1$
\end{algorithmic}
\end{algorithm}


This hierarchical process ensures each mask is refined to eliminate overlaps with smaller masks.

\subsection{Integration between SAM and Heatmap Clustering}
\label{sec:integration}
We propose a methodological framework to analyse the intersections between two distinct sets of masks, \(\mathcal{M}_1\) (from post-processed SAM) and \(\mathcal{M}_2\) (from Heatmap Clustering). This approach identifies the most relevant objects in an image, prioritising significance over size.
 
Large objects in an image may not necessarily be the most informative or relevant for a given analysis. For instance, in medical imaging, a small anomaly might be of far greater significance than larger but normal anatomical structures. In Figure~\ref{Fig:SAM_nor_right} the top 5 objects by size post-processing are shown, one can notice that the objects selected by size do not coincide with the areas found by the heatmap clustering. The main objective of the combination between the heatmap clustering and SAM is to select the top objects that the network finds relevant rather than selecting them by size.

% \begin{figure}[ht!]
% \begin{center}
% \includegraphics[width=\textwidth]{Figures/by_size.pdf}
% \end{center}
% \caption{Illustration of the top 10 objects identified by size in a sample image. This highlights the need for more nuanced selection criteria, as larger objects may not correspond to the most relevant or informative elements for analysis.}
% \label{Fig:large}
% \end{figure} 


\subsubsection{Sum of Overlaps} 

The approach measures the total intersection area between each mask in \(\mathcal{M}_1\) and all masks in \(\mathcal{M}_2\), evaluating how much each mask in \(\mathcal{M}_1\) overlaps with \(\mathcal{M}_2\). This method focuses on total overlap rather than the size of individual masks, making it suitable when overall coverage is more important than specific pairings.

For each mask \( M_1^i \) in \(\mathcal{M}_1\), the overlap is calculated by summing pixel-wise logical AND operations with each mask \( M_2^j \) in \(\mathcal{M}_2\):
\begin{equation*}
    O_1^i = \sum_{j} \sum_{k, l} \left( M_1^i(k, l) \land M_2^j(k, l) \right)
\end{equation*}
Masks in \(\mathcal{M}_1\) are then ranked by their overlap sum \( O_1^i \).


Figure~\ref{Fig:most_relevant} shows the primary regions identified by this method, which tends to select larger regions as they cover more points highlighted by the heatmap clustering. While larger regions are often chosen, the method also effectively identifies those most relevant to the heatmap's indications.
\begin{figure}[ht!]
\begin{center}
\includegraphics[width=0.9\textwidth]{Figures/summs_of_overlaps.pdf}
\end{center}
\caption{Depicted here are the key regions identified using the sum of overlapping areas method. This approach inherently gravitates towards larger regions, as they generally encompass more points from the heatmap clustering maps.}
\label{Fig:most_relevant}
\end{figure} 

\subsubsection{Maximum Percentage Overlap with $\mathcal{M}_1$} 
To overcome the limitations of focusing on total overlap, this method evaluates the proportion of each mask in \(\mathcal{M}_1\) that achieves maximal overlap with any single mask in \(\mathcal{M}_2\). This helps determine how well each object mask in \(\mathcal{M}_1\) is covered by a feature mask in \(\mathcal{M}_2\), making it ideal for assessing the completeness of coverage.

For each mask \( M_1^i \), calculate the percentage of its area that overlaps with each mask \( M_2^j \), and take the maximum:
\begin{equation*}
    P_1^i = \max_j \left( \frac{\sum_{k, l} \left( M_1^i(k, l) \land M_2^j(k, l) \right)}{\sum_{k, l} M_1^i(k, l)} \right)
\end{equation*}
The masks in \(\mathcal{M}_1\) are then ranked by their maximum percentage overlap \( P_1^i \).

Figure~\ref{Fig:covarage_mask1} shows the regions identified using this method, which tends to prioritise smaller areas. Smaller regions are more likely to align closely with heatmap clustering, leading to a strong correlation with the clustering outputs. The precision of this method is evident in the selected regions, which accurately reflect the spatial patterns highlighted by the heatmap clustering.

\begin{figure}[ht!]
\begin{center}
\includegraphics[width=\textwidth]{Figures/covarage_mask1.pdf}
\end{center}
\caption{This figure displays the selected regions identified by the method prioritising maximum percentage overlap of the masks from the heatmap-based clustering with those in $\mathcal{M}_1$. Notably, it reveals a tendency to highlight smaller regions, which is attributable to the proportionally higher coverage these regions achieve in alignment with the heatmap clustering.}
\label{Fig:covarage_mask1}
\end{figure} 

\subsubsection{Maximum Percentage Overlap with $\mathcal{M}_2$} 

This method reverses the perspective of the "Maximum Percentage Overlap with \(\mathcal{M}_1\)". It focuses on how well each heatmap-based cluster from \(\mathcal{M}_2\) is encapsulated by a single mask in \(\mathcal{M}_1\).

The maximum percentage overlap for each \( M_1^i \) is defined as:
\begin{equation*}
P_2^i = \max_j \left( \frac{\sum_{k, l} \left( M_1^i(k, l) \land M_2^j(k, l) \right)}{\sum_{k, l} M_2^j(k, l)} \right)
\end{equation*}
After calculating \( P_2^i \) for all \( i \), masks in \(\mathcal{M}_1\) are ranked by their coverage of \(\mathcal{M}_2\).

Figure~\ref{Fig:covarage_mask2} illustrates the regions identified using this method. It can detect objects that closely match heatmap clusters or larger objects that fully encapsulate the clusters, even if they partially overlap. This duality sometimes leads to selecting larger regions that align with the clusters but may introduce a discrepancy between the scale and specificity of the identified areas

\begin{figure}[ht!]
\begin{center}
\includegraphics[width=\textwidth]{Figures/covarage_mask2.pdf}
\end{center}
\caption{This figure illustrates the regions identified by emphasising the complete encapsulation of heatmap-based clusters by individual object masks in $\mathcal{M}_1$. It highlights the method's ability to accurately align with certain heatmap clusters while also indicating instances where larger $\mathcal{M}_1$ objects are selected for completely encompassing smaller heatmap clusters.}
\label{Fig:covarage_mask2}
\end{figure} 

\subsubsection{Combined Average Overlap Percentage}

This method offers a balanced metric that considers the overlap percentages between both \(\mathcal{M}_1\) and \(\mathcal{M}_2\), providing a bidirectional analysis. It accounts for how much each \(\mathcal{M}_1\) mask is covered by \(\mathcal{M}_2\) and vice versa, ensuring neither set dominates the outcome. This enables a more comprehensive interpretation of the overlap between heatmap clusters in \(\mathcal{M}_2\) and SAM object masks in \(\mathcal{M}_1\). For each mask \( M_1^i \), the average of the overlap percentages relative to both \( M_1^i \) and \( M_2^j \) is calculated:
\begin{equation*}
C^i = \max_j \left( \frac{1}{2} \left[ \frac{\sum_{k, l} \left( M_1^i(k, l) \land M_2^j(k, l) \right)}{\sum_{k, l} M_1^i(k, l)} + \frac{\sum_{k, l} \left( M_1^i(k, l) \land M_2^j(k, l) \right)}{\sum_{k, l} M_2^j(k, l)} \right] \right)
\end{equation*}

Figure~\ref{Fig:covarage_masks} shows the regions identified by this method, balancing precision and comprehensiveness. It highlights highly precise areas with strong alignment to heatmap clusters, as well as larger regions that provide broader coverage, encapsulating key areas. This approach captures both detailed and extensive features relevant to the analysis.
\begin{figure}[ht!]
\begin{center}
\includegraphics[width=\textwidth]{Figures/covarage_masks.pdf}
\end{center}
\caption{This figure illustrates regions selected by the combined average overlap percentage method, showcasing a balanced blend of precision and comprehensiveness.}
\label{Fig:covarage_masks}
\end{figure} 



\section{Conclusion}

This chapter has rigorously explored the challenge of isolating complex input features through clustering techniques on top of heatmaps. 

The heatmap clustering methodology extracts features that are deemed crucial by the network for its decision-making process. This direct link to the network’s internal representations provides insights into its classification mechanisms. However, a significant challenge arises when the extracted features, although vital for the network, lack immediate interpretability. This necessitates a more sophisticated approach that extracts comprehensible data points.

The latter part of the chapter combines object detection with the heatmap clustering approach. This not only pinpoints key features within images but also presents them in a way that significantly enhances our understanding of the aspects deemed important by the network.

The examples provided throughout this chapter serve primarily as illustrations to demonstrate the capabilities of the proposed approach. While they showcase the effectiveness of integrating object detection with heatmap clustering for feature isolation and interpretability, they do not constitute a systematic validation of the method. Comprehensive validation requires further empirical studies and quantitative analyses across diverse datasets and architectures, which remain as future work to establish the robustness and generalizability of this approach.

The methodologies and insights outlined in this chapter feed into the broader objective of assigning singular importance values to complex input features, a central theme in forthcoming discussions (see Chapters \ref{chapter:REVEAL} and \ref{chapter:revLRP}).

% \chapter{Relevance Distribution Tracing}
\label{chapter:assigning}

\section{Introduction}

The assignment of a singular value of importance to each complex input feature is a crucial
step in achieving a balance between the interpretability and faithfulness of the explanation.

Rather than depending on aggregation techniques that may obscure critical details, this chapter proposes a method that evaluates the actual influence of complex input features during the forward pass of the network. 

\section{Challenges with Aggregation of Relevance Scores}

\subsection{The Naive Approach}

The naive way to assign a single value is by aggregating the relevance scores provided by an interpretability method to the individual features within the cluster, thereby offering a summarized measure of the cluster's significance. Various aggregation methods, such as mean, median, or weighted sum, could be considered based on the application.

\subsection{Limitations of Aggregation Methods}

Aggregating relevance scores into a single value effectively compresses the diversity and granularity of the information contained within the cluster. This can be problematic when features have vastly different relevances that become obscured through aggregation. Not only can it lead to a loss of information, but it can also introduce biases or distortions. For example, taking a simple mean might be sensitive to outliers, while a median might not capture the influence of exceptionally important features well. The single value of relevance should be equally representative of all input features in the cluster, which could be problematic if the cluster contains a diverse range of features with different relevance scores, as no aggregation metric can capture this diversity with a single value.

In Figure~\ref{Fig:aggregating}, the aggregated scores of various clusters are presented. The method used to aggregate the scores involves both the median and the mean. No notable difference between median and mean can be seen; however, size and composition have a tremendous effect on the values assigned.

\begin{figure}[ht!]
	\centering
	\includegraphics[width=0.9\linewidth]{Figures/Mean_median.pdf}
	\caption{This figure is divided into two sections: ``Mean of Relevances'' (top) and ``Median of Relevances'' (bottom). It consists of three columns, with the left being the baseline image, the center showing a heatmap of regional relevance using Layer-wise Relevance Propagation (LRP), and the right displaying a color-coded relevance representation, where the image is overlaid with a gradient representing regions of interest, accompanied by a color scale legend for quantitative interpretation.}
	\label{Fig:aggregating}
\end{figure} 

For instance, smaller clusters, exemplified by those representing the pegs of the violin, showcase both a high mean and a high median. This suggests that the majority of input features within these clusters have high relevance, leading to their elevated aggregated scores.

Conversely, the larger cluster representing the body of the violin exhibits a different trend. The violin's cluster consists of areas with significant variations in relevance—some areas possess exceptionally high relevance while others do not. This heterogeneity within the cluster results in both a mean and median that are relatively low. This is particularly interesting as the mean and median values for the violin's cluster might not give a comprehensive view or representation of all the input features that constitute it. It underscores the importance of understanding the individual contributions within a cluster rather than just relying on aggregated metrics, as they might mask the nuances and variations present within larger, more diverse clusters.

\section{Interdependence of Features in Neural Networks}

Each feature in the input feature map undergoes various transformations as it passes through the layers of a network. These transformations are influenced by the surrounding features, affecting the ultimate relevance that a feature has. For example, pooling layers aggregate information from kernels of features, inherently making the output a function of multiple input features (see Section~\ref{section:avglayer}). Methods like batch normalization standardize features based on the distribution formed by all features, further intertwining their relevances (see Section~\ref{section:bn}). Current explainability methods that focus on properties such as fidelity, input invariance, handling saturation, and sensitivity assign the value of relevance of each feature in the input space given the output and its connection to other input features~\cite{SimonyanVZ13, SpringenbergDBR14, bach2015pixel, SelvarajuCDVPB20, ChattopadhyaySH18, abs-1908-01224, SmilkovTKVW17}. Therefore, each feature's assigned value is relative to every other feature in the input feature map. When aggregating the relevance from the same cluster, the relationship between the complex input feature relevance and other complex input feature relevances will not be the same as the relationships between the features that comprise it. A further problem with aggregation methods is that the relevance of distinct features in the input space can come from the \emph{same} more complex feature detected in the deeper layers of the network. Hence, in situations where a cluster contains highly correlated features, the aggregation might amplify this correlation, leading to overemphasis or underrepresentation of certain aspects in the data. Conversely, in cases where clusters contain uncorrelated or weakly related features, the aggregated score might dilute the individual feature's significance.

The inherent interdependence between features makes the process of untangling individual contributions challenging. For instance, when a convolutional layer identifies a specific pattern in an image, it doesn't just recognize a single pixel but rather a combination of neighboring pixels forming a pattern. So, if one were to assign importance to a single pixel in this scenario, it would not accurately reflect the true importance. Instead, the combined pattern of several pixels holds the true importance. When features are aggregated into clusters, this interdependency causes the problems described so far. This is particularly problematic in cases where the network has many layers and complex interactions between features, as some features may seem relevant but might not have a direct or substantial importance to the final output.

\section{Forward Pass Approach to Assigning a Single Value of Relevance to  A Complex Input Feature}

In light of these considerations, this thesis introduces a general approach for assigning an importance value to a complex input feature $CF$. Instead of determining the relevance by reversing the functions at each layer of the neural network~\cite{ZeilerKTF10, SpringenbergDBR14, SimonyanVZ13, bach2015pixel, LapuschkinBMMS16, ShrikumarGK17, SundararajanTY17}, we propose a \emph{forward-pass approach} that determines the influence only the complex input feature had during the inference (forward pass) through the network.

\section{Conclusion}

In this chapter, we have critically examined the challenges associated with assigning a singular importance value to complex input features in neural networks. We highlighted the limitations of aggregation methods, which often obscure the nuanced contributions of individual features due to loss of information, introduction of biases, and inability to account for the interdependence of features inherent in neural network architectures.

To address these challenges, we introduced the \emph{Contribution to Classification} (\CTC) method, a novel forward-pass approach that quantifies the influence of complex input features during the inference process. Unlike aggregation methods that rely on reversing the functions at each layer or aggregating relevance scores, the \CTC\ method provides a more faithful and interpretable measure of feature importance by evaluating how each complex input feature contributes to the network's output at every layer.

We demonstrated the differences between the \CTC\ method and traditional inference, particularly in the context of max pooling and ReLU layers, illustrating how the \CTC\ method captures feature contributions more accurately by considering the actual computational paths taken during the forward pass.

The introduction of the \CTC\ method offers a framework that preserves the fidelity of feature contributions while enhancing interpretability. In the subsequent chapters, specifically Chapters~\ref{chapter:revLRP} and~\ref{chapter:REVEAL}, we will delve into the specific rules and modifications to the forward pass that operationalize the \CTC\ method. We will explore how this method can be applied to different types of neural network architectures and discuss its implications for model interpretability and transparency.

\include{Chapters/SingleRelevanceValue}
\chapter{Relevance Distribution Tracing}
\label{chapter:revLRP}
\section{Introduction}

This chapter focuses on creating faithful post-hoc local explanations. It takes a set of complex input features as defined in the previous Chapter~\ref{chap:clustering} and assigns a single value of relevance to each complex input feature.

The chapter starts by exploring the limitation of using a naive approach of aggregating the relevance of a complex input feature. To address these challenges, this chapter further proposes \emph{Reverse Relevance Distribution Tracing} as a method that evaluates the actual influence of complex input features during the forward pass of the network, rather than depending on aggregation techniques. By tracing the distribution of relevance from specific input features through the network layers, it becomes possible to assign a single relevance value to each complex feature that more accurately reflects its contribution to the final output.

The contributions of this chapter are as follows: 
\begin{enumerate} 
\item Proposal of a novel method called \emph{Reverse Relevance Distribution Tracing}, which reverses Layer-wise Relevance Propagation (\LRP) to assign a single relevance value to a cluster of features. This method preserves the original faithfulness of \LRP\ while enhancing interpretability. 
\item Introduction of vector-based definitions for \LRP\ rules, transitioning from previous neuron-level descriptions using set theory to a more generalized and accessible framework. 
\item Development of a set of rules for reversing the relevance propagated by basic \LRP\ rules, enabling the backward tracing of relevance distribution through the network. 
\item Extension of the proposed method to the Alpha Beta rule, allowing for a more nuanced distribution of relevance scores and adaptability to different types of neural network layers and architectures. 

\end{enumerate}

\section{Challenges with Aggregation of Relevance Scores}

The naive way to assign a single value is by aggregating the relevance scores provided by an interpretability method to the individual features within the cluster, thereby offering a summarized measure of the cluster's significance. Various aggregation methods, such as mean, median, or weighted sum, could be considered based on the application.

Aggregating relevance scores into a single value effectively compresses the diversity and granularity of the information contained within the cluster. This can be problematic when features have vastly different relevances that become obscured through aggregation. Not only can it lead to a loss of information, but it can also introduce biases or distortions. For example, taking a simple mean might be sensitive to outliers, while a median might not capture the influence of exceptionally important features well. The single value of relevance should be equally representative of all input features in the cluster, which could be problematic if the cluster contains a diverse range of features with different relevance scores, as no aggregation metric can capture this diversity with a single value.

In Figure~\ref{Fig:aggregating}, the aggregated scores of various clusters are presented. The method used to aggregate the scores involves both the median and the mean. 
% No notable difference between median and mean can be seen; however, size and composition have a tremendous effect on the values assigned.

\begin{figure}[ht!]
	\centering
	\includegraphics[width=\linewidth]{Figures/Mean_median.pdf}
	\caption{This figure is divided into two sections: ``Mean of Relevances'' (top) and ``Median of Relevances'' (bottom). It consists of three columns, with the left being the baseline image, the center showing a heatmap of regional relevance using Layer-wise Relevance Propagation (LRP), and the right displaying a color-coded relevance representation, where the image is overlaid with a gradient representing regions of interest, accompanied by a color scale legend for quantitative interpretation.}
	\label{Fig:aggregating}
\end{figure} 

For instance, smaller clusters, exemplified by those representing the fruit that the insect is on top of, showcase both a high mean and a high median. This suggests that the majority of input features within these clusters have high relevance, leading to their high scores.

Conversely, the larger cluster representing the body of the insect exhibits a different trend. The insect's cluster consists of areas with significant variations in relevance—some areas possess exceptionally high relevance while others do not. 

% This heterogeneity within the cluster results in both a mean and median that are relatively low. This is particularly interesting as the mean and median values for the violin's cluster might not give a comprehensive view or representation of all the input features that constitute it. It underscores the importance of understanding the individual contributions within a cluster rather than just relying on aggregated metrics, as they might mask the nuances and variations present within larger, more diverse clusters.

\section{Interdependence of Features in Neural Networks}

Each feature in the input feature map undergoes various transformations as it passes through the layers of a network. These transformations are influenced by the surrounding features, affecting the ultimate relevance that a feature has. For example, pooling layers aggregate information from kernels of features, inherently making the output a function of multiple input features (see Section~\ref{section:avglayer}). Methods like batch normalization standardize features based on the distribution formed by all features, further intertwining their relevances (see Section~\ref{section:bn}). Current explainability methods that focus on properties such as fidelity, input invariance, handling saturation, and sensitivity assign the value of relevance of each feature in the input space given the output and its connection to other input features~\cite{SimonyanVZ13, SpringenbergDBR14, bach2015pixel, SelvarajuCDVPB20, ChattopadhyaySH18, abs-1908-01224, SmilkovTKVW17}. Therefore, each feature's assigned value is relative to every other feature in the input feature map. When aggregating the relevance from the same cluster, the relationship between the complex input feature relevance and other complex input feature relevances will not be the same as the relationships between the features that comprise it. A further problem with aggregation methods is that the relevance of distinct features in the input space can come from the \emph{same} more complex feature detected in the deeper layers of the network. Hence, in situations where a cluster contains highly correlated features, the aggregation might amplify this correlation, leading to overemphasis or underrepresentation of certain aspects in the data. Conversely, in cases where clusters contain uncorrelated or weakly related features, the aggregated score might dilute the individual feature's significance.

The inherent interdependence between features makes the process of untangling individual contributions challenging. For instance, when a convolutional layer identifies a specific pattern in an image, it doesn't just recognize a single pixel but rather a combination of neighboring pixels forming a pattern. So, if one were to assign importance to a single pixel in this scenario, it would not accurately reflect the true importance. Instead, the combined pattern of several pixels holds the true importance. When features are aggregated into clusters, this interdependency causes the problems described so far. This is particularly problematic in cases where the network has many layers and complex interactions between features, as some features may seem relevant but might not have a direct or substantial importance to the final output.


\section{Relevance Distribution Tracing}

To solve this, this chapter proposes a set of rules for \emph{Reverse Relevance Distribution Tracing}, which inspects parts of the input with respect to the distribution of relevance attributed to them by backward propagation explainability methods.

In this context, Layer-wise Relevance Propagation (\LRP)~\cite{bach2015pixel} is selected as the method for reversal, primarily because of its faithful representation of relevance distribution. \LRP is input-invariant and accounts for each neuron's contribution relative to others, which aids in understanding the impact of saturated features. Additionally, it produces different results when drastic changes occur in the input and output, enhancing its sensitivity to significant variations. By tracing the relevance of a part of the input and assigning a single relevance value to the entire region, \LRP preserves the method's original faithfulness while improving the interpretability of the explanation.

The reverse relevance distribution tracing approach utilizing \LRP\/ involves retracing the relevance assigned to each layer back from specific input features. By applying the reverse rules of \LRP\/ (see Section~\ref{rev_LRP}), one can gain insight not only into the feature's relevance to the final output but also the relevance attributed to that feature in the intermediate layers. Unlike other propagation techniques, which can be sensitive to the model's non-linearities and high-dimensional interactions, \LRP\/ is designed to conserve the relevance signal across layers. This conservation ensures that the relevance attributed to the output is precisely apportioned back through each layer, ultimately distributing it over the input features in a manner that reflects their contribution to the final decision. Therefore, by reversing the \LRP\/ rules, each complex input feature is assigned a single relevance value that reflects its influence on the network's decision-making process.

Figure~\ref{Fig:thee_stage} provides a comprehensive illustration of the neural network processes of inference, relevance assignment, and relevance distribution
tracing. The leftmost panel depicts the inference step, where input data is processed through successive layers to produce an output, with activation flow highlighted in yellow, where the brighter the colour the higher the activation. The center panel illustrates the backward pass, where relevance from the output is traced back to the input features using \LRP\/, visualized with red arrows moving from the output neuron back to the input layer. This step shows how contributions to the final decision are mapped back to individual input features through the application of \LRP\/ rules at each layer. The rightmost panel shows the novel relevance distribution tracing from selected input features, highlighting the process of reversing the \LRP\/ rules. Here, a subset of input features serves as the starting point, and the relevance corresponding to this subset is redistributed forward through the network, indicated by green arrows, from the input layer to the output until the contribution to the classification (\CTC\/) value is found.

\begin{figure}[ht!]
\begin{center}
\includegraphics[width=0.95\linewidth]{Figures/thee_stage.pdf}
\caption{Illustration of the three-phase neural network process, showcasing initial inference (left), Layer-wise Relevance Propagation (LRP) for relevance assignment (center), and the forward redistribution of relevance based on selected input features (right). Data propagation during inference is highlighted in yellow, the backward tracing of relevance via LRP is shown in red, and the forward redistribution of relevance from selected features is indicated in green, effectively mapping the contributions of input features to the network's decision-making process.}
\label{Fig:thee_stage}
\end{center}
\end{figure}

\section{Basic Rule}

To build on the foundation of \LRP\/, one needs to understand the layer-by-layer relevance propagation in detail. In this first subsection (see Section~\ref{lrp}), the \LRP\/ basic rule is broken down into multiple steps, where each part of the rule is not only described, but the reasoning behind it is explained. This enables the discussion of the contribution in this chapter (see Section~\ref{rev_LRP}) which computes the inverse of \LRP\/ and redistributes the relevance of a single complex feature from the input into the deeper layers of the network.

\subsection{Basic LRP in Detail}
\label{lrp}
The basic \LRP\/ rule introduced by~\cite{bach2015pixel} through which relevance is distributed to a given layer $\Lambda_j$ from the consecutive deeper layer $\Lambda_k$ can be decomposed into the following four steps. 

\begin{enumerate}
    \item In the first step, the authors define a vector-valued function $\vec{t}_k:\mathbb{R}^j\to\mathbb{R}^k$, that maps the activation vector from one layer of the neural network to another. The transformation is defined by the equation:
\begin{equation*}
    \vec{t}_k(\vec{x}) = W_j^\intercal\, \vec{x} + \vec{\epsilon}
\end{equation*}
   Here, $W_j^\intercal$ represents the transposed weight matrix associated with the connections from layer $\Lambda_j$ to layer $\Lambda_k$, and $\vec{\epsilon}$ is a tiny, nonzero bias vector introduced to ensure numerical stability. This bias vector is crucial because it prevents the possibility of dividing by zero in subsequent steps. The vector $\vec{t}_k$ refers to the weighted sum of the activations from the previous layer, slightly modified by the addition of $\vec{\epsilon}$. Note that this function also bypasses any activation function that may exist in the forward pass after the transformation has been completed.
   \item Once the modified weighted sum $\vec{t}_k$ is computed, the distribution of relevance scores from layer $\Lambda_k$ to layer $\Lambda_j$ is performed. The relevance vector $R_k$ from the deeper layer $\Lambda_k$ is element-wise divided by $\vec{t_k}$. This division determines how much relevance should be backpropagated to each unit of activation and weighted in the preceding layer. It is defined as:
\begin{equation*}
   \vec{s}_k = \vec{R}_k / \vec{t}_k.
\end{equation*} 
   The vector $\vec{s}_k$ contains the specific proportion of relevance that is attributed to each element of the activation weighted by $W_k$ in the layer $\Lambda_j$. This step is crucial for later understanding which parts of the preceding layer contributed most to the final decision.
   Steps 3) and 4) multiply the vector $\vec{s}_k$ that contains the relevances per unit with the activation$\times$weight, so that the final relevance $\vec{R}_j$ for layer $\Lambda_j$ is found.
   \item The third step multiplies the vector $\vec{s}_k$ with the weights $W_j$ between $\Lambda_j$ and $\Lambda_k$. This is defined as: 
\begin{equation*}
   \vec{c_{j}} =  W_j\, \vec{s_{k}}.
\end{equation*} 
This step is better achieved through calculating the gradient of \(\vec{t_{k}}(\vec{x})\) with respect to \(\vec{x}\). Given 
\begin{equation*}
    \nabla\left [\vec{t_{k}}(\vec{x}) \right] = \nabla\left [W_j^\intercal\, \vec{x} + \vec{\epsilon} \right],
\end{equation*} 
the gradient of \(\vec{t_{k}}(\boldsymbol{x})\) with respect to \(\vec{x}\) involves computing the partial derivatives of each component. Since \(\vec{\epsilon}\) is a constant vector, its gradient is zero. The gradient of a linear transformation \(W_j^\intercal\, \vec{x}\) with respect to \(\vec{x}\) is simply the matrix \(W_j^\intercal\) itself. This is because the derivative of a linear transformation is the transformation itself. Therefore, in the linear case
\begin{equation*}
    \nabla\left [\vec{t_{k}}(\vec{x}) \right] = W_j^\intercal.
\end{equation*}
However, in many neural network architectures, the transformations between layers can be non-linear. In such cases, the gradient \(\nabla\left [\vec{t_{k}}(\vec{x}) \right]\) is not simply the weight matrix, but includes derivatives of the non-linear functions as well. Computing the gradient explicitly allows for the application of the same rule in both linear and non-linear contexts. Modern machine learning frameworks (like TensorFlow, PyTorch, etc.) utilise automatic differentiation, which efficiently computes gradients of complex functions. This abstracts away the manual computation of derivatives and allows the framework to handle the complexities of gradient computation, regardless of whether the transformation is simple (like a linear layer) or complex (like a convolutional or recurrent layer). Therefore, the vector $\vec{c_{j}}$ is given through:
 \begin{equation*}
    \vec{c_{j}} = \nabla\left [\vec{t_{k}}(\boldsymbol{x}) \cdot \vec{s_{k}}\right],
\end{equation*}
which provides a generalised, flexible, and often more efficient approach to implementing neural network transformations, particularly for complex or non-linear layers.
   \item Finally, element-wise multiplication is performed to scale the intermediate relevance $\vec{c_{j}}$ to match the activations of the layer. The equation for this is as follows:
   \[
   \vec{R}_{j} = \vec{a}_{j} \odot \vec{c}_{j}
   \]
   Here, $\vec{a}_{j}$ is the activation vector of layer $\Lambda_j$, and $\vec{c}_{j}$ is the gradient vector computed in the previous step. The operation $\odot$ signifies element-wise multiplication, which scales the gradient by the activation, thus completing the relevance distribution to layer $\Lambda_j$.
\end{enumerate}

The entire process of \LRP\ can  be combined into a single (compact) equation
\begin{eqnarray*}
R_{j}= \vec{a}_{j} \odot \nabla\left [\vec{t_{k}}(\boldsymbol{x}) \cdot \frac{\vec{R}_k}{\vec{z_k}(\vec{a_j})}\right],
\end{eqnarray*}
which encapsulates the propagation of relevance from layer $\Lambda_k$ to $\Lambda_j$. However, understanding each step in detail as described in this subsection is critical for appreciating the intricacies of \LRP\ and building on top of it. 


\begin{figure}[ht!]
\begin{center}
\includegraphics[width=\textwidth]{Figures/LRP_illustration.pdf}
\end{center}
\caption{Illustration of Layer-wise Relevance Propagation (LRP) between Layers $\Lambda_j$ and $\Lambda_k$. This diagram depicts the sequential steps involved in the \LRP\/ algorithm, starting with the transformation of the activation vector from layer $\Lambda_j$ to $\Lambda_k$, followed by the computation of relevance scores, and culminating in the backpropagation of these scores to the preceding layer. Each step is clearly marked to demonstrate the flow and transformation of information, highlighting the critical aspects of \LRP\/, such as the element-wise division by the modified weighted sum, gradient computation, and the final element-wise multiplication with the activation vector. This visualisation serves as a discrete example of the methodology.}
\label{Fig:LRP_breakdown}
\end{figure} 


A detailed breakdown of the \LRP\ process, from the initial transformation of the activation vector to the final distribution of relevance to a preceding layer, is visualised in the accompanying Figure~\ref{Fig:LRP_breakdown}, which illustrates the application of \LRP\ between two layers, $\Lambda_j$ and $\Lambda_k$. In the example presented in Figure~\ref{Fig:LRP_breakdown}, the relevance matrix at layer \(\Lambda_k\), denoted as \(\vec{R}_k\), is initialized with values \([0.6, 0.2]\). The propagation begins by computing the transformation matrix \(\vec{t}_k(\vec{x})\) for the activations from the previous layer \(\Lambda_j\) using the equation \(\vec{t}_k(\vec{x}) = W_j^\intercal \vec{x} + \vec{\epsilon}\). Here, \(\vec{t}_k(\vec{x})\) results in \([0.2, 0.73]\). The relevance scores are then distributed using element-wise division of \(\vec{R}_k\) by \(\vec{t}_k(\vec{x})\), yielding the scaled relevance vector \(\vec{s}_k = [3, 0.146]\). This vector captures the proportion of relevance attributed to each activation unit. 

Next, the vector \(\vec{s}_k\) is multiplied by the weight matrix \(W_j\) to compute the intermediate gradient vector \(\vec{c}_j = W_j \vec{s}_k = [0.098, 1.026, 1.500, 0.014]\), which represents the relevance contributions distributed across the units in layer \(\Lambda_j\). Finally, element-wise multiplication with the activation vector \(\vec{a}_j = [0.3, 0.5, 0.1, 0.7]\) scales the gradient values, resulting in the relevance vector for layer \(\Lambda_j\), \(\vec{R}_j = [0.029, 0.513, 0.150, 0.009]\). This example illustrates how \LRP\ systematically redistributes relevance from deeper layers to preceding layers, aligning with the contributions of individual units to the overall decision. 

\subsection{Relevance Distribution
Tracing by Reversing Basic LRP Rules}
\label{rev_LRP}
The novel method proposed in this section reverses the propagated relevance of each complex input feature layer by layer starting from the input until the classification is reached. The final value that reaches the classification is the relevance value assigned to the complex input feature. This relevance distribution tracing is performed for all complex input features identified from the clustering algorithm described in Chapter~\ref{chap:clustering}. 


Given a trained neural network $\NN=\big(\Lambda,\passto,(f_k)_{k\in \Lambda}\big)$ with a set of layers $\Lambda=\{0,\dots, N\}$, with a single source $0\in \Lambda$, referred to as the input layer and single sink $N\in \Lambda$, referred to as the \emph{output layer}, each $f_k:\bbR^{n_j}\to \bbR^{m_k}$ describes a vector transformation associated with each layer, with the dimensionality conditions that $m_k=\sum_{j\passto k} n_j$, for all $k=1,\dots N$. 

The relevance of a given complex input feature at the input layer 0 $\in \Lambda$ is defined as:
\begin{equation*}
    R_0^\prime = \vec{a}_0 \odot \vec{m},
\end{equation*}
where $\vec{m}\in \{0,1\}^{n_0}$ is a mask over the complex input feature propagated as detected by the rules in Section~\ref{chap:clustering}, and $\odot$ denotes element-wise multiplication operation. Performing the multiplication with the mask sets to zero all the activations of neurons that are not in the region of the feature propagated.

\begin{enumerate}
\item
First the amount of relevance distributed during \LRP\ from layer $\Lambda_k$ to each neuron in layer $\Lambda_j$ is found. \LRP\ as described in Section~\ref{lrp} distributes relevance to $\Lambda_j$ by 
\begin{eqnarray*}
R_{j}= \vec{a}_{j} \odot \nabla\left [\vec{t_{k}}(\boldsymbol{x}) \cdot s_k\right].
\end{eqnarray*}
To find the amount of relevance distributed from each \emph{neuron} in layer $\Lambda_k$ to layer $\Lambda_j$, a Jacobian matrix $J_k$ (\ie a matrix of the first-order partial derivatives) of the function $\vec{z_k}$ (described in \LRP's step 1) with respect to the activations $\vec{a_j}$ in layer $\Lambda_j$ is scaled by the unit of relevance $\vec{s_k}$. Taking the Jacobian is an analog of performing the gradient operation $\nabla\left [\vec{t_{k}}(\boldsymbol{x}) \cdot \vec{s_{k}}\right]$ without summing across the $j$ dimension.
\begin{eqnarray*}
J_{kj}(\boldsymbol{x}) = \frac{\partial\left(\vec{t_{k}}(\boldsymbol{x}) \odot \vec{s_{k}}\right)}{\partial\vec{x}} 
= \left[\begin{array}{ccc}
s_{1}\,\frac{\partial z_{1}\, }{\partial x_{1}} & \cdots & s_{1}\frac{\partial z_{1}}{\partial x_{j}} \\
\vdots & \ddots & \vdots \\
 s_{k}\frac{\partial t_{k}}{\partial x_{1}} & \cdots & s_{k}\,\frac{\partial t_{k}}{\partial x_{j}}
\end{array}\right]
\end{eqnarray*}
\item
To find the full analogue of $\vec{a}_{j} \odot \nabla\left [\vec{t_{k}}(\boldsymbol{x}) \cdot s_k\right]$ the Jacobian matrix is scaled by the activation of layer $\Lambda_j$. To perform this element-wise product, a matrix $A_{jk}$ is defined as:
\begin{eqnarray*}
A_{jk} = \vec{a}_j\, \vec{1}_k^\intercal = \left[\begin{array}{ccc}
a_1 & \cdots & a_1 \\
\vdots & \ddots & \vdots \\
a_j & \cdots & a_j
\end{array}\right],
\end{eqnarray*}
which consists of the activations $\vec{a_j}$ in layer $\Lambda_j$. The multiplication with the scaled Jacobian matrix $J_{kj}(\vec{a_j})$ is performed:
\begin{eqnarray*}
R_{jk} = A_{jk} \odot J_{kj}(\vec{a_j})^\intercal\, 
\end{eqnarray*}
which results in a matrix $R_{jk}$ that holds the amount of relevance distributed from each neuron in layer $\Lambda_k$ to each neuron in $\Lambda_j$. To further understand the significance of $R_{jk}$, note that when summed across the $k$ dimension (\ie summing all the rows for each column), the result is the relevance vector $\vec{R}_j$. This is a crucial property as it shows that the total relevance assigned to each neuron in layer $\Lambda_j$ is a cumulative result of the contributions from all neurons in the succeeding layer $\Lambda_k$:
\begin{equation*}
    \vec{R}_j = \sum_{k} R_{jk}.
\end{equation*}
This property is vital for propagation of relevance, as it ensures that the total relevance in layer $\Lambda_j$ is preserved while being distributed among its neurons based on their contribution to the next layer's activations.

\item Next, the proportion of the relevance of each neuron in layer $\Lambda_k$ to the total relevance of each neuron in layer $\Lambda_j$ is calculated. This allows later to distribute the relevance $\vec{R}_j^\prime$ of the complex input feature at layer $\Lambda_j$ to represent a matrix that holds the amount of relevance each neuron in layer $\Lambda_k$ \emph{would have} distributed to each neuron in $\Lambda_j$, if the complex input feature was the only thing found relevant during the \LRP\ pass.

To find the proportion of the relevance of neurons in layer $\Lambda_k$ to layer $\Lambda_j$ the matrix $R_{jk}$ is divided element-wise by the total relevance $\vec{R}_{j}$ distributed to layer $\Lambda_j$:
\begin{eqnarray*}
P_{jk} = R_{jk} / \vec{R}_{j}.
\end{eqnarray*}
The resulting matrix contains only values between [0,1]. An important characteristic of the $P_{jk}$ matrix is that when summed across the $k$ dimension, the output is a vector of ones, with dimensionality $j$:
\begin{eqnarray*}
\sum_{k} P_{jk} = \vec{1}_j.
\end{eqnarray*}
This property is equivalent to $\vec{R}_j = \sum_{k} R_{jk}$, as $\sum_{k} P_{jk}$ being equal to 1s just shows that the percentages sums to 100\% of the $\vec{R}_j$ vector. 

\item The percent matrix $P_{jk}$ of how much each neuron in layer $\Lambda_k$ has contributed to the total amount of relevance for a neuron in layer $\Lambda_j$ is then multiplied by the new relevance $\vec{R}_j^\prime$. To perform this element-wise product, a matrix $R_{jk}^\prime$ is defined as:

\begin{eqnarray*}
R_{jk}^\prime = \vec{R}_j^\prime \, \vec{1}_k^\intercal = \left[\begin{array}{ccc}
R_1^\prime & \cdots &R_1^\prime \\
\vdots & \ddots & \vdots \\
R_j^\prime & \cdots & R_j^\prime
\end{array}\right],
\end{eqnarray*}

which consists of the relevances $R_{j}^\prime$ of the complex input feature in layer $\Lambda_j$. The multiplication with the matrix $P_{jk}$ is performed:
\begin{eqnarray*}
R_{jk}^\prime = P_{jk} \odot R_{jk}^\prime, 
\end{eqnarray*}
resulting in a matrix $R_{jk}^\prime$, which shows the amount of relevance each neuron in layer $\Lambda_k$ \emph{would have} distributed to each neuron in $\Lambda_j$, if the complex input feature was the only thing found relevant during the \LRP\ pass.

These relevance values cannot simply be summed across the $j$ dimension $\sum_{j}R_{jk}^\prime$, as the resulting vector despite having the same dimensionality as layer $\Lambda_k$ it does not take into account the function between $\Lambda_j$ and $\Lambda_k$. This is referred as the unscaled relevance of layer $\Lambda_k$.

\item To overcome this, the method calculates the aggregate ratio of the unscaled original relevance of layer $\Lambda_k$ and the unscaled complex input feature relevance of layer $\Lambda_k$. This involves summing the elements of both \( R_{jk}^\prime \) and \( R_{jk} \) across the \( j \) dimension, which yields two vectors, representing the aggregated complex input feature relevance and original relevance for each neuron in layer \( \Lambda_k \), represented by:
\begin{equation*}
\sum_{j} R_{jk}^\prime \quad \text{and} \quad \sum_{j} R_{jk}.
\end{equation*}
Next, a ratio of the unscaled complex input feature relevance to the unscaled original relevance is calculated:
\begin{equation*}
  Q_{k} = \frac{\sum_{j} R_{jk}^\prime}{\sum_{j} R_{jk}}.
\end{equation*}
This ratio \( Q_{k} \) represents the aggregated proportion of new relevance to old relevance for each neuron in layer \( \Lambda_k \). Stability in the calculation is ensured by assigning zero to $Q_{k}$ whenever \( \sum_{j} R_{jk} \) is zero. This ensures that if there is no relevance propagated from LRP to a neuron, then no relevance can be propagated during the complex input feature relevance propagation. The aggregate ratio \( Q_{k} \) is then used to scale the total relevance \( R_k \) of each neuron in layer \( \Lambda_k \). This is performed by multiplying \( Q_{k} \) element-wise with \( R_k \).
\begin{equation*}
    \vec{R}_k^\prime = Q_{k} \odot R_{k}
\end{equation*}
The vector \( \vec{R}_k^\prime \) represents the final redistributed relevance for each neuron in layer \( \Lambda_k \) from the complex input feature.

\end{enumerate}

\begin{figure}[ht!]
\begin{center}
\includegraphics[width=0.80\textwidth]{Figures/reverse_v2.pdf}
\end{center}
\caption{Concrete Example of the Inverse Layer-wise Relevance Propagation Process. This figure visually represents the step-by-step methodology of the inverse LRP as applied within a neural network.}
\label{Fig:Reverse_LRP_2}
\end{figure} 

In the example depicted in Figure~\ref{Fig:Reverse_LRP_2}, the process of redistributing relevance using inverse layer-wise relevance propagation begins with the relevance distribution matrix \(R_{jk}\), where values represent the relevance contributions from neurons in layer \(\Lambda_k\) to layer \(\Lambda_j\). 

For instance, the relevance matrix \(R_{jk}\) for this layer interaction includes values such as \([0.0, 0.029]\), \([0.45, 0.06]\), \([0.15, 0.0]\), and \([0.0, 0.009]\). These values are element-wise normalized against the total relevance of \(\Lambda_j\), resulting in the proportional relevance matrix \(P_{jk}\) with entries like \([0.0, 1.0]\), \([0.88, 0.12]\), \([1.0, 0.0]\), and \([0.0, 1.0]\). 

The relevance redistribution for a new input feature is then calculated using the proportional relevance \(P_{jk}\), scaled by the new relevance values at \(\Lambda_j\), denoted as \(\vec{R}_j'\). For this example, the new relevance matrix \(R_{jk}'\) becomes \([0.0, 0.0]\), \([0.563, 0.076]\), \([0.0, 0.0]\), and \([0.0, 0.07]\), illustrating the redistributed values back to layer \(\Lambda_k\). 

The final step scales the original relevance of layer \(\Lambda_k\), represented as \(\vec{R}_k = [0.6, 0.2]\), by the proportional aggregate relevance ratio \(Q_k = [0.93, 1.489]\). This results in the updated relevance vector \(\vec{R}_k' = [0.558, 0.297]\), signifying the final redistributed relevance for layer \(\Lambda_k\) aligned with the relevance of the input feature. This systematic process ensures the preservation of relevance while accurately redistributing it in reverse through the network layers. 

% This detailed breakdown of the inverse of LRP process, is further elucidated through a concrete example depicted in the Figure~\ref{Fig:reverse_LRP_breakdown}. This figure illustrates the practical application of the steps outlined in the reverse LRP method, showcasing how the redistribution of relevance is calculated and visualised in a neural network.
% \newpage
\section{The Alpha Beta Rule}
\label{section:more}

\LRP\ introduces a weighted version of the basic rule, often referred to as the Alpha Beta rule. The Alpha Beta rule offers several advantages over the basic rule, making it a more effective approach for many applications. One of the primary benefits of the Alpha Beta rule is its flexibility and adaptability to different types of neural network layers and architectures. This rule allows for a more nuanced distribution of relevance scores across the network, taking into account the positive and negative contributions of neurons to the final decision-making process. Unlike the basic rule, which treats all contributions equally, the Alpha Beta rule differentiates between positive (alpha) and negative (beta) contributions, allowing for a more detailed and accurate understanding of how different parts of the network contribute to its output.

Moreover, the Alpha Beta rule can mitigate some of the shortcomings of the basic rule, such as the potential for misleading relevance attributions in networks with mixed positive and negative activations. By separately considering positive and negative contributions, the Alpha Beta rule ensures a more balanced and realistic representation of the influence of each neuron. This is particularly important in complex networks where different layers may have varying impacts on the final output, and where understanding these impacts is crucial for interpreting the model's behaviour. The rule is presented as follows:
\begin{eqnarray*}
R_{j}= \alpha (R_{j}^+) - \beta(R_{j}^-)
\end{eqnarray*}
where
\begin{eqnarray*}
{R_{j}}^\pm = \vec{a}_{j} \odot \nabla\left [(\vec{t_{k}}(\boldsymbol{x}))^\pm\frac{R_{k}}{(\vec{z_k}(\vec{a_j}))^\pm}\right], 
\end{eqnarray*}

where \(\alpha\) and \(\beta\) are parameters that determine the proportion of positive and negative contributions, respectively, and their values are typically chosen such that \(\alpha - \beta = 1\), maintaining conservation of relevance scores throughout the network. The terms \({R_{j}}^+\) and \({R_{j}}^-\) represent the positive and negative contributions of a neuron \(j\) towards the activation of a higher-layer neuron \(k\). 

To reverse this rule through the method proposed in this chapter, the relevance $R_{j}$ is defined as:
\begin{eqnarray*}
R_{k}= \alpha (R_{k}^+) - \beta(R_{k}^-)
\end{eqnarray*}
where ${R_{j}}^{\pm}$ is defined as in the basic rule, but only by taking the positive or negative activation respectively:
\begin{eqnarray*}
{R_{k}}^{\prime\pm}=  R_{k}^\pm\, \odot \frac{\sum_{j}\frac{A_{jk}^\pm \odot J_{kj}(\vec{a_j}^\pm)^\intercal\,}{ \vec{R}_{j}} \odot \vec{R}_j^\prime \, \vec{1}_k^\intercal}{\sum_{j} R_{jk}}
\end{eqnarray*}

% \section{Results}

\section{Conclusion}

The method proposed in this chapter propagates relevance forward through reversing \LRP\ ~\cite{bach2015pixel}. This allows for examining the relevance of input features to the final output of a neural network. However, it comes with inherent complexities, primarily due to the computational resources required, especially for large, multi-layered networks.

A critical aspect of relevance propagation is the computation of Jacobians, which are mathematical representations used to understand how changes in inputs affect changes in outputs. In the context of this method, Jacobians are utilised to map the relevance propagated from each neuron in layer $\Lambda_k$ to each layer in $\Lambda_j$. In most neural networks, layers have four dimensions, corresponding to different aspects like input channels, output channels, and spatial dimensions (height and width in the case of image processing). When computing the Jacobian for such a network, the dimensionality effectively doubles, resulting in eight-dimensional vectors. This dramatic increase in dimensionality is one of the primary reasons for the heightened computational complexity and memory intensity. The method was tested on an 2xNvidia A100 80GB instance, but still threw an "Out of Memory Error" for a VGG16 model. This model is relatively large, but is still smaller than many of the state of the art models. It has 16 layers and approximately 138 million parameters, this equates to 19 quadrillion parameters when calculating Jacobians for the forward propagation. 


The high dimentionality of the Jacobians imposes high memory requirements. However the computation of these high-dimensional Jacobians is not just memory-intensive, but if the memory is not integrated, then accessing memory can significantly slow down the process --- i.e. both access to large-memory enabled instances and instances with a large amount of fast-access memory is needed. As a result of the need to compute Jacobians, the process of propagating relevance forward be considerably slow, acting as a limiting factor to the applicability of this method, especially in cases where real-time analysis or rapid computations are necessary. As neural networks become deeper and more complex, the scalability of relevance propagation methods increasingly becomes a concern. The exponential increase in computation and memory requirements can make it impractical to apply this method to state-of-the-art deep learning models, particularly those used in resource-constrained environments. The final issue is that the complexity of dealing with high-dimensional data can also impact the precision and accuracy of the relevance propagation. Numerical errors and approximations may accumulate, leading to less reliable interpretations of how inputs influence model outputs. 

Several strategies can be adopted to mitigate these computational challenges. Simplifying the model architecture, where feasible, can reduce the computational burden. This might involve using fewer layers or reducing the dimensionality of each layer. However, this may modify the function learned by the neural network and therefore make the interpretability method less precise. Employing approximation methods to compute Jacobians can help in reducing the computational load, although this also comes at a loss in precision. Making use of parallel processing techniques and hardware accelerators like specialised GPUs can significantly speed up the computation of high-dimensional Jacobians. Implementing efficient memory management techniques can help in handling the large memory requirements, such as using sparse matrix representations where appropriate. Further ways to address the computational challenges posed by this method are discussed in Section~\ref{dis:jac}.


While propagating relevance forward provides valuable insights into the inner workings of neural networks, the complexity and computational intensity, especially due to the requirement of computing high-dimensional Jacobians, pose significant challenges. Addressing these challenges requires a careful balance between model complexity, computational resources, and the precision of the interpretations derived from the relevance propagation process. The next chapter looks at a different way of assigning a single value of relevance to a complex input feature, which has a far smaller computational and memory requirement. 

% \include{Chapters/Contribution/Feature_Contribution}
\chapter{Contribution to Classification of Complex Input Features}
\chaptermark{Forward Pass Retracing}
\label{chapter:REVEAL}
\newcommand{\lo}[1][j]{l_{#1}}
\newcommand{\clo}[1][j]{\text{$c$-}\lo[#1]}
\newcommand{\djo}[1][j]{d_{#1}}
\newcommand{\cdjo}[1][j]{\text{$c$-}\djo[#1]}

\section{Introduction}

The previous chapter examined \emph{Relevance Distribution Tracing}, a technique that assigns importance to complex input features by reversing the flow of relevance through the network layers. This method enabled us to identify which features were most influential in the network's final prediction.

Despite its theoretical effectiveness, relevance distribution tracing relies on the computation of high-dimensional Jacobians, which require substantial computational resources and memory. These demands become prohibitive when it comes to implementing the method, limiting the method's practicality. Additionally, relevance-based approaches may overestimate the importance of certain input parts, as they consider the feature's importance within the context of the entire input and the specific classification output, potentially overlooking how features contribute during the actual forward pass.

To address these challenges, this chapter introduces a forward pass retracing method, that emphasizes the role of complex input features in contributing to the final classification. This approach focuses on calculating the \emph{Contribution To Classification (\CTC\/)} of a feature by retracing its influence through the network during the forward pass. By doing so, one can differentiate between a feature's direct contribution and its relevance within the broader input context, providing a more precise and computationally efficient explanation of the importance of complex input features during classification.

The contributions of this chapter are as follows:

\begin{enumerate}
  \item Introduced the \emph{Contribution to Classification} (\CTC\/) method, an interpretability technique that accurately traces the influence of complex input features through a neural network without the computational overhead associated with high-dimensional Jacobians.

  \item Provided a formal definition of \emph{Contribution} within a neural network framework, detailing how to compute it inductively at each layer based on activations and contributions from preceding layers.

  \item Developed a comprehensive set of \emph{Propagation Rules} for distributing contributions through various types of layers in a neural network, ensuring accurate tracing from input to output layers.

  \item Defined specific propagation rules for different layer types, including contribution of the first layer, masking after each layer, dense layer ,convolutional layer, max pooling layer, batch normalisation layer, concatenation layer and average pooling layer rules.

  \item Demonstrated the necessity of carefully adjusting the scaling of learned parameters (e.g., biases) to prevent exponential growth or shrinkage of the contribution signal, which could lead to distorted or misleading interpretations, especially in deep networks.

  \item Addressed the challenge of preserving the contribution signal when dealing with learned parameters that are added or subtracted by proposing a refined method for scaling these parameters. This method adjusts the scaling matrix to keep contributions within the natural variability of the layer outputs, ensuring that the influence of learned parameters remains balanced.
  \item Expansion of the iNNvestigate library to incorporate the newly developed \CTC\/ method. By integrating these techniques into a widely used GitHub repository, this work enhances the accessibility of these methods. 
  \item The \CTC\/ method is empirically validated using the ILSVRC 2012 dataset~\cite{ILSVRC15} across several well-known convolutional neural network architectures, including VGG16~\cite{SimonyanZ14a}, VGG19~\cite{SimonyanZ14a}, ResNet50~\cite{he2015deep}, InceptionV3~\cite{szegedy2015rethinking} and DenseNet121~\cite{huang2018densely}. This extensive evaluation shows the practical utility of this technique, involving both qualitative and quantitative analyses. The results show that the \CTC\/ method significantly improve the interpretability of neural network decisions while preserving the faithfulness of the explanations.
\end{enumerate}

\section{Motivation}

The motivation for developing contribution to classification arises from the need for an interpretability method that accurately reflects the contribution of complex input features without the computational overhead associated with high-dimensional Jacobians. Through retracing the activations caused by specific features during the forward pass a detailed understanding of how these features influence the network's computations at each layer is gained, ultimately affecting the final classification. 

\begin{Definition}{Contribution}{con}
GGiven a trained neural network $\NN=\big(\Lambda,\passto,(f_k)_{k\in \Lambda}\big)$, where $(\Lambda,\passto)$ is a directed graph over a set of \emph{layers} $\Lambda=\{0,\dots, N\}$, with a single source $0\in \Lambda$, referred to as the input layer and single sink $N\in \Lambda$, referred to as the \emph{output layer} as defined in~\Cref{def:nn}, each $f_k:\bbR^{n_j}\to \bbR^{m_k}$ describes a vector transformation associated with each layer, with the dimensionality conditions that $m_k=\sum_{j\passto k} n_j$, for all $k=1,\dots N$. For a given input vector $\vec{x}\in \bbR^d$ and a given feature $F\subseteq \bbR^2$, let the \emph{contribution} at layer $k\in \Lambda$ be inductively found, by taking:
\begin{equation*}
       \vec{c}_k = f_i^\prime\big(\vec{a}_k,[\vec{a}_j]_{j\passto k},[\vec{c}_j]_{j\passto k}\big)
\end{equation*}
where $f_i^\prime$ denotes the function that modifies the behaviour of the network at layer $\Lambda_k$, depending on the type of layer $k=0,\dots, N$, so that it calculates the contribution a feature had during the forward pass. The value of the contribution $\vec{c}_k$ depends on the contributions from all preceding layers, as well as activation $\vec{a}_k$ computed during the forward pass.
\end{Definition}

\subsection{Differences between CTC of a Feature and Classification of a Feature}

The contribution of a complex feature during the forward pass differs from merely classifying the feature. It operates on the premise that each feature can have varying degrees of influence at different layers of the neural network. The \CTC\ method evaluates the feature's contribution at each layer during the forward pass. This enables a more granular understanding of how the feature influences the neural network's computations at various stages, but it is not the same as classifying the feature in isolation.

An example where the contribution as defined in~\Cref{def:con} and the inference output differ is when calculating the contribution of the complex input feature to a max pooling layer. If the neuron that was most active in a kernel during the forward pass has received the most contribution from the complex input feature, then the outputs of both \CTC\ and inference are the same. However, in the case where the neuron with the largest contribution is not the same, \CTC\ will select the same neuron that was selected during the forward pass and propagate the contribution of \emph{that} neuron forward, not the most active one. In this case, if the unchanged inference function is applied taking the complex input feature values as input, it can result in neurons being selected that were not pooled during the forward pass, thereby violating the faithfulness of the explanation. Figure~\ref{Fig:max_pooling} illustrates the difference between the forward-pass max pooling layer and the contribution max pooling layer.

\begin{figure}[ht!]
	\centering
	\includegraphics[width=1\linewidth]{Figures/max_pooling.pdf}
	\caption{A comparative illustration of the forward-pass max pooling layer (left) and the contribution max pooling layer (right). While the forward-pass selects the neuron with the highest value (0.5), the contribution layer evaluates the contribution of the feature to each neuron and selects the neuron with the highest activation during the forward pass, despite that not being the same as the highest contribution. This highlights the potential differences between the calculation of activation values and their contributions.}
	\label{Fig:max_pooling}
\end{figure} 

Another example where \CTC\ and inference differ is in the case of a ReLU layer. For the activation of a neuron to pass through a ReLU layer, it has to be more than zero. When applying the \CTC\ method, if the partial activation (i.e., contribution) given to a neuron is negative and that neuron during the forward pass had an activation greater than zero, this negative contribution is still allowed to propagate through a ReLU layer as it signifies the influence the complex input feature has on the output. Figure~\ref{Fig:Relu} illustrates the distinction between the forward-pass ReLU layer and the contribution ReLU layer.

\begin{figure}[ht!]
	\centering
	\includegraphics[width=1\linewidth]{Figures/ReLU.pdf}
	\caption{Comparison between the forward-pass ReLU layer (left) and the contribution ReLU layer (right). The forward-pass ReLU layer clips all negative activations to zero, while the contribution ReLU layer evaluates and propagates both positive and negative contributions, providing insights into the influence of complex input features on the neural network's output.}
	\label{Fig:Relu}
\end{figure} 

By comparing these two layers, one can observe the nuanced differences between a neuron's activation and its contribution. While activations are purely feed-forward and determined by the input values and weights, contributions reflect the broader context of how a specific feature influences the overall decision-making process of the neural network.

This chapter outlines a set of rules for propagating activation from specific input regions through the hidden layers of the network, culminating in a singular \CTC\ value. The computation of a feature's \CTC\ value is facilitated by introducing the \emph{Contribution To a Neuron (\CTN)} value. This value represents the activation a neuron receives as a result of the feature during the forward pass, with the \CTC\ value of a feature being equivalent to the CTN value of the classification output neuron. This propagation of activation, which accounts for the feature's contribution during the forward pass, is distinct from a mere classification of the feature. This distinction is crucial due to the non-linear layers such as MaxPooling, which could otherwise pool the contribution of neurons not selected during the forward pass, compromising the explanation's faithfulness.


There is an important differentiation between the \CTC\ value of a feature and its relevance. While current interpretability methods focus on the importance of a feature as part of the entire input considering the specific classification output (\ie feature relevance), the approach described in this chapter isolates the importance of the \textit{feature alone} during the forward pass (\ie feature's \CTC\/). Figure~\ref{fig:contribution} illustrates the distinction between a feature's relevance and its \CTC\ value using a specific input. When considering each pixel as a distinct feature, techniques like Layer-wise Relevance Propagation (\LRP\/) (see Chapter~\ref{sec:backprop}) assign significant importance values within the context of the entire input. However, for an individual pixel, the \CTC\ value is generally insignificant, as no single pixel notably impacts the overall classification. In contrast, when assessing an amalgamation of simple features (pixels), or a complex feature, the \CTC\ value effectively shows the contribution of the entire region to the classification through a heatmap. To enhance interpretability, complex features are presented in varying colours.


% The chapter posits that the \CTC\ value provides a more useful explanation of a feature's importance to the classification than its relevance. In convolutional neural networks, while edges detected from the input are relevant for decision-making, their contribution to the classification can vary significantly. Relevance-based approaches may overestimate the importance of certain input parts, as they may not actively contribute much to the classification. This is particularly evident in deep convolutional neural networks with extensive convolutional layers, where almost all edges are likely to be deemed important. Figure~\ref{fig:guitar_dog} contrasts relevance-propagation methods with this contribution approach, highlighting the difference in the importance assigned to various input parts.


\begin{figure}[h]
\centering
\includegraphics[width=\linewidth]{Figures/chimp_contributions.pdf}
	\caption{ Illustration of the difference between contribution to classification and relevance on a pixel level. The explanations were computed on VGG16~\cite{szegedy2016rethinking} and the relevance heatmap is of LRP~\cite{bach2015pixel}}
	\label{fig:contribution}
\end{figure}


% \begin{figure}[h]
% \centering
% \includegraphics[width=\linewidth]{Figures/tiger_contribution.pdf}
% 	\caption{ \LRP\ identifies most of the edges in the image as relevant when explaining the ``Electric Guitar'' class including the head of the dog. In contrast, \REVEAL\ attributes only 0.73\% relevance to the head of the dog and 53\% relevance to the actual guitar.}
% 	\label{fig:guitar_dog}
% \end{figure}
\section{Propagation Rules}

This section defines a set of rules which allow for the contribution to be distributed layer by layer starting from the input layer until the output layer is reached. These rules differ from just performing a forward-pass on a complex input feature as they compute the activation that is a direct result of the presence of the complex input feature during the forward pass, as opposed to computing the activations that this complex input feature has by itself. Given the presence of different layer types, different contribution propagation rules must be defined which respect and reflect the underlying structure of the network and functions that comprise it.


\subsection{Contribution of First Layer}
The contribution of a region at the input layer 0 $\in \Lambda$ is the same as the activation's of that region (i.e the pixel values of the three input channels), so the contribution $\vec{c}_0$ is defined as:
\begin{equation*}
    \vec{c}_0 = \vec{a}_0 \odot \vec{m},
\end{equation*}
where $\vec{m}\in \{0,1\}^{n_0}$ is a mask over the region propagated as found by the method described in Section~\ref{chap:clustering}, and $\odot$ denotes element-wise multiplication operation and $\vec{a}_0$ are the pixel values of the three input channels in the first layer. Performing the multiplication with the mask sets to zero all the activations of neurons that are not in the region of the feature that is being propagated. This allows only for the activations that are part of the complex input region to be propagated. 

\subsection{Masking After Each Layer}
Only neurons that were active during the inference step should have contributions distributed to them during the feature contribution propagation. Thereby after each layer's function is performed, a mask $\vec{m}_k\in \{0,1\}^{n_k}$ is extracted over the activations values in $\vec{a}_k$, as follows 
\begin{equation*}
    \label{eq:masking}
    m_k(i) = \begin{cases}
    1 & \mbox{if $a_k(i)\not=0$}\\
    0 & \mbox{otherwise}
    \end{cases} 
\end{equation*}
for all $i \in \{1,\dots, n_k\}$, so that every position that had a non-zero values during the forward pass after the function of the layer was computed has a value of one and otherwise has a value of zero. The contributions are then masked, by:
    \begin{equation*}
    \label{masking}
        \vec{c}_k = \vec{c}_k\odot \vec{m}_k,
    \end{equation*}
where $\vec{m_k}\in \{0,1\}^{n_k}$ is a mask over the activations values $\vec{a}_k$ in layer $\Lambda_k$. Note that neurons that were active during the classification but did not receive any contribution from the feature are still zero. An example of this rule being needed in the case of a ReLU layer, where no negative activations should be propagated, but one can have positive contribution attributed to a negative activation. This contribution should be set to zero as the activation is set to zero and not propagated forward. An example of that can be seen in Figure~\ref{Fig:Relu}.

The practice of distributing contributions only to neurons that were active during the inference step is not just a procedural detail, but a fundamental aspect of the feature contribution propagation process. Omitting this step could lead to inaccurate or misleading interpretations of a neuron's contribution to the final output. For instance, if contributions were assigned to inactive neurons, it would distort the understanding of how each feature influences the model's decision-making process.


\subsection{Dense Layer Rule}
The dense layer represented by a vector valued function $f_k: \bbR^{n_j}\to \bbR^{m_k}$ of the form $f_k(\vec{x})= W_k \vec{x} + \vec{b}_k$, for some matrix of weights $W_k \in \bbR^{n_j\times m_k}$ and some vector of biases $\vec{b}_k \in \bbR^{m_k}$. This function is often followed by an activation function, where the output of the the dense layer is often refereed to as the net value $\net$. The explanation of the function $f_k$ is broken down in the section below into its individual components. This approach aims to clearly illustrate the impact of the dense layer on the classification of a part of the input during the forward pass of the network.

\subsubsection{Weighting the contribution}
In the dense layer function, the first step involves the weighting of the activations that have reached the layer. Contribution refers to the amount of activation that is distributed from a specific region in the input space during the forward pass. Since weighting is a multiplicative operation, it proportionally scales each unit of activation relative to the weight. Thus, it is deemed sufficient for the contributions to be weighted in the same manner as the activations were during the forward pass, as detailed below:
\begin{equation}
    \net^- = W_k [\vec{a}_j]_{j\passto k}\quad\mbox{and}\quad
   \cnet^- = W_k[\vec{c}_j]_{j\passto k},
\label{eq:cnet}
\end{equation}
where the left hand-side equation shows the weighting of the activation $\net^-$ and the right hand-side one shows the weighting of the contribution $\cnet^-$.
\subsubsection{Adding the bias}
\label{section:adding_the_bias}
To calculate the net value $\net$ during the forward pass a learned bias term $\vec{b}_k$ of the layer is added to the weighted sum $\net^-$. 
The bias term holds a critical role in shaping the function learned by a neural network. During the backpropagation process not only does the bias influence the update of the weights, but the weight adjustments also affect the bias. This makes the bias an integral component of the neural network's learned function. Neglecting the bias during inference can lead to markedly different outputs. 
% A striking example of this is illustrated by an experiment shown in Figure~\ref{fig:bias_no_bias}, where the classification outcome for a guinea pig dramatically shifts from the most positive to the second most negative class simply due to the exclusion of biases. Moreover, the bias contributes to the model's ability to recognise patterns in data that don't centre around zero. It does this by inducing a learned shift in the activation function, either towards the positive or negative side. Omitting this shift can substantially alter the function of the neural network, underscoring the importance of the bias in its overall operation.

% \begin{figure}[ht!]
% 	\begin{center}
% 		\includegraphics[width=\linewidth]{Figures/guini_without_bias.pdf}
% 	\end{center}
% 	\caption{The figure shows an experiment of how the classification of an image can change completely when the learned bias in the network is removed. This experiment was done on VGG16~\ref{SimonyanZ14a}. In this example it changes from the correctly classified guinea pig to a upright piano, which isn't present in the image. Guinea pig also becomes the second most negative classification, so the bias' effect on such image is difficult to be overstated.}
% 	\label{fig:bias_no_bias}
% \end{figure} 
% \noindent

When advancing the contribution forward, a straightforward approach is to add the bias as it is done in the forward pass. However, this method needs refinement, as the bias is calibrated for the contribution from the entire input space, not just a single feature. This mismatch can lead to two issues:
\begin{enumerate}
    \item If the contribution from a specific input region to a neuron is just a small fraction of the total input's contribution, adding the original bias can disproportionately amplify its effect on the output.
    \item Conversely, if the contribution from a single neuron is much larger than it was during the forward pass (perhaps due to the absence of input regions that are irrelevant or negatively contributing), the original bias might not adequately adjust the mean as it did during the forward pass.
\end{enumerate}

\begin{figure}[ht!]
	\begin{center}
		\includegraphics[width=0.8\linewidth]{Figures/AandB.png}
	\end{center}
	\caption{(A) illustration of the effect of the bias when the contribution of an input region with respect to a particular neuron is a very small fraction of the the entire input's contribution. (B) illustration of the converse effect, where the contribution of a single neuron might be far greater than the entire input' contribution.}
	\label{fig:Neurons_with_bias}
\end{figure} 

The bias is originally tuned to work within a specific range of values. In the first scenario, adding the same bias term can excessively influence the output, as illustrated in Figure~\ref{fig:Neurons_with_bias}~A, where the relevance is a fraction of the activation. In the second scenario, illustrated in Figure~\ref{fig:Neurons_with_bias}~B, the relevance exceeds the activation. Here, adding the bias of 0.5 results in an output minimally affected by the bias, contrasting with its more significant role during the forward pass. These examples highlight the need for a nuanced approach to incorporating the bias when propagating contributions.

Given that the \CTC\/ method proposed in this chapter involves propagating only a portion of the input (\ie the complex input feature) through the network, maintaining the integrity and the influence of this small subset of activations, without it getting diluted or lost presents a significant challenge. This general way of adjustment can be formulated as:
\begin{equation}
   \vec{c}_k  = \cnet^- +  P_{k}{b}_k.
\label{eq:ck} 
\end{equation} 

where the proportion of the bias ${b}_k$ to be added to the scaled contributions $\cnet^-$ is defined by a matrix $P_{k}$ that when multiplied with the original bias $\vec{b}_k$ yields an adjusted bias matrix~\footnote{The exact way $P_{k}$ is defined can be seen in Section~\ref{sec:scale}, which presents a discussion of how the relevance signal can be preserved when dealing with learned network parameters.}

\subsection{Convolutional Layer Rules}
Convolutional layers comprise of multiple operations stacked after one another. Suppose that $k\in \Lambda$ is an convolutional layer with $f_k$ of the form $f_k(\vec{x})=\act(W_k\ast\vec{x} + \vec{b}_k)$, for some matrix kernel $W_k:\bbR^{d_k\times d_k'}$, some vector of biases $\vec{b}_k\in \bbR^{m_k}$, and some (component-wise) activation function $\act:\bbR^{m_k}\to \bbR^{m_k}$, such as a ReLU, Sigmoid, Hyperbolic Tangent or any activation function that may be used in a given neural network.

\subsubsection{Weighting the contribution}
First the convolutional layer divides the input data into smaller regions called \emph{local receptive fields}. Each receptive field corresponds to one element or a small neighbourhood of elements in the input data. Next the convolution operation is performed by sliding the filters over the input data, applying element-wise multiplication between the filter and the receptive field, and summing up the results. This produces a single value, which represents the activation of a neuron in this resulting layer. This process is repeated for each location from the input to the layer, which helps in capturing the same features regardless of their location (see Section~\ref{sec:conv}).

The sum of each locally receptive field multiplication with a filter is a smaller (local for each locally receptive field) version of taking the weighted sum in a dense layer. Similarly as in the dense layer, weighting is a multiplicative operation, it scales each unit of activation in a locally receptive field proportionally to the weight in the filter. Therefore it is sufficient for the contributions to be
weighted in the same way as activations were during the forward pass. As detailed
\begin{equation}
    \net^- = W_k\ast[\vec{a}_j]_{j\passto k}\quad\mbox{and}\quad
   \cnet^- = W_k\ast[\vec{c}_j]_{j\passto k},
\label{eq:conv_cnet}   
\end{equation}
where the left hand-side equation shows the weighting of the activation $\net^-$ and the right hand-side one shows the weighting of the contribution $\cnet^-$.
\subsubsection{Adding the Bias}

The bias vector \(\vec{b}_k \in \bbR^{m_k}\) is added to the output of the convolution operation. The process enhances the layer’s ability to learn patterns in data and adjust its activation function appropriately. Adding bias in a convolutional layer is crucial as it allows the network to adjust the output of the layer independently of its inputs. This is particularly useful in cases where the data should not centre around zero, as it helps in shifting the activation function. This general way of adjusting the biases in convolutional layers is expressed as:
\begin{equation}
   \vec{c}_k =\cnet^- +  P_{k}{b}_k, 
\end{equation} 
which is the same in Equation~\ref{eq:ck}.
\subsubsection{Activation}
The convolutional layer often has an activation function as its last operation. If the function is ReLU it is bypassed by the contribution rules. This allows for negative relevances to pass through and accurately show that a complex input feature had a negative relevance on a neuron that had an overall positive activation. Note that the masking after each layers ensures that relevances assigned to non-active neurons during the forward pass are assigned zero.  


\subsection{Max Pooling Layer Rule}

The process of pooling in neural networks can be described as a transformation via a vector-valued function, denoted as $f_k:\bbR^{n_j}\to\bbR^{m_k}$, where the dimensionality $n_j$ of the input layer is reduced to $m_k$ in the subsequent layer, with $n_j \geq m_k$ (see Section~\ref{section:max}). The core objective of the contribution propagation during the new pooling function is to ascertain the significance of each complex input feature in relation to the neurons that exhibit maximum values during the network's forward pass. This evaluation of importance is crucial in understanding how each feature contributes to the model's decision-making process. A key aspect to consider is that the maximum value within a given patch identified during the classification phase might not align with the value deemed most significant during the contribution propagation phase. Hence, it is not feasible to simply apply max pooling on the relevance vector, as this would overlook the dynamic nature of the most relevant features across different phases of the network's operation.

\subsubsection{Finding the Maximum Elements During Inference}
To refine the pooling process for contributions, a method is introduced that more precisely mirrors the actions taken during the forward pass. This method involves the creation of a binary mask, $\vec{m} \in \{0,1\}^{n_j}$, which is applied over the inputs to layer \(k\). In this mask, elements corresponding to inputs that were not pooled during the forward pass are set to zero, while those that were pooled are set to one. This approach allows for an exact replication of the pooling behaviour observed in the forward pass. The mask \(\vec{m}\) is created by taking the gradient of the function \(\vec{f}_k\) with respect to the activations from the preceding layer(s), denoted as \([\vec{a}_j]_{j\passto k}\). This is represented as:
\begin{equation}
    \vec{m_j} = \nabla f_k\big([\vec{a}_j]_{j\passto k}\big).
\end{equation}
where, the subscript \(j\) refers to the input layers that feed into layer \(k\), with \(j \passto k\) indicating the connection from layer \(j\) to \(k\). The mask \(\vec{m_j}\) filters out the non-maximum elements, as all neurons that was not pooled during the inference step will have a gradient of zero. When this mask is applied to the contributions from layer \(j\), it results in a new vector, \(\vec{c_j}^\prime\), which retains only the contributions of the neurons that were actually pooled. This is expressed as:
\begin{equation}
    \vec{c_j}^\prime = [\vec{c}_{j}]_{j\passto k} \odot \vec{m_j}.
\end{equation}
The resulting vector, \(\vec{c_j}^\prime\), contains contribution values for the neurons that were pooled, with zeros in all other positions. 

\subsubsection{Pooling the Relevance Values at the Maximally Activated Neurons During Inference}

When assessing the new contribution vector \(\vec{c_j}^\prime\), if all contributions are positive, determining the most significant contributions becomes straightforward. This involves performing a max pooling operation on \(\vec{c_j}^\prime\), wherein the largest values indicate the most significant contributions of the pooled neurons. However, it is important to note that a region's contribution to a neuron can be negative, even if that neuron was maximally active when considering the entire input. To pool the relevances no matter whether they are positive or negative one can perform the max pooling operation on the absolute values of the $\vec{c_j}^\prime$ vector 
\begin{equation}
    \label{pool}
    \vec{c_{k}}^- = f_{k}(||\vec{c_j}^\prime||)
\end{equation}
and then add the sign of the contribution to the resulting vector by:
\begin{equation}
    \label{reasign}
    \vec{c_{k}} = \begin{cases}
    c_{k}^-(i) & \mbox{if $f_{k}(\vec{c_j}^\prime)\not=0$}\\
    -1\times c_{k}^-(i)& \mbox{otherwise}.
    \end{cases}
\end{equation}

The sign re-assignment in Equation~\ref{reasign} is attained by pooling the non-absolute contributions after masking $\vec{c_j}^\prime$ (\ie taking $f_{k}(\vec{c_j}^\prime)$). The output of $f_{k}(\vec{c_j}^\prime)$ will be non-zero in the cases where the contribution at the position pooled is positive and zero in all other positions. This allows to establish the positions where the contribution is negative and the sign needs to be reassigned, as those positions will have a non-zero value after preforming $f_{k}(||\vec{c_j}^\prime||)$, but a zero value when performing $f_{k}(\vec{c_j}^\prime)$. Therefore a minus sign is assigned for $c_{k}(i)$ whenever the pooled value is zero.


\subsection{Batch Normalisation Layer Rule}
A batch normalisation layer $k\in \Lambda$ during the forward pass can be expressed as a vector-valued function $f_k:\bbR^{n_k}\to \bbR^{m_k}$ ($n_k=m_k$) given~by:
\begin{equation*}
    f_k(\vec{x}) = \gamma \frac{(\vec{x}-\mu)}{\sqrt{\sigma_x^2 +\epsilon}} + \beta ,
\end{equation*}
where $\gamma,\beta \in \mathbb{R}$ are learned parameters that scale and shift the normalised input respectively. This layer adjusts each feature dimension of the input \(\vec{x}\) based on the learned mean \(\mu\) and standard deviation \(\sigma\) from the training process, incorporating a small constant \(\epsilon > 0\) to maintain numerical stability.

As batch normalisation does not alter the contribution of features but just shifts and scales them, this layer can safely be bypassed. The contribution of layer $\Lambda_k$ is therefore defined as:
\begin{equation*}
    \vec{c}_k = \vec{c}_j
\end{equation*}

% Focusing on first normalising the activations to have an approximate mean of zero and a standard deviation of one, the step can be trivially rewritten as:
% \begin{equation*}
%   \vec{a_k}^- = \frac{1}{\sqrt{\sigma_j^2 +\epsilon}}(\vec{a_j}-\mu)
% \end{equation*}
% making it easy to notice that the normalisation is equivalent to shifting the activations by the learned mean $\mu$ and scaling the resulting value by the learned standard deviation $\sigma$. The shifting of the activation with respect to the learned mean is the same as adding the bias in a dense layer, where the activation in each feature dimension are also shifted by a constant learned value (see Section~\ref{section:dense}). Given that the mean $\mu$ is learned as an approximate mean of the training examples seen, it is tuned for values within a certain distribution and magnitude. To preserve the signal of the complex input feature distributed (which may be just a small portion of the entire input) it is important for this learned value to be scaled. If it is not it can cause problems, as it can have either a disproportionately high or low effect on the output. Therefore when shifting the activations to be centred roughly around zero (note that the shift is not precise as the value is learned and is not the exact mean of this input) one needs to weight the mean with respect to the how much contribution of the feature has been propagated in comparison to the activation, as follows:
% \begin{equation*}
%   \vec{c_j}^\prime= \vec{c_j}- P_j\mu
% \end{equation*}
% The scaling matrix $P_j$ is defined in a similar way to the matrix that scales the bias in dense layers. Methods for defining the matrix $P_j$ for weighting the added constant are described in Section~\ref{sec:scale}. Note that if one takes the new mean of the relevances and normalises them this mean value, the signal of the relevance will be lost. This is the case as no matter how big or small the feature is, if a separate mean for each is calculated, it will result in all features being centred around zero and therefore losing all the signal they carry. Scaling the mean allows for activations that were positive and now have a smaller relevance than the original activation values to have a positive relevance distributed forward. 


% The resulting vector must be divided by the standard deviation. Normally division scales each unit of activation proportionally and it should be sufficient for the contributions to be weighted in the same way as activations were during the forward pass. However, the standard deviation is also a learned parameter during training and is a result of the typical deviation from the mean. As the mean and the standard deviation are learned together, the standard deviation is a direct result from it. Therefore, for the purpose of preserving the contribution signal through the network a new standard deviation needs to be used that is equivalent to the deviation from the new mean. The new standard deviation uses the following new variance
% \begin{equation*}
% \sigma_j^{\prime2} = \sigma_j^{2} \frac{\frac{1}{M}\sum_{i=1}^{M}(c_i - P_j\mu)^2}{\frac{1}{N}\sum_{i=1}^{N}(x_i - \mu)^2},
% \end{equation*}
% where the learned variance is scaled by the ratio between the deviation of the activations to their mean and the deviation of the relevances to the new mean. 

% This new variance is used to re-scale the already centred contribution vector $\vec{c_j}^-$, as follows:
% \begin{equation*}
%   \vec{c_k}^-= \frac{\vec{c_j}^\prime}{\sqrt{\sigma^{\prime2} +\epsilon}}
% \end{equation*}

% The resulting vector $\vec{c_k}^-$ is scaled and shifted. Now it needs to be scaled by $\gamma$ and get the new mean $\beta$:
% \begin{equation}
%     \vec{c_k} = \gamma \vec{c_k}^- + \beta.
% \end{equation}


\subsection{Concatenation \& Average Pooling Layers}

There are two types of layer to be considered: concatenation layers and average pooling layers. Both the concatenation layer and average pooling layer perform functions that do not influence by the input. Therefore the functions for distributing the contribution forward is exactly the same as during the forward pass, where the activation is distributed. Each concatenations layer $k\in \Lambda$ is given by a function, where $a_k = f_k\big([\vec{a}_j]_{j\passto k}\big) = [\vec{a}_j]_{j\passto k}$. 

Therefore, it is sufficient to take the contribution of layer $k$ to be the concatenation of the contributions of the same collection of layers $j\in \Lambda$ such that $j\passto k$: 
\begin{equation*}
    \vec{c}_k = f_k\big([\vec{c}_j]_{j\passto k}\big) = [\vec{c}_j]_{j\passto k}.
\end{equation*}

The average pooling operator involves calculating the average for each patch of the feature map. This operation is pivotal in reducing the spatial dimensions of the input feature map, effectively summarising the information contained in each patch. Given an average pooling layer $l$, the operation can be defined as follows:

\begin{equation*}
    \vec{c}_k = \frac{1}{|\mathcal{P}_l|} \sum_{a \in \mathcal{P}_l} a(i),
\end{equation*}

where $\mathcal{P}_l$ represents the set of pixels in each patch of the feature map that the pooling layer $\Lambda_j$ processes, and $a(i)$ is the activation of a neuron $i$ in the patch $\mathcal{P}_l$.


\section{Preserving The Contribution Signal Through The Network by Scaling Learned Parameters}
\label{sec:scale}

The focus of traditional neural network interpretability methods is on evaluating the entire input data, given the complete classification output to determine the importance of different input components. These methods use the entire activation signal that was present during the forward pass. This means that values that were learned by the network are still receiving data in the same distribution. However, the approach described in this chapter diverges from this norm. It involves propagating only a portion of the input through the network. This selective propagation presents a unique challenge: maintaining the integrity and the influence of a small subset of activations as they pass through the various layers of the network, without the relevance getting diluted or lost, thereby accurately reflecting the significance in the overall decision-making process of the model. When the operation is multiplicative, the signal is preserved, as the relevances get scaled proportionately to how activations were scaled during the forward pass. The big challenge is introduced by learned parameters that are added or subtracted to the activation during the forward pass. Adding such parameters directly to the relevances leads to a degraded relevance signal. This can be seen in dense layers and convolutional layers when the bias is added.


What all of these learned parameters have in common is that they represent a shift in either the positive or the negative direction. This means that the scaling matrix $P_{k}$ has to be positive. Consequently, when adding or subtracting the learned parameter to the contribution, the scaled parameter should maintain its directional influence --- a positive bias or mean should continue to exert a positive shift, and similarly, a negative bias or mean should remain negative. In the following subsections, several methods for determining the values of the adjustment matrix are examined. The process might appear straightforward, but even minor modifications in this calculation can lead to significant changes in the network's output, especially in larger and more complex architectures due to repeated application of the same function in sequential layers (see Section~\ref{sec:parallels}).

\subsection{Naïve Weighting of The Learned Parameters}
\label{naive}

A naïve approach to define $P_{k}$ is as an element-wise absolute ratio between the relevances and the activations. As defined by
\begin{equation*}
    P_{k} = \dfrac{\cnet^-}{\net^-} ,
\end{equation*}
where in the case of dense layers this is defined as a ratio between the elements of the weighted sum of the contribution matrix $\cnet^- =\vec{W_k(j)}\,[c_j(i)]_{j\passto k}$ and the weighted sum of the activation matrix $\net^- = \vec{W_k(j)}\,[a_j(i)]_{j\passto k}$ as in Equation~\ref{eq:cnet}. In the case of convolutional layers it represents the ratio between the contribution element after the convolution operation $\cnet^- =\vec{W_k(j)}\ast\,[c_j(i)]_{j\passto k}$
and the activation elements after the convolution operation $\net^- = \vec{W_k(j)} \ast \,[a_j(i)]_{j\passto k}$ as in Equation~\ref{eq:conv_cnet}. 

Using this definition of the ratio has the same effect as calculating the bias per unit of activation and then multiplying that by the amount of contributions in the region.

\begin{equation*}
   c_k = W_k \vec{c_j} + \left| \dfrac{\vec{b}_k}{W_k\,[\vec{a}_j]_{j\passto k}}\right| W_k\,[\vec{c}_j]_{j\passto k}, \quad \mbox{and}\quad c_k = W_k \ast \vec{c_j} + \left| \dfrac{\vec{b}_k}{W_k \ast \,[\vec{a}_j]_{j\passto k}}\right| W_k\ast \,[\vec{c}_j]_{j\passto k},
\end{equation*}
\newline
\newline
This method of weighting the bias works for a single layer in isolation, but whenever the activation and the relevance are of considerably different magnitudes, the bias will become very big or very small. Here big and small refer to the orders of magnitude of the value. This may be the case even when the relevances are in the same distribution as the activations. For example, a neuron that has an activation of 0.05 and a relevance of 0.5, will lead to a ratio of 10, which after multiplied with the bias and added to the contribution, will lead to a very big contribution that is out of distribution. This contribution will then be used in following layers and for calculating a ratio that will push the next relevances to become even further out of distribution. As layers that have learned parameters will be repeated numerous times within a single neural network, the order of magnitude between the activation and the relevance amplifies and grows exponentially. This pushes the bias to be either extremely small or extremely large, particularly in deeper networks with multiple sequential layers (see Figure~\ref{fig:exponential}). Given that the contribution of a subsequent layer is a function of the bias in the previous layer, an exponentially growing (or shrinking) bias term results in an exponential change in the value of the relevances too. 

\begin{figure}[ht!]
	\begin{center}
		\includegraphics[width=\linewidth]{Figures/Logarithms_of_the_ratios.pdf}
	\end{center}
	\caption{The logarithmic scale comparison of maximum (left) and minimum (right) ratio values in VGG16 neural network layers. The left graph illustrates how the logarithm of the ratio can become exceedingly large, indicating a very high bias when the activation is significantly smaller than the relevance. Conversely, the right graph shows instances where the logarithm of the ratio is highly negative, reflecting an extremely low bias when the activation is much larger than the relevance. These extremes demonstrate the instability of the ratio in the network.}
	\label{fig:exponential}
\end{figure} 

\subsection{Drawing Parallels to Exploding and Vanishing Gradients in Backpropagation}
\label{sec:parallels}

The challenges in preserving the contribution signal through a neural network, especially when managing learned parameters, bear striking parallels to the phenomena of exploding and vanishing gradients in backpropagation~\cite{DBLP:journals/tnn/BengioSF94}. These similarities are crucial for understanding and addressing the stability and effectiveness in deep neural networks.

Similar to the exploding and vanishing gradients problem, the relevance signal in the network can either exponentially increase or decrease as it is processed through multiple layers. This is a direct parallel to how gradients can grow or shrink during backpropagation, leading to instability in the network's learning dynamics. Just as deeper networks are more susceptible to the compounding effects of gradients, they are equally prone to the distortion of relevance signals due to the repeated application of biases and scaling parameters across layers. This necessitates a careful approach to manage these effects in larger architectures. The issues of exploding gradients can cause numerical instability and hinder convergence, while vanishing gradients can impede learning. Analogously, an uncontrolled relevance signal can distort the interpretability and functionality of the network, misrepresenting the importance of inputs or activations. Both scenarios underscore the importance of considered parameter management. Techniques like normalisation are vital in backpropagation, just as sophisticated methods for calculating and adjusting the scaling matrix \( P_{k} \) are critical for maintaining a stable and accurate relevance signal. The next subsections demonstrate how a version of normalisation can be applied in the context of calculating the scaling matrix \(P_{k}\).

\subsection{Weighting of the bias within distribution}

A more sophisticated approach would be to scale the orders of magnitude of the ratio into an acceptable standard deviation. However as the learned parameters do not have a distribution (\eg there is one bias per channel of activations), it is difficult to find the acceptable range.  For ease of comprehension, the term magnitude henceforth refers to the order of magnitude of a value with respect to its mean. Scaling the magnitudes of the ratios to an acceptable standard deviation limits the exponential magnification or shrinking of the contribution when functions that have learned parameters are called repeatedly throughout sequential layers of the neural network. 

To answer the question of what is acceptable standard deviation of the distribution of the magnitude of the ratio, it helps to think of the magnitude of the output values. The magnitude of the scaled learned parameter can deviate from its mean to the same extend that the output's magnitude can deviate from its mean. The deviation of the bias by this limited amount will not lead to any noticeable magnitude change after the scaled learned parameter is added to the relevances. The activation output magnitude can therefore be used as the maximum standard deviation allowed for the learned parameter's magnitude standard deviation. 

The learned parameter is scaled through the scaling matrix \(P_{k}\), so one needs to find an equivalent relationship between the scaling matrix \(P_{k}\) and another ratio $Q_{k}$ as the relationship between the learned parameter's magnitude is to the activation output magnitude. The ratio $Q_{k}$ is therefore defined as the ratio between the activation output and its mean:
\begin{equation*}
    Q_{k} = \dfrac{\vec{a_k}}{\mu_{a_k}},
\end{equation*}
where $\vec{a_k}$ is the activation output vector (\ie after the learned parameter is added or subtracted) and $\mu_{a_k}$ is the mean of the $\vec{a_k}$ activation output vector. This ratio captures the variability of the activation's output. When \( P_{k} \) is multiplied by the learned parameters, the resulting values should have a magnitude deviation that is no greater than when the ratio $Q_{k}$ is multiplied with the mean of the activation output. Therefore, \( Q_{k} \) serves as a benchmark for determining how much the learned parameter, once scaled by \( P_{k} \), can deviate in magnitude. This constraint ensures that the scaling process does not introduce disproportionate amplification or reduction of the learned parameters, which could otherwise lead to an unfaithful explanation. The key idea here is to ensure that the magnitude deviation of the learned parameters, after being scaled, does not exceed the magnitude deviation of the layer's outputs. This is crucial because it aligns the influence of the learned parameters with the natural scale of the network's activations, thus maintaining a balanced and stable operation of the network.


It is important to note that scaling the standard deviation of the magnitudes of the \(P_{k}\) matrix to match the standard deviation of the magnitudes of the $Q_{k}$ matrix is not the same as transforming the \(P_{k}\) distribution to match the $Q_{k}$, as that would limit the magnitude of \(P_{k}\) from becoming exponentially large, but would not limit the magnitude of \(P_{k}\) from becoming exponentially small. 


To keep the orders of magnitude of the \(P_{k}\) matrix and $Q_{k}$ within the same distribution, one first needs to get the magnitudes of both matrices. This is achieved by taking the natural logarithm of the activation and the contribution values, with all zero values being discarded and set to zero (given that the magnitude of zero is undefined). 
\begin{equation}
\label{eq:log}
    \lo (i) = \begin{cases}
    \ln |Q_k(i)| & \mbox{if $Q_k(i)\not=0$}\\
    0 & \mbox{otherwise}
    \end{cases}
    \quad\mbox{and}\quad 
    \clo(i) =  \begin{cases}
    \ln |P_k(i)| & \mbox{if $P_k(i)\not=0$}\\
    0 & \mbox{otherwise}
    \end{cases} 
\end{equation}
\newline
The logarithm vectors will have a value higher than 0 if the number is greater than 1, and it will have a value less than 0 if the number is between 0 and 1. The more positive the logarithm of ratio is, the larger the ratio, similarly the more negative the logarithm is the smaller the ratio.

Next, the standard deviation of both the magnitude of the scaling matrix \(P_{k}\) and the standard deviation of the magnitude of \(Q_{k}\) are calculated. These standard deviations provide a quantitative measure of how spread out the values are in each matrix. The standard deviation of the magnitudes of \(Q_{k}\), denoted as \(\sigma_{Q_k}\), reflects the variability in the activation outputs relative to their mean. Similarly, the standard deviation of the magnitudes of \(P_{k}\), denoted as \(\sigma_{P_k}\), reflects the possible variability in the scaled learned parameter relative to its mean.

\begin{equation}
\sigma_{Q_k} = \sqrt{\frac{1}{N} \sum_{i=1}^{N} (\lo(i) - \mu_{\lo})^2}
\quad \text{and} \quad
\sigma_{P_k} = \sqrt{\frac{1}{M} \sum_{i=1}^{N} (\clo(i) - \mu_{\clo})^2},
\end{equation}

where \(\mu_{\lo}\) and \(\mu_{\clo}\) represent the mean of the logarithmic magnitudes of \(Q_{k}\) and \(P_{k}\) respectively, and \(N\) and \(M\) denote the number of non-zero elements in each matrix $Q_k$ and $P_k$ respectively. These standard deviations are crucial for the next step: ensuring that the scaled learned parameters maintain a magnitude deviation within an acceptable range.

The goal is to scale \(P_{k}\) in such a way that its standard deviation of magnitudes, \(\sigma_{P_k}\), does not exceed the standard deviation of magnitudes of \(Q_{k}\), \(\sigma_{Q_k}\). This constraint is vital because it ensures that the contribution of the scaled learned parameters remains within the natural variability of the layer’s outputs, as represented by \(Q_{k}\). To achieve this, if  \(\sigma_{P_k}\) is bigger than  \(\sigma_{Q_k}\) an equivalent to batch normalisation function is performed. This involves normalising the magnitudes \( \clo(i) \) of the scaling matrix \( P_{k} \), then scaling them to match the magnitudes \( \lo(i) \) of \( Q_{k} \), followed by a shift back to the original mean \( \mu_{\clo} \) of \( \clo(i) \). This can be broken down into the following steps:
\begin{enumerate}
    \item First, \( \clo(i) \) is normalised to zero mean and a unit variance:
\begin{equation*}
    \clo'(i) = \frac{\clo(i) - \mu_{\clo}}{\sigma_{P_k}}.
\end{equation*}
    \item Then, the normalised \( \clo'(i) \) is scaled to have the same standard deviation as \( \lo(i) \), after which it is shifted back to the original mean \( \mu_{\clo} \), by:
\begin{equation*}
    \clo''(i) = (\clo'(i) \cdot \sigma_{Q_k}) + \mu_{\clo}.
\end{equation*}
In this way, the adjusted \(\clo''(i)\) now has the same standard deviation as \(\lo(i)\) and is centred around its original mean $\mu_{\clo}$. Centring around the original mean of $\mu_{\clo}$ is very important as it preserves the difference between different complex input features.
    \item Finally, the adjusted magnitude matrix \(\clo''(i)\) can now be scaled back to the new scaling matrix \(P_{k}^\prime\), as such:
\begin{equation*}
    P_{k}^\prime(i) = e^{\clo''(i)}
\end{equation*}
This exponential transformation converts the logarithmic magnitudes back to their original scale. The scaling matrix \(P_{k}^\prime\) now effectively embodies the adjusted scaling factors, ensuring that the influence of the learned parameters is in line with the network's natural output variability.
\end{enumerate}

This methodology achieves two primary goals. Firstly, it adjusts the scaling matrix \(P_{k}^\prime\) to preserve the natural range of the activation outputs. This balance ensures that the learned parameters significantly influence the explanation without overwhelming it. Secondly, by centring around the original mean \(\mu_{\clo}\), the approach retains the distinctions among complex input features, which is crucial for the network's interpretive accuracy, particularly in cases where minor differences in inputs are significant.

Illustrated in Figure~\ref{fig:scaled}, the logarithmic scale of ratios after adjusting \(P_{k}^\prime\) shows that the natural layer output variability of the network is maintained. This result is consistent across the same mask, network, and example. Notably, the signal's general shape remains similar to that in Figure~\ref{fig:exponential} where ratios are not adjusted, yet the scale stays within the layer’s output variability. In this scenario, the maximum outlier has a logarithm below 12 with the adjusted \(P_{k}^\prime\), as opposed to over 25,000 when unbounded. Similarly, the minimum log is -22 with adjustment, in contrast to less than -1600 when unbounded.

\begin{figure}[ht!]
	\begin{center}
		\includegraphics[width=\linewidth]{Figures/Logarithms_of_the_ratios_scaled.pdf}
	\end{center}
	\caption{The logarithmic scale comparison of maximum and minimum ratio values in VGG16 neural network layers after \(P_{k}^\prime\) adjustment. While maintaining the signal, it mirrors the shape of the unbounded signal of Figure~\ref{fig:exponential}  but on a significantly smaller scale.}
	\label{fig:scaled}
\end{figure}

Thus, this approach enables a more measured and balanced integration of learned parameters into the network. It ensures these parameters are scaled effectively to keep the network interpretable. The incorporation of \(P_{k}^\prime\) is vital for ensuring that these scaled parameters contribute meaningfully and precisely to the model's decision-making process, avoiding distortions from disproportionate scaling. This method is especially beneficial in deep neural networks, where layer-wise parameter interactions can greatly affect both performance and interpretability.

The rule described in this section for scaling the learned parameters is a result of taking a ratio between the relevances and the activations. In the cases where either the relevances or the activations are very small, round-off error can occur due to the inherent limitations of computers in representing decimal numbers. This rounding can lead to small inaccuracies in calculations. These errors accumulate over successive computational steps (similar to how the the ratio can become exponentially small or big as discussed in Chapter~\ref{sec:parallels}), sometimes significantly impacting the final result. To remedy this in the implementation of forward pass retracing the activation and relevance are quantised before the ratio is taken so that such errors are minimised. The quantised activation and relevance are only used for the computation of the ratio and therefore this does not effect the overall contribution of features and the methods accuracy.


\section{Results}
\label{sec:results}

This section presents the results VGG16~\cite{SimonyanZ14a}, VGG19~\cite{SimonyanZ14a}, ResNet50~\cite{he2015deep}, DenseNet121~\cite{huang2018densely} and InceptionV3~\cite{szegedy2015rethinking} and the evaluation is based on the widely recognised ImageNet Large Scale Visual Recognition Challenge 2012 (ILSVRC 2012) dataset~\cite{ILSVRC15}, which has been used in training these models. To facilitate a standardised and consistent analysis, the \CTC\/ method described in this chapter is implemented as part of the iNNvestigate framework~\cite{inn}, which provides a common interface and out-of-the-box implementation for many analysis methods. This choice not only streamlines the evaluation process but also allows for a comparative analysis with the leading interpretability methods in the field.


To test if the \CTC\/ method provides an interpretable explanations with high fidelity, the explanations are evaluated both qualitatively and quantitatively. In line with standard practices~\cite{SundararajanTY17, ShrikumarGK17, SelvarajuCDVPB20, SmilkovTKVW17, AnconaCO018, kindermans2017learning} the qualitative analysis compares the \CTC\ method's contribution to classification against established attribution methods, including Input$\times$Gradient~\cite{SimonyanVZ13}, DeconvNet~\cite{ZeilerKTF10}, Guided BackProp~\cite{SpringenbergDBR14}, and LRP-$\alpha_1\beta_0$~\cite{bach2015pixel}. Arguably, showing the single value of \CTC\ makes the comparison between different features and models easier (corroborating results reported in \cite{Ribeiro0G16}). The methodology through which the method was evaluated involved applying the \CTC\ method and the established attribution methods to a diverse set of input images processed by various neural network architectures~\cite{SimonyanZ14a, SimonyanZ14a, he2015deep, huang2018densely, szegedy2015rethinking} and compare the visual explanations generated by each method for both correctly and incorrectly classified examples.

The quantitative analysis is essential to ascertain the relative standing of the \CTC\ method in the context of existing literature, particularly focusing on key interpretability properties such as input invariance and sensitivity, as detailed in Section~\ref{lit:discussion}. The evaluation of these interpretability properties is done by employing fidelity metrics to assess the accuracy of the explainability model. This ensures that the resulting method is not only more interpretable due to the reduced information shown to the user, but also retained the faithfulness of state of the art models.


\subsection{Qualitative Results}
\label{sec:qualitative}

Unique to the approach of selecting complex input features is the visualisation technique employed. It diverges from traditional heatmaps that use seismic colour patterns to represent relevance. Instead, the explanations are assigned a distinct colour to each complex feature's pixels. These colours are then reflected in a colour-coded legend adjacent to the heatmap, clearly illustrating the \CTC\ values and offering a more intuitive understanding of the model's decision-making process.

As the \CTC\ value is assigned to a complex input feature rather than on a pixel level like in other relevance methods, different ways by which the complex input feature is selected can lead to different results. This chapter selecting the top features by size rather thaN selecting the top features by the amount of relevance they have received (as described in Chapter~\ref{chap:clustering}). Selecting features by size does not rely on any relevance distribution method to select it is features, making it a more independent evaluation and a fair comparison to other methods in the literature.

\subsection{CTC Evaluated Against Different Post-hoc Explanation Methods}

\subsubsection{Correctly Classified Examples}
\label{sec:correct}

Figure~\ref{Fig:vgg16_correct} provides five examples classified on VGG16~\cite{SimonyanZ14a} and compares the explanations provided by \CTC\ with Input$\times$Gradient~\cite{SimonyanVZ13}, Integrated Gradients~\cite{SundararajanTY17}, DeconvNet~\cite{ZeilerKTF10}, and LRP-$\alpha_1\beta_0$~\cite{bach2015pixel}. Each column presents the visualized results for different explainability methods, while each row corresponds to analyses for a specific input sample. To the right of the \CTC\ method, numerical values indicate the contribution of the most prominent objects in the input.

\begin{figure}[ht!]
	\begin{center}
		\includegraphics[width=0.95\linewidth]{Figures/vgg16_correct.pdf}
	\end{center}
	\caption{Comparison of visual explanations for correctly classified examples using different explainability methods. The \CTC\ method stands out by providing clear, segmented visualizations of the most relevant features for the model's predictions, along with numerical values that quantify the importance of individual features.}
	\label{Fig:vgg16_correct}
\end{figure}

The primary advantage of the \CTC\ method lies in its ability to assign a clear, singular contribution value to the most significant features. Unlike traditional explainability methods, which often produce noisy or diffuse visual explanations, \CTC\ delivers a focused and interpretable perspective on the features deemed most critical by the model. Furthermore, \CTC\ provides a distinct ordering of features with associated numerical values, enabling a more detailed understanding of feature importance.

For instance, in the Saluki image (second example), the dog's body is assigned the highest contribution score of 13.15 (relative to the classification score of 20.81). The second-highest contribution, the toy bear’s ear, is much lower at 3.31, reflecting an importance nearly four times smaller than the dog. The rest of the bear, as well as the background, exhibit negligible importance with scores as low as 0.13. Similarly, in the Alp scene (fifth example), the snow and mountain ranges are assigned equally high relevance scores of around 4-5 (relative to the classification score of 15.626), contributing significantly to the classification, while features such as the pants and jacket of one skier are far less important with scores around 1.

The \CTC\ method excels in providing a segmented visualization of the key regions contributing to the model's predictions. For example, in the Chimpanzee image, \CTC\ clearly highlights the chimpanzee, the piano, and even the chairs in the background. In the Saluki image, the method successfully identifies the dog, the toy bear, and the blanket as relevant features. In contrast, other methods such as Input$\times$Gradient and DeconvNet produce noisy and less interpretable results, failing to distinctly localize features that contribute to the model's decision.

Although Guided Backpropagation and \LRP\ improve upon Input$\times$Gradient and DeconvNet by capturing edges and shapes, they lack contextual understanding and proper importance weighting. For instance, in the Saluki image, \LRP\ highlights all edges indiscriminately, making it unclear how the dog compares to the background or the toy in terms of importance. Similarly, in the Crib example, while Guided Backpropagation and \LRP\ successfully highlight relevant areas, they fail to provide a clear segmentation or quantitative breakdown of feature importance.

\subsubsection{Incorrectly Classified Examples}

Figure~\ref{Fig:vgg16_incorrect} showcases five examples that were misclassified by VGG16. The figure contrasts how different explainability methods—Input$\times$Gradient, Integrated Gradients, DeconvNet, LRP-$\alpha_1\beta_0$, and \CTC—attribute relevance to features in these examples.

\begin{figure}[ht!] 
\begin{center} 
\includegraphics[width=0.95\linewidth]{Figures/vgg16_incorrect.pdf} \end{center} 
\caption{Comparison of visual explanations for incorrectly classified examples by VGG16~\cite{SimonyanZ14a}. The figure highlights how \CTC\ provides clearer insights into the features driving misclassifications compared to other methods} 
\label{Fig:vgg16_incorrect}
\end{figure}

The first example depicts a grasshopper on a small red fruit. The classifier incorrectly identifies the image as a rose hip (a red/black fruit), rather than the insect itself.  Input$\times$Gradient and DeconvNet have noisy relevance distributed along the input. Giuded Propagation and LRP in contrast primarily focus on the head of the insect with a diffuse and blurry relevance across the rest of the image. These methods largely ignore the fruit beneath the insect, which likely contributed to the model’s misclassification. In contrast, \CTC\ identifies the fruit as the dominant contributor, assigning it a relevance score of 9 out of 11.031, with additional relevance to the surrounding fruits in the background.

Established methods often struggle to assign relevance to background features, instead favouring the central object and its edges. This limitation becomes particularly evident in examples where the classifier's decision is driven by the background rather than the central object. For instance, in the fourth example (a swimmer next to a pool, misclassified as a ballplayer), \LRP\ spreads relevance indiscriminately across the majority of edges in the image, including the swimmer and surrounding environment. However, \CTC\ identifies the grass as the most relevant features contributing to the misclassification with a score of 7.97 out of the classification score of 9.780. This suggests that the model may have incorrectly associated the grassy background with a sports context, such as a ball game.

In another example, a framed car (third input) is misclassified by VGG16 as a monitor. Guided Backpropagation and \LRP\ highlight most of the edges in the image, particularly those of the car, without providing a clear explanation of why the model arrived at this decision. By contrast, \CTC\ identifies the frame as the most significant feature, correctly revealing that the model's misclassification likely stemmed from the visual similarity between the car frame and a monitor bezel.

The final input image depicts a person standing near a set of pillars but is misclassified as a banister. Guided Backpropagation and \LRP\ once again attribute relevance to nearly all detectable edges, including the person in the foreground and the pillars in the background. This broad attribution fails to clarify the cause of the misclassification. In contrast, \CTC\ assigns four times higher relevance to the pillars in the background than any other object.

Arguably, \CTC\ not only simplifies the interpretation process but also provides a more precise and meaningful representation of what the model deems important for classification. In contrast, earlier methods, while progressively reducing noise in their visual explanations, often overemphasize the significance of edges in the classification process. Since convolutional layers inherently detect most (if not all) edges in an image, it is natural for these edges to be marked as relevant by these methods. However, this approach often fails to discern which features are genuinely critical to the classification decision versus those that are merely detectable. In contrast, \CTC\ delineates a more nuanced and accurate landscape of feature importance by identifying which features truly drive the classification, avoiding an over-generalized attribution of relevance to all detectable features.

\begin{figure}[ht!]
	\begin{center}
		\includegraphics[width=1\linewidth]{Figures/busy_results.pdf}
	\end{center}
	\caption{Analysis regarding the predicted class by VGG19~\cite{SimonyanZ14a} as computed by the selected analysers. The example chosen are very busy scenes, where the method can provide significant benefit.}
	\label{Fig:busy}
\end{figure} 

Figure~\ref{Fig:busy} illustrates additional examples of highly complex, cluttered scenes in which traditional explainability methods (Input x Gradient, DeconvNet, SmoothGrad, Guided Backpropagation, and LRP) produce explanations that highlight numerous regions across the input, making it challenging to interpret the model's focus or prioritize the relevance of specific features. These methods distribute importance broadly, resulting in visualizations that lack clarity and fail to provide meaningful insights into what the classifier deems most important.

In stark contrast, the \CTC\/ method stands out by delivering precise and easily interpretable results. \CTC\/ assigns clear and hierarchical importance to objects within the scene, visualizing the magnitude of relevance for different areas or objects. For instance, in the pillow example, \CTC\/ distinctly emphasizes key areas with a well-ordered hierarchy of relevance, where the pillow itself is 10 times more important than anything else in the scene, while other methods scatter importance across the entire image. Similarly, for the cab, harp, shoe shop, and crane examples, \CTC\/ consistently demonstrates its ability to isolate and prioritize critical regions, avoiding the overwhelming and indiscriminate attributions observed in other methods.

\subsection{\CTC\/ Evaluated on Different Networks}

The following subsections evaluate the performance of Input$\times$Gradient, DeconvNet, LRP-$\alpha_1\beta_0$ and \CTC\/ in showing the difference in what is found important between different network given the same input. This section shows the scalability and adaptability of the \CTC\/ method across a range of neural network architectures. The diversity in architectural design, exemplified by models such as VGG16~\cite{SimonyanZ14a}, VGG19~\cite{SimonyanZ14a}, ResNet50~\cite{he2015deep}, DenseNet121~\cite{huang2018densely}, and InceptionV3~\cite{szegedy2015rethinking}, presents unique challenges to explainability methods. The goal is to understand how the \CTC\/ method performs in different architectural contexts, considering the inherent design and operational differences of these networks. 

\subsubsection{Correctly Classified Examples}

This section evaluates the ability of various explainability methods to reveal what each network identifies as relevant features. By comparing these methods, the aim is to demonstrate whether \CTC\/ offers a clearer and more accurate representation of feature importance. To illustrate this, the first three inputs from Figure~\ref{Fig:vgg16_incorrect} were used as examples.

\begin{figure}[ht!]
	\begin{center}
		\includegraphics[width=0.95\linewidth]{Figures/all_networks_correct_1.pdf}
	\end{center}
	\caption{While Input$\times$Gradient, DeconvNet, and Guided Backpropagation highlight similar regions across networks, \LRP\ varies significantly between models. In contrast, \CTC\/ consistently identifies the chimpanzee as the most relevant feature across all networks, demonstrating its clarity and consistency.}
	\label{Fig:compare_models_same_1}
\end{figure} 

Figure~\ref{Fig:compare_models_same_1} highlights a chimpanzee playing a piano. Methods such as Input$\times$Gradient, DeconvNet, and Guided Backpropagation consistently highlight the same regions of the image across all networks. In contrast, \LRP\/ shows variability: for VGG16 and VGG19, it identifies nearly all edges of the chimpanzee and piano as relevant, emphasizing the chimpanzee. For ResNet50, \LRP\/ produces sparse relevance, while for InceptionV3 and DenseNet121, it marks the piano as highly relevant.

In comparison, \CTC\/ assigns significantly higher relevance to the chimpanzee—at least three times more than any other feature—across all networks. Moreover, \CTC\/ maintains a similar ranking and importance for other features consistently. Notably, the clusters remain consistent across all networks, except for InceptionV3, which uses 299x299 input dimensions compared to 224x224 for the other networks.



\begin{figure}[ht!]
	\begin{center}
		\includegraphics[width=0.85\linewidth]{Figures/all_networks_correct_2.pdf}
	\end{center}
	\caption{Visualization of relevance attribution across models and explanation methods for a dog and snowmobile, highlighting variations in focus and consistency.}
	\label{Fig:compare_models_same_2}
\end{figure} 

Figure~\ref{Fig:compare_models_same_2} illustrates the next two examples from Figure~\ref{Fig:vgg16_correct}—a dog and a snowmobile—analyzed across different neural networks. Both for the dog and snowmobile example, Input$\times$Gradient and DeconvNet produce similar heatmaps across all networks, though their explanations remain noisy.

In the dog example, Guided Backpropagation consistently identifies the dog as the most relevant feature, with a particular focus on its eyes, which is especially pronounced in ResNet50, DenseNet121, and InceptionV3. In contrast, \LRP\/ assigns importance to both the dog and the bear in VGG16 and VGG19, while providing unclear, noisy explanations in ResNet50. For InceptionV3 and DenseNet121, \LRP\/ shifts relevance predominantly to the dog's eyes. The \CTC\/ method finds the dog over four times more relevant than the bear in VGG16, VGG19, and ResNet50, and nearly twice as relevant in DenseNet121. However, in InceptionV3, the bear is deemed twice as relevant as the dog, which may reflect a limitation of the \CTC\/ method. This discrepancy could also be attributed to InceptionV3’s tendency to memorize training data due to its wide architecture \cite{nguyen2020wide}. Further, both Guided Backpropagation and \LRP\/ consistently highlight the dog’s eyes as important, and no \CTC\/ cluster contain the eyes, which may change the relevance of the dog cluster. 

In the snowmobile example, Guided Backpropagation and \LRP\ both assign relevance to the person, the snowmobile, and the surrounding snow tracks and trees, though the emphasis varies by model. ResNet50 prioritizes the person more than the snowmobile, whereas InceptionV3 focuses predominantly on the snowmobile. DenseNet121, on the other hand, distributes relevance more evenly between the edges of the person, the snowmobile, and the surrounding trees. Unlike these methods, \CTC\/ assigns primary importance to the snow itself, especially in cases where Guided Backpropagation and \LRP\ focus on edges. As discussed in Section~\ref{sec:correct}, Guided Backpropagation and \LRP\/ struggle to assign relevance to background elements effectively. By prioritizing edges, these methods can mislead interpreters into believing that the highlighted object is the key factor in the input, when in reality, the relevance might lie in the background or other contextual features. While the correct answer is unknown in this instance, it is plausible that the snow is the decisive feature driving the classification. Notably, in InceptionV3, all three methods—Guided Backpropagation, \LRP\/, and \CTC\/—consistently identify the snowmobile as the most important feature for classification.


\subsubsection{Incorrectly Classified Examples}

This section examines how various explainability methods perform in identifying relevant features for incorrectly classified examples. By analysing cases where networks fail to make the correct prediction, the aim to uncover potential biases or limitations in the models’ reasoning processes and assess the reliability of the explainability methods in these challenging scenarios. Specifically, this section explores whether \CTC\/ can provide more insightful and interpretable explanations in instances where other methods might generate noisy or misleading relevance maps. 

To illustrate this, three examples of misclassifications were examined further from Figure~\ref{Fig:vgg16_incorrect}, comparing how each method highlights features across different networks. By focusing on these challenging cases, the aim is to better understand the strengths and weaknesses of each explainability method and highlight the role of \CTC\/ in providing more consistent and interpretable insights.


\begin{figure}[ht!]
	\begin{center}
		\includegraphics[width=1\linewidth]{Figures/all_networks_incorrect_1.pdf}
	\end{center}
	\caption{Analysis regarding the actually predicted class by different classifier as computed by the selected analysers.}
	\label{Fig:compare_models_same_missclassified_1}
\end{figure} 

The first example is the misclassified grasshopper shown in Figure~\ref{Fig:compare_models_same_missclassified_1}, which originates from Figure~\ref{Fig:vgg16_incorrect}. The input image features a grasshopper perched on top of red, green, and black fruit. VGG16 and VGG19 misclassify the image as a rose hip, likely influenced by the fruit’s resemblance to a rose hip. ResNet50 and InceptionV3, on the other hand, correctly classify the image as a grasshopper, while DenseNet121 misclassifies it as a cricket. Despite these varying classifications across networks, Input$\times$Gradient and DeconvNet generate similar heatmaps with consistently noisy explanations across all networks.

Guided Backpropagation highlights the grasshopper’s head as the most relevant feature in all networks. However, in VGG16 and VGG19, it also assigns some relevance to the fruit, though this is not the primary focus. A similar pattern is observed in \LRP\/, where VGG16 and VGG19 attribute some importance to the fruit while primarily focusing on the grasshopper’s head. For ResNet50, \LRP\ produces a relatively noisy explanation but still emphasizes the head of the grasshopper. On InceptionV3, \LRP\ generates an almost empty explanation, while for DenseNet121, the grasshopper's head is again identified as the most relevant feature.

In contrast to these methods, \CTC\/ assigns the most significant relevance to the fruit for VGG16 and VGG19, reflecting its likely contribution to the networks’ misclassification as a rose hip. For VGG16, the fruit is assigned relevance scores of 9.24 in the foreground and 9.088 in the background, given the total classification score of 14.983 for a rose hop. Similarly, for VGG19, the fruit scores 8.48 in the foreground and 4.79 in the background, with a classification score of 11.031 for rose hop. In both cases, the grasshopper receives far less relevance, with scores of 3.04 and 3.19 respectively—less than half of the fruit's assigned importance.

In contrast, in the networks that correctly classify the insect (ResNet50 and InceptionV3) or misclassify it as another insect (DenseNet121), \CTC\/ assigns the highest relevance to the grasshopper itself, reflecting the networks’ focus on the insect as the primary object in the image. This highlights \CTC\/’s capacity to align relevance attribution with the networks’ specific classification outcomes arguably better than other methods.

Figure~\ref{Fig:compare_models_same_missclassified_2} shows the next two examples from Figure~\ref{Fig:vgg16_incorrect} of a water buffalo and pickup truck. Similarly to all previous examples Input$\times$Gradient and DeconvNet generate similar heatmaps across networks. Input$\times$Gradient in the water buffalo case focuses on the animal and in the truck input they focus on the truck. DeconvNet produces very noisy explanations for all of them. 

\begin{figure}[ht!]
	\begin{center}
		\includegraphics[width=0.85\linewidth]{Figures/all_networks_incorrect_2.pdf}
	\end{center}
	\caption{Analysis regarding the actually predicted class by different classifier as computed by the selected analysers.}
	\label{Fig:compare_models_same_missclassified_2}
\end{figure} 

In the water buffalo example, both guided backpropagation and \LRP\ consistently highlight only the animal as relevant, regardless of the network used. Despite differences in classification across networks, the explanations remain unchanged. In contrast, \CTC\/ produces notably distinct explanations for different networks, offering a clear ranking of features. This allows the explainee to understand the factors driving variations in classification.

In the truck example, VGG16 and VGG19 misclassify the image as monitor and sports car respectively. \CTC\/ effectively identifies the vehicle features that could plausibly lead to these errors, such as a monitor-like display or the sleek body shape of a sports car. The relevance scores for these regions are significantly higher than for other parts of the image, demonstrating that the method accurately captures the features driving these misclassifications.

For InceptionV3, the pickup portion of the truck is deemed most relevant, with minimal relevance assigned to the frame. However, for ResNet50 and DenseNet, the frame gains considerable relevance even though the image is classified as a cab. This might be as  the truck is effectively divided into distinct parts during the propagation process. If the entire truck were selected as a cohesive feature, its relevance might have been higher.

Interestingly, although the background achieves the highest relevance in some examples, its contribution is still much lower compared to instances of misclassification, such as VGG16's monitor prediction. In that case, the monitor's relevance score is 18, whereas for the cab classification in ResNet50 and DenseNet, the corresponding relevance scores drop significantly to 3.05 and 5.18, respectively. This underscores how \CTC\/ adapts its explanations based on the classification outcome and feature contributions.

A critical observation is that the explanations provided by other post-hoc interpretability methods show minimal variation across different networks, rendering it challenging to discern the distinct functional characteristics of each classifier. Furthermore, while LRP-$\alpha_1\beta_0$~\cite{bach2015pixel} generally produces similar explanations across different models, it notably struggles with networks like ResNet50. This inconsistency is likely attributable to the network’s unique structural features, such as residual connections, which impact the method's ability to generate clear explanations.

An interesting facet demonstrated is the effect of network size on the explanations generated. There seems a slight negative correlation between the amount of relevance present in other methods' heatmaps and the network size. This is a pressing problem --- network architectures are tending to grow in size in recent years. \CTC\/ demonstrably maintains the amount of information provided to the user.

In contrast to all other methods explanations, the \CTC\/ method's approach to feature ordering provides a more straightforward and insightful analysis. Its ability to distinctly outline and prioritise features makes it easier to identify and understand the differences in the functions learned by various classifiers. This advantage becomes even more pronounced in scenarios depicted in Figure~\ref{Fig:compare_models_same_missclassified_2}, where inputs are not uniformly classified across all networks. In such cases, an interpretability method like \CTC\/ becomes invaluable. It aids in pinpointing differences in the learned functions between correctly and incorrectly classifying networks, offering critical insights into each model's decision-making process.

\section{Quantitative Analysis}
In this subsection, a quantitative analysis is conducted to assess the input invariance and sensitivity of various attribution methods. The focus is on how these methods respond to alterations in the input data, which are introduced to the input through Gaussian noise, Gaussian blurring, and Uniform noise. This evaluation is crucial in determining the robustness of each method against subtle or impactful changes in the input, thereby providing insight into their reliability and consistency in different scenarios. As in the previous section to allow for independent evaluation and fair comparison to other methods in the literature, the contribution is calculated on the biggest masks as identified by SAM. This is opposed to selecting the masks with the most relevance attributed to them. Given that the change in relevance is what is being measured, the most relevant mask may change, tying the \CTC\/ method results with the relevance method used for mask selection. The goal of this analysis is to identify how faithful the \CTC\/ rules are by introducing noise to the input, and measuring the variance of the output. Therefore, steps were taken to ensure that any variance in the output is as a result of \CTC\/ and the underlying network.


Similarly, the quantitative analysis aims to test the fidelity of the importance value assigned to features, not the clustering technique. To evaluate the \CTC\/ value independently of the clustering the biggest masks identified by SAM on the original input are fixed and when evaluated on the noisy inputs. Ultimately, it was decided not to recompute feature-masks as all other methods have a fixed number of input features, they present one value for each pixel, always evaluated on all pixels, in the case of image classification. As such, it stands to reason that the fairest evaluation method would be to measure the difference in explanations over the same input features --- in other words, were new input features to be computed with each injection of noise, it is not trivial to identify which features have mapped to other features from which to compute the distance. Particularly given that \CTC\/ can be employed on top of any existing clustering method, it stands to reasons that fixing masks, and allowing the variance in output to be derived solely from the noise and forward propagation is the most reliable way of assessing the methods robustness and reliability to noise. 

The quantitative analysis tests the input invariance and sensitivity of the explainability method. For a method to be input invariant the change in explanations should be low for changes in the input that are small and do not change the models output. For a method to be sensitive the change in explanations should be high for changes in the input that are big and do change the models classification.

\subsection{Input invariance}
\label{input_inv}
Input invariance describes the resilience of an attribution method against certain changes to the input that do not modify the model's output~\cite{YehHSIR19}. Essentially, a robust interpretation method should demonstrate minimal sensitivity to these alterations, meaning it should consistently provide nearly identical explanations even when slight variations are made to the input. To assess the input invariance of the \CTC\/ method and how it compares the input invariance of Input$\times$Gradient~\cite{SimonyanVZ13}, Integrated Gradients~\cite{SundararajanTY17}, DeconvNet~\cite{ZeilerKTF10}, Guided BackProp~\cite{SpringenbergDBR14}, and \LRP\-$\alpha_1\beta_0$~\cite{bach2015pixel}, three types of slight variations of the input are introduced.


The first type of input variation is introduced through Gaussian noise using a function that generates noise with a mean of zero and standard deviation of 0.1, which is added to the original image. The second method applies a Gaussian blur, which smooths the image by averaging pixel values in a manner weighted by their proximity to each pixel being processed. The degree of blurring is governed by the size of the Gaussian kernel used in the process, which is $5 \times 5$ in the images generated to test the input invariance of the method. This allows for the blurring to not be too strong. Unlike the Gaussian noise added previously, Gaussian blurring affects the image in a more uniform and systematic way, altering the spatial relationships within the image without introducing extraneous pixel-level variability. The final method introduces Uniform noise. Unlike Gaussian noise, Uniform noise affects all pixels with the same probability and intensity, making it a starkly different test for the attribution method's robustness. The noise is randomly generated within a specified range, which is [-255,255] in the case of images. It is scaled by the intensity parameter, which suits as a noise level. The noise level can hold values between zero and one, where zero means no noise is added to the image and one means that the image is converted only to noise. The noise level parameter was set to 0.02. All resulting noisy images are clipped to ensure all pixel values remain within the [0,255] range, suitable for standard image formats.

\begin{figure}[ht!]
	\begin{center}
		\includegraphics[width=0.95\linewidth]{Figures/small_noise.pdf}
	\end{center}
	\caption{Examples of images from the ILSVRC 2012 dataset after the introduction of the three types of image alteration --- Gaussian noise addition, Gaussian blurring and Uniform noise.}
	\label{Fig:noisy_images}
\end{figure} 

These three methods of image alteration --- Gaussian noise addition, Gaussian blurring and Uniform noise --- provide distinct challenges to the attribution methods being evaluated. The noise images are generated from the ImageNet Large Scale Visual Recognition Challenge 2012 (ILSVRC 2012) data set~\cite{ILSVRC15} and examples of all three types of noisy images can be seen in Figure~\ref{Fig:noisy_images}. As one may observe, the Gaussian noise introduces pixel-level changes, which are too small for the human eye to notice. The Gaussian blurring alters the image's overall texture and detail and smooths the edges of objects present in the image. Finally, the uniform noise affects all pixels with the same probability and intensity, but as the noise level parameter is quite small it can be mostly observed in the lighter part of images. After generating the modified dataset, explanations were created for each input and its variation --- original, Gaussian blurring, Gaussian noise, and Uniform Noise --- using Input$\times$Gradient, Integrated Gradients, DeconvNet, Guided BackProp, \LRP\-$\alpha_1\beta_0$ and the \CTC\/ method. This is followed by a normalisation step before comparing the original and modified inputs, ensuring the sum of explanations equals one. This is especially important for the \CTC\/ method, which tends to assign larger values to regions compared to the individual pixel attributions by other methods. The three types of noise present a comprehensive evaluation of how well the \CTC\/ method, along with others like Input$\times$Gradient, Integrated Gradients, DeconvNet, Guided BackProp, and \LRP\-$\alpha_1\beta_0$, can maintain consistent interpretations in the presence of subtle input variations. Such resilience is key in ensuring reliable and robust model explanations.


The comparative results, shown in Figure~\ref{Fig:norm}, Figure~\ref{Fig:blur} and Figure~\ref{Fig:uniform}, are based on 500 distinct original inputs. When measuring the input invariance the results show the difference between original inputs explanation and the noisy input explanation only for the noisy images that do \emph{not} change the classification. As input invariance measures the change in explanations when the input has a small change, here if the noisy image changes the classification the noise added is not considered a small change. The distance between the original input's and the noisy input's explanations is calculated using Euclidean distance. Given two $n$-dimensional vectors \( \vec{v}_1 = (v_{1,1}, v_{1,2}, ..., v_{1,n}) \) and \( \vec{v}_2 = (v_{2,1}, v_{2,2}, ..., v_{2,n}) \), where $\vec{v}_1$ is the original image explanations and $ \vec{v}_2$ is the noisy input explanations, the Euclidean distance \( d \) between them is given by:
\begin{equation*}
    d(\vec{v}_1, \vec{v}_2) = \sqrt{\sum_{i=1}^{n} (v_{1,i} - v_{2,i})^2}
\end{equation*}
Here, the summation sums up the squares of the differences of the corresponding components of the two vectors, and the square root is taken of the sum. In all figures the top graph presents the variability of all six methods, bellow a description of each distance vector $\vec{d}$ shows the mean, standard deviation, minimum, 25\%, 50\%, 75\% and the maximum value of each distance vector $\vec{d}$. The bottom left graph shows a zoomed version of all six methods, which ignores the outliers in some methods, such as Guided Backpropagation and makes it easier to see the mean and the standard deviation of the majority of distances. The bottom right graph shows the distances between the original explanations and explanation for the noisy image only for \LRP\/ and the \CTC\/ methods. As the distance between the original explanations and explanation for the blurry image for those two methods is very small, it is hard to see their mean and standard deviation. This zoomed graph makes it easier to see this information.  

\subsubsection{Gaussian Noise ($\mu = 0, \sigma = 0.1$)}

The application of Gaussian noise, characterized by a mean ($\mu$) of 0 and a standard deviation ($\sigma$) of 0.1, provides insight into the robustness of attribution methods under moderate noise conditions. As illustrated in Figure~\ref{Fig:norm}, this type of noise has minimal impact on methods like \LRP\-$\alpha_1\beta_0$ and \CTC\/, which continue to produce consistent interpretability. 
\begin{figure}[ht!]
	\begin{center}
		\includegraphics[width=0.92\linewidth]{Figures/minor_noise_normal_stats.pdf}
	\end{center}
	\caption{\CTC\/ and \LRP-$\alpha_1\beta_0$ are robust to Gaussian noise, maintaining consistent interpretability with minimal differences in explanations across noisy and original inputs.}
	\label{Fig:norm}
\end{figure} 

The results show that out of 500 inputs subjected to Gaussian noise, 477 retained their original classifications, demonstrating the limited effect of this type of noise on classification accuracy. From the perspective of explanation consistency, \CTC\/ exhibits the smallest mean difference between explanations generated from the original and noise-affected inputs. The standard deviation of the differences in \CTC\/ explanations is also low, second only to \LRP\-$\alpha_1\beta_0$. This highlights \CTC\/'s ability to distribute attribution values over broader regions of the input, reducing sensitivity to localized variations introduced by noise. In contrast, pixel-level methods like Input$\times$Gradient and Integrated Gradients show higher sensitivity, as their explanations are directly tied to the intensity of individual pixels. These methods struggle to maintain invariance under noisy conditions, often amplifying discrepancies introduced by the noise.

\subsubsection{Gaussian Blur ($k = (5 \times 5)$)}

The effect of Gaussian blur, with a kernel size of $5 \times 5$, is visualized in Figure~\ref{Fig:blur}. This noise type subtly alters the fine-grained details in the input image by smoothing edges and softening textures. Unlike Gaussian noise, which adds pixel-specific randomness, Gaussian blur systematically reduces high-frequency details, challenging the interpretability of methods that rely heavily on edge information. Out of the 500 inputs tested, 429 retained their original classifications after applying Gaussian blur, showing a slightly greater sensitivity of the network to this noise compared to Gaussian noise.


Edge-focused methods like Guided BackProp and \LRP\-$\alpha_1\beta_0$ exhibit greater differences in their explanations between the original and blurred inputs. The lesser edge sharpness due to Gaussian blur disproportionately impacts these methods, leading to almost double the differences compared to Gaussian noise. Other pixel-sensitive methods such as Input$\times$Gradient, Integrated Gradients, and DeconvNet show similar trends, with their attribution values deviating significantly under blurred conditions.

In contrast, \CTC\/ demonstrates remarkable resilience, maintaining consistency in its explanations regardless of the introduction of Gaussian blur. Its standard deviation remains low, on par with Gaussian noise conditions. This robustness can be attributed to \CTC\/'s ability to assign attribution values across broader regions rather than focusing on fine-grained details, making it less susceptible to noise-induced variations. The consistency of \CTC\/ across different types of noise further solidifies its position as a robust interpretability method.

\begin{figure}[ht!]
	\begin{center}
		\includegraphics[width=0.92\linewidth]{Figures/minor_blur_stats.pdf}
	\end{center}
	\caption{Guided BackProp, \LRP-$\alpha_1\beta_0$ and \CTC\/ are resilient to  Gaussian blur. However Guided BackProp and \LRP-$\alpha_1\beta_0$ are twice more sensitive to it then Gaussian noise, where as \CTC\/ is minimally impacted.}
	\label{Fig:blur}
\end{figure} 

\subsubsection{Small Uniform Noise ($l = 0.02$)}

The impact of small uniform noise, applied with a range of $[-0.02, 0.02]$, is presented in Figure~\ref{Fig:uniform}. Unlike Gaussian noise, which varies based on a standard deviation, uniform noise affects all pixels evenly within the specified range, disproportionately impacting lighter regions due to its consistent intensity. This type of noise had the highest impact on classification among the noise types analysed, with 401 of the 500 inputs retaining their original classifications. This indicates a greater sensitivity of the network to uniform noise, even though its intensity is relatively low.

For the 401 inputs with unchanged classifications, the consistency of explanations varied across methods. Input$\times$Gradient, Integrated Gradients, DeconvNet, Guided BackProp, and \LRP\-$\alpha_1\beta_0$ all exhibited significant differences in their explanations compared to Gaussian noise, although these differences were less pronounced than for Gaussian blur. \CTC\/ once again outperformed other methods in terms of robustness, showing minimal changes in its explanations under uniform noise. This suggests that \CTC\/ effectively minimizes the influence of uniform noise by focusing on broader input regions, rather than being overly reliant on pixel-level intensities.

\begin{figure}[ht!]
	\begin{center}
		\includegraphics[width=0.93\linewidth]{Figures/minor_noise_uniform_stats.pdf}
	\end{center}
	\caption{Guided BackProp, \LRP-$\alpha_1\beta_0$ and \CTC\/ are resilient to Small Uniform Noise. This is the type of noise that has the biggest impact on the network.}
	\label{Fig:uniform}
\end{figure} 

\subsubsection{Comparative Analysis of Noise Types}

The results across noise types reveal a consistent pattern in network sensitivity and the robustness of interpretability methods:

\begin{itemize}

    \item \textbf{Highly Sensitive Methods:} \textbf{LRP-$\alpha_1\beta_0$} and the \CTC\/ method consistently demonstrate high sensitivity across all noise types, reliably distinguishing between cases where classification changes occur and where it does not. \textbf{LRP-$\alpha_1\beta_0$} is particularly reactive to Gaussian blur due to its reliance on edge features, while the \CTC\/ method balances global and local feature sensitivity, making it effective against both random noise (Gaussian and Uniform) and structural alterations (blur).

    \item \textbf{Less Sensitive Methods:} \textbf{Guided Backpropagation} and \textbf{DeconvNet} exhibit minimal sensitivity to all noise types, often failing to reflect significant changes in input data. \textbf{Integrated Gradients} and \textbf{Input$\times$Gradient} display moderate sensitivity, particularly excelling with Gaussian blur and Gaussian noise, but are less effective for Uniform noise, indicating limitations in detecting distributed noise variations.
\end{itemize}


Overall, the results reinforce the importance of selecting attribution methods that can preserve interpretability under various noise conditions, with \CTC\/ emerging as the most robust option across all tested scenarios.

\subsection{Sensitivity}
Sensitivity in interpretability methods refers to how methods react to significant changes in input data. It is crucial that these methods provide distinctly different explanations when the model's classification alters~\cite{NielsenDRRB22}. This section evaluates the sensitivity of the \CTC\/ method by comparing it with several other approaches: Input$\times$Gradient~\cite{SimonyanVZ13}, Integrated Gradients~\cite{SundararajanTY17}, DeconvNet~\cite{ZeilerKTF10}, Guided BackProp~\cite{SpringenbergDBR14}, and LRP-$\alpha_1\beta_0$~\cite{bach2015pixel}. To conduct this comparison, the same three significant types of input variations are introduced, but with much higher levels of noise than the ones when testing input invariance.

The first variation adds Gaussian noise with a much higher standard deviation ($\sigma$) than previously used for assessing input invariance. In this case, the standard deviation $\sigma$ is set at 50, compared to the standard deviation of 0.1 used for input invariance evaluation. The second variation creates a blurred version of the original input. The degree of blurring is governed by the size of the Gaussian kernel used in the process, which is $17 \times 17$ for the sensitivity test, in comparison to the $5 \times 5$ Gaussian kernel used for input invariance. The final variation introduces Uniform noise at a level of 0.3, again substantially higher than the 0.02 level used in the input invariance tests. These increased noise levels are designed to rigorously test the sensitivity of the \CTC\/ method against state-of-the-art methods. The parameters are set arbitrary to lead to a change in the classification, while still preserving part of the original image.

To visually illustrate the impact of these input variations, Figure \ref{Fig:noisy_images_2} presents examples of images from the ILSVRC 2012 dataset after the introduction of the three types of significant image alterations. These images serve as a clear representation of the dramatic changes in input data, providing a concrete basis for assessing the sensitivity and robustness of the \CTC\/ method in comparison to established techniques. The differences observed in these images are crucial for understanding how each method responds to substantial variations in input. As one may observe, the introduction of blur, Gaussian noise, and Uniform noise to the ILSVRC 2012 dataset images results in significant visual alterations, but the inputs are still somewhat recognisable. After generating the modified dataset, explanations were created for each input image variation --- original, Gaussian noise, Gaussian blurring and Uniform Noise --- using Input$\times$Gradient, Integrated Gradients, DeconvNet, Guided BackProp, LRP-$\alpha_1\beta_0$ and the \CTC\/ method.  

\begin{figure}[ht!]
	\begin{center}
		\includegraphics[width=0.95\linewidth]{Figures/big_noise.pdf}
	\end{center}
	\caption{Examples of images from the ILSVRC 2012 dataset after the introduction of the three types of big image alteration --- Blur ($k = (17 \times 17)$), Gaussian noise addition ($\mu= 0; \sigma=50$) and Uniform noise ($l = 0.3$).}
	\label{Fig:noisy_images_2}
\end{figure} 

Similarly to when testing input invariance, the explanations are normalised ensuring the sum of explanations equals one. The three types of noise present a comprehensive evaluation of how well the \CTC\/ method, along with others like Input$\times$Gradient, Integrated Gradients, DeconvNet, Guided BackProp, and \LRP\-$\alpha_1\beta_0$, can reflect big changes in the input. Showing a significantly different explanation when the input has been significantly changed is is key in ensuring reliable model explanations. The comparative results, shown in Figure~\ref{Fig:big_gaus}, Figure~\ref{Fig:big_blur} and Figure~\ref{Fig:big_noise_uniform}, are based on the distance between 500 distinct original inputs and their noisy counterparts. As highlighted in Section~\ref{input_inv} certain methodologies exhibit significant alterations in their explanations even with minimal noise introduction. This section aims to evaluate the relative sensitivity of these methods by examining the variance in explanations between examples where noise did not alter the classification and those where it did. This approach allows for an assessment of explanation changes in relation to a baseline alteration for negligible changes. Each figure presents the variability of all six methods between the inputs that changed the classification and the ones that did not. To validate if the difference is statistically significant, a Welch's t-test was performed~\cite{welch1947generalization}. Welch's t-test, also known as the unequal variances t-test, is a statistical test used to compare the means of two groups to determine if they are significantly different from each other. It is an adaptation of the standard Student's t-test~\cite{student1908probable} and is more reliable in real-world scenarios where the assumption of equal variances between groups is often violated.


\subsubsection{Gaussian Noise Addition ($\mu = 0, \sigma = 50$)}
The distances between the original explanation and the explanation of the input, which has a big amount of Gaussian Noise added is shown in Figure~\ref{Fig:big_gaus}. This blurring alters image details and should result in attribution methods generating explanations different from the original input.


The analysis of the impact of Gaussian noise addition ($\mu = 0, \sigma = 50$) reveals significant differences in the sensitivity of various interpretability methods to substantial input alterations. Among the tested approaches, LRP-$\alpha_1\beta_0$ and the \CTC\/ method exhibit the highest sensitivity, with clear and statistically significant distinctions in their explanation distances between cases of "Big change" and "Insignificant change". 

\begin{figure}[ht!]
	\begin{center}
		\includegraphics[width=0.92\linewidth]{Figures/big_gaus.pdf}
	\end{center}
	\caption{The bigger the change the better the sensitivity to big noise, here the sensitivity of Input$\times$Gradient~\cite{SimonyanVZ13}, Integrated Gradients~\cite{SundararajanTY17}, DeconvNet~\cite{ZeilerKTF10}, Guided BackProp~\cite{SpringenbergDBR14}, and LRP-$\alpha_1\beta_0$~\cite{bach2015pixel} and the \CTC\/ method is illustrated.}
	\label{Fig:big_gaus}
\end{figure} 

LRP-$\alpha_1\beta_0$ and the \CTC\/ methods effectively capture the effects of input noise, providing reliable and meaningful explanations when classification changes occur. In contrast, methods such as Guided Backpropagation show minimal sensitivity, with little variation in explanation distances under substantial noise compare to insignificant amount of noise, indicating limited utility for capturing meaningful input changes. Integrated Gradients and Input$\times$Gradient demonstrate moderate-to-high sensitivity, reflecting their robustness in explaining significant alterations. Overall, this evaluation highlights the \CTC\/ method's strong performance in maintaining interpretability and sensitivity under noisy conditions, making it a competitive option compared to state-of-the-art techniques.


\subsubsection{Gaussian Blur ($k = (17 \times 17)$)}
The distances between the original explanation and the explanation of the input, which has a big amount of Gaussian blur is shown in Figure~\ref{Fig:big_blur}. This blurring alters image details and should result in attribution methods generating explanations different from the original input.


The analysis of Gaussian blur ($k = 17 \times 17$) reveals significant variations in the sensitivity of interpretability methods to input alterations caused by blurring. Methods such as Integrated Gradients, Input$\times$Gradient, and LRP-$\alpha_1\beta_0$ exhibit strong sensitivity, with clear and statistically significant distinctions in explanation distances when classification changes occur. These methods effectively capture the impact of blurring on input features and model behavior. The \CTC\/ method demonstrates moderate sensitivity, providing reliable explanation differences, although it is slightly less responsive compared to the most sensitive methods. In contrast, Guided Backpropagation and DeconvNet show minimal sensitivity, with negligible changes in explanation distances even for significant classification alterations, suggesting a limited ability to reflect the effects of Gaussian blur. 

The results for the \CTC\/ method are unsurprising, given its lower input invariance to edges, as previously discussed. Unlike methods that heavily rely on early-layer activations, such as LRP-$\alpha_1\beta_0$, \CTC\/ demonstrates a lower sensitivity across the network. LRP and Guided Backpropagation often overemphasize the first convolutional layers, which are highly sensitive to edges and low-level image features. This over-exaggeration makes these methods particularly reactive to Gaussian blur, which primarily alters edge details. Interestingly Guided Backpropagation is not sensitive to the blur in comparison to small blur.
% shows varying degrees of resilience among the methods, with some, like \CTC, effectively preserving their interpretative accuracy despite the blurring. Note that out of the 500 inputs this change was tested on 429, which did not result in a change in classification. 
% The top graph 
\begin{figure}[ht!]
	\begin{center}
		\includegraphics[width=0.92\linewidth]{Figures/big_blur.pdf}
	\end{center}
	\caption{The bigger the change the better the sensitivity to blurry input. Comparison of method sensitivity to Gaussian blur ($k = 17 \times 17$), highlighting significant variability in explanation distances across interpretability techniques."}
	\label{Fig:big_blur}
\end{figure} 

\subsubsection{Uniform Noise Addition ($l = 0.3$)}
The distances between the original explanation and the explanation of the input, which has a big amount of Uniform Noise added is shown in Figure~\ref{Fig:big_noise_uniform}. 

The analysis of uniform noise addition ($l = 0.3$) highlights significant differences in the sensitivity of interpretability methods to random input alterations. Methods such as \LRP\/, Integrated Gradients and the \CTC\/ demonstrate strong sensitivity, with clear and statistically significant distinctions in explanation distances when classification changes occur. These methods effectively capture the impact of uniform noise on input features, providing reliable and meaningful explanations. Input$\times$Gradient also shows moderate sensitivity, reflecting explanation differences for altered classifications, though less prominently than the top-performing methods. In contrast, Guided Backpropagation and DeconvNet exhibit low sensitivity, with little to no variation in explanation distances between cases of significant and insignificant classification changes. This insensitivity suggests limited reliability for these methods in scenarios involving substantial noise. Overall, the results reaffirm the robustness of methods like \LRP\/, Integrated Gradients and the \CTC\/ method in responding to noisy inputs while underscoring the limitations of less sensitive approaches.

\begin{figure}[ht!]
	\begin{center}
		\includegraphics[width=0.95\linewidth]{Figures/big_noise_uniform.pdf}
	\end{center}
	\caption{The figure shows the challenge big amount of Uniform noise addition poses to various attribution methods.}
	\label{Fig:big_noise_uniform}
\end{figure} 


\subsubsection{Comparative Analysis of Noise Types}

The results across noise types reveal a consistent pattern in the sensitivity of interpretability methods and their ability to reflect significant input changes:

\begin{itemize}
    \item \textbf{Highly Sensitive Methods:} \textbf{LRP-$\alpha_1\beta_0$} and the \CTC\/ method consistently show high sensitivity across all noise types, reliably generating distinct explanations when classification changes occur. While \LRP\/ is particularly reactive to Gaussian blur due to its emphasis on edge features, the \CTC\/ method balances global and local feature attributions, demonstrating strong sensitivity to both random noise (Gaussian and Uniform) and structural noise (blur).

    \item \textbf{Less Sensitive Methods:} \textbf{Guided Backpropagation} and \textbf{DeconvNet} show minimal sensitivity across all noise types, often failing to reflect significant input alterations. Integrated Gradients and Input$\times$Gradient display moderate sensitivity, particularly excelling with Gaussian blur and Gaussian noise, but are less effective for Uniform noise, highlighting limitations in their ability to adapt to distributed noise variations.
\end{itemize}



% \subsection{Discussion}

% The key objective of this chapter is to demonstrate that the novel interpretability method, which gives a single value of contribution to the classification (\CTC) to each identified complex input feature, is capable of providing faithful insight into the inner workings of models while delivering these insights in an easily graspable manner. A particular concern is that the simplification of evaluating entire features might negatively harm fidelity. To address this, the chapter looks at fidelity metrics to compare the performance of \CTC\/ against established interpretability techniques. The chapter focuses on two essential attributes: \emph{sensitivity} and \emph{input invariance}. Both of these qualities were assessed by introducing noise to the input and observing the resultant variations in the explanations provided. An ideal interpretability method should exhibit low sensitivity to minor noise perturbations, thereby demonstrating input invariance, particularly under the assumption of a robust network. Conversely, a substantial introduction of noise should lead to a noticeable change in explanation. 



% The sensitivity of an interpretability method is assessed by measuring how much its explanation fluctuates with the introduction of noise. This chapter aims to provide a nuanced understanding of these fluctuations. The significance of large changes in explanations is assessed by establishing a baseline of minor changes, which serves as a reference point for evaluating the validity of a single explanation. The relevance of a major alteration in the input becomes evident only when contrasted with minor, inconsequential changes. For instance, if an insignificant modification in the input, one that doesn't alter the model's classification, leads to a substantial shift in the explanation, it implies that a major input change would almost certainly result in a significant explanation shift. To contextualise sensitivity, the explanations of inputs with \emph{insignificant} changes is compared--- largest amount of noise that does not affect the classification --- and \emph{significant} changes --- those where the introduced noise does alter the classification, and thus, should also trigger a marked change in the explanation. This approach is part of a broader conversation about the efficacy and intent of these metrics, which is further explored in Chapter~\ref{evl}.

% The results clearly indicate that \CTC\ presents a statistically significant change between the explanations with a significant and insignificant noise for all three types of noise presented. Notably, blurring led to the biggest change in input classification. This is expected as it significantly reduces fine details and alters edge information, which are crucial for the network to make it's predictions. In contrast, Gaussian and Uniform Noise tend to add random pixel-level variations that do not fundamentally alter the underlying structures and edges as drastically as blurring does. In this type of noise the integrated gradients and Input $\times$ Gradient were disproportionately effected compared to distance in explanations using Gaussian Noise and Uniform Noise. Gradient methods are sensitive to spatial information and take the gradient with respect to the input. Blurring disrupts the spatial relationships between pixels by averaging them over a region. This smoothing effect can drastically change their explanations, leading to higher sensitivity. As discussed, to evaluate the \CTC\ value independently of the clustering the biggest masks identified by SAM on the original input are fixed and when evaluated on the noisy inputs used without recomputing. This injects more ordered information, as the mask often are dictated by where there used to be edges in the input. The \CTC\ algorithm is therefore less effected by the smoothing operation than Gaussian Noise and Uniform Noise. That being said, it is still statistically different from the inputs that do not change the classification. Further, the \CTC\ method has the second best sensitivity after \LRP\ on inputs with Gaussian Noise and Uniform Noise. 


% In summation, it is evident that a quantitative analysis of empirical results shows that \CTC\ exhibits the qualities of sensitivity and input invarience. Indeed, even despite the considerably simplified input, the method demonstrably maintains fidelity to a degree comparable to other leading methods of interpretability. In light of the goals specified by the thesis (simplifying the explanations to reduce the cognitive load upon the explainee while maintaining fidelity), it is fair to conclude that \CTC\ meets the criteria of preserving faithfulness, despite the clear reduction in explanation dimensionality and granularity. Therefore, under the assumption that reducing the number of input features by clustering/segmenting the input space into relevant features does simplify the explanation, the results provide evidence that the research objectives have been met.  


\section{Conclusion}

The method proposed in this chapter assigns a single value of relevance to a complex input feature by propagating its contribution forward. The contribution to the classification (\CTC\/) is calculated by modified versions of the layers of the network. These modified layers allow for the behaviour of inference step of the neural network to be replicated, but with only part of the input present. 

This chapter demonstrates that \CTC\/, is capable of providing deep insights into the inner workings of models while delivering these insights in an easily graspable manner. A particular concern is that the simplification of evaluating entire features might negatively harm fidelity. To address this, the chapter employed fidelity metrics to compare the performance of \CTC\/ against established interpretability techniques. Two essential attributes were examined: \emph{sensitivity} and \emph{input invariance}. Both of these qualities were assessed by introducing noise to the input and observing the resultant variations in the explanations provided. An ideal interpretability method should exhibit low sensitivity to minor noise perturbations, thereby demonstrating input invariance, particularly under the assumption of a robust network. Conversely, a substantial introduction of big amount of noise should lead to a noticeable change in explanation. 

The empirical results demonstrates that \CTC\/ exhibits the qualities of sensitivity and input invariance. Indeed, even despite the considerably simplified input, the method demonstrably maintains fidelity to a degree comparable to other leading methods of interpretability and in some cases better. In light of the goals specified by the thesis (simplifying the explanations to reduce the cognitive load upon the explainee while maintaining fidelity), it is fair to conclude that \CTC\/ meets the criteria of preserving faithfulness, despite the clear reduction in explanation dimensionality and granularity. 


 

% Chapter~\ref{chapter:results} delves into the practical implementation of forward pass retracing, as well as the interpretability and fidelity of the contribution to classification (\CTC\/) values. Through the comprehensive evaluation on various types of neural network architectures and datasets, the next chapter shows the versatility and effectiveness of the \CTC\ method described in this chapter by comparing it to state-of-the-art methods. The chapter further discusses how different architectures and parameter choices affect the contribution values. Through these investigations, the aim is to uncover patterns, correlations, and potential areas for improvement of the \CTC\ method.
% \chapter{Results \& Discussion}
\label{chapter:results}
\section{Introduction}
The previous chapter discussed an efficient way of finding the contribution of an entire region (\ie complex input feature) with respect to the classification. Building on top of this work, this chapter presents the outcomes of a comprehensive evaluation of the framework proposed in Chapter~\ref{chap:framework}, where the complex input features are selected as described in Chapter~\ref{chap:clustering} and the contribution propagation rules outlined in Chapter~\ref{chapter:REVEAL} are used for assigning a single value of relevance to each complex input feature. The methods proposed in this thesis aim to provide deep insights into the inner workings of models while delivering these insights in an easily graspable manner. To test if the methods provide a interpretable explanations with high fidelity, the explanations are evaluated both qualitatively and quantitatively. 

The novel explainability method is applied on renowned models such as VGG16~\cite{SimonyanZ14a}, VGG19~\cite{SimonyanZ14a}, ResNet50~\cite{he2015deep}, DenseNet121~\cite{huang2018densely} and InceptionV3~\cite{szegedy2015rethinking} and the evaluation is based on the widely recognised ImageNet Large Scale Visual Recognition Challenge 2012 (ILSVRC 2012) dataset~\cite{ILSVRC15}, which has been used in training these models. To facilitate a standardised and consistent analysis, the method described in this chapter is integrated within the iNNvestigate framework~\cite{inn}. This choice not only streamlines the evaluation process but also allows for a comparative analysis with the leading interpretability methods in the field.

In line with standard practices the qualitative analysis compares the \CTC\ method's contribution to classification against established attribution methods, including Input$\times$Gradient~\cite{SimonyanVZ13}, DeconvNet~\cite{ZeilerKTF10}, Guided BackProp~\cite{SpringenbergDBR14}, and LRP-$\alpha_1\beta_0$~\cite{bach2015pixel}. Arguably, showing the single value of \CTC\ makes the comparison between different features and models easier (corroborating results reported in \cite{Ribeiro0G16}). The quantitative analysis is essential to ascertain the relative standing of the \CTC\ method in the context of existing literature, particularly focusing on key interpretability properties such as input invariance and sensitivity, as detailed in Section~\ref{lit:discussion}. The evaluation of these interpretability properties is done by employing fidelity metrics to assess the accuracy of the explainability model. This ensures that the resulting method is not only more interpretable due to the reduced information shown to the user, but also retained the faithfulness of state of the art models.


\section{Qualitative Analysis}

Unique to the approach of selecting complex input features is the visualisation technique employed. It diverges from traditional heatmaps that use seismic colour patterns to represent relevance. Instead, the explanations are assigned a distinct colour to each complex feature's pixels. These colours are then reflected in a colour-coded legend adjacent to the heatmap, clearly illustrating the \CTC\ values and offering a more intuitive understanding of the model's decision-making process.

As the \CTC\ value is assigned to a complex input feature rather than on a pixel level like in other relevance methods, the way the complex input feature is selected influences the results. This chapter presents two ways of selecting the complex input features --- selecting the top features by size, and selecting the top features by the amount of relevance they have received (as described in Chapter~\ref{chap:clustering}). The former doesn't rely on any relevance distribution method to select its features, making it a more independent evaluation and a fairer comparison to other methods in the literature. The latter is particularly useful when there are many features in the input and the relevance serves as a heuristic indicating which ones may be of highest interest to examine. However, due to its dependency on the underlying relevance method for the feature selection, the results are presented separately.  

\subsection{CTC Evaluated Against Different Post-hoc Explanation Methods}

This section focuses on evaluating the performance of the \CTC method on two specific models: VGG16~\cite{SimonyanZ14a} and VGG19~\cite{SimonyanZ14a}. The goal is to compare the explanations generated by the \CTC\ method with those produced by other established post-hoc explanation methods for the same input and models.

\begin{figure}[ht!]
	\begin{center}
		\includegraphics[width=1\linewidth]{Figures/compare_qualitive.pdf}
	\end{center}
	\caption{This figure shows the analysis regarding the predicted class by VGG16~\cite{SimonyanZ14a} as computed by the selected analysers. Each column shows the visualised results for different analysers and each row shows the analyses with respect to the input sample. To the right of the CTC method the value of the contribution of the biggest objects in the input.}
	\label{Fig:quality}
\end{figure} 

Figure~\ref{Fig:quality} provides five examples classified on VGG16~\cite{SimonyanZ14a} and compares the explanations provided by \CTC\ with  Input$\times$Gradient~\cite{SimonyanVZ13}, Integrated Gradients~\cite{SundararajanTY17}, DeconvNet~\cite{ZeilerKTF10}, Guided BackProp~\cite{SpringenbergDBR14}, and LRP-$\alpha_1\beta_0$~\cite{bach2015pixel} methods. The primary advantage of the \CTC\ method lies in its ability to assign a clear, singular value of contribution to the most significant features. Unlike traditional explainability methods, which often yield noisy visual explanations, \CTC\ provides offers a clear and focused perspective on the features deemed most critical by the model.


Arguably \CTC\ not only simplifies the interpretation process but also provides a more accurate representation of what the model deems important. In contrast, earlier methods, while evolving to produce less noisy visual explanations to increase clarity, often overemphasise the importance of edges to the classification. As convolutional layers detect most (if not all edges) it is expected for them to be found relevant. In contrast, \CTC\ delineates a more accurate landscape of feature importance by pinpointing which features truly drive the classification, rather than simply identifying all detectable features. Figure~\ref{Fig:edges} shows three examples, which are classified incorrectly on VGG16 and are explained by Guided Backpropagation, \LRP\ and \CTC.

\begin{figure}[ht!]
	\begin{center}
		\includegraphics[width=0.8\linewidth]{Figures/Edges.pdf}
	\end{center}
	\caption{This figure shows the analysis on the predicted class as computed by the selected analysers. The examples shown were not correctly classified by the VGG16 model.}
	\label{Fig:edges}
\end{figure} 

Established methods do not find it easy to assign relevance to the background, as opposed to the central object and edges in the input. Figure~\ref{Fig:edges} has a set of selected examples, where given the network's classification, the class label is likely driven by the background of the input image rather than the central object. In this example the focus on edges presumably leads to incorrect assumptions. The first input image is of a swimmer next to a pool, which is classified as a ballplayer. Here, \LRP\ provides importance to the majority of edges, while \CTC\ detects the most important feature as being the grass. Similarly, when presented with an input of a framed car, VGG16 classifies the input as a monitor. Guided Backpropagation and \LRP's explanation detects most of the edges and seemingly provide most importance to the car, where as \CTC\ arguably correctly assigns the most importance to the frame itself. The final input is classified as a banister, which is likely due to the pillars at the back rather than the person in the centre of the input. Yet the explanations from Guided Backpropagation and \LRP\ attribute relevance to everything that can be detected. Figure~\ref{Fig:vgg19} provides further examples of the performance of the \CTC\ method compared to the established attribution methods on the VGG19 classifier~\cite{SimonyanZ14a}. 


\begin{figure}[ht!]
	\begin{center}
		\includegraphics[width=1\linewidth]{Figures/compare_qualitive_vgg19.pdf}
	\end{center}
	\caption{This figure shows the analysis regarding the actually predicted class by VGG19~\cite{SimonyanZ14a} as computed by the selected analysers.}
	\label{Fig:vgg19}
\end{figure} 

\subsection{CTC Evaluated on Different Networks}
This section shows the scalability and adaptability of the \CTC\ method across a range of neural network architectures. The diversity in architectural design, exemplified by models such as VGG16~\cite{SimonyanZ14a}, VGG19~\cite{SimonyanZ14a}, ResNet50~\cite{he2015deep}, DenseNet121~\cite{huang2018densely}, and InceptionV3~\cite{szegedy2015rethinking}, presents unique challenges to explainability methods. The goal is to understand how the \CTC\ method performs in different architectural contexts, considering the inherent design and operational differences of these networks. 

Figure~\ref{Fig:compare_models_same} showcases the classification of the same input across various networks, which all correctly classify the input. Crucially, the figure also compares the \CTC method explanation to the established attribution methods.

\begin{figure}[ht!]
	\begin{center}
		\includegraphics[width=1\linewidth]{Figures/methods_differnet_network_same.pdf}
	\end{center}
	\caption{This figure shows the analysis regarding the actually predicted class by different classifier as computed by the selected analysers.}
	\label{Fig:compare_models_same}
\end{figure} 

Figure~\ref{Fig:compare_models_not_same} showcases the classification of the same input across various networks, which are not all classified correctly.

\begin{figure}[ht!]
	\begin{center}
		\includegraphics[width=1\linewidth]{Figures/methods_differnet_network_notsame.pdf}
	\end{center}
	\caption{This figure shows the analysis regarding the actually predicted class by different classifier as computed by the selected analysers.}
	\label{Fig:compare_models_not_same}
\end{figure} 


\subsection{CTC of the Most Relevant Complex Input Features}

This section explores the effectiveness of the \CTC\ method when applied to the most relevant complex input features, as identified by the selection methods in Chapter~\ref{chap:clustering}. By segmenting objects and finding the contribution of the biggest objects one might miss which part of the object makes contributes the most to the classification. Finding small regions of high importance is something state-of-the-art methods excel at. The assignment of contribution to the smaller objects could help the process of pinpointing the exact part of the input that makes an object important.

Figure~\ref{Fig:small_parts} provides examples of the \CTC\ method applied on VGG16, InceptionV3 and ResNet50 where the most relevant complex input feature is selected for relevance distribution.
\begin{figure}[ht!]
	\begin{center}
		\includegraphics[width=1\linewidth]{Figures/relevant_masks.pdf}
	\end{center}
	\caption{This figure shows the CTC value attributed to the most relevant input feature}
	\label{Fig:small_parts}
\end{figure} 



\subsection{Discussion}

\subsubsection{Comparison to other methods}

A critical observation of recent post-hoc explanation methods is the tendency of certain methods, notably Layer-wise Relevance Propagation (LRP), Guided Backpropagation and GradCAM, to disproportionately emphasise the importance of edges in their heatmap representations. This propensity often leads to the overstatement of edges' relevance to the classification decision, potentially obscuring other crucial features. This phenomenon is not merely a visual artefact but a reflection of a deeper interpretative bias within these methods.

This observation aligns with the findings of Adebayo et al.~\cite{AdebayoGMGHK18}, who noted that methods like Guided BackProp~\cite{SpringenbergDBR14} and GradCAM~\cite{ShrikumarGK17} demonstrate a lack of sensitivity to the network's final layers. This insensitivity suggests that these methods are inclined to highlight features such as objects and edges identified by the network's earlier layers, which might not be directly pertinent to the classification's final outcome. 

This is further supported by the results presented in Figure~\ref{Fig:edges}, where LRP and Guided Backpropagation neglect background regions, which may be instrumental in the classification process. This oversight is particularly striking in cases where the background or less prominent features play a key role in the correct classification of the input. For example, in Figure~\ref{Fig:quality}, the bottom image is one of a Chimpanzee playing the piano. One can see that nearly all features (including the piano, a chair, ad other objects in the background) are highlighted by the majority of other methods --- leading the user to assume that nearly all features played a similar role in leading towards the classification. This might lead a user to also assume the network has simply memorised the input, rather than paying much attention to the Chimpanzee. However, this particular image is not part of the ILSVRC 2012 dataset~\cite{ILSVRC15} that the network has been trained on. In contrast \CTC\ presents an ordering of objects and features, and demonstrates that despite the edges of a piano/chair appearing in generic heatmap methods, the most important feature, by several orders of magnitude is the Chimpanzee itself. This is a clear example of the value of \CTC\ in providing a more faithful explanation, and a more interpretable explanation.

This point is perhaps better evidenced in situations of incorrect classification, as shown in Figure~\ref{Fig:edges}. All images have been incorrectly classified, however a generic heatmap does not offer a large amount of information as to \textit{why} this is the case. The tendency of generic heatmap methods to highlight edges and objects, rather than features, means that the user is left with little information as to why the network has made the wrong classification. In contrast, \CTC\ highlights features in the background, or indeed lend relevance to featureless (or edgeless) proportions of the image. An example of this is the background of the Truck/Moniter image in Figure~\ref{Fig:edges}, and on the image of a crib in Figure~\ref{Fig:quality}, which is considerably busy, however a major influence to the network (as found by \CTC) is the blanket. This is hard to spot in a generic heatmap, but the colouring of the blanket is clearly visible in the \CTC\ explanation. Indeed, the feature selection, along with clear ordering of the features provides the user with far clearer and informative explanations.

While conventional methods often struggle to distinctly prioritise the significance of numerous overlapping or closely situated features, the \CTC\ method effectively highlights the most influential features in the classification process. It does so by aggregating and representing the collective importance of feature clusters, thus providing a streamlined and coherent interpretation. By simplifying the complexity inherent in feature-rich inputs, the \CTC\ method unveils the underlying patterns and relationships that may otherwise be obscured in a more granular or dispersed representation. The ability of the \CTC\ method to offer a clear ordering between features is invaluable, especially in scenarios where discerning the relative importance of features is crucial.



\subsubsection{Comparison over different networks}

A critical observation is that the explanations provided by other post-hoc interpretability methods show minimal variation across different networks, rendering it challenging to discern the distinct functional characteristics of each classifier. Furthermore, while LRP-$\alpha_1\beta_0$~\cite{bach2015pixel} generally produces similar explanations across different models, it notably struggles with networks like ResNet50. This inconsistency is likely attributable to the network’s unique structural features, such as residual connections, which impact the method's ability to generate clear explanations.

An interesting facet demonstrated is the effect of network size on the explanations generated. The networks are ordered in Figure~\ref{Fig:compare_models_same} and Figure~\ref{Fig:compare_models_not_same} in ascending order of size, with the smallest network being VGG16, and the deepest with most parameters being DenseNet121. There seems a slight negative correlation between the amount of relevance present in other methods' heatmaps and the network size. This is a pressing problem --- network architectures are tending to grow in size in recent years. \CTC\ demonstrably maintains the amount of information provided to the user. Further, particularly in heatmaps with little information, the segmentation and ordering of relevance of features presents a welcome contrast --- for instance, in Figure~\ref{Fig:compare_models_not_same} the heatmap generated on InceptionV3 is almost entirely blank, with the exception of a small amount of information in the centre of the image. In contrast, \CTC\ extracts an explanation which, in line with \LRP\ and other methods, highlights the head of the water buffalo as the most important feature. However, \CTC\ also highlights the water, and the muddy bank, as being relevant, even if they are of considerably less importance.

The ability of \CTC\ to adapt to larger networks is desired quality, as larger networks, and in particular wider networks, like InceptionV3 are known for memorising components in the training set~\cite{nguyen2020wide}. This is a particular problem in the field of medical imaging, where the training set is often limited, and the network is prone to memorising components of the training set~\cite{alzubaidi2021novel}. A possible example of memorisation is given by the feature-map generated to explain InceptionV3's classification Saluki (a type of dog) in Figure~\ref{Fig:compare_models_same} --- where despite the correct classification being achieved, the most relevant feature is the toy dog/bear next to the Saluki.

In contrast to all other methods explanations, the \CTC\ method's approach to feature ordering provides a more straightforward and insightful analysis. Its ability to distinctly outline and prioritise features makes it easier to identify and understand the differences in the functions learned by various classifiers. This advantage becomes even more pronounced in scenarios depicted in Figure~\ref{Fig:compare_models_not_same}, where inputs are not uniformly classified across all networks. In such cases, an interpretability method like \CTC\ becomes invaluable. It aids in pinpointing differences in the learned functions between correctly and incorrectly classifying networks, offering critical insights into each model's decision-making process.


\subsubsection{Comparison over Relevant Complex Input Features}

A major influence on the quality of \CTC\ heatmaps is the method of feature selection. Most results presented in this chapter demonstrate the selection of masks selected based on size, as opposed to relevance. However, this introduces a bias towards larger features.

Figure~\ref{Fig:small_parts} presents examples of mask selection by a relevance-based ordering. The author suggests this to be the method of choice when applied to complex images with multiple features, wherein the likely cause for a classification is a small, but highly relevant feature. For instance, the top-left hand image of a shoe-store demonstrates a large amount of relevance for a single pair of shoes, which proportionally makes up a very small amount of a feature rich image.

This is further demonstrated by a miss-classification. The second image at the top-left hand of Figure~\ref{Fig:small_parts} presents a photo of a tripod in a chapel, but VGG16 miss-classifies the image as a "swab". When choosing the most relevant complex input features a clear explanation is provided for where the mistake of the reasoning of the classifier lies. A feature selection method with a bias towards larger complex input features, is likely not to have identified this crucial element.

However, this is not always the case. By selecting masks by relevance-based ordering, the shortcomings of other relevance metrics can negatively affect the performance of the \CTC\ explanation generated. For instance, the bottom left image in Figure~\ref{Fig:small_parts} fails to correctly identify the grass-hopper as the major feature in the image, instead showing a preference for smaller features which were attributed a high degree of relevance by the underlying attribution method (\LRP). The explanation in this case is still valid, as these small regions are assigned very low relevance, but it is not explaining which parts do contribute to the classifiers decision. The author suggest that in cases where the most relevant features do not provide an informative explanation the biggest complex input features may provide a better explanation.

% A key strength of the Contribution to Classification (CTC) method becomes apparent in scenarios where inputs possess a high density of features. The CTC method's unique ability to assign a singular relevance value to an entire cluster of features facilitates a clearer and more hierarchically structured interpretation of these features' importance. This aspect of the CTC method is particularly evident when analysing complex inputs, as seen in Figure~\ref{Fig:quality} and Figure~\ref{Fig:vgg19}.

% One of the most significant advantages of the Contribution to Classification (CTC) method is its capability to facilitate clear differentiation and ordered representation of features. This attribute is particularly crucial when analysing and contrasting the performance of different classifiers. In Figure~\ref{Fig:compare_models_same}, the same input is being processed by various networks, all of which correctly classify the input. Here, the figure not only showcases the classifications but also juxtaposes the explanations generated by the CTC method with those from other post-hoc interpretability methods, such as Input$\times$Gradient~\cite{SimonyanVZ13}, DeconvNet~\cite{ZeilerKTF10}, and Guided BackProp~\cite{SpringenbergDBR14}.

% In these figures, we observe inputs processed by the VGG16 model, characterised by their rich feature presence. Here, the CTC method demonstrates its prowess, offering a stark contrast to traditional attribution methods. While conventional methods often struggle to distinctly prioritise the significance of numerous overlapping or closely situated features, the CTC method effectively highlights the most influential features in the classification process. It does so by aggregating and representing the collective importance of feature clusters, thus providing a streamlined and coherent interpretation.


% This clarity in feature prioritisation is not just a matter of visual neatness; it embodies a deeper understanding of the model's decision-making process. By simplifying the complexity inherent in feature-rich inputs, the CTC method unveils the underlying patterns and relationships that may otherwise be obscured in a more granular or dispersed representation. The ability of the CTC method to offer a clear ordering between features is invaluable, especially in scenarios where discerning the relative importance of features is crucial. This characteristic makes the CTC method particularly suited for applications in which a concise yet comprehensive overview of the model's reasoning is necessary.


% A critical observation of recent post-hoc explanation methods is the tendency of certain methods, notably Layer-wise Relevance Propagation (LRP), Guided Backpropagation and GradCAM, to disproportionately emphasise the importance of edges in their heatmap representations. This propensity often leads to the overstatement of edges' relevance to the classification decision, potentially obscuring other crucial features. This phenomenon is not merely a visual artefact but a reflection of a deeper interpretative bias within these methods.

% This observation aligns with the findings of Adebayo et al.~\cite{AdebayoGMGHK18}, who noted that methods like Guided BackProp~\cite{SpringenbergDBR14} and GradCAM~\cite{ShrikumarGK17} demonstrate a lack of sensitivity to the network's final layers. This insensitivity suggests that these methods are inclined to highlight features such as objects and edges identified by the network's earlier layers, which might not be directly pertinent to the classification's final outcome. This is further supported by the results presented in Figure~\ref{Fig:edges}, where LRP and Guided Backpropagation neglect background regions, which may be instrumental in the classification process. This oversight is particularly striking in cases where the background or less prominent features play a key role in the correct classification of the input.

% One of the most significant advantages of the Contribution to Classification (CTC) method is its capability to facilitate clear differentiation and ordered representation of features. This attribute is particularly crucial when analysing and contrasting the performance of different classifiers. In Figure~\ref{Fig:compare_models_same}, the same input is being processed by various networks, all of which correctly classify the input. Here, the figure not only showcases the classifications but also juxtaposes the explanations generated by the CTC method with those from other post-hoc interpretability methods, such as Input$\times$Gradient~\cite{SimonyanVZ13}, DeconvNet~\cite{ZeilerKTF10}, and Guided BackProp~\cite{SpringenbergDBR14}. A critical observation is that the explanations provided by these other methods show minimal variation across different networks, rendering it challenging to discern the distinct functional characteristics of each classifier. Furthermore, while LRP-$\alpha_1\beta_0$~\cite{bach2015pixel} generally produces similar explanations across different models, it notably struggles with networks like ResNet50. This inconsistency is likely attributable to the network’s unique structural features, such as residual connections, which impact the method's ability to generate clear explanations.

% In contrast to all other methods explanations, the CTC method's approach to feature ordering provides a more straightforward and insightful analysis. Its ability to distinctly outline and prioritise features makes it easier to identify and understand the differences in the functions learned by various classifiers. This advantage becomes even more pronounced in scenarios depicted in Figure~\ref{Fig:compare_models_not_same}, where inputs are not uniformly classified across all networks. In such cases, a robust interpretability method like CTC becomes invaluable. It aids in pinpointing differences in the learned functions between correctly and incorrectly classifying networks, offering critical insights into each model's decision-making process.


\section{Quantitative Analysis}
In this subsection, a quantitative analysis is conducted to assess the input invariance and sensitivity of various attribution methods. The focus is on how these methods respond to alterations in the input data, which are introduced to the input through Gaussian noise, Gaussian blurring, and Uniform noise. This evaluation is crucial in determining the robustness of each method against subtle or impactful changes in the input, thereby providing insight into their reliability and consistency in different scenarios. As in the previous section to allow for independent evaluation and fair comparison to other methods in the literature, the contribution is calculated on the biggest masks as identified by SAM.  This is opposed to selecting the masks with the most relevance attributed to them. Given that the change in relevance is what is being measured, the most relevant mask may change, tying the \CTC\ method results with the relevance method used for mask selection. The goal of this analysis is to identify how faithful the \CTC\ rules are by introducing noise to the input, and measuring the variance of the output. Therefore, steps were taken to ensure that any variance in the output is as a result of \CTC\ and the underlying network.


Similarly, the quantitative analysis aims to test the fidelity of the importance value assigned to features, not the clustering technique. To evaluate the \CTC\ value independently of the clustering the biggest masks identified by SAM on the original input are fixed and when evaluated on the noisy inputs. Ultimately, it was decided not to recompute feature-masks for two major reasons: fairness and computational cost. The major consideration is fairness. All other methods have a fixed number of input features, they present one value for each pixel, always evaluated on all pixels, in the case of image classification. As such, it stands to reason that the fairest evaluation method would be to measure the difference in explanations over the same input features --- in other words, were new input features to be computed with each injection of noise, it is not trivial to identify which features have mapped to other features from which to compute the distance. Particularly given that \CTC\ can be employed on top of any existing clustering method, it stands to reasons that fixing masks, and allowing the variance in output to be derived solely from the noise and forward propagation is the most reliable way of assessing the methods robustness and reliability to noise. The second consideration, be it minimal, is that recomputing the feature masks with each introduction of noise would have increased the computational cost of the experiments. While forwards propagation is a relatively cheap operation, the cost of computing the feature masks is not, and would have substantially increased the run-time and cost of the experiments. Particularly given that the results show the evaluation of the \CTC\ method rather than the clustering technique, it was decided that the computational cost exceeded the likely benefits of recomputing feature masks.

\subsection{Input invariance}
\label{input_inv}
Input invariance describes the resilience of an attribution method against certain changes to the input that do not modify the model's output~\cite{YehHSIR19}. Essentially, a robust interpretation method should demonstrate minimal sensitivity to these alterations, meaning it should consistently provide nearly identical explanations even when slight variations are made to the input. To assess the input invariance of the \CTC\ method and how it compares the input invariance of Input$\times$Gradient~\cite{SimonyanVZ13}, Integrated Gradients~\cite{SundararajanTY17}, DeconvNet~\cite{ZeilerKTF10}, Guided BackProp~\cite{SpringenbergDBR14}, and \LRP\-$\alpha_1\beta_0$~\cite{bach2015pixel}, three types of slight variations of the input are introduced.


The first type of input variation is introduced through Gaussian noise using a function that generates noise with a mean of zero and standard deviation of 0.1, which is added to the original image. The second method applies a Gaussian blur, which smooths the image by averaging pixel values in a manner weighted by their proximity to each pixel being processed. The degree of blurring is governed by the size of the Gaussian kernel used in the process, which is $5 \times 5$ in the images generated to test the input invariance of the method. This allows for the blurring to not be too strong. Unlike the Gaussian noise added previously, Gaussian blurring affects the image in a more uniform and systematic way, altering the spatial relationships within the image without introducing extraneous pixel-level variability. The final method introduces Uniform noise. Unlike Gaussian noise, Uniform noise affects all pixels with the same probability and intensity, making it a starkly different test for the attribution method's robustness. The noise is randomly generated within a specified range, which is [-255,255] in the case of images. It is scaled by the intensity parameter, which suits as a noise level. The noise level can hold values between zero and one, where zero means no noise is added to the image and one means that the image is converted only to noise. The noise level parameter was set to 0.02. All resulting noisy images are clipped to ensure all pixel values remain within the [0,255] range, suitable for standard image formats.

\begin{figure}[ht!]
	\begin{center}
		\includegraphics[width=1\linewidth]{Figures/small_noise.pdf}
	\end{center}
	\caption{Examples of images from the ILSVRC 2012 dataset after the introduction of the three types of image alteration --- Gaussian noise addition, Gaussian blurring and Uniform noise.}
	\label{Fig:noisy_images}
\end{figure} 

These three methods of image alteration --- Gaussian noise addition, Gaussian blurring and Uniform noise --- provide distinct challenges to the attribution methods being evaluated. The noise images are generated from the ImageNet Large Scale Visual Recognition Challenge 2012 (ILSVRC 2012) data set~\cite{ILSVRC15} and examples of all three types of noisy images can be seen in Figure~\ref{Fig:noisy_images}. As one may observe, the Gaussian noise introduces pixel-level changes, which are too small for the human eye to notice. The Gaussian blurring alters the image's overall texture and detail and smooths the edges of objects present in the image. Finally, the uniform noise affects all pixels with the same probability and intensity, but as the noise level parameter is quite small it can be mostly observed in the lighter part of images. After generating the modified dataset, explanations were created for each input and its variation --- original, Gaussian blurring, Gaussian noise, and Uniform Noise --- using Input$\times$Gradient, Integrated Gradients, DeconvNet, Guided BackProp, \LRP\-$\alpha_1\beta_0$ and the \CTC\ method. This is followed by a normalisation step before comparing the original and modified inputs, ensuring the sum of explanations equals one. This is especially important for the \CTC\ method, which tends to assign larger values to regions compared to the individual pixel attributions by other methods. The three types of noise present a comprehensive evaluation of how well the \CTC\ method, along with others like Input$\times$Gradient, Integrated Gradients, DeconvNet, Guided BackProp, and \LRP\-$\alpha_1\beta_0$, can maintain consistent interpretations in the presence of subtle input variations. Such resilience is key in ensuring reliable and robust model explanations.


The comparative results, shown in Figure~\ref{Fig:norm}, Figure~\ref{Fig:blur} and Figure~\ref{Fig:uniform}, are based on 500 distinct original inputs. When measuring the input invariance the results show the difference between original inputs explanation and the noisy input explanation only for the noisy images that do \emph{not} change the classification. As input invariance measures the change in explanations when the input has a small change, here if the noisy image changes the classification the noise added is not considered a small change. The distance between the original input's and the noisy input's explanations is calculated using Euclidean distance. Given two $n$-dimensional vectors \( \vec{v}_1 = (v_{1,1}, v_{1,2}, ..., v_{1,n}) \) and \( \vec{v}_2 = (v_{2,1}, v_{2,2}, ..., v_{2,n}) \), where $\vec{v}_1$ is the original image explanations and $ \vec{v}_2$ is the noisy input explanations, the Euclidean distance \( d \) between them is given by:
\begin{equation*}
    d(\vec{v}_1, \vec{v}_2) = \sqrt{\sum_{i=1}^{n} (v_{1,i} - v_{2,i})^2}
\end{equation*}
Here, the summation sums up the squares of the differences of the corresponding components of the two vectors, and the square root is taken of the sum. In all figures the top graph presents the variability of all six methods, bellow a description of each distance vector $\vec{d}$ shows the mean, standard deviation, minimum, 25\%, 50\%, 75\% and the maximum value of each distance vector $\vec{d}$. The bottom left graph shows a zoomed version of all six methods, which ignores the outliers in some methods, such as Guided Backpropagation and makes it easier to see the mean and the standard deviation of the majority of distances. The bottom right graph shows the distances between the original explanations and explanation for the noisy image only for \LRP\ and the \CTC\ methods. As the distance between the original explanations and explanation for the blurry image for those two methods is very small, it is hard to see their mean and standard deviation. This zoomed graph makes it easier to see this information.  

\subsubsection{Gaussian Noise ($\mu = 0, \sigma = 0.1$)}
Figure~\ref{Fig:norm} displays the effects of Gaussian noise. The figure highlights the robustness of specific methods like \LRP\-$\alpha_1\beta_0$ and \CTC, which manage to maintain consistent interpretations in the presence of such noise. Note that out of the 500 inputs this change was tested on 477 noisy inputs which did not result in a change in classification. 
\begin{figure}[ht!]
	\begin{center}
		\includegraphics[width=0.85\linewidth]{Figures/minor_noise_normal_stats.pdf}
	\end{center}
	\caption{The figure highlights how consistent, low-intensity noise, such as the one introduced by the Uniform Noise challenges the robustness of various attribution methods.}
	\label{Fig:norm}
\end{figure} 
\subsubsection{Gaussian Blur ($k = (5 \times 5)$)} The distances between the original explanation and the explanation of the input, which has a small amount of Gaussian blur is shown in Figure~\ref{Fig:blur}. 
\begin{figure}[ht!]
	\begin{center}
		\includegraphics[width=0.85\linewidth]{Figures/minor_blur_stats.pdf}
	\end{center}
	\caption{The figure shows the ability of methods to maintain consistent interpretations in the presence of Gaussian blur.}
	\label{Fig:blur}
\end{figure} 
This blurring, while only subtly altering image details, challenges the attribution methods to maintain accurate interpretations. The figure demonstrates varying degrees of resilience among the methods, with some, like \CTC, effectively preserving their interpretative accuracy despite the blurring. Note that out of the 500 inputs this change was tested on 429, which did not result in a change in classification.
\subsubsection{Small Uniform Noise ($l = 0.02$)}
Figure~\ref{Fig:uniform} focuses on the implications of Uniform noise, which uniformly alters all pixels but with a low intensity, especially affecting lighter image regions. 
\begin{figure}[ht!]
	\begin{center}
		\includegraphics[width=0.85\linewidth]{Figures/minor_noise_uniform_stats.pdf}
	\end{center}
	\caption{The figure highlights how the Uniform Noise challenges the robustness of various attribution methods.}
	\label{Fig:uniform}
\end{figure} 
\newpage
\subsection{Sensitivity}
Sensitivity in interpretability methods refers to how methods react to significant changes in input data. It's crucial that these methods provide distinctly different explanations when the model's classification alters~\cite{NielsenDRRB22}. This section evaluates the sensitivity of the \CTC\ method by comparing it with several other approaches: Input$\times$Gradient~\cite{SimonyanVZ13}, Integrated Gradients~\cite{SundararajanTY17}, DeconvNet~\cite{ZeilerKTF10}, Guided BackProp~\cite{SpringenbergDBR14}, and LRP-$\alpha_1\beta_0$~\cite{bach2015pixel}. To conduct this comparison, the same three significant types of input variations are introduced, but with much higher levels of noise than the ones when testing input invariance.

The first variation adds Gaussian noise with a much higher standard deviation ($\sigma$) than previously used for assessing input invariance. In this case, the standard deviation $\sigma$ is set at 50, compared to the standard deviation of 0.1 used for input invariance evaluation. The second variation creates a blurred version of the original input. The degree of blurring is governed by the size of the Gaussian kernel used in the process, which is $17 \times 17$ for the sensitivity test, in comparison to the $5 \times 5$ Gaussian kernel used for input invariance. The final variation introduces Uniform noise at a level of 0.3, again substantially higher than the 0.02 level used in the input invariance tests. These increased noise levels are designed to rigorously test the sensitivity of the \CTC\ method against state-of-the-art methods. The parameters are set arbitrary to lead to a change in the classification, while still preserving part of the original image.

To visually illustrate the impact of these input variations, Figure \ref{Fig:noisy_images} presents examples of images from the ILSVRC 2012 dataset after the introduction of the three types of significant image alterations. These images serve as a clear representation of the dramatic changes in input data, providing a concrete basis for assessing the sensitivity and robustness of the \CTC\ method in comparison to established techniques. The differences observed in these images are crucial for understanding how each method responds to substantial variations in input. As one may observe, the introduction of blur, Gaussian noise, and Uniform noise to the ILSVRC 2012 dataset images results in significant visual alterations, but the inputs are still somewhat recognisable. After generating the modified dataset, explanations were created for each input image variation --- original, Gaussian noise, Gaussian blurring and Uniform Noise --- using Input$\times$Gradient, Integrated Gradients, DeconvNet, Guided BackProp, LRP-$\alpha_1\beta_0$ and the \CTC\ method.  Similarly to when testing input invariance, the explanations are normalised ensuring the sum of explanations equals one. The three types of noise present a comprehensive evaluation of how well the \CTC\ method, along with others like Input$\times$Gradient, Integrated Gradients, DeconvNet, Guided BackProp, and \LRP\-$\alpha_1\beta_0$, can reflect big changes in the input. Showing a significantly different explanation when the input has been significantly changed is is key in ensuring reliable model explanations. 

\begin{figure}[ht!]
	\begin{center}
		\includegraphics[width=1\linewidth]{Figures/big_noise.pdf}
	\end{center}
	\caption{Examples of images from the ILSVRC 2012 dataset after the introduction of the three types of big image alteration --- Blur ($k = (17 \times 17)$), Gaussian noise addition ($\mu= 0; \sigma=50$) and Uniform noise ($l = 0.3$).}
	\label{Fig:noisy_images}
\end{figure} 

The comparative results, shown in Figure~\ref{Fig:big_gaus}, Figure~\ref{Fig:big_blur} and Figure~\ref{Fig:big_noise_uniform}, are based on the distance between 500 distinct original inputs and their noisy counterparts. As highlighted in Section~\ref{input_inv} certain methodologies exhibit significant alterations in their explanations even with minimal noise introduction. This section aims to evaluate the relative sensitivity of these methods by examining the variance in explanations between examples where noise did not alter the classification and those where it did. This approach allows for an assessment of explanation changes in relation to a baseline alteration for negligible changes. Each figure presents the variability of all six methods between the inputs that changed the classification and the ones that did not. To validate if the difference is statistically significant, a Welch's t-test was performed~\cite{welch1947generalization}. Welch's t-test, also known as the unequal variances t-test, is a statistical test used to compare the means of two groups to determine if they are significantly different from each other. It is an adaptation of the standard Student's t-test~\cite{student1908probable} and is more reliable in real-world scenarios where the assumption of equal variances between groups is often violated.


\subsubsection{Gaussian Noise Addition($\mu = 0, \sigma = 50$)}
The distances between the original explanation and the explanation of the input, which has a big amount of Gaussian Noise added is shown in Figure~\ref{Fig:big_gaus}. This blurring alters image details and should result in attribution methods generating explanations different from the original input.

% Figure~\ref{Fig:noise} displays the effects of Gaussian noise. This type of noise introduces a fine level of randomness, which can potentially mislead attribution methods. The figure highlights the robustness of specific methods like LRP-$\alpha_1\beta_0$ and \CTC, which manage to maintain consistent interpretations in the presence of such noise.

\begin{figure}[ht!]
	\begin{center}
		\includegraphics[width=0.95\linewidth]{Figures/big_gaus.pdf}
	\end{center}
	\caption{The figure shows the challenge big amount of Gaussian noise addition poses to various attribution methods.}
	\label{Fig:big_gaus}
\end{figure} 

\subsubsection{Gaussian Blur ($k = (17 \times 17)$)}
The distances between the original explanation and the explanation of the input, which has a big amount of Gaussian blur is shown in Figure~\ref{Fig:big_blur}. This blurring alters image details and should result in attribution methods generating explanations different from the original input.



% demonstrates varying degrees of resilience among the methods, with some, like \CTC, effectively preserving their interpretative accuracy despite the blurring. Note that out of the 500 inputs this change was tested on 429, which did not result in a change in classification. 
% The top graph 
\begin{figure}[ht!]
	\begin{center}
		\includegraphics[width=0.95\linewidth]{Figures/big_blur.pdf}
	\end{center}
	\caption{The figure shows the challenge a very blurry input poses to various attribution methods.}
	\label{Fig:big_blur}
\end{figure} 

\subsubsection{Uniform Noise Addition($l = 0.3$)}
The distances between the original explanation and the explanation of the input, which has a big amount of Uniform Noise added is shown in Figure~\ref{Fig:big_noise_uniform}. 


\begin{figure}[ht!]
	\begin{center}
		\includegraphics[width=0.95\linewidth]{Figures/big_noise_uniform.pdf}
	\end{center}
	\caption{The figure shows the challenge big amount of Uniform noise addition poses to various attribution methods.}
	\label{Fig:big_noise_uniform}
\end{figure} 

\newpage

\subsection{Discussion}

The key objective of this chapter is to demonstrate that the novel interpretability method, which gives a single value of contribution to the classification (\CTC) to each identified complex input feature, is capable of providing deep insights into the inner workings of models while delivering these insights in an easily graspable manner. A particular concern is that the simplification of evaluating entire features might negatively harm fidelity. To address this, we employed fidelity metrics to compare the performance of \CTC\ against established interpretability techniques. We focused on two essential attributes: \emph{sensitivity} and \emph{input invariance}. Both of these qualities were assessed by introducing noise to the input and observing the resultant variations in the explanations provided. An ideal interpretability method should exhibit low sensitivity to minor noise perturbations, thereby demonstrating input invariance, particularly under the assumption of a robust network. Conversely, a substantial introduction of noise should lead to a noticeable change in explanation. 

The \CTC\ method distinctively assigns a single value to a broad region, inherently minimising its susceptibility to significant changes in response to noise. This stands in marked contrast to other methods, which assign values to individual pixels and, as a result, have a broader scope for discrepancies between the explanations of the original and noise-affected inputs. However, this approach also implies a caveat: if even a small segment of the chosen complex input feature significantly influences the \CTC\ value, it leads to the attribution of an extreme value over a larger area. Such instances can result in notably greater differences in certain explanations compared to other methods, which would only reflect changes in the limited region that has an extreme importance.

Continuing this line of thought, it's noteworthy that overall, the \CTC\ method demonstrates the smallest mean difference between the original and noise-influenced explanations. In some instances, particularly with specific types of noise, this mean difference is over ten times smaller than that of Layer-wise Relevance Propagation (\LRP), which is the second most input invariant method after \CTC. While there is a marginally higher standard deviation observed in \CTC, owing to instances where a small part of a larger feature has a disproportionate impact on the \CTC\ value, these variations remain relatively minimal. The three types of noise and comprehensive evaluation to other methods evidence that the \CTC\ method surpasses others in terms of input invariance, marking it the most stable method in this respect within current literature. 

When accessing the input invariance of the techniques, three types of noise were used: Gaussian noise, small uniform noise, and small blur. The results of only the noisy inputs that did not change the classification are taken. However, the amount of images that did not change the classification was not equal for each type of noise. This indicates that the network is more sensitive to some types of noise than others. In the VGG16 case the biggest impact was seen by small Uniform noise, followed by blurring and Gaussian noise. Further, the impact of the noise to the network could be seen by the change in explanation by \CTC\ and \LRP. The change in explanation was the highest for small uniform noise, followed by blurring and then small Gaussian noise. This was not the case for other methods such as Input $\times$ Gradient, Integrated Gradients, and Guided Backprop and DecovNet. The fact that both \LRP\ and \CTC\ more closely reflect the network's sensitivity to noise further supports the argument that they present more faithful explanations than other methods.


The sensitivity of an interpretability method is assessed by measuring how much its explanation fluctuates with the introduction of noise. This chapter aims to provide a nuanced understanding of these fluctuations. We assess the significance of large changes in explanations by establishing a baseline of minor changes, which serves as a reference point for evaluating the validity of a single explanation. The relevance of a major alteration in the input becomes evident only when contrasted with minor, inconsequential changes. For instance, if an insignificant modification in the input, one that doesn't alter the model's classification, leads to a substantial shift in the explanation, it implies that a major input change would almost certainly result in a significant explanation shift. To contextualise sensitivity, we compare the explanations of inputs with \emph{insignificant} changes --- largest amount of noise that does not affect the classification --- and \emph{significant} changes --- those where the introduced noise does alter the classification, and thus, should also trigger a marked change in the explanation. This approach is part of a broader conversation about the efficacy and intent of these metrics, which is further explored in Chapter~\ref{evl}.

The results clearly indicate that \CTC\ presents a statistically significant change between the explanations with a significant and insignificant noise for all three types of noise presented. Notably, blurring led to the biggest change in input classification. This is expected as it significantly reduces fine details and alters edge information, which are crucial for the network to make it's predictions. In contrast, Gaussian and Uniform Noise tend to add random pixel-level variations that do not fundamentally alter the underlying structures and edges as drastically as blurring does. In this type of noise the integrated gradients and Input $\times$ Gradient were disproportionately effected compared to distance in explanations using Gaussian Noise and Uniform Noise. Gradient methods are sensitive to spatial information and take the gradient with respect to the input. Blurring disrupts the spatial relationships between pixels by averaging them over a region. This smoothing effect can drastically change their explanations, leading to higher sensitivity. As discussed, to evaluate the \CTC\ value independently of the clustering the biggest masks identified by SAM on the original input are fixed and when evaluated on the noisy inputs used without recomputing. This injects more ordered information, as the mask often are dictated by where there used to be edges in the input. The \CTC\ algorithm is therefore less effected by the smoothing operation than Gaussian Noise and Uniform Noise. That being said, it is still statistically different from the inputs that do not change the classification. Further, the \CTC\ method has the second best sensitivity after \LRP\ on inputs with Gaussian Noise and Uniform Noise. 


In summation, it is evident that a quantitative analysis of empirical results demonstrates that \CTC\ exhibits the qualities of sensitivity and input invarience. Indeed, even despite the considerably simplified input, the method demonstrably maintains fidelity to a degree comparable to other leading methods of interpretability. In light of the goals specified by the thesis (simplifying the explanations to reduce the cognitive load upon the explainee while maintaining fidelity), it is fair to conclude that \CTC\ meets the criteria of preserving faithfulness, despite the clear reduction in explanation dimensionality and granularity. Therefore, under the assumption that reducing the number of input features by clustering/segmenting the input space into relevant features does simplify the explanation, the results provide evidence that the research objectives have been met.  


% \section{Conclusion}


    % \item Throughout the results, images were included where the introduction of noise did not result in a change in the classification --- however, one can see that the average change in explanation was \textit{larger} than in comparison to a different type of noise

% \subsection{A meta-analysis of the experimental process}
% There are number of potential reasons why this is a shitty way of evaluating results.
% \begin{itemize}
%     \item Variance/sensitivity of metrics with respect to the size of the features being evaluated. E.g. in the case where there is already noise in the image? 
%     \item When does small noise turn into big noise -- what is the cut-off point where you'd like to see a big change in the explanation as opposed to a small change in the explanation? And how does this relate to the injection of noise? Would the injection of other features make more sense (is it still a ball player if there's an alien and a dog going heels-to-jesus in the background?)
%     \item Variance in the explanations with respect to the variance in the classification perhaps? 
%     \item Distance metrics --- there is some evidence to suggest that the traditional way of measuring distance between two explanations (traditional difference metrics) is not the best way of doing it. E.g. An increase in brightness might be a massive change in euclidean space, but in feature-space, this could be very small. 
%     \item Discussion over the differences between types of noise? We hvae noted that some networks are more sensitive to some types of noise than others. Why would this be the case? 
%     \item Discussion of computational cost in generating explanations? 
% \end{itemize}


% degree of input invariance between \CTC\ and Input$\times$Gradient, Integrated Gradients, DeconvNet, Guided BackProp, and LRP-$\alpha_1\beta_0$. The distance metric used to 


% The 















% employs a function to introduce uniform noise to the images. Unlike Gaussian noise, uniform noise affects all pixels with the same probability and intensity, making it a starkly different test for the attribution method's robustness. The noise is randomly generated within a specified range, which is -255 and 255 in the case of images. It is scaled by the intensity parameter, which suits as a noise level. The noise level can hold values between zero and one, where zero means no noise is added to the image and one means that the image is converted only to noise. The noise level parameter was set to 0.3 The result is then clipped to maintain valid pixel values.



% To asses the input invariance, saturation handling, and sensitivity of the \CTC\ method 

% The 
 

% The \CTC\ value of each complex feature is then found by propagating it through the network starting from the input layer until the classification is reached. We colour all the pixels that belong to each complex feature in one colour and then use that colour in the colour-coded legend next to the heatmap to show the \CTC\ value as a percentage of the overall classification. This deviates from traditional heatmaps, where seismic colours are used to show positive relevance with the intensity of red pixels, and negative relevance with the intensity of blue.
% \begin{figure}[t!]
% \includegraphics[width=\linewidth]{Chapters/Results/contribution_vs_relevance.pdf}
% 	\caption{Examples showcasing the difference between relevance and the \CTC\ value (contribution) of a feature.}
% 	\label{fig:differnce}
% \end{figure}
% % We argue that the single value of \CTC\ makes the comparison between different features easier. For example in Figure~\ref{fig:all}, when looking at \REVEAL\/'s explanation, each stocking (second row from the top) has a  \CTC\ value in isolation. One in particular (the central stocking) contributes far more than those on either side. This distinction cannot be drawn from any other technique and presents a visible difference even when each stocking is not clearly separated, as when clustering on top of \LRP\/. A similar phenomenon can be seen for the bottom plot of \ref{fig:all}, which is classified as a sports car, with the windscreen seemingly playing an equal part to the front wheel-arch and headlamp in all benchmarks when, in fact, the \CTC\ value is far greater for the wheel -- this is seen when detecting clusters on top of both Input$\times$Gradient and LRP-$\alpha_1\beta_0$.

% In Figure~\ref{fig:differnce} we show more examples that expose the difference between relevance and contribution. We have found that our technique is particularly useful when the input has features that are from two different classes that the network recognises. This brings back the point of relevance based techniques recognising edges that are relevant, but do not contribute much to the classification, as mentioned in section~\ref{subsection:contribution}. In the first example of Figure~\ref{fig:differnce} the classifier labels the input as matchsticks, \LRP\ finds both the matchsticks and the candle relevant, while putting more relevance on the candle. In contrast, \REVEAL\ shows that the matchsticks contribute more to the classification. This can also be seen in third example, where \LRP\ gives almost no relevance to the feature that contributes the most, as presented by \REVEAL\/.  

% The \CTC\ value as a percentage of the overall classification shows us how much of the classification can be explained by the feature alone. Note that the percentages don't total 100\%. As mentioned in Section~\ref{subsection:faithfulness}, this is intentional, as two distinct features may contribute to the same hidden neurons in the network and therefore have an overlapping contribution to the classification.

% As the clusters formed and evaluated by \REVEAL\ are concrete and have a single value of relevance, there is a far smaller chance of inconsistency of the interpretability. This property, as mentioned in section~\ref{subsection:contribution}, requires different users to comprehend the same information from a single explanation. While quantitative analysis is challenging as users find different explanations useful, \REVEAL\ is the only system that has predetermined and quantified complex features making it the only one that achieves that property. 
% \section{Conclusion}

% \include{Chapters/IdentifyingRelationships}
% \include{Chapters/LabelingFeatures}
% \chapter{Professional Issues}
\chapter{Conclusion and Future Work}
\label{chapter:conclusion}
\section{Implications of the Research}
In this thesis, a significant gap in the field of explainability of deep neural networks is investigated: the need for interpretability methods that increases the ease of interpretability, while preserving faithfulness. This thesis proposes a methodology for finding complex input features (see Chapter~\ref{chap:clustering}) and two different approaches for assigning a single value of relevance to them (see Chapter~\ref{chapter:revLRP} and Chapter~\ref{chapter:REVEAL}). This clustering reduces the cognitive load typically associated with interpreting the vast array of features that DNNs find relevant. Further, this thesis goes beyond traditional binary interpretations of complex input features (`important' or `not important'), which, while easy to understand, often lack fidelity to the actual workings of the model due to the complex non-linear function learned by the model. Instead, the new framework provides a single importance score for each feature cluster, which offers higher fidelity to the model's predictions and allows for a \emph{ranking} of the importance of feature clusters, which adds another layer of interpretability. This approach presents a manageable volume of information, allowing for an intuitive understanding without losing the nuanced detail of each feature's contribution to the model's decision-making process.

The implications of the proposed framework and the methods that implement it amount to a theoretical and empirical contribution. Reverse relevance distribution tracing in Chapter~\ref{chapter:revLRP}, while computationally intensive, offers a solution to practically attribute relevance to specific features. Further, forward pass retracing through contribution to classification (\CTC\/) emerges as a more efficient and scalable alternative, particularly suitable for larger and more complex networks. The findings from the evaluation of forward pass retracing performed on the isolated complex features reveal that the method's fidelity is not compromised despite the simplification of the explanation. 

\section{Discussion with Respect to the Wider Field}
Chapter~\ref{chap:lit} describes relevant literature in the field of explainable AI. Relevant methods and techniques were divided into a number of categories. \CTC\/ falls into the categories of \textit{post-hoc} and \textit{propagation-based} explainable AI techniques.  However, the framework encompassing heatmap clustering and \CTC\/ described in this thesis extends beyond traditional propagation-based techniques. Clear parallels can be drawn to other methods, which are discussed below.

Conceptually, the overarching framework described could be seen as a post-hoc attention-based technique. Attention-based techniques typically enforce that a model learns to focus on specific parts of the input. Rather than enforcing attention during learning, \CTC\/ enforces that the explanation focuses on specific parts of the input during explanation generation. In particular, Xaio et al. \cite{XiaoXYZPZ15} introduce a framework which uses a combination of attention on certain `patches' of the input, derived from a bottom up proposal, and then uses top-down filtering (i.e. introducing some prior knowledge) in order to make a classification. The resulting classifier uses clear features to make a decision making interpretable. This technique however relies on application during training, and cannot be applied post-training to any generic CNN. Despite this difference, there are clear parallels here to \CTC\/. In essence, one could interpret \CTC\/ as a post-hoc counterpart to \cite{XiaoXYZPZ15}'s framework, where the initial heatmap generation proposes candidate patches (the bottom-up approach), and the use of SAM or clustering replaces the presence of top-down/prior knowledge to select meaningful features. Moving beyond conceptual similarities to \cite{XiaoXYZPZ15}'s work, it seems fair to refer to \CTC\/ as an attention-based explanation, as it fundamentally draws attention to distinct features, which are selected by the clustering mechanism --- relying less on any anthropomorphic interpretation of features of traditional pixel-wise explanations provided by propagation-based techniques.


Unlike many other fields of AI research, explainable AI is still in its infancy, and as such questions over methodology and evaluation are open. This is primarily driven by the lack of an objective measure of ground-truth. Fidelity refers to how accurate an explanation is at illustrating the decision-making process used by the model. It is noted repeatedly throughout the literature that it is a desired quality, and yet there is no way to directly measure fidelity.  In practice, it is evaluated by adding noise to the input as in Section~\ref{sec:results} of the \CTC\/ chapter or by perturbation-based methods, which remove part of the input and monitor the resulting change in the output. Yeh~\textit{et. al.}~\cite{YehHSIR19} note that simple removal of pixels is not a reliable measurement of fidelity, as omitting parts of the input is not an accurate measurement of the importance of features. To understand the contribution of a small part of the input, propagation rules that are true to the models functioning during the original instance inference are needed. The rules defined in Chapter~\ref{chapter:REVEAL} propose a more faithful way in which this can be achieved than simply performing inference on the modified input. Ultimately, there is a need for novel metrics to accurately assess fidelity and research into forward propagation techniques is likely to provide an alternative fidelity measure. 

Ribiero's LIME \cite{Ribeiro0G16} builds a local surrogate model, which is trained based on the removal of parts from the input and monitoring the resulting change in the output. This model can then be queried for the importance of complex input features. This surrogate model demonstrates a loss of fidelity, as discussed by~\cite{YehHSIR19}. However, LIME similar to the method combining identification of complex input features and \CTC\/ described in this thesis --- indicates a relevance score by region or segment, as opposed to pixel-level. Indeed, \cite{Ribeiro0G16} reports studies with human subjects including both graduate ML students and non-technical subjects from Mechanical Turk, which demonstrate that explanations of this sort are preferred by users, and are able to lead to better decision-making in model-selection, feature-engineering and identifying irregularities in model-classification. Given that the explanations of the method, combining identification of complex input features and \CTC\/, are functionally equivalent to LIME from the user's perspective, it stands to reason that \CTC\/ would also be preferred by users, while also offering a more computationally efficient, and crucially more faithful explanation.

Computational cost is a considerable factor in any explanation method. While \CTC\/ employs only a single forward to generate explanations, the majority of computational complexity and time is spent in computing the feature masks. In terms of overall time required for an explanation, this renders \CTC\/ as fast as most gradient based methods~\cite{SimonyanVZ13, SimonyanVZ13, SpringenbergDBR14, bach2015pixel, SelvarajuCDVPB20, SelvarajuCDVPB20, ChattopadhyaySH18, abs-1908-01224, SmilkovTKVW17}. Even by taking the feature masks generation into account the computational time required is still far less than Integrated Gradients~\cite{SundararajanTY17} and SmoothGrad~\cite{SmilkovTKVW17}'s, which require 50 to 200 steps to compute an explanation or methods like SHAP~\cite{LundbergL17} and LIME~\cite{Ribeiro0G16}, which require numerous perturbations. It is pertinent to note that the \CTC\/ implementation at present is not optimised. The author expects an optimised implementation to be capable of achieving a speed-up of 10--20 times faster than the current implementation and a speed-up of up to 52 times faster than the current implementation of the clustering. While this is likely to still fall behind some methods with respect to computational cost, this is likely to further increase the usable capacity of \CTC\/ with feature selection.



\begin{table}[h!]
\small
\centering
\begin{tabularx}{\textwidth}{cXXXXXXXXX}
\hline
\textbf{Method} & \textbf{Input Invariant} & \textbf{Handles Saturation} & \textbf{Input Sensitive} & \textbf{Fast Generation} & \textbf{Easy Interpretability}\\
\hline
Saliency Maps & no & no & yes  & yes & no\\
Input$\times$Gradient & no & no & yes  & yes & no\\
SmoothGrad & partially & partially & yes  & no & no\\
Guided Backprop. & no & no & yes  & yes & no\\
Grad-CAM & yes & partially & yes & yes & no\\
Integrated Grad. & no & yes & yes  & no & no\\
DeepLIFT & partial & partially & yes  & yes & no\\
LRP & yes & yes & yes  & yes & no\\
Relevance Tracing & yes & -- & yes  & no & yes\\
\CTC\/ & yes & yes & yes &  yes & yes\\

\hline
\end{tabularx}
\caption{Comparison of explanation methods across different properties.}
\label{tab:comparison_2}
\end{table}

Table~\ref{tab:comparison_2} highlights the key desiderata outlined in the literature review, with a focus on Relevance Distribution Tracing and \CTC\/ methods. Both \CTC\/ and Relevance Tracing produce interpretable explanations, but they differ significantly in practicality. While Relevance Tracing is computationally demanding to the point of being impractical for many applications, \CTC\/ achieves comparable interpretability with computational efficiency akin to other distribution-based methods. 

Faithfulness is evaluated through properties such as input invariance, saturation handling, and input sensitivity. Both \CTC\/ and Relevance Tracing are input invariant and sensitive, aligning with the high standards of faithfulness. However, as Relevance Tracing effectively reverses the LRP process, it inherits LRP's strengths and limitations. As shown in Section~\ref{sec:results}, LRP and \CTC\/ consistently demonstrate high sensitivity across various noise types. They reliably distinguish between scenarios where classification changes occur and those where they do not. LRP is particularly reactive to Gaussian blur due to its dependence on edge features, while \CTC\/ balances global and local feature sensitivity, making it robust against both random noise (Gaussian and Uniform) and structural changes (blur).

All methods in Table~\ref{tab:comparison_2} are marked as input-sensitive, as they exhibit statistically significant responses to large noise levels compared to minor perturbations. However, CTC and LRP stand out with the most pronounced differences in explanation quality between cases of output changes and minor input variations. Saturation handling, while not explicitly tested, is another key criterion. Since \CTC\/ avoids gradients in its computations, it is inherently resistant to saturation issues. In contrast, Relevance Tracing, which relies on Jacobians, may be susceptible to this limitation. These distinctions further underscore the versatility and effectiveness of \CTC\/ for faithful and computationally efficient explanations.

A significant limitation of any approach that presents relevances as a group of features—whether through heatmaps or other representations—is their inability to address high-level concepts or reasoning. While these methods effectively highlight areas of importance in the input, they do not inherently capture abstract relationships or broader contextual understanding. For instance, in medical diagnosis, distinct features like certain biomarkers or imaging characteristics may be associated with multiple outcomes, and a diagnosis might still be accurate even when none of the identified features is explicitly present in the input. This limitation underscores a gap in current approaches, which often focus on feature-level relevance without delving into the reasoning processes that underpin complex decision-making.

To bridge this gap, methods that construct relationships on top of feature relevances—such as causal models or graph-based reasoning—can be employed~\cite{abs-1911-10500, heinzedeml2017causalstructurelearning, DBLP:journals/corr/abs-1904-12584}. These methods can encode and interpret dependencies between features and outcomes, potentially addressing situations where features are shared across outcomes or where outcomes arise independently of specific features. In Chapters 6 and 7 of this thesis, concrete results demonstrate the utility of feature clustering and post-hoc interpretability frameworks in addressing certain complexities, providing a foundation for extending the capabilities of current methods. These contributions are directly relevant to promising areas of future work, particularly in addressing contemporary concerns regarding the reasoning capabilities of large language models (LLMs). The increasing deployment of LLMs in domains requiring high-level reasoning and contextual understanding amplifies the need for methods that go beyond feature-based explanations to include reasoning frameworks.

\section{Future Work}
The advancements made through the unified framework and the techniques of feature isolation, reverse relevance distribution tracing, and forward pass retracing, open several options for future research. In this section a number of possible research avenues are discussed with motivations and possible limitations for each.

\subsection{Feature Isolation}
The feature isolation approach defined in Chapter~\ref{chap:clustering} integrates object detection with advanced heatmap-based clustering to isolate and analyse features in neural network models. This methodology holds significant potential for simplifying the explanations of different relevance methods. Even without the integration of the proposed methods for assigning a single value of contribution to the features identified, merely taking the sum of the relevances withing the feature can provide a more easily comprehensible explanations. There are several promising directions for further research in this area.

One key area is the exploration of how various types of attribution methods can be effectively combined with object detection techniques. Different types of attribution methods have a different spread of relevances, different mean and median of intensity, some have only positive relevances, while some have negative relevances too. By clustering the heatmaps generated from different attribution methods, there is an opportunity to experiment with different clustering parameters such as the distance metric or the threshold for cluster formation. In the presence of negative contributions one could split the clustering in two, where one part clusters the negative contributions, while another clusters the positives, or take the absolute value and only consider the intensity of the relevance. Patterns and relationships within the data that might be detected by the relevance propagation methods may remain hidden, if not analysed in detail after the explanation is generated. 

Additionally, investigating different clustering approaches other than DBSCAN as used in Chapter~\ref{chap:clustering} can offer further advancements in this field. Techniques like hierarchical clustering, density-based clustering, or graph-based clustering could provide alternative perspectives on the data. It is important to consider the input type of the data when choosing the clustering mechanism. When dealing with images as input, the number of clusters is unknown and the shapes of the clusters are not spherical. Therefore clustering techniques that have a centroid point and a predefined number of clusters are not appropriate. However, if the task is better defined, such as separating background noise from not background noise, a clustering technique that has a predefined number of classes might be more appropriate. In some cases isolated features may be part of a bigger feature, if the extraction of such relationships is important, the use of hierarchical clustering could provide this extra information, allowing the inspection of features at different levels of the clustering process. Each of these methods has unique strengths and can be particularly suited to different types of input data, different applications and different neural networks.

\subsection{Reverse Relevance Distribution}
\label{dis:jac}
The proposed reverse relevance distribution tracing in Chapter~\ref{chapter:revLRP} could significantly contribute to the interpretability of neural networks. The core issue with the method lies in its intensive computational requirements. This method demands a significant amount of memory and processing power due to its reliance on computing Jacobian matrices. This computation is both memory-intensive and time-consuming, posing a barrier to its application in larger, more complex networks and applications where an explanation on the fly is needed. To address these challenges, it is worth exploring innovative approaches that maintain the depth and fidelity of interpretations while reducing computational overhead. One such approach is to develop techniques that bypass the need for Jacobian calculations, aiding in the reduction of the memory load. This could involve leveraging alternative mathematical frameworks or simplifying the relevance tracing process while retaining its interpretative power.

% One promising avenue to explore is the use of approximation methods for computing Jacobians. Finite-difference methods are a class of technique that transform ordinary differential equations (ODEs) or partial differential equations (PDEs), which might be nonlinear, into a set of linear equations. This conversion enables the use of matrix algebra methods for solving these equations. With the power of modern computers, executing these linear algebra operations is highly efficient, therefore making the approximate Jacobian computation possible. It is important to consider the accuracy and reliability when deploying this solution for finding the relevance of a feature. Finite-difference methods are subject to two primary types of errors: round-off error and truncation error, also known as discretisation error. The round-off error occurs due to the inherent limitations of computers in representing decimal numbers. Computers can only store a finite number of digits, and as a result, they often round these numbers to fit their storage capabilities. This rounding can lead to small inaccuracies in calculations, known as round-off errors. These errors accumulate over successive computational steps (often the case in deep neural networks), sometimes significantly impacting the final result. The truncation error arises from the method itself. Finite difference methods approximate continuous differential equations by discretising them – breaking them down into a finite number of intervals or steps. The exact solution of the original differential equation and the solution obtained through this discretised approximation will not perfectly align. This deviation is called the truncation or discretisation error. It's the difference between the true, continuous solution and the solution as calculated using the finite difference method, assuming that there are no round-off errors. The size of this error depends on how the differential equation is discretised and the size of the steps taken in the approximation.


One promising avenue to explore is the use of sparse approximation, which seeks to represent the signal with a minimal number of active components or features, which can reduce the computational complexity. This often involves approximating the original data with a combination of a small number of basis elements. Low-rank matrix estimation~\cite{1102314} provides another alternative, which aims is to find a matrix that is close to the original matrix in terms of some distance measure, but with a lower number of linearly independent rows or columns (\ie lower rank). Parallel processing techniques also offer a viable solution. By distributing the computation across multiple processors or using specialised hardware, the process can be expedited. These methods, while potentially sacrificing some accuracy, can dramatically lessen the computational burden. 


However, the memory requirement is still prominent with these solutions. Memory-efficient data structures, such as boom filters, which test whether an element is a member of a set are highly space-efficient, but are probabilistic and can have false positives. Sparse arrays are a common memory-efficient data structure that already has implementations in TensorFlow~\cite{tensorflow2015} and PyTorch~\cite{NEURIPS2019_9015}. In such arrays only non-zero elements are stored. This can save significant memory particularly for the layers closer to the input, as only part of the input's relevance is being reversed and therefore all parts that are not part of the feature are zero. Segmenting tensors into subparts and processing each in isolation could be a divide-and-conquer approach for memory efficiency. Each segment's relevance can be computed independently, reducing the overall memory requirements. As relevance is distributed layer by layer each segment of a layer would need to be completed before the next layer is started, which is likely to create a computational overhead.
This strategy has not been proposed before and may be difficult to implement. It is worth mentioning that the integration of emerging hardware technologies designed for high-intensity computing tasks could make reverse relevance distribution tracing a feasible method of interpretability. New processors and computing architectures, specifically tailored for AI and machine learning applications, could provide the necessary power to run these intensive computations more efficiently without any change in the implementation of the reverse relevance propagation method.

\subsection{Forward Pass Retracing}
\label{futurectc}
Chapter~\ref{chapter:REVEAL}'s novel forward pass retracing method proposes a set of rules for finding a single value of relevance to a complex input feature. These rules mimic the behaviour of each layer's function during the forward pass given the entire input. As the rules propagate a smaller portion of the input than the original activations, the rules are designed to preserve the signal during the propagation. This is particularly challenging for layers with learned parameters. The scaling of learned parameter was the biggest challenge encountered during this research. Given that learned parameters do not change after training, scaling them changes the behaviour of the network, but is a necessary step in preserving the signal of the feature being propagated. Other methods for preserving the signal of the contributions through layers may be necessary. It is worth exploring scaling the learned parameters differently across layers. An example of this can be seen by \LRP~\cite{bach2015pixel}, which proposed a composite strategy~\cite{SamekBLM17} where different rules are used at parts of the network. This is very likely to be a good research direction for improving the contribution to classification (\CTC\/) value found for a complex feature. Contributions are sparse in the layers close to the input, as all other relevances that are not part of the complex feature are zeroed, so scaling the learned parameters more heavily at these layers may reduce the issues of preserving the signal of the contributions. Layers closer to the output, due to the presence of dense and convolutional layers, have contributions distributed in a uniform way, so such layers may need less parameter scaling. Another direction worth investigating is the scaling of contributions rather than learned parameter values. This can also have a different strength across different parts of the network. The alpha beta rules proposed by \LRP\/ present another way to scale contributions. It can be a beneficial strategy if one is more interested in the positive than the negative contributions. The novel rules proposed in Chapter~\ref{chapter:REVEAL} are the ``basic rules'', which can suit as the foundation for different types of contribution scaling.


The research in this thesis primarily focuses on convolutional neural networks (CNNs) like VGG16~\cite{SimonyanZ14a}, VGG19~\cite{SimonyanZ14a}, and InceptionV3~\cite{szegedy2015rethinking}. These models have been extensively studied in this thesis due to their widespread use in image processing and computer vision tasks. However, the landscape of deep learning encompasses a much broader array of architectures, each with unique characteristics and applications. Extending beyond CNNs to gain a comprehensive understanding of the performance of the method across various models is a natural extension of the research proposed in this thesis. One promising direction is the exploration of interpretability in recurrent neural networks (RNNs). RNNs, known for their effectiveness in handling sequential data like text and time series~\cite{10.1162/neco.1997.9.8.1735}, pose a fundamentally different architecture from CNNs. The temporal aspect of RNNs makes the signal of the relevance harder to trace through the network. Issues of expanding and vanishing contributions are likely and further investigation as to how they can be managed in recurrent architectures is needed. Adapting to RNNs could provide valuable test for the interpretability offered by forward pass tracing method and the robustness of the proposed solution for 
preserving the contribution signal through the network.


Similarly, transformer models, which have revolutionised the field of natural language processing~\cite{rothman2021transformers, chernyavskiy2021transformers} and have also been applied to vision (ViTs)~\cite{dosovitskiy2021imageworth16x16words}, present another avenue for extending the forward pass retracing method. Transformers, particularly known for their self-attention mechanisms~\cite{VaswaniSPUJGKP17}, require a different approach to trace and understand the contribution of input features. Currently, methods for calculating the contribution of complex features in such layers are nonexistent. Therefore, a significant future contribution is the development and refining of rules for these specialised layers. Transformers are the largest models currently being deployed~\cite{VaswaniSPUJGKP17}, so the issues described of signal preservation and learned parameter scaling are a significant part of the difficulty in integrating forward pass retracing for such architectures. Further, the feedback loop for the development of rules for the specialised layers in these architectures would be slow and the computational resources needed for such experimentation would be large. The development of the forward pass retracing as a method for interpreting transformers would involve not only devising new rules but also validating their efficacy in providing faithful and interpretable explanations. 

Generative adversarial networks (GANs) also offer a unique landscape for interpretability research. As models that learn to generate data that is indistinguishable from real data, understanding the decision-making process in GANs could provide insights into their generative capabilities and limitations. This is particularly relevant in applications like image generation, where understanding the contribution of different features to the generated output can be of significant interest. However, applying the forward pass tracing method to these architectures is not straightforward. Each type of neural network layer, be it in RNNs, transformers, or GANs, requires specialised rules to correctly evaluate how complex input features contribute to the classification or output generation. While common layers like dense, convolutional, max pooling, average pooling, and batch normalisation have been explored in this thesis (as detailed in Chapter~\ref{chapter:REVEAL}), layers unique to RNNs and transformers pose new challenges. 

Recent advances in deep learning architectures such as 
Diffusion models~\cite{ho2020denoisingdiffusionprobabilisticmodels}, which have gained prominence in generative tasks, also represent a compelling direction for extending the forward pass tracing method. These models iteratively refine noise to generate high-quality data, such as images or audio. Understanding the interpretability of these iterative processes is critical for assessing how input noise and intermediate steps contribute to the final output. Diffusion models operate with unique mechanisms, such as noise addition and de-noising, requiring novel interpretability rules for these operations. The iterative nature of these models introduces additional complexity, as the contribution of each iteration to the final result must be traced and contextualised. 

\subsection{Evaluation}
\label{evl}
Chapter~\ref{sec:results} evaluates the forward pass retracing both qualitatively and quantitatively. It is worth exploring forward pass retracing on a wider range of networks and analysing the performance of the method across them. While the qualitative evaluation offers valuable insights into the interpretability of the method, a more thorough quantitative analysis is needed. This should involve extending the evaluation to additional networks beyond VGG16~\cite{SimonyanZ14a}. A critical aspect of this quantitative analysis is to examine the input invariance and sensitivity of the forward pass retracing method on different network architectures. Understanding how the method responds to varying inputs and its sensitivity to changes can provide crucial information about its reliability and consistency.

An interesting area of exploration is the method's response to different types and intensities of noise in the input. In the current evaluation, we assessed the method's performance with two levels of noise intensity: a lower level representing a minor perturbation and a higher level indicative of substantial noise. Further research could involve a more nuanced analysis of these noise levels. By calculating and comparing the differences in explanations under varying noise intensities, we can gain insights into the threshold at which noise significantly impacts the explanations. This analysis can be crucial for understanding the robustness of the method and determining its reliability under different conditions. Moreover, it is important to evaluate how the significance of noise varies across different network architectures. The point at which noise begins to substantially affect the explanation might differ from one network to another, depending on factors like network depth, complexity, and the type of data it is trained on. 

\subsection{Interpretability and Usability}
\label{study}
This thesis is dedicated to enhancing the interpretability of explanations provided by deep neural networks. It draws on research underscoring human cognitive limitations, notably the challenge in extracting meaningful insights when confronted with more than five significant items simultaneously, as highlighted in seminal works by Cowan~\textit{et al.}~\cite{cowan2001magical}, Starkey~\textit{et al.}~\cite{starkey1995development}, and Morris~\textit{et al.}~\cite{morris2018human}. Addressing these cognitive constraints is central to the thesis. To empirically validate the effectiveness of the proposed interpretability method, a structured user study is proposed here as a direction for future work. This study intricately examines how different interpretability methods influence user comprehension and decision-making. Participants, selected to represent a diverse range of familiarity with neural networks, will be presented with a series of explanations generated by various interpretability methods.

Each participant will engage in a task where they are required to rank the importance of different areas of the input data, based on the explanation they are provided. This task is designed to be intuitive yet comprehensive, ensuring that participants of varying expertise levels can understand and effectively engage with the material. To facilitate this, a brief training session and clear instructions will be provided before the commencement of the study. The core assessment metric in this study is the consistency of the areas chosen by the participants as being of biggest importance to the network as identifyed by the interpretability method. This consistency will be quantitatively evaluated, with statistical methods applied to analyse the rankings across different participants. The interpretability method that yields the highest consistency in the ranking among participants will be considered the most effective. Furthermore, the study will include a qualitative component where participants can provide feedback on each explanation method. This feedback will be instrumental in understanding the subjective aspects of interpretability, such as clarity, relevance, and overall user satisfaction. 


To enhance the usability and accessibility of neural network interpretability methods for a broader audience, including those without deep technical expertise, the development of intuitive, user-friendly interfaces and visualisation tools is essential. The goal is to demystify the complex internal workings of neural networks and make them more transparent and understandable to non-experts. Future work in this area could focus on designing interactive platforms which allow users to select specific parts of the input data they are curious about, and then find out how much this part contributes to the model's output by employing the methods described in the thesis. For instance, in an image recognition task, a user could select a particular object in the image and receive a visualisation showing how that object influenced the model's classification decision. Such tools would not only make it easier for users to grasp the model's reasoning but also promote a more hands-on approach to explainability.



\section{Final Thoughts}
As we navigate the rapidly evolving realm of artificial intelligence, the importance of interpretability in AI systems cannot be overstated. The development of more powerful computers and the explosion of data available for analysis allowed AI to grow in sophistication and practicality.

As AI systems become more advanced and autonomous, the implications of their operations grow in complexity and importance. The research in this thesis is aligned with the global movement towards explainable AI, advocating for systems that are not just efficient and powerful, but also transparent and fair. As AI continues to evolve and become an ever-growing part of our lives, the work done in this thesis contributes to a future, where AI is not an inscrutable black box, but a transparent, understandable, and trustworthy tool used to further progress and innovation.

%%%%%%%%%%%%%%%%%%%%%%%%%%%%%%%%%
% References
%%%%%%%%%%%%%%%%%%%%%%%%%%%%%%%%%
\bibliographystyle{plain}
\bibliography{references.bib}
%%%%%%%%%%%%%%%%%%%%%%%%%%%%%%%%%
% Appendices
%%%%%%%%%%%%%%%%%%%%%%%%%%%%%%%%%
\appendix
\include{Appendices/appendix}
\include{Appendices/UserGuide}
\include{Appendices/SourceCode}
\end{document}
